% !TeX root = ../tfg.tex
% !TeX encoding = utf8

\chapter{Fundamentos del álgebra homológica}
La teoría de homología es una rama de la topología que trata de resolver
problemas topológicos en el ámbito del álgebra. Por este motivo es importante conocer
muy bien algunas herramientas algebraicas que iremos utilizando con frecuencia.
En todo el capítulo usaremos como referencia principal
\cite{maclane2012homology}.

\section{Módulos}

La estructura de módulo surge con la idea de generalizar el concepto de espacio vectorial
sobre un cuerpo a un anillo. Nuestro interés en ellos radica en que la teoría de
homología se construye sobre módulos y por ello es necesario hacer una
introducción al campo. Esta sección recoge algunas definiciones y resultados de interés
vistos en la asignatura de Álgebra Moderna y complementada con los contenidos de
\cite{dummit2004abstract}.

\begin{definicion}
	Sea \(R\) un anillo con elemento identidad \(1 \neq 0\). Un \textbf{\(R\)-módulo
		izquierdo} \(M\) es un grupo abeliano aditivo junto con una función \(p: R \times
	M \rightarrow M\) con \((r, m) \to rm\) tal que dados \(r,r' \in R\), \(m,m' \in A\)
	se tiene
	\begin{enumerate}
		\item \((r+r') m = r m + r' m\),
		
		\item \((rr') m = r (r' m)\),
		
		\item \(r(m+m') = rm + rm'\),
		
		\item \(1m = m\).
	\end{enumerate}
\end{definicion}

De la definición anterior se sigue que \(0m = 0\) y \((-1)m = -m\).

De manera análoga, definimos \textbf{\(R\)-módulo derecho} donde el anillo actúa
por la derecha en vez de por la izquierda de forma que
\(p: M \times R \rightarrow M\). Si \(R\) es un anillo conmutativo, los \(R\)-módulos
izquierdos y derechos coinciden y les llamamos simplemente \(R\)-módulos. Como los
resultados de \(R\)-módulos izquierdos y derechos son análogos, trabajaremos con
los \(R\)-módulos izquierdos y nos referiremos a ellos como \textbf{\(R\)-módulos} o
\textbf{módulos} a menos que se indique explícitamente lo contrario.

\begin{ejemplo}
	El interés de los \(R\)-módulos subyace en la cantidad de estructuras conocidas que
	engloba. Si por ejemplo consideramos el \(K\)-módulo donde \(K\) es un cuerpo, éste
	adquiere la estructura de \textbf{espacio vectorial}. Ahora sea \(M\) un \(\mathbb{Z}\)-módulo.
	Definimos el producto \(p\) de forma que para \(n \in \mathbb{Z}\) y \(m \in M\) con
	\(n>0\), \(nm = m + m + \ldots + m\) (\(n\) veces), \(0m=0\) y \((-n)m = -(nm)\). Entonces
	\(M\) ha de tener estructura de \textbf{grupo abeliano}. En particular, si \(R\)
	es un anillo entonces es también un \(R\)-módulo.
\end{ejemplo}

\begin{definicion}
	Sea \(M\) un \(R\)-módulo izquierdo y \(N\) un subconjunto de \(M\). Diremos que \(N\)
	es un \textbf{submódulo} de \(M\), esto es, \(N \subset M\), si \(N\) es cerrado
	respecto a la suma y si \(r \in R\), \(n \in N\) entonces \(rn \in N\).
\end{definicion}
De la definición anterior se deduce que \(N\) es un \(R\)-módulo.
\begin{definicion}
	Sea \(R\) un \(R\)-módulo. Si un submódulo de \(R\) es un subconjunto \(I \subset R\)
	cerrado respecto a la suma tal que
	\(\langle I \rangle = \{ri \ : \ i \in I\} \subset I\) para todo \(r \in R\), lo llamaremos
	\textbf{ideal} de \(R\). En particular, si \(I\) consta de un único elemento
	\(i \in I\), diremos que es el \textbf{ideal generado por \(i\)} y lo denotaremos
	por \(\langle i \rangle\).
\end{definicion}
%Tomando un ideal izquierdo \(I\) de \(R\) y un \(R\)-módulo izquierdo \(M\), definimos
%el producto del ideal \(I\) por el módulo \(M\)
%\[
%	IM = \left\lbrace \sum_{j=0}^{n} i_{j}m_{j} \ : \ i_{j} \in I,\ m_{j} \in M,\ n
%	\in \N \right\rbrace
%\]
%donde \(IM\) es un submódulo de \(M\). En particular, el producto de dos ideales
%izquierdos \(II'\) es también un ideal izquierdo y \((II')M = I(I'M)\).

\begin{definicion}
	Sea \(M\) un \(R\)-módulo y sea \(m \in M\). El conjunto
	\(\langle m \rangle = \{r m : r \in R\}\) es un submódulo de \(M\) que
	denominaremos \textbf{submódulo cíclico generado por \(m\)}.
\end{definicion}
\begin{observacion}
	Nótese que si \(R\) es un \(R\)-módulo, el submódulo cíclico generado por un
	elemento es el ideal generado por el mismo elemento.
\end{observacion}

\begin{definicion}
	Sea \(M\) un \(R\)-módulo y sean \(S\) un subconjunto de \(M\). Sea
	\(\langle S \rangle\) el submódulo formado por la intersección de todos los
	submódulos de \(M\) que contienen a \(S\). Diremos entonces que \(\langle S \rangle\)
	es el \textbf{submódulo generado por \(S\)} y los elementos de $S$ los llamaremos \textbf{generadores} de \(\langle S \rangle\).
\end{definicion}

\begin{definicion}
	Sea \(M\) un \(R\)-módulo. Un submódulo \(N\) de \(M\) es \textbf{finitamente generado}
	si existe un subconjunto finito \(S \subset M\) tal que \(N = \langle S \rangle\).
\end{definicion}

\begin{lema}
	Sea \( M \) un \( R \)-módulo. \( M \) es finitamente generado por \( n \) elementos si, y sólo si, existe un epimorfismo \( \phi : R^n \to M \).
\end{lema}
%\begin{proof}
%	Sea \( M \) un módulo generado por un conjunto finito de elementos \( \{m_1, \ldots, m_n\} \). Consideremos el homomorfismo \( \phi: R^n \rightarrow M \) definido por
%	\[
%	\phi(r_1, \ldots, r_n) = \sum_{i=1}^n r_i m_i.
%	\]
%	Este homomorfismo \( \phi \) es claramente sobreyectivo, ya que cada elemento \( m \) en \( M \) puede ser expresado como \( \phi(r_1, \ldots, r_n) \) para algunos \( r_1, \ldots, r_n \in R \).
%	
%	Por otro lado, si existe un homomorfismo sobreyectivo \( \phi: R^n \rightarrow M \), entonces, para cada \( m \in M \) existe una \( n \)-tupla \( (r_1, \ldots, r_n) \) en \( R^n \) tal que \( \phi(r_1, \ldots, r_n) = m \). Los elementos \( \phi(e_i) \), donde \( e_i \) es el \( i \)-ésimo vector de la base canónica de \( R^n \), generan \( M \). De aquí se sigue que \( M \) es finitamente generado.
%\end{proof}

\begin{definicion}
	Sea \( M \) un \( R \)-módulo finitamente generado por \( n \) elementos y sea \( \phi : R^n \to M \) un epimorfismo. Diremos que \( M \) es \textbf{finitamente presentado} si \( \ker \phi \) es finitamente generado.
\end{definicion}

\begin{definicion}
	Sean \(M\), \(N\) \(R\)-módulos. Definimos el \textbf{homomorfismo de \(R\)-módulos}
	de \(M\) a \(N\) como la aplicación \(\alpha: M \rightarrow N\) tal que
	\begin{enumerate}
		\item \(\alpha(m+m') = \alpha(m) + \alpha(m')\),
		
		\item \(\alpha(rm) = r\alpha(m)\)
	\end{enumerate}
	para todo \(m,m' \in M\), \(r \in R\).
\end{definicion}

Cuando \(\alpha: M \rightarrow N\) sea un homomorfismo de \(R\)-módulos, diremos que
\(M\) es el \textbf{dominio} y \(N\) el \textbf{rango}. La \textbf{imagen} de
\(\alpha\) es el conjunto \(\im(\alpha) = \{ \alpha(m) : m \in M \}\). El \textbf{núcleo}
será el conjunto de elementos que se anulan en su imagen, esto es,
\(\ker(\alpha) = \{ m \in M : \alpha(m) = 0 \}\). Diremos que \(\alpha\) es un \textbf{epimorfismo}
cuando \(\alpha\) sea sobreyectiva, un \textbf{monomorfismo} cuando \(\alpha\) sea
inyectiva y un \textbf{isomorfismo} si \(\alpha\) es un epimorfismo y un monomorfismo
a la vez. Si existe un isomorfismo entre \(M\) y \(N\) diremos que son \textbf{isomorfos}
y lo notaremos \(A \cong B\). Un homomorfismo \(\alpha: M \rightarrow M\) lo
llamaremos \textbf{endomorfismo}.

Dado que el núcleo y la imagen de un homomorfismo de \(R\)-módulos coincide con el
de los grupos abelianos subyacentes, la siguiente caracterización es inmediata
de la ya conocida para grupos:

\begin{proposicion}
	Sea \(\alpha: M \to N\) un homomorfismo de \(R\)-módulos. Entonces
	\begin{enumerate}
		\item \(\alpha\) es un monomorfismo si, y sólo si, \(\ker(\alpha) = 0\).
		
		\item \(\alpha\) es un epimorfismo si, y sólo si, \(\im(\alpha) = N\).
	\end{enumerate}
\end{proposicion}

Es frecuente escribir el homomorfismo de \(R\)-módulos \(\alpha: M \rightarrow N\) como
\(M \xrightarrow{\alpha}N\). Respecto a la notación de la imagen de un elemento
\(m \in M\) por \(\alpha\), pondremos \(\alpha(m)\) o simplemente \(\alpha m\). En
cuanto a la imagen de \(A\) por \(\alpha\), lo representaremos de manera análoga por
\(\alpha(M)\) o \(\alpha M\).

Dados dos homomorfismos de \(R\)-módulos \(\alpha_{1}, \alpha_{2}: M \rightarrow N\),
su \textbf{suma} \(\alpha_{1}+ \alpha_{2}\) la definimos como \((\alpha_{1}+ \alpha_{2}
)(m) = \alpha_{1}(m) + \alpha_{2}(m)\) para todo \(m \in M\). Además, dados dos homomorfismos
de \(R\)-módulos \(\alpha: M \rightarrow N\), \(\beta: N \rightarrow P\), su \textbf{composición}
\(\beta \circ \alpha: M \rightarrow P\) es también un homomorfismo de \(R\)-módulos.
Nótese que para que la composición sea posible, el rango de \(\alpha\) tiene que
ser igual al dominio de \(\beta\). En ocasiones usaremos la notación por yuxtaposición
\(\alpha\beta = \alpha \circ \beta\). Llamaremos \textbf{inversa} (por ambos lados)
de \(\alpha : M \rightarrow N\) al homomorfismo \(\alpha^{-1}: N \rightarrow M\) tal
que \(\alpha^{-1}\circ \alpha = \id_{M}\) y \(\alpha \circ \alpha^{-1}= \id_{N}\).
Una \textbf{inversa izquierda} de \(\alpha\) es una función \(\gamma: N \rightarrow
M\) tal que \(\gamma \circ \alpha = \id_{M}\). De manera análoga, el homomorfismo \(\theta
: M \to N\) es \textbf{inversa derecha} de \(\alpha\) si
\(\alpha \circ \theta = \id_{N}\).

Si \(T \subseteq N\), el conjunto \(\alpha^{-1}T = \{ m \in M \ : \ \alpha(m) \in T
\}\) es un submódulo de \(M\), llamado la \textbf{imagen inversa} (completa) de \(T\).
En particular, \(\ker \alpha = \alpha^{-1}0\), donde \(0\) denota el submódulo de
\(N\) que consiste solo en el elemento cero.

Sea \(T \subseteq N\) donde \(N\) es un \(R\)-módulo, llamaremos \textbf{inclusión} o
\textbf{inyección canónica} al homomorfismo \(i: T \rightarrow N\) tal que \(i(t) =
t\) para todo \(t \in T\). En particular, \(i\) es un monomorfismo. Las \textbf{clases
	laterales} de \(T\) en \(N\) son los conjuntos \(n + T = \{n + t : t \in T\}\) donde \(n
\in N\). Dos clases laterales \(n_{1}+ T\), \(n_{2}+ T\) son iguales si \(n_{1}- n_{2}\in
T\). Como \(T\) es un submódulo, el grupo abeliano \(N/T\) se convierte en un \(R\)-módulo
cuando \(r(n+T) = rn + T\) para todo \(r \in R\). A este \(R\)-módulo lo llamaremos el
\textbf{módulo cociente} de \(N\) sobre \(T\). El homomorfismo
\(\pi: N \rightarrow N/ T\) tal que \(\pi(n) = n + T\) es un epimorfismo que
llamaremos \textbf{proyección canónica} de \(N\).

\begin{proposicion}
	[Teorema de factorización] \label{prop:teo-fact} Sea \(\beta: M \rightarrow M'\)
	un homomorfismo de módulos con \(T \subset \ker \beta\). Existe entonces un
	único homomorfismo de módulos \(\beta': M/T \rightarrow M'\) con
	\(\beta'\pi = \beta\); es decir, el siguiente diagrama con \(\beta(T) = 0\)
	\begin{equation}
		\label{firstisothm}\xymatrix{ N \ar[r]^\pi \ar[dr]^\beta & M/T \ar[d]^{\beta'} \\ & M' }
	\end{equation}
	es conmutativo. Al homomorfismo \(\beta'\) lo llamaremos \textbf{homomorfismo
		inducido} por \(\beta\).
\end{proposicion}
%\begin{proof}
%	Consideremos el \(R\)-módulo \(B\) y definamos \(\beta'(b + T) = \beta(b)\). Por
%	estar \(T\) contenida en el núcleo de \(\beta\), la función está bien definida. En
%	efecto, si \(a,b \in B\) entonces \(a+T = b+T\) lo que implica que \(a-b \in T \subset
%	\ker \beta\). Esto es, \(\beta(a-b) = 0\) y por tanto, \(\beta(a)=\beta(b)\). Como \(\beta\)
%	es un homomorfismo,
%	\[
%		\beta'((ra + sb) + T) = \beta(ra+sb) = r\beta(a) + s\beta(b) = r \beta'(a+T)
%		+ s \beta'(b+T)
%	\]
%	para todo \(r,s \in R\), \(a,b \in B\). En consecuencia, \(\beta'\) también es un homomorfismo
%	de \(R\)-módulos.
%\end{proof}
%En particular, si \( \beta: B \rightarrow B' \) es un epimorfismo con núcleo \( T \), \( \beta': B/T \rightarrow B' \) es un isomorfismo. Esta afirmación puede expresarse de la siguiente manera: cada \( \beta \) con \( \beta(T) = 0 \) \textit{factoriza de manera única} a través de la proyección \( \pi \). Esta propiedad caracteriza a \( \pi: B \rightarrow B/T \) hasta un isomorfismo de \( B/T \), de la siguiente manera:
%
%\begin{proposicion}
%	Si \( T \subset B \) y \( \eta: B \rightarrow D \) es tal que \( \eta(T) = 0 \) y cada \( \beta: B \rightarrow B' \) con \( \beta(T) = 0 \) factoriza de manera única a través de \( \eta \), entonces hay un isomorfismo \( \theta: B/T \rightarrow D \) con \( \theta \pi = \eta \).
%\end{proposicion}
%
%\begin{proof}
%	Factorizamos \(\eta\) a través de \(\pi\) y \(\pi\) a través de \(\eta\), así que \(\eta = (\eta' \pi) \eta = 1_\eta\). Pero \(\eta\) factoriza \textit{únicamente} a través de \(\pi\), así que \(\eta' \pi = 1\). Simétricamente, \(\pi' \eta = 1\). Por lo tanto \(\pi' = (\eta')^{-1}\) y \(\eta'\) es el isomorfismo deseado \(\theta\).
%\end{proof}
%
\begin{teorema}
	[Primer teorema de isomorfía] \label{teo:first-iso} Sea \(\beta : M \to M'\) un homomorfismo
	de \(R\)-módulos. Entonces
	\[
	\frac{M}{\ker \beta}\cong \im \beta.
	\]
\end{teorema}
%\begin{proof}
%	La demostración es análoga a la ya vista para grupos. Construimos un
%	homomorfismo de \(R\)-módulos \(\beta: B \to \im \beta\) donde aplicando el
%	\nameref{prop:teo-fact} con \(T = \ker \beta\), obtenemos el homomorfismo
%	inducido \(\beta' : B/\ker \beta \to \im \beta\) tal que
%	\(\beta'(b+\ker \beta) = \beta(b)\). Veamos que está bien definida y que es biyectiva.
%	Para ello, consideremos que existe otro \(c \in B\) tal que \(b + \ker B = c + \ker
%	B\) lo cual se cumple si, y sólo si, \((b + \ker B) - (c + \ker B) = (b - c) + \ker
%	\beta = 0\). Esto es, que \(b - c \in \ker \beta\) y por tanto \(\beta(b-c) = \beta
%	(b) - \beta(c) = 0\). Es decir, \(\beta(b) = \beta(c)\). En consecuencia, la aplicación
%	\(\beta'\) está bien definida y además es inyectiva. Claramente
%	\(\beta: B \to \im \beta\) es sobreyectiva.
%\end{proof}
%Cada homomorfismo \(\alpha: A \rightarrow B\) determina dos módulos cociente
%\[\text{Coim } \alpha = A / \text{Ker } \alpha, \quad \text{Coker } \alpha = B / \text{Im } \alpha,\]
%llamados la \textbf{coimagen} y el \textbf{conúcleo} de \(\alpha\). Esta definición provee dos sucesiones exactas cortas
%\[ \text{Ker } \alpha \hookrightarrow A \twoheadrightarrow \text{Coim } \alpha, \quad \text{Im } \alpha \hookrightarrow B \twoheadrightarrow \text{Coker } \alpha, \]
%un isomorfismo \(\text{Coim } \alpha \cong \text{Im } \alpha\) y una sucesión exacta más larga
%\[ 0 \rightarrow \text{Ker } \alpha \xrightarrow{i} A \xrightarrow{\alpha} B \rightarrow \text{Coker } \alpha \rightarrow 0. \]
%Por la \autoref{prop:first_iso}, \(\beta \alpha = 0\) implica que \(\beta\) factoriza de manera única a través de \(\pi\) como \(\beta = \beta' \pi\). Dualmente, si algún \(\gamma': A' \rightarrow A\) tiene \(\alpha \gamma' = 0\), entonces \(\gamma'\) factoriza a través de \(i\) como \(\gamma' = i \gamma''\) para un único \(\gamma'': A' \rightarrow \ker \alpha\). Esta propiedad caracteriza \(i: \ker \alpha \rightarrow A\) como un isomorfismo de \(\text{Ker } \alpha\). Observa las afirmaciones duales: \(\alpha\) es un monomorfismo si y solo si \(\ker \alpha = 0\), y es un epimorfismo si y solo si \(\text{Coker } \alpha = 0\).
%
%Para \(K \subseteq S \subseteq A\) el módulo \(S/K\) es llamado un \textbf{subcociente} de \(A\). Este es, a su vez, un módulo cociente del submódulo \(S\) de \(A\) y un submódulo del módulo cociente \(A/K\). Además, si \(K' \subseteq K \subseteq S' \subseteq S \subseteq A\), entonces \(K'/K\) es un submódulo de \(S'/K\) y la proyección compuesta \(S' \rightarrow (S'/K)/(K'/K)\) tiene núcleo \(K'\), por lo tanto el isomorfismo familiar \((S'/K)/(K'/K) \cong S'/K'\). Esto nos permite escribir cada subcociente \((S'/K)/(K'/K)\) de un subcociente \(S/K\) directamente como un subcociente de \(A\). Si \( \alpha: A \rightarrow A'\) tiene \(\alpha S \subseteq S'\) y \(\alpha K \subseteq K'\), entonces \(\alpha s + K'\) es una clase lateral de \(S'/K'\) determinada de manera única por la clase lateral \(s+K\) de \(S/K\). Por lo tanto \(\alpha_{\ast}(s+K) = \alpha s + K'\) define un homomorfismo
%\[\alpha_{\ast}: S/K \rightarrow S'/K'\]
%\[(\alpha S \subseteq S', \alpha K \subseteq K')\]
%llamado el homomorfismo \textbf{inducido} por \(\alpha\) en los subcocientes dados.
%
%Si \(S\) y \(T\) son submódulos de \(A\), su \textbf{intersección} \(S \cap T\) (como conjuntos) es también un submódulo, así como su \textbf{unión} \(S + T\), consistiendo de todas las sumas \(s + t\) para \(s \in S\), \(t \in T\). El \textbf{teorema del isomorfismo de Noether} afirma que \(1_{A}\) induce un isomorfismo
%
%\[ 1_{\ast}: S/(S \cap T) \cong (S + T)/T. \]
%\subsection{Módulos libres}

\begin{definicion}
	Sea \(\{M_{i}\}_{i \in I}\) una familia de \(R\)-módulos indexada por \(I\).
	Definimos el \textbf{producto directo} o \textbf{producto directo externo} de
	\(\{M_{i}\}_{i \in I}\) como el producto cartesiano
	\[
	\prod_{i \in I}M_{i}= \{(x_{i})_{i \in I}: x_{i}\in M_{i}\},
	\]
	donde las operaciones se definen componente a componente:
	\begin{align*}
		(x_{i})_{i \in I}+ (y_{i})_{i \in I} & = (x_{i}+y_{i})_{i \in I}, \\
		r(x_{i})_{i \in I}                   & = (rx_{i})_{i \in I},
	\end{align*}
	para todo \(r \in R\), \(x_{i},y_{i}\in M_{i}\), \(i \in I\).
\end{definicion}

\begin{definicion}
	Sea \(\{M_{i}\}_{i \in I}\) una familia de \(R\)-módulos indexada por \(I\).
	Definimos la \textbf{suma directa} o \textbf{suma directa interna} de
	\(\{M_{i}\}_{i \in I}\) como el submódulo de \(\prod_{i \in I}M_{i}\) tal que
	\[
	\bigoplus_{i \in I}M_{i}= \{(x_{i})_{i \in I}: x_{i}= 0 \ \text{p.c.t.}\ i \in
	I\}.
	\]
\end{definicion}
\begin{nota}
	Recordemos que una condición se cumple "para casi todo" (p.c.t.) elemento de
	un conjunto si se cumple para todo elemento en él salvo en un subconjunto
	finito de elementos.
\end{nota}
\begin{definicion}
	Sea \(B\) un conjunto y sea \(M\) el \(R\)-módulo tal que \(M = \bigoplus_{b \in B}R_{b}\)
	donde \(R_{b}=R\) para todo \(b \in B\). Llamaremos a dicho \(R\)-módulo el \textbf{\(R\)-módulo
		libre de base \(B\)} y lo notaremos por \(R \langle B \rangle\). De esta forma, cada
	\(x \in R \langle B \rangle\) se representa por \(x = \sum_{b\in B}\lambda_{b} b\) donde \(\lambda_{b}\in R\) son coeficientes no nulos en un número finito de posiciones
	\(b\).
\end{definicion}

\begin{teorema}
	[Propiedad universal de los módulos libres] \label{teo:univ-prop-free-mod} Sean
	\(B\) un conjunto, \(M\) un \(R\)-módulo y \(\varphi : B \to M\) una aplicación entre
	conjuntos. Entonces existe un único homomorfismo de \(R\)-módulos \(\phi : R \langle
	B \rangle \to M\) de forma que \(\phi(b) = \varphi(b)\) para todo \(b \in B\). Es
	decir, el diagrama
	\[
	\xymatrix{ B \ar@{->}[r]^{\varphi} \ar@{->}[d]^{i} & M \\ R \langle B \rangle \ar@{->}[ru]^{\phi} & }
	\]
	conmuta.
\end{teorema}

\begin{definicion}
	Sea \(M\) un \(R\)-módulo libre. Si para toda base \(B\) de \(M\), \(B\) tiene la misma cardinalidad,
	entonces decimos que \(M\) tiene \textbf{rango} \(\rg \, M = \# B\), donde \(\# B\)
	es la cardinalidad alguna base de \(M\).
\end{definicion}
\begin{definicion}
	Sea \(x\) un elemento de un \(R\)-módulo. Decimos que \(x\) es un\textbf{ elemento de
		torsión} si existe un \(r \in R \setminus \{0\}\) tal que \(rx = 0\). Por otro lado,
	\(x\) es un \textbf{elemento sin torsión} si el único elemento \(r \in R\) que satisface
	\(rx = 0\) es \(r = 0\). Un \(R\)-módulo se clasifica como \textbf{módulo de torsión}
	si cada uno de sus elementos es un elemento de torsión. Análogamente, un \textbf{módulo
		sin torsión} es aquel cuyos elementos no nulos son elementos sin torsión.
\end{definicion}

\begin{definicion}
	Sea \(M\) un \(R\)-módulo. Definimos el \textbf{anulador de \(M\)} como el submódulo
	\(\ann(M) = \{r \in R : rm=0, \ \forall m \in M\}\). De manera análoga, llamaremos
	\textbf{anulador de \(m \in M\)} al submódulo
	\(\ann(M) = \{r \in R : rm=0, \ \forall m \in M\}\).
\end{definicion}

\begin{definicion}
	Definimos el \textbf{submódulo de torsión} de un \(R\)-módulo \(M\) como el
	conjunto \(\tor(M)= \{x \in M : \ann(x) \neq \{0\}\}\). Es decir, el conjunto de
	todos los elementos de torsión de \(M\).
\end{definicion}
\begin{teorema}
	[Descomposición cíclica primaria] \label{teo:desc-cicl-prim} Sea \(R\) un dominio de ideales principales y sea
	\(M\) un \(R\)-módulo finitamente generado. Entonces \(M\) se descompone como la
	suma directa
	\[
	M \cong R^{f}\oplus \bigoplus_{i = 1}^{k}\frac{R}{\langle r_{i}\rangle}
	\]
	donde \(R^{f}\) es un módulo libre de rango \(f\) y
	\(R/\langle r_{1}\rangle, \ldots, R/\langle r_{k}\rangle\) son módulos cíclicos
	con anuladores \(\langle r_{1}\rangle, \ldots, \langle r_{k}\rangle\). Además,
	\(f\) y los ideales \(\langle r_{1}\rangle, \ldots, \langle r_{k}\rangle\) de \(R\)
	generados por \(r_{1}, \ldots, r_{k}\in R\) están determinados de manera única
	salvo el orden por \(M\).
\end{teorema}
\begin{ejemplo}
	Consideremos un espacio vectorial $V$. Entonces podemos considerar, por ejemplo, su base canónica y generar todo $V$ a partir de ella. En consecuencia, $V$ es un módulo libre. Además, aplicando el teorema de \nameref{teo:desc-cicl-prim}, es claro que la parte libre de la descomposición es isomorfa a $V$ y por tanto $V$ carece de parte cíclica.
\end{ejemplo}

\section{Sucesiones exactas}

\begin{definicion}
	Sea \(\{A_{i}, \alpha_{i}\}_{i \in \mathbb{Z}}\) una familia de \(R\)-módulos \(A_{i}\)
	y homomorfismos entre ellos tal que \(\alpha_{i}: A_{i}\rightarrow A_{i+1}\).
	Diremos que la sucesión
	\[
	\cdots \xrightarrow{\alpha_{i-2}}A_{i-1}\xrightarrow{\alpha_{i-1}}A_{i}\xrightarrow
	{\alpha_{i}}A_{i+1}\xrightarrow{\alpha_{i+1}}\cdots
	\]
	es \textbf{exacta} en \(A_{i}\) cuando \(\im \alpha_{i}= \ker \alpha_{i+1}\). Si la sucesión es exacta para todo $i \in \Z$, diremos que la sucesión es \textbf{exacta larga} o simplemente exacta.
\end{definicion}

\begin{definicion}
	Sean \(A,B\) y \(C\) \(R\)-módulos y \(\sigma: A \rightarrow B\),
	\(\gamma: B \rightarrow C\) homomorfismos entre ellos. Diremos que la \textbf{sucesión
		exacta} es \textbf{corta} si
	\[
	(\sigma, \gamma): 0 \rightarrow A \xrightarrow{\sigma}B \xrightarrow{\gamma}C
	\rightarrow 0.
	\]
	Es decir, una sucesión exacta de cinco \(R\)-módulos con los dos módulos
	exteriores siendo cero (y por lo tanto las dos funciones exteriores triviales).
\end{definicion}

\begin{proposicion}
	Sean \(A,B\) y \(C\) \(R\)-módulos y \(\sigma: A \rightarrow B\),
	\(\gamma: B \rightarrow C\) homomorfismos entre ellos. Entonces
	\begin{enumerate}
		\item La sucesión \(0 \to A \overset{\sigma}{\to}B\) es exacta (en \(A\)) si, y
		sólo si, \(\sigma\) es inyectiva.
		
		\item La sucesión \(B \to C \overset{\gamma}{\to}0\) es exacta (en \(C\)) si, y
		sólo si, \(\gamma\) es sobreyectiva.
	\end{enumerate}
\end{proposicion}
\begin{proof}
	El único homomorfismo que cumple \(0 \to A\) tiene imagen \(0\) en \(A\) y por tanto,
	el núcleo de \(\sigma\) será este si, y sólo si, \(\sigma\) es inyectiva. De manera
	similar, el único homomorfismo \(C \to 0\) es el homomorfismo nulo para todo
	elemento de \(C\), que es la imagen de \(\gamma\) si, y sólo si, \(\gamma\) es sobreyectiva.
\end{proof}
\begin{corolario}
	La sucesión \(0 \rightarrow A \xrightarrow{\sigma}B \xrightarrow{\gamma}C \rightarrow
	0\) es exacta si, y sólo si, \(\sigma\) es inyectiva, \(\gamma\) es sobreyectiva y
	\(\im \sigma = \ker \gamma\).
\end{corolario}

% PUEDE MEJORARSE CON DUMMIT PROPOSICION 22 DE MODULOS
Como acabamos de probar, la exactitud en \(A\) significa que \(\sigma\) es un monomorfismo,
en \(B\) significa que \(\sigma A = \ker \gamma\) y en \(C\) que \(\gamma\) es un
epimorfismo. Así la sucesión exacta corta puede escribirse como \(A \xrightarrow{\sigma}
B \xrightarrow{\gamma}C\), con exactitud en \(B\). Ahora \(\sigma\) induce un isomorfismo
\(\sigma': A \to \sigma A\) y \(\gamma\) un isomorfismo \(\gamma': B/\sigma A \to C\);
juntos estos proveen un isomorfismo de sucesiones exactas cortas, en la forma de
un diagrama conmutativo
\begin{equation}
	\label{exactshortseq-equiv}\xymatrix{ 0 \ar[r] & A \ar[r]^{\sigma} \ar[d]^{\sigma'} & B \ar[r]^\gamma \ar@{=}[d] & C \ar[r] \ar[d]^{(\gamma')^{-1}} & 0 \\ 0 \ar[r] & \sigma A \ar[r]_i & B \ar[r] & B/\sigma A \ar[r] & 0. }
\end{equation}

En resumen, una sucesión exacta corta es simplemente otro nombre para un submódulo
y su cociente.

\begin{ejemplo}
	Respecto al \nameref{prop:teo-fact}, la inclusión \(i\) y la proyección \(\pi\) producen
	una sucesión exacta corta.
	\[
	0 \rightarrow T \xrightarrow{i}M \xrightarrow{\pi}M/T \rightarrow 0.
	\]
\end{ejemplo}
%
%\begin{definicion}
%	Sea \(0 \rightarrow A \xrightarrow{\sigma} B \xrightarrow{\gamma} C \rightarrow 0\) una sucesión exacta corta. Diremos que la sucesión es \textbf{escindible} si existe un homomorfismo \(\rho: C \to B\) tal que \(\gamma \circ \rho = \id_C\).
%\end{definicion}
%\begin{observacion}
%	En las condiciones anteriores, \(\rho\) tiene que ser inyectiva.
%\end{observacion}
%
%\begin{lema}[Lema de escisión]
%	\label{lem:escision}
%	Sea \(0 \rightarrow A \xrightarrow{\sigma} B \xrightarrow{\gamma} C \rightarrow 0\) una sucesión exacta corta. Son equivalentes las siguientes afirmaciones
%	\begin{enumerate}
	%		\item La sucesión es escindible.
	%		\item B = \(\im \sigma \oplus C'\) donde \(C'\) es un submódulo de \(B\) isomorfo a \(C\). En particular, \(B \cong A \oplus C\).
	%		\item Existe \(\eta : B \to A\) tal que \(\eta \circ \sigma = \id_A\).
	%	\end{enumerate}
%\end{lema}
%\begin{proof}
%	\((a) \implies (b)\). Por ser la sucesión escindible el diagrama
%	\[
%		\xymatrix{
	%			C \ar@{->}[r]^{\rho} \ar@/_/@{->}[rr]_{\id} & B \ar@{->}[r]^{\gamma} & C
	%		}
%	\]
%	implica que \(B = \im \rho \oplus \ker \gamma\) ya que \(x = \rho(\gamma(x)) + (x - \rho(\gamma(x)))\) para todo \(x \in B\). En consecuencia, \(B = C' \oplus \im \rho\). Además, por la exactitud de la sucesión deducimos que \(\ker \gamma = \sigma(A)\).
%
%	\((b) \implies (c)\). Es sufiente considerar \(B \overset{\pi}{\to} \sigma(A) \overset{\sigma^{-1}}{\to}\) donde \(\pi\) es la proyección canónica. Por ser la sucesión exacta, \(\sigma\) es inyectiva.
%
%	\((c) \implies (a)\). Siguiendo un razonamiento análogo al anterior, el diagrama
%	\[
%	\xymatrix{
	%		A \ar@{->}[r]^{\sigma} \ar@/_/@{->}[rr]_{\id} & B \ar@{->}[r]^{\eta} & A
	%	}
%	\]
%	implica que \(B = \sigma(A) \oplus \ker \eta\). Por el \nameref{teo:first-iso} y como \(\ker \gamma = \sigma(A)\), tenemos que \(\ker \eta \cong B/\ker(\gamma) \overset{\overline{\gamma}}{\to} \im \gamma = C\) donde \(\overline{\gamma}\) es el isomorfismo inducido por \(\gamma\). Finalmente, \(\rho = \overline{\gamma}^{-1}\).
%\end{proof}
%
%Para realizar la prueba del siguiente resultado usaremos una estrategia típica en
%álgebra homológica conocida como persecución de diagramas. Este método consiste
%en buscar composiciones de aplicaciones equivalentes en diagramas conmutativos, y
%en explotar las propiedades de los homomorfismos inyectivos, sobreyectivos y
%biyectivos, así como de las sucesiones exactas.
%
%\begin{lema}
%	[Lema de los cinco] \label{lem:cinco} Consideremos el siguiente diagrama conmutativo
%	de \(R\)-módulos
%	\[
%		\xymatrix{ A_1 \ar@{->}[r]^{\sigma_1} \ar@{->}[d]^{\phi_1} & A_2 \ar@{->}[r]^{\sigma_2} \ar@{->}[d]^{\phi_2} & A_3 \ar@{->}[r]^{\sigma_3} \ar@{->}[d]^{\phi_3} & A_4 \ar@{->}[r]^{\sigma_4} \ar@{->}[d]^{\phi_4} & A_5 \ar@{->}[d]^{\phi_5} \\ B_1 \ar@{->}[r]^{\gamma_1} & B_2 \ar@{->}[r]^{\gamma_2} & B_3 \ar@{->}[r]^{\gamma_3} & B_4 \ar@{->}[r]^{\gamma_4} & B_5 }
%	\]
%	cuyas filas son exactas. Si además \(\phi_{2}, \phi_{4}\) son isomorfismos, \(\phi
%	_{1}\) es un epimorfismo y \(\phi_{5}\) es un monomorfismo, entonces \(\phi_{3}\)
%	es un isomorfismo.
%\end{lema}
%\begin{proof}
%	Comencemos viendo la inyectividad de \(\phi_{3}\). Sea \(x \in A_{3}\) tal que
%	\(\phi_{3}(x) = 0\). Considerando la conmutatividad y la exactitud de las filas,
%	obtenemos
%	\[
%		\phi_{4}(\sigma_{3}(x)) = \gamma_{3}(\phi_{3}(x)) = \gamma_{3}(0) = 0.
%	\]
%	Dado que \(\phi_{4}\) es inyectiva, concluimos que \(\sigma_{3}(x) = 0\). La
%	exactitud en \(A_{3}\) implica la existencia de \(y \in A_{2}\) tal que
%	\(\sigma_{2}(y) = x\). Aplicando \(\phi_{2}\) y utilizando la conmutatividad,
%	tenemos que
%	\[
%		\gamma_{2}(\phi_{2}(y)) = \phi_{3}(\sigma_{2}(y)) = \phi_{3}(x) = 0.
%	\]
%	La exactitud en \(B_{2}\) implica que existe \(z \in B_{1}\) tal que
%	\(\gamma_{1}(z) = \phi_{2}(y)\). Como \(\phi_{1}\) es sobreyectiva, existe
%	\(w \in A_{1}\) con \(\phi_{1}(w) = z\). Nuevamente, la conmutatividad del
%	diagrama y la inyectividad de \(\phi_{2}\) implican que
%	\[
%		\phi_{2}(\sigma_{1}(w)) = \gamma_{1}(\phi_{1}(w)) = \gamma_{1}(z) = \phi_{2}(
%		y),
%	\]
%	y por tanto, \(\sigma_{1}(w) = y\). Por la exactitud en \(A_{2}\), concluimos \(x =
%	\sigma_{2}(y) = \sigma_{2}(\sigma_{1}(w)) = 0\).
%
%	Veamos ahora la sobreyectividad de \(\phi_{3}\). Dado cualquier \(x \in B_{3}\), como
%	\(\phi_{4}\) es sobreyectiva, existe \(p \in A_{4}\) tal que
%	\(\phi_{4}(p) = \gamma_{3}(x)\). La exactitud y la inyectividad de \(\phi_{5}\) nos
%	llevan a
%	\[
%		\phi_{5}(\sigma_{4}(p)) = \gamma_{4}(\phi_{4}(p)) = \gamma_{4}(\gamma_{3}(x))
%		= 0,
%	\]
%	implicando que \(\sigma_{4}(p) = 0\). Por exactitud, existe \(r \in A_{3}\) con \(\sigma
%	_{3}(r) = p\), y utilizando la conmutatividad, obtenemos
%	\[
%		\gamma_{3}(\phi_{3}(r)) = \phi_{4}(\sigma_{3}(r)) = \phi_{4}(p) = \gamma_{3}(
%		x).
%	\]
%	Entonces, \(\phi_{3}(r) - x\) pertenece a \(\ker \gamma_{3} = \im \gamma_{2}\). En
%	consecuencia, existe \(s \in B_{2}\) tal que \(\gamma_{2}(s) = \phi_{3}(r) - x\). Utilizando
%	la sobreyectividad de \(\phi_{2}\), encontramos \(t \in A_{2}\) con
%	\(\phi_{2}(t) = s\), y por lo tanto,
%	\[
%		\phi_{3}(r - \sigma_{2}(t)) = \phi_{3}(r) - \gamma_{2}(\phi_{2}(t)) = \phi_{3}
%		(r) - \gamma_{2}(s) = x.
%	\]
%\end{proof}

\section{Categorías y funtores}
\label{sec:cat}

La teoría de categorías fue introducida por primera vez por Samuel Eilenberg y Saunders
MacLane en \cite{eilenberg1945general}. En particular, las categorías son estructuras
algebraicas que capturan la noción de composición. Gracias a ellas podemos
analizar y comparar estructuras algebraicas, permitiendo sacar conclusiones comunes
y trasladar problemas complejos a otros espacios donde resolverlos es más sencillo.
En esta sección haré una breve introducción de las categorías apoyándome en
\cite{mac2013categories}.

\begin{definicion}
	Una \textbf{categoría} \(\mathcal{C}\) es una tripleta
	\((\mathcal{O}, \hom, \circ)\) formada por:
	\begin{enumerate}
		\item Una clase \(\mathcal{O}\), cuyos elementos denominamos \textbf{objetos} de
		\(\mathcal{C}\) y notamos por \(Obj(\mathcal{C})\).
		
		\item Por cada par de objetos \((A,B)\) de \(\mathcal{C}\), un conjunto
		\(\hom(A,B)\) cuyos elementos son llamados \textbf{morfismos} de \(A\) a \(B\).
		Si \(f \in \hom(A,B)\), normalmente escribiremos \(f: A \rightarrow B\) o \(A \xrightarrow
		{f}B\).
		
		\item Una \textbf{ley de composición} que asocia a cada morfismo
		\(f: A \rightarrow B\) y a cada morfismo \(g: B \rightarrow C\) un morfismo
		\(g \circ f : A \rightarrow C\) que satisface
		\begin{itemize}
			\item \textbf{Asociatividad}. Si \(f: A \rightarrow B\),
			\(g: B \rightarrow C\) y \(h : C \rightarrow D\) son morfismos de
			\(\mathcal{C}\), entonces \(h \circ (g \circ f) = (h \circ g) \circ f\).
			
			\item \textbf{Identidad}. A cada objeto \(B\) le podemos asociar un morfismo
			identidad \(\id_{B}: B \rightarrow B\) tal que si \(f: A \rightarrow B\) y
			\(g: B \rightarrow C\) entonces \(g \circ \id_{B}= g\) y \(\id_{B}\circ f =
			f\).
		\end{itemize}
		Llamaremos a este morfismo la \textbf{composición} de \(f\) y \(g\).
	\end{enumerate}
\end{definicion}

\begin{ejemplo}
	Como veremos a continuación, la definición anterior nos va a permitir trabajar
	con un gran número de espacios matemáticos que ya conocemos en el contexto de
	la teoría de categorías. Algunos de ellos son:
	\begin{itemize}
		\item \textbf{La categoría de conjuntos} \(\Cat{Set}\), cuyos objetos son todos
		los conjuntos y sus morfismos todas las aplicaciones entre conjuntos.
		
		\item \textbf{La categoría de grupos} \(\Cat{Grp}\), donde los objetos son todos
		los grupos y los morfismos todos los homomorfismos de grupos.
		
		\item \textbf{La categoría de espacios topológicos} \(\Cat{Top}\), donde los objetos
		son todos los espacios topológicos y los morfismos todas las aplicaciones continuas
		entre espacios topológicos \(f: X \rightarrow Y\).
		
		\item \textbf{La categoría de \(R\)-módulos} \(R\)-\(\Cat{Mod}\), donde los objetos
		son todos los \(R\)-módulos y los morfismos todos los homomorfismos de módulos.
		
		\item \textbf{La categoría de sucesiones exactas de \(R\)-módulos de longitud
			n}. Los objetos son dichas sucesiones \(S: A_{1}\rightarrow \cdots \rightarrow
		A_{n}\). Para dos sucesiones \(S\) y \(S'\), los morfismos son de la forma \(\Gamma
		: S \rightarrow S'\) tal que \(\Gamma = (\gamma_{1}, \ldots, \gamma_{n})\) es
		una tupla donde los \(\gamma_{i}: A_{i}\rightarrow A_{i}'\) son homomorfismos
		de \(R\)-módulos tal que
		\begin{equation}
			\label{exactseqcat}\xymatrix{ A_1 \ar[r] \ar[d]_{\gamma_1} & A_2 \ar[r] \ar[d]_{\gamma_2} & \cdots \ar[r] & A_{n-1} \ar[r] \ar[d]_{\gamma_{n-1}} & A_n \ar[d]^{\gamma_n} \\ A'_1 \ar[r] & A'_2 \ar[r] & \cdots \ar[r] & A'_{n-1} \ar[r] & A'_n }
		\end{equation}
		conmuta para todo \(i \in \{ 1, \cdots, n \}\).
	\end{itemize}
\end{ejemplo}

También es posible encontrar colecciones propias de una categoría que siguen manteniendo dicha estructura.

\begin{definicion}
	Una \textbf{subcategoría} \(\mathcal{D}\) de una categoría \(\mathcal{C}\) consiste en:
	\begin{enumerate}
		\item Una subclase de objetos de \(\mathcal{C}\), denotada por \(Obj(\mathcal{D})\), tal que cada objeto de \(\mathcal{D}\) es también un objeto de \(\mathcal{C}\).
		\item Para cada par de objetos \(A, B \in Obj(\mathcal{D})\), un subconjunto de morfismos \(\hom_{\mathcal{D}}(A, B) \subseteq \hom_{\mathcal{C}}(A, B)\) cuyos elementos son también morfismos en \(\mathcal{C}\).
		\item La ley de composición en \(\mathcal{D}\) es inducida por la ley de composición en \(\mathcal{C}\), y esta composición es cerrada en \(\mathcal{D}\).
	\end{enumerate}
	Además, diremos que \(\mathcal{D}\) es una \textbf{subcategoría plena} si para cada par de objetos \(A, B \in Obj(\mathcal{D})\), \(\hom_{\mathcal{D}}(A, B) = \hom_{\mathcal{C}}(A, B)\).
\end{definicion}

\begin{definicion}
	Sea \(f \in \hom(A,B)\) un morfismo en la categoría \(\mathcal{C}\). Diremos que
	\(f\) es una \textbf{equivalencia} en \(\mathcal{C}\) si existe en \(\mathcal{C}\) otro
	morfismo \(g \in \hom(B,A)\) tal que \(g \circ f = \id_{A}\) y \(f \circ g = \id_{B}\).
\end{definicion}
\begin{observacion}
	Nótese que si \(f \in \hom(A,B)\) es una equivalencia en \(\mathcal{C}\),
	\(g \in \hom(B,A)\) debe ser única. En efecto, si suponemos que existe
	\(g' \in \hom(B,A)\) tal que \(g' \circ f = \id_{A}\), entonces \(g = g'\circ f \circ
	g = g' \circ \id_{B}= g'\).
\end{observacion}

Dentro de la teoría de categorías, los funtores tienen un papel principal, pues
nos van a permitir llevar objetos y morfismos de una categoría a otra preservando
identidades y composiciones.

\begin{definicion}
	Sean \(\mathcal{C}, \mathcal{D}\) dos categorías. Un \textbf{funtor covariante} de
	\(\mathcal{C}\) a \(\mathcal{D}\) es una pareja de funciones denotadas por
		la misma letra \(T\) tal que:
	\begin{enumerate}
		\item Una \textbf{función objeto} que asigna a cada objeto
		\(C \in \mathcal{C}\) un objeto \(T(C) \in \mathcal{D}\).
		
		\item Una \textbf{función de morfismos} qu asigna a cada morfismo
		\(\gamma: C \rightarrow C'\) de \(\mathcal{C}\) un morfismo
		\(T(\gamma): T(C) \rightarrow T(C')\) de \(\mathcal{D}\). Este par de funciones
		satisfacen las siguientes condiciones:
		\begin{equation}
			T(1_{C}) = \id_{T(C)}, \qquad C \in \mathcal{C},
		\end{equation}
		\begin{equation}
			T(\beta \gamma) = T(\beta)T(\gamma), \qquad \beta \gamma \text{ definido
				en }\mathcal{C}.
		\end{equation}
	\end{enumerate}
\end{definicion}

Es decir, un funtor covariante \(T: \mathcal{C}\rightarrow \mathcal{D}\) es una
aplicación que preserva el rango, dominio, identidades y composiciones de
\(\mathcal{C}\) en \(\mathcal{D}\).

\begin{definicion}
	Sean \(\mathcal{C}, \mathcal{D}\) dos categorías. Diremos que \(\mathcal{C}\) y \(\mathcal{D}\) son \textbf{isomorfas} si existen funtores covariantes $F: \mathcal{C} \to \mathcal{D}$ y $G: \mathcal{D} \to \mathcal{C}$ tales que $G \circ F = \id_\mathcal{C}$ y  $F \circ G = \id_\mathcal{D}$.
\end{definicion}

%\section{Transformación natural}
%
%Mediante el uso de estructuras algebraicas como módulos y complejos de cadenas,
%la homología asigna a cada espacio topológico una serie de grupos de homología, que
%reflejan características clave como agujeros y vacíos en diferentes dimensiones.
%Estos grupos permiten no sólo discernir la estructura interna de los espacios, sino
%también compararlos de manera abstracta. Usaremos de referencia
%\cite{maclane2012homology}.
\section{Módulos diferenciales}

Comenzaremos definiendo lo que es un módulo de homología y estableceremos la terminología
que emplearemos cuando trabajemos con ellos.

\begin{definicion}
	Sea \(C\) un \(R\)-módulo junto a un endomorfismo \(d: C \rightarrow C\) tal que \(d^{2}
	= d \circ d = 0\). Diremos entonces que \(C\) es un \textbf{módulo diferencial} y
	llamaremos a \(d\) \textbf{operador borde} de \(C\).
\end{definicion}

Llamaremos a los elementos de \(C\) \textbf{cadenas}. El submódulo de \textbf{ciclos}
será \(Z(C) = \ker d\), y el submódulo de \textbf{bordes} \(B(C) = \im d\). Si nos
fijamos, el requisito \(d^{2}= 0\) es equivalente a exigir que \(\im{d}\subset \ker{d}\).

\begin{definicion}
	Sea \(C\) un módulo diferencial. Definimos el \textbf{\(R\)-módulo de homología} de
	\(C\) como el módulo cociente \(H(C;R)\) tal que
	\[
	H(C;R) = \frac{Z(C)}{B(C)}
	\]
\end{definicion}

En particular, cuando \(C\) sea un \(\Z\)-módulo diferencial, lo llamaremos \textbf{grupo
	diferencial} y notaremos \(H(C;\Z)\) simplemente por \(H(C)\).

Por tanto, el módulo de homología de un módulo diferencial \(C\) está formado por
las clases laterales \([c] = c + B(C)\) donde \(c\) es un ciclo de \(C\). A los elementos
de \(H(C;R)\) los llamaremos \textbf{clases de homología}. Dos ciclos \(c\) y \(c'\) diremos
que son \textbf{homólogos} si ambos pertenecen a la misma clase de homología, esto
es, \(c \sim c'\).

\begin{definicion}
	Sean \(C\) y \(C'\) dos módulos diferenciales y \(d, d'\) sus respectivos operadores
	borde. Diremos que \(f: C \rightarrow C'\) es un \textbf{homomorfismo de módulos
		diferenciales} si \(f\) es un homomorfismo de módulos y además \(d'f = fd\).
\end{definicion}

La anterior definición nos permite preservar la estructura algebraica del módulo
diferencial. De esta forma, si tomamos una cadena \(c \in C\), que sea un ciclo o
un borde, y \(f\colon C \to C'\) es un homomorfismo de módulos diferenciales,
\(f(c) \in C'\) seguirá siendo un ciclo o un borde de manera correspondiente. En efecto,
si \(z \in Z(C)\), entonces
\[
d'f(z) = f(dz) = f(0) = 0.
\]
Esto es, \(f(z) \in \ker d'\). Ahora, si \(b \in B(C)\), entonces existe \(c \in C\) tal
que \(dc = b\). En consecuencia,
\[
d'f(c) = f(dc) = f(b),
\]
y por tanto, \(f(b) \in \operatorname{im}d'\).

Los grupos diferenciales definen una categoría $\mathcal{C}$ donde los objetos son los módulos
diferenciales y los morfismos son los homomorfismos de módulos diferenciales. Tomemos
como ley de composición interna la composición de dichos homomorfismos.
Claramente es asociativa pues si \(C, C', \bar{C}, \tilde{C}\in \obj(\mathcal{C})\), y \(f\colon
C \to C'\), \(g \colon C' \to \bar{C}\), \(h\colon \bar{C}\to \tilde{C}\), entonces
\(h \circ (g \circ f)\) se cumple si, y sólo si,
\begin{gather*}
	\tilde{d}(h \circ (g \circ f)) = (\tilde{d}h) \circ (g \circ f) = (h\bar{d}) \circ
	(g \circ f) = h \circ (\bar{d}g) \circ f \\ = h \circ (g d') \circ f = h \circ
	g \circ (d'f) = h \circ g \circ (fd) = (h \circ g) \circ fd \\ = ((h \circ g) \circ
	f) d
\end{gather*}
y por tanto \(h \circ (g \circ f) = (h \circ g) \circ f\). La propiedad de identidad
se sigue de existir el homomorfismo identidad de módulos.

\begin{definicion}
	Sean \(C, C'\) módulos diferenciales y \(f:C \rightarrow C'\) un homomorfismo de módulos
	diferenciales. Definimos la función \(f_{*}= H(f): H(C;R) \rightarrow H(C';R)\)
	tal que
	\[
	f_{*}([c]) = [f(c)]
	\]
	Diremos que \(H(f)\) es el \textbf{homomorfismo inducido} por \(f\).
\end{definicion}
\begin{proposicion}
	En estas condiciones, \(H\) es un funtor covariante de la categoría de módulos
	diferenciales a la categoría de módulos.
\end{proposicion}
\begin{proof}
	Por la definición dada del módulo de homología, es claro que la función objeto
	\(H\) asigna a cada grupo diferencial \(C\) un grupo de homología \(H(C;R)\). En cuanto
	a la función de morfismos, la identidad de grupos diferenciales se preserva
	pues \(H(\id)([c]) = \id_{*}([c]) = [\id(c)] = [c]\) para todo \(c \in C\). Además,
	si \(f,g \in \hom(C)\), entonces
	\begin{align*}
		H(g \circ f)([c]) & = (g \circ f)_{*}([c]) = [(g \circ f)(c)] = [g (f (c))]         \\
		& = g_{*}([f(c)]) = g_{*} ( f_{*}([c])) = (H(g) \circ H(f)) ([c])
	\end{align*}
	para todo \(c \in C\), manteniendo la ley de composición.
\end{proof}

\section{Complejos de cadenas}

\begin{definicion}
	Sea \(R\) un anillo. Un \textbf{complejo de cadenas} \(C_{\bullet}\) de \(R\)-módulos
	es una familia \(\{C_{n}, \partial_{n}\}\) donde \(C_{n}\) son \(R\)-módulos y \(\partial
	_{n}: C_{n}\rightarrow C_{n-1}\) homomorfismos de \(R\)-módulos tales que \(\partial
	_{n}\partial_{n+1}= 0\) para todo \(n \in \Z\).
\end{definicion}
\begin{nota}
	Usualmente notaremos directamente \(\partial\) al homomorfismo \(\partial_{n}\) independientemente
	del valor de \(n\) siempre y cuando se sobrentienda por el contexto.
\end{nota}
\begin{observacion}
	La última condición es equivalente a que \(\im{\partial_{n+1}}\subset \ker{\partial_n}\).
\end{observacion}
Un complejo \(C_{\bullet}\) es por tanto una sucesión doblemente infinita
\[
C_{\bullet}: \cdots \rightarrow C_{1} \xrightarrow{\partial_1} C_{0} \xrightarrow{\partial_0} C_{-1} \rightarrow \cdots
\]
donde toda composición de homomorfismos de dicha familia es el homomorfismo nulo.
La \textbf{homología} \(H(C_{\bullet})\) es la familia de \(R\)-módulos
\[
H_{n}(C_{\bullet}) = \frac{\ker \partial_{n}}{\im \partial_{n+1}}
\]
donde \(H_{n}(C_{\bullet})\) es el \textbf{\(n\)-ésimo módulo de homología} de
\(C_{\bullet}\).

Luego \(H_{n}(C_{\bullet})=0\) implica que la sucesión \(C_{\bullet}\) es exacta en
\(C_{n}\). A los elementos de \(C_{n}\) los llamaremos \textbf{n-cadenas} o \textbf{cadenas
	de dimensión n}. Un \textbf{n-ciclo} o \textbf{ciclo de dimensión n} de
\(C_{\bullet}\) es un elemento del submódulo
\(Z_{n}(C_{\bullet}) = \ker \partial_{n}\). Un \textbf{n-borde} o \textbf{borde de
	dimensión n} es un elemento de \(B_{n}(C_{\bullet})= \im{\partial_{n+1}}\). La
clase lateral de un ciclo \(c\) la notaremos por \([c] = c + \partial_{n+a}C_{n+1}\).
Dos \(n\)-ciclos \(c,c' \in C_{n}\) pertenecientes a la misma clase lateral \([c]\) decimos
que son \textbf{homólogos}, es decir, \(c \sim c'\).

\begin{nota}
	Si la dimensión se sobrentiende en estos casos, no la indicaremos de manera
	explícita.
\end{nota}

\begin{definicion}
	Sea \(\{C_{\bullet}^{i}, \partial^{i}\}_{i \in I}\) una familia de complejos de cadenas.
	Su \textbf{suma directa} la definimos como el complejo de cadenas \(\bigoplus_{i
		\in I}C^{i}_{\bullet}\) cuyos operadores borde vienen dados por \(\bigoplus_{i
		\in I}{\partial}^{i}_{n}: \bigoplus_{i \in I}C^{i}_{n}\to \bigoplus_{i \in I}C^{i}
	_{n-1}\) para todo \(n \in \Z\).
\end{definicion}
\begin{proposicion}
	\label{prop:hom-free-commute} Sea \(\{C_{\bullet}^{i}, \partial^{i}\}_{i \in I}\)
	una familia de complejos de cadenas. Entonces su homología conmuta con la suma
	directa, esto es,
	\(H_{n}(\bigoplus_{i \in I}C_{\bullet}^{i}) \cong \bigoplus_{i \in I}H_{n}(C_{\bullet}
	^{i})\)
	para todo \(n \in \Z\).
\end{proposicion}
\begin{proof}
	Para demostrar que la homología conmuta con sumas directas, queremos mostrar
	que para una colección de complejos de cadenas
	\(\{C_{\bullet}^{i}, \partial^{i}\}_{i \in I}\), los homomorfismos
	\begin{align*}
		\phi: H_{n}\left( \bigoplus_{i \in I}C_{\bullet}^{i}\right) \rightarrow \bigoplus_{i \in I}H_{n}(C_{\bullet}^{i}) : [(c_{i})] \mapsto ([c_{i}]), \\
		\psi: \bigoplus_{i \in I}H_{n}(C_{\bullet}^{i}) \rightarrow H_{n}\left( \bigoplus_{i \in I}C_{\bullet}^{i}\right) : ([c_{i}]) \mapsto [(c_{i})],
	\end{align*}
	están bien definidos y son inversos uno del otro.
	
	En primer lugar, para comprobar que dichas aplicaciones están bien definidas, observemos
	que \([(c_{i})] = [(c_{i}')]\) si, y sólo si, \([0] = [(c_{i}- c_{i}')]\). Esto
	ocurre si, y sólo si, existe un \(b_{i}\in C_{\bullet}^{i}\) tal que
	\(\partial_{i}(b_{i}) = (c_{i}- c_{i}')\), lo cual es equivalente a \(c_{i}+ \partial
	_{i}(b_{i}) = c_{i}'\) para cada \(i \in I\). Por lo tanto, \([(c_{i})] = [(c_{i}')
	]\) si, y sólo si, \(\phi([(c_{i})]) = \phi([(c_{i}')]) = [(c_{i}' + \partial_{i}
	(b_{i}))] = [(c_{i}')]\). De manera análoga, \([(c_{i})] = [(c_{i}')]\) si, y
	sólo si, \(\psi([(c_{i})]) = \psi([(c_{i}')])\). Esto implica que tanto \(\phi\)
	como \(\psi\) están bien definidos.
	
	En segundo lugar, es claro que \(\phi\) y \(\psi\) son homomorfismos de \(R\)-módulos.
	Además, \(\phi\) lleva la clase de equivalencia \([(c_{i})]\) a \(([c_{i}])\),
	mientras que \(\psi\) lleva \(([c_{i}])\) a \([(c_{i})]\), lo que demuestra que son inversos
	el uno del otro.
\end{proof}

\begin{definicion}
	Sean \(C_{\bullet},C_{\bullet}'\) complejos de cadenas. Una \textbf{aplicación
		de cadenas} o \textbf{morfismo de cadenas}
	\(f: C_{\bullet}\rightarrow C_{\bullet}'\) es una familia de homomorfismos de
	\(R\)-módulos \(f_{n}: C_{n}\rightarrow C_{n}'\) tal que
	\(\partial_{n}'f_{n}= f_{n-1}\partial_{n}\) para todo \(n \in \Z\).
\end{definicion}
\begin{equation}
	\label{chaincomplex}
	\xymatrix{
		\cdots \ar@{->}[r] & C_{n+1} \ar@{->}[r]^{\partial_{n+1}} \ar@{->}[d]^{f_{n+1}} & C_n \ar@{->}[r]^{\partial_{n}} \ar@{->}[d]^{f_n} & C_{n-1} \ar@{->}[r] \ar@{->}[d]^{f_{n-1}} & \cdots \\
		\cdots \ar@{->}[r] & C'_{n+1} \ar@{->}[r]^{\partial'_{n+1}} & C'_n \ar@{->}[r]^{\partial'_{n}} & C'_{n-1} \ar@{->}[r] & \cdots
	}
\end{equation}
Cuando se sobrentienda del contexto, notaremos simplemente por \(\partial\) a los correspondientes
\(\partial_{n}\) y \(\partial_{n}'\).

Los complejos de cadenas, junto con sus morfismos, forman una categoría que denotaremos
por \(R\textbf{-}\Cat{Ch_\bullet}\). Esto se debe a que existe la identidad y se
cumple la propiedad de asociatividad para la composición de morfismos. En efecto,
para tres morfismos \(f : C_{\bullet} \to C_{\bullet}'\), \(g : C_{\bullet}' \to C_{\bullet}
''\) y \(h : C_{\bullet}'' \to \overline{C}_{\bullet}\), se tiene que
\[
h_{n} \circ (g_{n} \circ f_{n}) = (h_{n} \circ g_{n}) \circ f_{n}
\]
para cada \(n \in \Z\), ya que son homomorfismos de módulos diferenciales. En
consecuencia, \(h \circ (g \circ f) = (h \circ g) \circ f\).

Sea \(f : C_{\bullet} \to C_{\bullet}'\) un morfismo de complejos de cadenas.
Definimos \(H_{n}(f) = f_{*} : H_{n}(C_{\bullet}) \to H_{n}(C_{\bullet}')\) dado que 
\[
f_{*}([c]) = f_{*}(c + \partial C_{n+1}) = fc + \partial C'_{n+1} = [f(c)].
\]
Entonces \(f_{*}\) es un homomorfismo de \(R\)-módulos, como recoge el siguiente resultado.

\begin{proposicion}
	Cada \(H_{n}\) es un funtor covariante de la categoría de complejos de cadenas y
	morfismos de cadenas a la categoría de \(R\)-módulos.
\end{proposicion}
\begin{proof}
	Como vimos anteriormente, \(H_{n}\) asigna a cada complejo de cadenas
	\(C_{\bullet}\) un \(R\)-módulo. Para demostrar que \(H_{n}\) también asigna a cada morfismo
	de cadenas un homomorfismo de \(R\)-módulos, consideremos
	\([c], [c'] \in H_{n}(C_{\bullet})\) y \(r, s \in R\). Entonces, podemos ver que
	\begin{align}
		f_{*}(r[c] + s[c']) & = f_{*}(r(c + \partial C_{n+1}) + s(c' + \partial C_{n+1}))        \\
		& = f_{*}(rc + \partial C_{n+1}+ sc' + \partial C_{n+1})             \\
		& = f_{*}(rc + sc' + \partial C_{n+1})                               \\
		& = f(rc + sc') + \partial C'_{n+1}                                  \\
		& = rf(c) + sf(c') + \partial C'_{n+1}                               \\
		& = r(f(c) + \partial C'_{n+1}) + s(f(c') + \partial C'_{n+1})       \\
		& = r(f_{*}(c + \partial C_{n+1})) + s(f_{*}(c' + \partial C_{n+1})) \\
		& = rf_{*}([c]) + sf_{*}([c']).
	\end{align}
	Además, si consideramos \(f = \id\) la identidad, es claro que \(\id_{*}\) es la
	identidad de \(R\)-módulos.
\end{proof}

\begin{definicion}
	\label{def:chain_homotopy} Sean \(C_{\bullet},C_{\bullet}'\) complejos de cadenas
	y \(f,g: C_{\bullet}\rightarrow C_{\bullet}'\) dos aplicaciones de cadenas entre
	ellos. Una \textbf{homotopía de cadenas} u \textbf{homotopía algebraica} \(s\)
	es una familia de homomorfismos de módulos \(s_{n}: C_{n}\rightarrow C_{n+1}'\)
	para cada \(n \in \Z\) tal que
	\begin{equation}
		\partial_{n+1}'s_{n}+ s_{n-1}\partial_{n}= f_{n}- g_{n}.
	\end{equation}
	Diremos entonces que \(f\) y \(g\) son \textbf{algebraicamente homotópicas} y escribiremos
	\(f \simeq g\).
\end{definicion}

\begin{teorema}
	\label{teo:homot-cad-misma-homologia} Si \(s\) es una homotopía de cadenas entre
	\(f,g: C_{\bullet}\rightarrow C_{\bullet}'\), entonces
	\[
	H_{n}(f) = H_{n}(g) : H_{n}(C_{\bullet}) \rightarrow H_{n}(C_{\bullet}').
	\]
\end{teorema}
\begin{proof}
	Si \(c\) es un ciclo de \(C_{n}\), tenemos que \(\partial_{n}c = 0\). Por la \autoref{def:chain_homotopy}
	se cumple que \(f_{n}c-g_{n}c = \partial s_{n}c\). Como consecuencia \(f_{n}c\) y
	\(g_{n}c\) son homólogos lo que implica que \([f_{n}c] = [g_{n}c]\) en
	\(H_{n}(C_{\bullet}')\), como queríamos demostrar.
\end{proof}

\begin{definicion}
	Una aplicación de cadenas \(f: C_{\bullet}\rightarrow C_{\bullet}'\) es una \textbf{equivalencia
		de cadenas} si existe otra aplicación
	\(h: C_{\bullet}' \rightarrow C_{\bullet}\) y homotopías
	\(s: h \circ f \rightarrow{\id_C}_{\bullet}\), \(t: f \circ h \rightarrow{\id_C'}_{\bullet}\)
	tales que \(h \circ f \simeq{\id_C}_{\bullet}\),
	\(f \circ h \simeq{\id_{C'}}_{\bullet}\).
\end{definicion}

Como \(H_{n}({\id_C}_{\bullet}) = \id_{H_n(C_{\bullet})}\), del anterior teorema
se deduce lo siguiente.

\begin{corolario}
	\label{cor:equiv-homot} Si \(f: C_{\bullet}\rightarrow C_{\bullet}'\) es una equivalencia
	de cadenas, la aplicación inducida
	\(H_{n}(f): H_{n}(C_{\bullet}) \rightarrow H_{n}(C_{\bullet}')\) es un
	isomorfismo para cada \(n \in \Z\).
\end{corolario}

\begin{proposicion}
	Sean \(f,g: C_{\bullet}\rightarrow C_{\bullet}'\) y \(f',g': C_{\bullet}' \rightarrow
	C_{\bullet}''\) aplicaciones de cadenas. Sean \(s: f \rightarrow g\), \(s': f' \rightarrow
	g'\) homotopías de cadenas entre ellas tales que \(f \simeq g\), \(f' \simeq g'\). Entonces
	la composición
	\[
	f' s + s' g: f' \circ f \rightarrow g' \circ g \qquad g' \circ g : C_{\bullet}
	\rightarrow C_{\bullet}''
	\]
	es una homotopía de cadenas.
\end{proposicion}
\begin{proof}
	Por ser \(s,s'\) homotopías de cadenas tenemos que \(\partial s + s\partial = f-g\)
	y \(\partial s' + s'\partial = f'-g'\). Aplicando \(f'\) a la izquierda de la
	primera expresión y \(g\) a la derecha de la segunda nos queda
	\begin{equation}
		\begin{cases}
			f'\partial s + f's\partial = f' \circ f-f' \circ g,   \\
			\partial s' g + s'\partial g = f' \circ g-g' \circ g.
		\end{cases}
	\end{equation}
	Sumando ambas igualdades
	\begin{align*}
		f'\partial s + f's\partial + \partial s' g + s'\partial g    & = f' \circ f-f' \circ g + f' \circ g-g' \circ g, \\
		f'\partial s + f's\partial + \partial s' g + s'\partial g    & = f' \circ f - g' \circ g,                       \\
		\partial f' s + f's \partial + \partial s' g + s' g \partial & = f' \circ f - g' \circ g,
	\end{align*}
	donde finalmente queda
	\begin{align*}
		\partial (f' s + s' g) + (f's + s' g) \partial & = f' \circ f - g' \circ g .
	\end{align*}
\end{proof}

\section{Subcomplejos y complejos cociente}

\begin{definicion}
	Un \textbf{subcomplejo} \(S_{\bullet}\) de \(C_{\bullet}\) es una familia de
	submódulos \(S_{n}\subset C_{n}\) tal que para cada \(n \in \Z\), \(\partial S_{n}\subset
	S_{n-1}\).
\end{definicion}

Por tanto, \(S_{\bullet}\) es un complejo en sí con el operador borde \(\partial\) inducido
de \(C_{\bullet}\) y la inclusión \(i: S_{\bullet}\rightarrow C_{\bullet}\) es una aplicación
de cadenas.

\begin{definicion}
	Sea \(S_{\bullet}\) un subcomplejo de \(C_{\bullet}\). El \textbf{complejo
		cociente} \(C_{\bullet}/S_{\bullet}\) es la familia \((C_{\bullet}/S_{\bullet})_{n}
	= C_{n}/S_{n}\) de módulos cocientes con operador borde \(\partial'_{n}: C_{n}/S_{n}
	\rightarrow C_{n-1}/S_{n-1}\) inducido por \({\partial_C}_{\bullet}\).
\end{definicion}

%La proyección es la aplicación de cadenas \(\pi: C_{\bullet} \rightarrow C_{\bullet}/S\) y la sucesión corta \(S_n\) FLECHARARA \(C_n FLECHA (C_{\bullet}/S)_n\) es exacta para cada \(n \in \Z\).
%
%Si \( f: C_{\bullet} \to C_{\bullet}' \) es una aplicación de cadenas, entonces \( \ker(f) = \{\ker(f_n)\} \) es un subcomplejo de \( C_{\bullet} \), \( \text{Im}(f) = \{f_n(C_n)\} \) es un subcomplejo de \( C_{\bullet}' \), mientras que \( C_{\bullet}'/\text{Im}(f) \) es el cociente de \( f \) y \( C_{\bullet}/\ker(f) \) el conúcleo.

\begin{definicion}
	Sean \(f: C_{\bullet}\rightarrow C_{\bullet}'\),
	\(g: C_{\bullet}' \rightarrow C_{\bullet}''\) aplicaciones de cadenas. La sucesión
	de complejos
	\[
	C_{\bullet}\xrightarrow{f}C_{\bullet}' \xrightarrow{g}C_{\bullet}''
	\]
	es \textbf{exacta} en \(C_{\bullet}'\) si \(\im(f) = \ker(g)\); es decir, si cada
	sucesión \(C_{n}\xrightarrow{f_n}{C'}_{n}\xrightarrow{g_n}{C''}_{n}\) de módulos
	es exacta en \({C'}_{n}\).
\end{definicion}

%Para cualquier \( f: C_{\bullet} \to C_{\bullet}', \)
%\[ 0 \to \ker(f) \to C_{\bullet} \xrightarrow{f} C_{\bullet}' \to \text{Coker}(f) \to 0 \]
%es una sucesión exacta de complejos.

\begin{definicion}
	Un complejo \(C_{\bullet}\) es \textbf{positivo} si \(C_{n}= 0\) para todo \(n < 0\)
	con \(n \in \Z\). Su \(n\)-ésimo módulo de homología es entonces positivo ya que \(H
	_{n}(C_{\bullet}) = 0\) para todo \(n < 0\). De manera análoga, un complejo \(C_{\bullet}\)
	es \textbf{negativo} si \(C_{n}= 0\) para todo \(n > 0\) con \(n \in \Z\).
\end{definicion}
%
%Los complejos negativos suelen notarse con índices superiores positivos donde \(C_{-n}\)
%se sustituye por \(C^{n}\) y \(\partial_{-n}: C_{-n}\rightarrow C_{-n-1}\) por \(\delta
%^{n}: C^{n} \rightarrow C^{n+1}\) quedando así
%\[
%0 \to C^{0} \xrightarrow{\delta^0}C^{1} \xrightarrow{\delta^1}C^{2} \to \cdots,
%\quad \delta \circ \delta = 0
%\]
%donde el \(n\)-ésimo módulo de homología \(H^{n}(C_{\bullet}) = \ker(\delta^{n})/\text{Im}
%(\delta^{n-1})\) es positivo en los índices superiores. A dicho complejo lo
%llamaremos \textbf{complejo de cocadenas}.

\begin{definicion}
	Sea \(C_{\bullet}\) un complejo positivo de \(R\)-módulos. Denominaremos \textbf{aumento
		de \(C_{\bullet}\)} al homomorfismo sobreyectivo \(\varepsilon : C_{0}\to R\) de forma
	que \(\varepsilon \circ \partial_{1}= 0\).
\end{definicion}

\begin{definicion}
	Sea \(C_{\bullet}\) un complejo de cadenas positivo, \(\varepsilon : C_{0}\to R\)
	un aumento de \(C_{\bullet}\) y sea \(n \in \Z\). Consideremos el complejo positivo
	\(\widetilde{C}_{\bullet}\) tal que \(\widetilde{C}_{n}= C_{n}\) para todo \(n \geq
	0\), \(\widetilde{C}_{n}= 0\) para todo \(n < -1\) y \(\widetilde{C}_{-1}=R\). Consideremos
	también \(\widetilde{\partial}_{n}= \partial_{n}\) para todo \(n \geq 1\) y \(\widetilde
	{\partial}_{0}= \varepsilon\). Llamaremos a este complejo \textbf{complejo
		aumentado} de \(C_{\bullet}\).
\end{definicion}
%
%\begin{definicion}
%Sean \(f\) y \(g\) aplicaciones de cadenas de \(C_{\bullet}\) a \(C_{\bullet}'\) y sea \(s\) una homotopía de cadenas entre ellas. Diremos que \(s\) es una \textbf{homotopía de cocadenas} si  está escrita con índices superiores. Esto es, \( s^n: C^n \to C'^{n-1} \) con \( \delta s + s \delta = f - g \).
%\end{definicion}

\begin{definicion}
	Sea \(A\) un módulo. Definimos el siguiente complejo positivo donde \(A_{0}= A\), \(A
	_{n}= 0\) para \(n \neq 0\) y \(\partial = 0\). Un \textbf{complejo sobre \(A\)} es un
	complejo positivo \(C_{\bullet}\) junto con una aplicación de cadenas \(\varepsilon
	: C_{\bullet}\to A\) donde \(\varepsilon\) es un homomorfismo de módulos \(\varepsilon
	: C_{0}\to A\) tal que \(\varepsilon \partial = 0: C_{1}\to A\).
\end{definicion}

\begin{definicion}
	\label{def:contr_homotopy} Una \textbf{homotopía contráctil} para
	\(\varepsilon: C_{\bullet}\to A\) es una aplicación de cadenas
	\(f: A \to C_{\bullet}\) tal que \(\varepsilon f = \id_{A}\) junto con una
	homotopía \(s:{\id_C}_{\bullet}\rightarrow f \varepsilon\) donde
	\({\id_C}_{\bullet}\simeq f \varepsilon\). En otras palabras, una homotopía
	contráctil consiste en homomorfismos de módulos \(f: A \to C_{0}\) y una homotopía de cadenas formada por los homomorfismos 
	\(s_{n}: C_{n}\to C_{n+1} \) donde $n \in \N \cup \{0\}$ tal que
	\[
	\varepsilon f = \id_{A}, \quad \partial_{1}s_{0}+ f \varepsilon = \id_{C_0},
	\quad \partial_{n+1}s_{n}+ s_{n-1}\partial_{n}= \id_{C_n},
	\]
	para todo $n > 0$.
\end{definicion}

Podemos extender el complejo estableciendo \(C_{-1}= A\), \(\partial_{0}= \varepsilon
: C_{0}\to C_{-1}\) y \(s_{-1}= f\). Por la definición anterior, la homotopía contráctil \(s:{\id_C}
_{\bullet}\rightarrow 0\) es una homotopía de cadenas. Si
\(\varepsilon: C_{\bullet}\to A\) tiene una homotopía contráctil, sus grupos de homología
son isomorfos por \(\varepsilon_{*}: H_{0}(C_{\bullet}) \rightarrow A\) para \(n = 0\)
y \(H_{n}(C_{\bullet}) = 0\) para \(n > 0\).

Consideremos un complejo de cadenas \(C_{\bullet}= \{ C_{n}, d_{n}\}_{n \in
	\mathbb{Z}}\), donde cada \(C_{n}\) es un \(\mathbb{Z}\)-módulo libre y
\(d_{n}: C_{n}\to C_{n-1}\) es el operador diferencial de \(C_{\bullet}\) que cumple
\(d_{n-1}\circ d_{n}= 0\) para todo \(n\). Este tipo de complejos aparece frecuentemente
en el estudio de espacios topológicos.

Supongamos además que cada \(C_{n}\) es finitamente generado. Entonces, el n-ésimo
grupo de homología de \(C_{\bullet}\), definido como
\[
H_{n}(C_{\bullet}) = \frac{\ker(d_{n})}{\im(d_{n+1})},
\]
es un grupo abeliano finitamente generado. Este resultado se sigue del hecho de que
el núcleo y la imagen de los morfismos entre \(\mathbb{Z}\)-módulos libres finitamente
generados son también finitamente generados.

El teorema de \nameref{teo:desc-cicl-prim} y, en particular, el teorema de
estructura para grupos abelianos finitamente generados afirma que cualquier
grupo abeliano finitamente generado \(G\) puede expresarse como una suma directa
de grupos cíclicos de la forma
\[
G \cong \mathbb{Z}^{\beta}\oplus \mathbb{Z}_{m_1}\oplus \ldots \oplus \mathbb{Z}
_{m_k},
\]
donde \(\beta\) es el rango de \(G\) y cada \(\mathbb{Z}_{m_i}\) es un grupo cíclico
de orden \(m_{i}\), donde \(m_{i}\) divide a \(m_{i+1}\) para cada \(i \in \{1, \ldots,
k\}\). Aplicando este teorema al $n$-ésimo módulo de homología \(H_{n}(C_{\bullet})\),
obtenemos que
\[
H_{n}(C_{\bullet}) \cong \mathbb{Z}^{\beta_n}\oplus \mathbb{Z}_{m_1}\oplus \ldots
\oplus \mathbb{Z}_{m_k},
\]
donde \(\beta_{n}\) es el rango de \(H_{n}(C_{\bullet})\), conocido como el \textbf{n-ésimo
	número de Betti} de \(C_{\bullet}\), y los \(m_{i}\) son los \textbf{n-ésimos
	coeficientes de torsión}, donde \(m_{i}\) divide a \(m_{i+1}\) para cada
\(i \in \{1, \ldots, k\}\).

\begin{definicion}
	Sea \(C_{\bullet}\) un complejo de cadenas y \(n\) un entero no negativo. El
	\textbf{\(n\)-ésimo número de Betti}, \(\beta_{n}(C_{\bullet})\), se define como el
	rango del \(n\)-ésimo módulo de homología de \(C_{\bullet}\) sobre el dominio de ideales
	principales \(R\), \(H_{n}(C_{\bullet};R)\). Esto es,
	\(\beta_{n}(C_{\bullet}) = \rg(H_{n}(C_{\bullet};R))\). Si no hay lugar a confusión
	lo notaremos simplemente por \(\beta_{n}\).
\end{definicion}

\begin{definicion}
	Dado un complejo de cadenas \(C_{\bullet}\) y considerando la descomposición en sumas
	directas del \(n\)-ésimo módulo de homología \(H_{n}(C_{\bullet};R)\) sobre un dominio
	de ideales principales \(R\), definimos los \textbf{\(n\)-ésimos coeficientes de
		torsión} como los \(m_{i} \in R\) donde \(i \in \{1, \ldots, k\}\) que aparecen en
	la descomposición
	\[
	H_{n}(C_{\bullet};R) \cong R^{\beta_n}\oplus \frac{R}{\langle m_{1} \rangle}\oplus
	\ldots \oplus \frac{R}{\langle m_{k} \rangle},
	\]
	donde cada \(m_{i}\) divide a \(m_{i+1}\).
\end{definicion}

Los números de Betti \(\beta_{n}\) proporcionan una medida de la dimensionalidad
de la n-ésima homología, mientras que los coeficientes de torsión \(\{m_{i}\}\) representan
los órdenes de los sumandos cíclicos de torsión en la descomposición de homología
y reflejan la estructura torsional de \(H_{n}(C_{\bullet};R)\).

La homología con coeficientes en \(\mathbb{Z}\) es fundamental en topología algebraica, proporcionando una rica estructura para el estudio de espacios topológicos. Sin embargo, el uso de coeficientes en otros anillos, particularmente en cuerpos, puede simplificar significativamente los cálculos y aún así ofrecer información valiosa para ciertas aplicaciones. La elección de un cuerpo como coeficientes, por ejemplo \(\Z_2\), reduce la complejidad de los cálculos al evitar problemas relacionados con la torsión, facilitando la manipulación algebraica. Al estudiar la homología con estos coeficientes alternativos, aunque se obtiene una visión menos detallada del espacio, a menudo es suficiente para resolver problemas específicos dentro de un contexto dado, tales como la detección de características topológicas esenciales o la simplificación de la clasificación de espacios topológicos.

\endinput
%--------------------------------------------------------------------
% FIN DEL CAPÍTULO.
%--------------------------------------------------------------------