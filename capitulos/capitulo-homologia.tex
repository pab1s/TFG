% !TeX root = ../tfg.tex
% !TeX encoding = utf8

\chapter{Homología}

Mediante el uso de estructuras algebraicas como grupos y complejos de cadenas, la homología asigna a cada espacio topológico una serie de grupos de homología, que reflejan características clave como agujeros y vacíos en diferentes dimensiones. Estos grupos permiten no sólo discernir la estructura interna de los espacios, sino también compararlos de manera abstracta. Usaremos de referencia \cite{maclane2012homology}.

\section{Grupos diferenciales}

Comenzaremos definiendo lo que es un grupo de homología y estableceremos la terminología que emplearemos cuando trabajemos con ellos.

\begin{definicion}
	Sea $C$ un grupo abeliano junto a un endomorfismo  $d: C \rightarrow C$ tal que $d^2 = d \circ d = 0$. Diremos entonces que $C$ es un \textbf{grupo diferencial} y llamaremos a $d$ \textbf{operador borde} de $C$.
\end{definicion}

Llamaremos a los elementos de $C$ \textbf{cadenas}. El subgrupo de \textbf{ciclos} será $Z(C) = \ker d$,  y el subgrupo de \textbf{bordes} $B(C) = \im d$. Si nos fijamos, el requisito $d^2 = 0$ es equivalente a exigir que $\im{d} \subset \ker{d}$.

\begin{definicion}
	Sea $C$ un grupo diferencial. Definimos el \textbf{grupo de homología} de $C$ como el grupo cociente $H(C)$ tal que 
	\[ H(C) = \frac{Z(C)}{B(C)}\]
\end{definicion}

Por tanto, el grupo de homología de un grupo diferencial $C$ está formado por las clases laterales $[c] = c + B(C)$ donde $c$ es un ciclo de $C$. A los elementos de $H(C)$ los llamaremos \textbf{clases de homología}. Dos ciclos $c$ y $c'$ diremos que son \textbf{homólogos} si ambos pertenecen a la misma clase de homología, esto es, $c \sim c'$.

\begin{definicion}
	Sean $C$ y $C'$ dos grupos diferenciales y $d, d'$ sus respectivos operadores borde. Diremos que $f: C \rightarrow C'$ es un \textbf{homomorfismo de grupos diferenciales} si $f$ es un homomorfismo de grupos y además $d'f = fd$.
\end{definicion}

La anterior definición nos permite preservar la estructura algebraica del grupo diferencial. De esta forma, si tomamos una cadena $c \in C$ que sea un ciclo o un borde y $f:C \rightarrow C'$ es un homomorfismo de grupos diferenciales, $f(c) \in C'$ seguirá siendo un ciclo o un borde de manera correspondiente.

Los grupos diferenciales definen una categoría donde los objetos son los grupos diferenciales y los morfismos son los homomorfismos de grupos diferenciales. Tomamos como ley de composición interna la composición de dichos homomorfismos. Claramente es asociativa pues si $C,C',\bar{C}, \tilde{C} \in \obj()$ y $f: C \to C'$, $g: C' \to \bar{C}$, $h: \bar{C} \to \tilde{C}$, entonces $h \circ (g \circ f)$ se cumple si, y sólo si,
\begin{gather*}
	\tilde{d}(h \circ (g \circ f)) 
	= (\tilde{d}h) \circ (g \circ f) \\
	=(h\bar{d}) \circ (g \circ f)) 
	=h \circ (\bar{d} g) \circ f \\
	=h \circ (g d') \circ f
	=h \circ g \circ (d' f) \\
	=h \circ g \circ (fd)
	=(h \circ g) \circ fd \\
	=((h \circ g) \circ f) d
\end{gather*}
y por tanto $h \circ (g \circ f) = (h \circ g) \circ f$. La propiedad de identidad se sigue de existir el homomorfismo identidad de grupos.
\begin{definicion}
	Sean $C, C'$ grupos diferenciales y $f:C \rightarrow C'$ un homomorfismo de grupos diferenciales. Definimos la función $f_* = H(f): H(C) \rightarrow H(C')$ tal que 
	\[f_*([c]) = [f(c)] \]
	Diremos que $H(f)$ es el \textbf{homomorfismo inducido} por $f$.
\end{definicion}
\begin{proposicion}
	En estas condiciones, $H$ es un funtor covariante de la categoría de grupos diferenciales a la categoría de grupos.
\end{proposicion}
\begin{proof}
	Por la definición dada del grupo de homología, es claro que la función objeto $H$ asigna a cada grupo diferencial $C$ un grupo de homología $H(C)$. En cuanto a la función de morfismos, la identidad de grupos diferenciales se preserva pues $H(\id)([c]) = \id_*([c]) = [\id(c)] = [c]$ para todo $c \in C$. Además, si $f,g \in \hom(C)$, entonces $H(g \circ f)([c]) = (g \circ f)_*([c]) = [(g \circ f)(c)] = [g (f (c))] = g_*([f(c)]) = g_* ( f_*([c])) = (H(g) \circ H(f)) ([c])$ para todo $c \in C$, manteniendo la ley de composición.
\end{proof}

\section{Complejos de cadenas}

\begin{definicion}
	Sea $R$ un anillo. Un \textbf{complejo de cadenas} $C_{\bullet}$ de $R$-módulos es una familia $\{C_n, \partial_n\}$ donde $C_n$ son $R$-módulos y $\partial_n : C_n \rightarrow C_{n-1}$ homomorfismos de $R$-módulos tales que $\partial_n \partial_{n+1} = 0$ para todo $n \in \Z$.
\end{definicion}
\begin{nota}
	Usualmente notaremos directamente $\partial$ al homomorfismo $\partial_n$ independientemente del valor de $n$ siempre y cuando se sobrentienda por el contexto.
\end{nota}
\begin{observacion}
	La última condición es equivalente a que $\im{\partial_{n+1}} \subset \ker{\partial_n}$.
\end{observacion}
Un complejo $C_{\bullet}$  es por tanto una sucesión doblemente infinita
\[ C_{\bullet} : \cdots \rightarrow C_{1} \rightarrow C_0 \rightarrow C_{-1} \rightarrow \cdots \]
donde toda composición de homomorfismos de dicha familia es el homomorfismo nulo. La \textbf{homología} $H(C_{\bullet})$ es la familia de $R$-módulos
\[ H_n(C_{\bullet}) = \frac{\ker \partial_n}{\im \partial_{n+1}} \]
donde $H_n(C_{\bullet})$ es el \textbf{$n$-ésimo módulo de homología} de $C_{\bullet}$.

Luego $H_n(C_{\bullet})=0$ implica que la sucesión $C_{\bullet}$ es exacta en $C_n$. A los elementos de $C_n$ los llamaremos \textbf{n-cadenas} o \textbf{cadenas de dimensión n}. Un \textbf{n-ciclo} o \textbf{ciclo de dimensión n} de $C_{\bullet}$ es un elemento del submódulo $Z_n(C_{\bullet}) = \ker \partial_n$. Un \textbf{n-borde} o \textbf{borde de dimensión n} es un elemento de $B_n(C_{\bullet})= \im{\partial_{n+1}}$. Si la dimensión se sobrentiende en estos casos, no es necesario indicarla de manera explícita. La clase lateral de un ciclo $c$ la notaremos por $[c] = c + \partial C_{n+1}$. Dos $n$-ciclos $c,c' \in C_n$ pertenecientes a la misma clase lateral $[c]$ decimos que son \textbf{homólogos}, es decir, $c \sim c'$.

\begin{definicion}HACER BIEN
	Sea $\{C_{\bullet}^i, \partial_i\}_{i \in I}$ una familia de complejos de cadenas. Su \textbf{suma directa} \( \bigoplus_{i \in I} C^i \) la definimos como el complejo de cadenas $\bigoplus_{i \in I} C^i_{\bullet}$ cuyos operadores borde vienen dados por $\bigoplus_{i \in I} {\partial}^i_n : \bigoplus_{i \in I} C^i_n \to \bigoplus_{i \in I} C^i_{n-1}$ para todo $n \in \Z$.
\end{definicion}
\begin{proposicion}
	\label{prop:hom-free-commute}
	Sea $\{{C_{\bullet}}_i, \partial_i\}_{i \in I}$ una familia de complejos de cadenas. Entonces su homología conmuta con la suma directa, esto es, $H_n(\bigoplus_{i \in I} {C_{\bullet}}_i) \cong \bigoplus_{i \in I} H_n({C_{\bullet}}_i)$ para todo $n \in \Z$.
\end{proposicion}
\begin{proof}
	Para demostrar que el homomorfismo $\log$ conmuta con sumas directas, queremos mostrar que para una colección de complejos de cadenas $\{ (C_i, d_i) \}_{i \in I}$, el isomorfismo
	\[ \phi: H_n \left( \bigoplus_{i \in I} C_i \right) \rightarrow \bigoplus_{i \in I} H_n (C_i) : [(c_i)] \mapsto ([c_i]), \]
	y su inverso
	\[ \psi: \bigoplus_{i \in I} H_n (C_i) \rightarrow H_n \left( \bigoplus_{i \in I} C_i \right) : ([c_i]) \mapsto [(c_i)], \]
	están bien definidos y son inversos mutuos.
	
	En primer lugar, para establecer la bien definición, observamos que $[(c_i)] = [(c_i')]$ si y solo si $[0] = [(c_i - c_i')]$. Esto ocurre si y solo si existe un $b_i \in C_i$ tal que $d_i(b_i) = (c_i - c_i')$, lo cual es equivalente a $c_i + d_i(b_i) = c_i'$ para cada $i \in I$. Por lo tanto, $[(c_i)] = [(c_i')]$ si y solo si $\phi([(c_i)]) = \phi([(c_i')]) = [(c_i' + d_i(b_i))] = [(c_i')]$. De manera similar, $[(c_i)] = [(c_i')]$ si y solo si $\psi([(c_i)]) = \psi([(c_i')])$. Por lo tanto, tanto $\phi$ como $\psi$ están bien definidos.
	
	En segundo lugar, es evidente que tanto $\phi$ como $\psi$ son homomorfismos de grupos abelianos. Además, $\phi$ mapea $[(c_i)]$ a $([c_i])$, y $\psi$ mapea $([c_i])$ a $[(c_i)]$, lo que demuestra que son inversos entre sí.
	
	Por lo tanto, $\phi$ es un isomorfismo con $\psi$ como su inverso, estableciendo que $\bigoplus_{i \in I} H_n (C_i) \cong H_n \left( \bigoplus_{i \in I} C_i \right)$.
\end{proof}
\begin{definicion}
	Sean $C_{\bullet},C_{\bullet}'$ complejos de cadenas. Una \textbf{aplicación de cadenas} o \textbf{morfismo de cadenas} $f: C_{\bullet} \rightarrow C_{\bullet}'$ es una familia de homomorfismos de $R$-módulos $f_n: C_n \rightarrow C_n'$ tal que $\partial_n'f_n = f_{n-1}\partial_n$ para todo $n \in \Z$.
\end{definicion}

\begin{equation}
	\label{chaincomplex}
	\xymatrix{
		\cdots & C_{n-1} \ar[l] \ar[d]^{f_{n-1}} \ar@{..>}[dr] & C_n \ar[l]_{\partial_{n}} \ar[d]^{f_n} \ar@{..>}[dr] & C_{n+1} \ar[l]_{\partial_{n+1}} \ar[d]^{f_{n+1}} & \cdots \ar[l] \\
		\cdots & C'_{n-1} \ar[l] & C'_n \ar[l]^{\partial'_{n}} & C'_{n+1} \ar[l]^{\partial'_{n+1}} & \cdots \ar[l]
    }
\end{equation}

Cuando se sobrentienda del contexto, notaremos simplemente por $\partial$ a los correspondientes $\partial_n$ y $\partial_n'$.

La función $H_n(f) = f_*$ definida por $f_*([c]) = f_*(c + \partial C_{n+1}) = fc + \partial C'_{n+1}$ es un homomorfismo $H_n(f): H_n(C_{\bullet}) \rightarrow H_n(C_{\bullet}')$. Así mismo, cada $H_n$ es un funtor covariante de la categoría de complejos de cadenas y morfismos de cadenas a la categoría de módulos.

\begin{definicion}
	\label{def:chain_homotopy}
	Sean $C_{\bullet},C_{\bullet}'$ complejos de cadenas y $f,g: C_{\bullet} \rightarrow C_{\bullet}'$ dos aplicaciones de cadenas entre ellos. Una \textbf{homotopía de cadenas} u \textbf{homotopía algebraica} $s$ es una familia de homomorfismos de módulos $s_n: C_n \rightarrow C_{n+1}'$ para cada $n \in \Z$ tal que
	\begin{equation}
		\partial_{n+1}'s_n + s_{n-1} \partial_n = f_n - g_n
	\end{equation}
	Diremos entonces que $f$ y $g$ son \textbf{algebraicamente homotópicas} y escribiremos $f \simeq g$.
\end{definicion}

\begin{teorema}
	\label{teo:homot-cad-misma-homologia}
	Si $s$ es una homotopía de cadenas entre $f,g: C_{\bullet} \rightarrow C_{\bullet}'$, entonces
	\[ H_n(f) = H_n(g) : H_n(C_{\bullet}) \rightarrow H_n(C_{\bullet}') \]
\end{teorema}
\begin{proof}
	Si $c$ es un ciclo de $C_n$, tenemos que $\partial_n c = 0$. Por la \autoref{def:chain_homotopy} se cumple que $f_nc-g_nc =  \partial s_n c$. Como consecuencia $f_n c$ y $g_n c$ son homólogos lo que implica que $[f_n c] = [g_n c]$ en $H_n(C_{\bullet}')$, como queríamos demostrar.
\end{proof}

\begin{definicion}
	Una aplicación de cadenas $f: C_{\bullet} \rightarrow C_{\bullet}'$ es una \textbf{equivalencia de cadenas} si existe otra aplicación $h: C_{\bullet}' \rightarrow C_{\bullet}$ y homotopías $s: h \circ f \rightarrow {\id_C}_{\bullet}$, $t: f \circ h \rightarrow {\id_C'}_{\bullet}$ tales que $h \circ f \simeq {\id_C}_{\bullet}$, $f \circ h \simeq {\id_{C'}}_{\bullet}$.
\end{definicion}

Como $H_n({\id_C}_{\bullet}) = \id_{H_n(C_{\bullet})}$, del anterior teorema se deduce lo siguiente.

\begin{corolario}
	\label{cor:equiv-homot}
	Si $f: C_{\bullet} \rightarrow C_{\bullet}'$ es una equivalencia de cadenas, la aplicación inducida $H_n(f): H_n(C_{\bullet}) \rightarrow H_n(C_{\bullet}')$ es un isomorfismo para cada $n \in \Z$.
\end{corolario}

\begin{proposicion}
	Sean $f,g: C_{\bullet} \rightarrow C_{\bullet}'$ y $f',g': C_{\bullet}' \rightarrow C_{\bullet}''$ aplicaciones de cadenas. Sean $s: f \rightarrow g$, $s': f' \rightarrow g'$ homotopías de cadenas entre ellas tales que $f \simeq g$, $f' \simeq g'$. Entonces la composición
	\[ f' s + s' g: f' \circ f \rightarrow g' \circ g \qquad g' \circ g : C_{\bullet} \rightarrow C_{\bullet}'' \]
	es una homotopía de cadenas.
\end{proposicion}
\begin{proof}
	Por ser $s,s'$ homotopías de cadenas tenemos que $\partial s + s\partial = f-g$ y $\partial s' + s'\partial = f'-g'$. Aplicando  $f'$ a la izquierda de la primera expresión y $g$ a la derecha de la segunda nos queda
	\begin{equation}
		\begin{cases}
			 f'\partial s + f's\partial = f' \circ f-f' \circ g \\
			 \partial s' g + s'\partial g = f' \circ g-g' \circ g 
		\end{cases}
	\end{equation}
	Sumando ambas igualdades
	\[ f'\partial s + f's\partial + \partial s' g + s'\partial g = f' \circ f-f' \circ g + f' \circ g-g' \circ g \]
	\[ f'\partial s + f's\partial + \partial s' g + s'\partial g = f' \circ f - g' \circ g \]
	\[ \partial f' s + f's \partial + \partial s' g + s' g \partial = f' \circ f - g' \circ g \]
	\[ \partial (f' s + s' g) + (f's + s' g) \partial = f' \circ f - g' \circ g \]
\end{proof}

\section{Subcomplejos y complejos cociente}

\begin{definicion}
	Un \textbf{subcomplejo} $S_{\bullet}$ de $C_{\bullet}$ es una familia de submódulos $S_n \subset C_n$ tal que para cada $n \in \Z$, $\partial S_n \subset S_{n-1}$.
\end{definicion}

Por tanto, $S_{\bullet}$ es un complejo en sí con el operador borde $\partial$ inducido de $C_{\bullet}$ y la inclusión $i: S_{\bullet} \rightarrow C_{\bullet}$ es una aplicación de cadenas.

\begin{definicion}
	Sea $S_{\bullet}$ un subcomplejo de $C_{\bullet}$. El \textbf{complejo cociente} $C_{\bullet}/S_{\bullet}$ es la familia $(C_{\bullet}/S_{\bullet})_n = C_n/S_n$ de módulos cocientes con operador borde $\partial'_n: C_n/S_n \rightarrow C_{n-1}/S_{n-1}$ inducido por ${\partial_C}_{\bullet}$.
\end{definicion}

%La proyección es la aplicación de cadenas $\pi: C_{\bullet} \rightarrow C_{\bullet}/S$ y la sucesión corta $S_n$ FLECHARARA $C_n FLECHA (C_{\bullet}/S)_n$ es exacta para cada $n \in \Z$.
%
%Si \( f: C_{\bullet} \to C_{\bullet}' \) es una aplicación de cadenas, entonces \( \ker(f) = \{\ker(f_n)\} \) es un subcomplejo de \( C_{\bullet} \), \( \text{Im}(f) = \{f_n(C_n)\} \) es un subcomplejo de \( C_{\bullet}' \), mientras que \( C_{\bullet}'/\text{Im}(f) \) es el cociente de \( f \) y \( C_{\bullet}/\ker(f) \) el conúcleo. 

\begin{definicion}
Sean \( f: C_{\bullet} \rightarrow C_{\bullet}' \), \( g: C_{\bullet}' \rightarrow C_{\bullet}'' \) aplicaciones de cadenas. La sucesión de complejos \[ C_{\bullet} \xrightarrow{f} C_{\bullet}' \xrightarrow{g} C_{\bullet}'' \] es \textbf{exacta} en \( C_{\bullet}' \) si \( \im(f) = \ker(g) \); es decir, si cada sucesión \( C_n \xrightarrow{f_n} {C'}_n \xrightarrow{g_n} {C''}_n \) de módulos es exacta en \( {C'}_n \).
\end{definicion}

%Para cualquier \( f: C_{\bullet} \to C_{\bullet}', \)
%\[ 0 \to \ker(f) \to C_{\bullet} \xrightarrow{f} C_{\bullet}' \to \text{Coker}(f) \to 0 \]
%es una sucesión exacta de complejos.

\begin{definicion}
Un complejo \( C_{\bullet} \) es \textbf{positivo} si \( C_n = 0 \) para todo \( n < 0 \) con $n \in \Z$. Su $n$-ésimo módulo de homología es entonces positivo  ya que \( H_n(C_{\bullet}) = 0 \) para todo \( n < 0 \). De manera análoga, un complejo \( C_{\bullet} \) es \textbf{negativo} si \( C_n = 0 \) para todo \( n > 0 \) con $n \in \Z$.
\end{definicion}

Los complejos negativos suelen notarse con índices superiores positivos donde $C_{-n}$ se sustituye por $C^n$ y $\partial_{-n} : C_{-n} \rightarrow C_{-n-1}$ por $\delta^n: C^n \rightarrow C^{n+1}$ quedando así
\[ 0 \to C^0 \xrightarrow{\delta^0} C^1 \xrightarrow{\delta^1} C^2 \to \cdots, \quad \delta \circ \delta = 0 \]
donde el $n$-ésimo módulo de homología \( H^n(C_{\bullet}) = \ker(\delta^n)/\text{Im}(\delta^{n-1}) \) es positivo en los índices superiores. A dicho complejo lo llamaremos \textbf{complejo de cocadenas}.

\begin{definicion}
	Sea $C_{\bullet}$ un complejo positivo de $R$-módulos. Denominaremos \textbf{aumento de $C_{\bullet}$} al homomorfismo sobreyectivo $\varepsilon : C_0 \to R$ de forma que $\varepsilon \circ \partial_1 = 0$.
\end{definicion}

\begin{definicion}
	Sea $C_{\bullet}$ un complejo de cadenas positivo, $\varepsilon : C_0 \to R$ un aumento de $C_{\bullet}$ y sea $n \in \Z$. Consideremos el complejo positivo $\widetilde{C}_{\bullet}$ tal que $\widetilde{C}_n = C_n$ para todo $n \geq 0$, $\widetilde{C}_n = 0$ para todo $n < -1$ y $\widetilde{C}_{-1}=R$. Consideremos también $\widetilde{\partial}_n = \partial_n$ para todo $n \geq 1$ y $\widetilde{\partial}_0= \epsilon$. Llamaremos a este complejo \textbf{complejo aumentado} de $C_{\bullet}$.
\end{definicion}

\begin{definicion}
Sean $f$ y $g$ aplicaciones de cadenas de $C_{\bullet}$ a $C_{\bullet}'$ y sea $s$ una homotopía de cadenas entre ellas. Diremos que $s$ es una \textbf{homotopía de cocadenas} si  está escrita con índices superiores. Esto es, \( s^n: C^n \to C'^{n-1} \) con \( \delta s + s \delta = f - g \).
\end{definicion}

\begin{definicion}
	Sea $A$ un módulo. Definimos el siguiente complejo positivo donde \( A_0 = A \), \( A_n = 0 \) para \( n \neq 0 \) y \( \partial = 0 \). Un \textbf{complejo sobre \( A \)} es un complejo positivo \( C_{\bullet} \) junto con una aplicación de cadenas \( \varepsilon: C_{\bullet} \to A \) donde \( \varepsilon \) es un homomorfismo de módulos \( \varepsilon: C_0 \to A \) tal que \( \varepsilon \partial = 0: C_1 \to A \).
\end{definicion}

\begin{definicion}
\label{def:contr_homotopy}
Una \textbf{homotopía contráctil} para \( \varepsilon: C_{\bullet} \to A \) es una aplicación de cadenas \( f: A \to C_{\bullet} \) tal que \( \varepsilon f = \id_A \) junto con una homotopía \( s: {\id_C}_{\bullet} \rightarrow f \varepsilon \) donde \( {\id_C}_{\bullet} \simeq f \varepsilon \). En otras palabras, una homotopía contráctil consiste en homomorfismos de módulos \( f: A \to C_0 \) y \( s_n: C_n \to C_{n+1}, n = 0, 1, \dots \), tal que

\[ \varepsilon f = \id_A, \quad \partial_{1} s_0 + f \varepsilon = \id_{C_0}, \quad \partial_{n+1} s_n + s_{n-1} \partial_n = \id_{C_n} \quad n > 0 \]
\end{definicion}

Podemos extender el complejo estableciendo \( C_{-1} = A \), \( \partial_0 = \varepsilon: C_0 \to C_{-1} \) y \( s_{-1} = f \). Aplicando la \autoref{def:contr_homotopy}, \( s: {\id_C}_{\bullet} \rightarrow 0 \) es una homotopía de cadenas. Si \( \varepsilon: C_{\bullet} \to A \) tiene una homotopía contráctil, sus grupos de homología son isomorfos por \( \varepsilon_*: H_0(C_{\bullet}) \rightarrow A \) para \( n = 0 \) y \( H_n(C_{\bullet}) = 0 \) para \( n > 0 \).

Los complejos \( C_{\bullet} \) de $\Z$-módulos libres surgen en topología. Si cada \( C_n \) es finitamente generado, entonces cada \( H_n(C_{\bullet}) \) es un grupo abeliano finitamente generado. El teorema de estructura para tales grupos presenta \( H_n(C_{\bullet}) \) como una suma directa
\[ \Z \oplus \ldots \oplus \Z \oplus \ldots \oplus \Z_{m_1} \oplus \ldots \oplus \Z_{m_k}, \]
donde el número \( \beta_n \) de sumandos cíclicos infinitos y los enteros \( m_1, \ldots, m_k \) (cada uno divisor del siguiente) dependen solo de \( H_n(C_{\bullet}) \). El entero \( \beta_n \) lo llamaremos el \textbf{n-ésimo número de Betti} de \( C_{\bullet} \) y a los \( \{m_i\} \) los \textbf{n-ésimos coeficientes de torsión}.

%\section{Sucesión de Mayer-Vietoris}

\endinput
%--------------------------------------------------------------------
% FIN DEL CAPÍTULO. 
%--------------------------------------------------------------------
