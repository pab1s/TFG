% !TeX root = ../tfg.tex
% !TeX encoding = utf8

\chapter{Homología}

Mediante el uso de estructuras algebraicas como grupos y complejos de cadenas, la homología asigna a cada espacio topológico una serie de grupos de homología, que reflejan características clave como agujeros y vacíos en diferentes dimensiones. Estos grupos permiten no sólo discernir la estructura interna de los espacios, sino también compararlos de manera abstracta. Usaremos de referencia \cite{maclane2012homology}.

\section{Grupos de homología}

Comenzaremos definiendo lo que es un grupo de homología y estableceremos la terminología que emplearemos cuando trabajemos con ellos.

\begin{definicion}
	Sea $C$ un grupo abeliano junto a un endomorfismo  $d: C \rightarrow C$ tal que $d^2 = d \circ d = 0$. Diremos entonces que $C$ es un \textbf{grupo diferencial} y llamaremos a $d$ \textbf{operador borde} de $C$.
\end{definicion}

Llamaremos a los elementos de $C$ \textbf{cadenas}. El subgrupo de \textbf{ciclos} será $Z(C) = \ker d$,  y el subgrupo de \textbf{bordes} $B(C) = \im d$. Si nos fijamos, el requisito $d^2 = 0$ es equivalente a exigir que $\im{d} \subset \ker{d}$.

\begin{definicion}
	Sea $C$ un grupo diferencial. Definimos el \textbf{grupo de homología} de $C$ como el grupo cociente $H(C)$ tal que 
	\[ H(C) = \frac{Z(C)}{B(C)}\]
\end{definicion}

Por tanto, el grupo de homología de un grupo diferencial $C$ está formado por las clases laterales $[c] = c + B(C)$ donde $c$ es un ciclo de $C$. A los elementos de $H(C)$ los llamaremos \textbf{clases de homología}. Dos ciclos $c$ y $c'$ diremos que son \textbf{homólogos} si ambos pertenecen a la misma clase de homología, esto es, $c \sim c'$.

\begin{definicion}
	Sean $C$ y $C'$ dos grupos diferenciales y $d, d'$ sus respectivos operadores borde. Diremos que $f: C \rightarrow C'$ es un \textbf{homomorfismo de grupos diferenciales} si $f$ es un homomorfismo de grupos y además $d'f = fd$.
\end{definicion}

La anterior definición nos permite preservar la estructura algebraica del grupo diferencial. De esta forma, si tomamos una cadena $c \in C$ que sea un ciclo o un borde y $f:C \rightarrow C'$ es un homomorfismo de grupos diferenciales, $f(c) \in C'$ seguirá siendo un ciclo o un borde de manera correspondiente.

\begin{definicion}
	Sean $C, C'$ grupos diferenciales y $f:C \rightarrow C'$ un homomorfismo de grupos diferenciales. Definimos la función $f_* = H(f): H(C) \rightarrow H(C')$ tal que 
	\[f_*([c]) = [f(c)] \]
	Diremos que $H(f)$ es el \textbf{homomorfismo inducido} por $f$.
\end{definicion}

En estas condiciones, $H$ es un funtor covariante de la categoría de grupos diferenciales a la categoría de grupos.

\section{Complejos de cadenas}

\begin{definicion}
	Sea $R$ un anillo. Un \textbf{complejo de cadenas} $K$ de $R$-módulos es una familia $\{K_n, \partial_n\}$ donde $K_n$ son $R$-módulos y $\partial_n : K_n \rightarrow K_{n-1}$ homomorfismos de $R$-módulos tales que $\partial_n \partial_{n+1} = 0$ para todo $n \in \Z$.
\end{definicion}
\begin{observacion}
	La última condición es equivalente a que $\im{\partial_{n+1}} \subset \ker{\partial_n}$.
\end{observacion}

Un complejo $K$  es por tanto una secuencia doblemente infinita
\[ K : \cdots \rightarrow K_{1} \rightarrow K_0 \rightarrow K_{-1} \rightarrow \cdots \]
donde toda composición de homomorfismos de dicha familia es el homomorfismo nulo. La \textbf{homología} $H(K)$ es la familia de módulos
\[ H_n(K) = \frac{\ker \partial_n}{\im \partial_{n+1}} \]
Luego $H_n(K)=0$ implica que la secuencia $K$ es exacta en $K_n$. A los elementos de $K_n$ los llamaremos \textbf{n-cadenas} o \textbf{cadenas de dimensión n}. Un \textbf{n-ciclo} o \textbf{ciclo de dimensión n} de $K$ es un elemento del submódulo $C_n(K) = \ker \partial_n$. Un \textbf{n-borde} o \textbf{borde de dimensión n} es un elemento de $\partial_{n+1}K_{n+1}$. Si la dimensión se sobrentiende en estos casos, no es necesario indicarla de manera explícita. La clase lateral de un ciclo $c$ la notaremos por $[c] = c + \partial K_{n+1}$. Dos $n$-ciclos $c,c' \in C_n(K)$ pertenecientes a la misma clase lateral $[c]$ decimos que son \textbf{homólogos}, es decir, $c \sim c'$.

\begin{definicion}
	Sean $K,K'$ complejos de cadenas. Una \textbf{aplicación de cadenas} o \textbf{morfismo de cadenas} $f: K \rightarrow K'$ es una familia de homomorfismos de $R$-módulos $f_n: K_n \rightarrow K_n'$ tal que $\partial_n'f_n = f_{n-1}\partial_n$ para todo $n \in \Z$.
\end{definicion}

\begin{equation}
	\label{chaincomplex}
	\xymatrix{
		\cdots & K_{n-1} \ar[l] \ar[d]^{f_{n-1}} \ar@{..>}[dr] & K_n \ar[l]_{\partial_{n}} \ar[d]^{f_n} \ar@{..>}[dr] & K_{n+1} \ar[l]_{\partial_{n+1}} \ar[d]^{f_{n+1}} & \cdots \ar[l] \\
		\cdots & K'_{n-1} \ar[l] & K'_n \ar[l]^{\partial'_{n}} & K'_{n+1} \ar[l]^{\partial'_{n+1}} & \cdots \ar[l]
    }
\end{equation}

Cuando se sobrentienda del contexto, notaremos simplemente por $\partial$ a los correspondientes $\partial_n$ y $\partial_n'$.

La función $H_n(f) = f_*$ definida por $f_*([c]) = f_*(c + \partial K_{n+1}) = fc + \partial K'_{n+1}$ es un homomorfismo $H_n(f): H_n(K) \rightarrow H_n(K')$. Así mismo, cada $H_n$ es un funtor covariante de la categoría de complejos de cadenas y morfismos de cadenas a la categoría de módulos.

\begin{definicion}
	\label{def:chain_homotopy}
	Sean $K,K'$ cadenas de complejos y $f,g: K \rightarrow K'$ dos aplicaciones de cadenas entre ellos. Una \textbf{homotopía de cadenas} u \textbf{homotopía algebraica} $s$ es una familia de homomorfismos de módulos $s_n: K_n \rightarrow K_{n+1}'$ para cada $n \in \Z$ tal que
	\begin{equation}
		\partial_{n+1}'s_n + s_{n-1} \partial_n = f_n - g_n
	\end{equation}
	Diremos entonces que $f$ y $g$ son \textbf{algebraicamente homotópicas} y escribiremos $f \simeq g$.
\end{definicion}

\begin{teorema}
	Si $s$ es una homotopía de cadenas entre $f,g: K \rightarrow K'$, entonces
	\[ H_n(f) = H_n(g) : H_n(K) \rightarrow H_n(K') \]
\end{teorema}
\begin{proof}
	Si $c$ es un ciclo de $K_n$, tenemos que $\partial_n c = 0$. Por la \autoref{def:chain_homotopy} se cumple que $f_nc-g_nc =  \partial s_n c$. Como consecuencia $f_n c$ y $g_n c$ son homólogos lo que implica que $[f_n c] = [g_n c]$ en $H_n(K')$, como queríamos demostrar.
\end{proof}

\begin{definicion}
	Una aplicación de cadenas $f: K \rightarrow K'$ es una \textbf{equivalencia de cadenas} si existe otra aplicación $h: K' \rightarrow K$ y homotopías $s: h \circ f \rightarrow \id_K$, $t: f \circ h \rightarrow \id_{K'}$ tales que $h \circ f \simeq \id_K$, $f \circ h \simeq \id_{K'}$.
\end{definicion}

Como $H_n(\id_K) = \id_{H_n(K)}$, del anterior teorema se deduce lo siguiente.

\begin{corolario}
	Si $f: K \rightarrow K'$ es una equivalencia de cadenas, la aplicación inducida $H_n(f): H_n(K) \rightarrow H_n(K')$ es un isomorfismo para cada $n \in \Z$.
\end{corolario}

\begin{proposicion}
	Sean $f,g: K \rightarrow K'$ y $f',g': K' \rightarrow K''$ aplicaciones de cadenas. Sean $s: f \rightarrow g$, $s': f' \rightarrow g'$ homotopías de cadenas entre ellas tales que $f \simeq g$, $f' \simeq g'$. Entonces la composición
	\[ f' s + s' g: f' \circ f \rightarrow g' \circ g \qquad g' \circ g : K \rightarrow K'' \]
	es una homotopía de cadenas.
\end{proposicion}
\begin{proof}
	Por ser $s,s'$ homotopías de cadenas tenemos que $\partial s + s\partial = f-g$ y $\partial s' + s'\partial = f'-g'$. Aplicando  $f'$ a la izquierda de la primera expresión y $g$ a la derecha de la segunda nos queda
	\begin{equation}
		\begin{cases}
			 f'\partial s + f's\partial = f' \circ f-f' \circ g \\
			 \partial s' g + s'\partial g = f' \circ g-g' \circ g 
		\end{cases}
	\end{equation}
	Sumando ambas igualdades
	\[ f'\partial s + f's\partial + \partial s' g + s'\partial g = f' \circ f-f' \circ g + f' \circ g-g' \circ g \]
	\[ f'\partial s + f's\partial + \partial s' g + s'\partial g = f' \circ f - g' \circ g \]
	\[ \partial f' s + f's \partial + \partial s' g + s' g \partial = f' \circ f - g' \circ g \]
	\[ \partial (f' s + s' g) + (f's + s' g) \partial = f' \circ f - g' \circ g \]
\end{proof}

\section{Subcomplejos y complejos cociente}

\begin{definicion}
	Un \textbf{subcomplejo} $S$ de $K$ es una familia de submódulos $S_n \subset K_n$ tal que para cada $n \in \Z$, $\partial S_n \subset S_{n-1}$.
\end{definicion}

Por tanto, $S$ es un complejo en sí con el operador borde $\partial$ inducido de $K$ y la inclusión $i: S \rightarrow K$ es una aplicación de cadenas.

\begin{definicion}
	Sea $S$ un subcomplejo de $K$. El \textbf{complejo cociente} $K/S$ es la familia $(K/S)_n = K_n/S_n$ de módulos cocientes con operador borde $\partial': K_n/S_n \rightarrow K_{n-1}/S_{n-1}$ inducido por $\partial_K$.
\end{definicion}

%La proyección es la aplicación de cadenas $\pi: K \rightarrow K/S$ y la secuencia corta $S_n$ FLECHARARA $K_n FLECHA (K/S)_n$ es exacta para cada $n \in \Z$.
%
%Si \( f: K \to K' \) es una aplicación de cadenas, entonces \( \ker(f) = \{\ker(f_n)\} \) es un subcomplejo de \( K \), \( \text{Im}(f) = \{f_n(K_n)\} \) es un subcomplejo de \( K' \), mientras que \( K'/\text{Im}(f) \) es el cociente de \( f \) y \( K/\ker(f) \) el conúcleo. 

\begin{definicion}
Sean \( f: K \rightarrow K' \), \( g: K' \rightarrow K'' \) aplicaciones de cadenas. La secuencia \( K \xrightarrow{f} K' \xrightarrow{g} K'' \) es \textbf{exacta} en \( K' \) si \( \im(f) = \ker(g) \); es decir, si cada secuencia \( K_n \xrightarrow{f_n} K'_n \xrightarrow{g_n} K''_n \) de módulos es exacta en \( K'_n \).
\end{definicion}

%Para cualquier \( f: K \to K', \)
%\[ 0 \to \ker(f) \to K \xrightarrow{f} K' \to \text{Coker}(f) \to 0 \]
%es una secuencia exacta de complejos.

Un complejo \( K \) es \textbf{positivo} si \( K_n = 0 \) para \( n < 0 \). Su homología es entonces positiva  ya que (\( H_n(K) = 0 \) para \( n < 0 \). De manera análoga, un complejo \( K \) es \textbf{negativo} si \( K_n = 0 \) para \( n > 0 \). Los complejos negativos suelen notarse con índices superiores positivos donde $K_{-n}$ se sustituye por $K^n$ y $\partial_n : K_{-n} \rightarrow K_{-n-1}$ por $\delta^n: K^n \rightarrow K^{n+1}$ quedando así
\[ 0 \to K^0 \xrightarrow{\delta^0} K^1 \xrightarrow{\delta^1} K^2 \to \cdots, \quad \delta \circ \delta = 0 \]
donde la homología \( H^n(K) = \ker(\delta^n)/\text{Im}(\delta^{n-1}) \) es positiva en los índices superiores. A dicho complejo lo llamaremos \textbf{complejo de cocadenas}.

\begin{definicion}
Sean $f$ y $g$ aplicaciones de cadenas de $K$ a $K'$ y sea $s$ una homotopía de cadenas entre ellas. Diremos que $s$ es una \textbf{homotopía de cocadenas} si  está escrita con índices superiores. Esto es, \( s^n: K^n \to K'^{n-1} \) con \( \delta s + s \delta = f - g \).
\end{definicion}

\begin{definicion}
	Sea $A$ un módulo. Definimos el siguiente complejo positivo donde \( A_0 = A \), \( A_n = 0 \) para \( n \neq 0 \) y \( \partial = 0 \). Un \textbf{complejo sobre \( A \)} es un complejo positivo \( K \) junto con una aplicación de cadenas \( \varepsilon: K \to A \) donde \( \varepsilon \) es un homomorfismo de módulos \( \varepsilon: K_0 \to A \) tal que \( \varepsilon \partial = 0: K_1 \to A \).
\end{definicion}

\begin{definicion}
\label{def:contr_homotopy}
Una \textbf{homotopía contráctil} para \( \varepsilon: K \to A \) es una aplicación de cadenas \( f: A \to K \) tal que \( \varepsilon f = \id_A \) junto con una homotopía \( s: \id_K \rightarrow f \varepsilon \) donde \( \id_K \simeq f \varepsilon \). En otras palabras, una homotopía contráctil consiste en homomorfismos de módulos \( f: A \to K_0 \) y \( s_n: K_n \to K_{n+1}, n = 0, 1, \dots \), tal que

\[ \varepsilon f = \id_A, \quad \partial_{1} s_0 + f \varepsilon = \id_{K_0}, \quad \partial_{n+1} s_n + s_{n-1} \partial_n = \id_{K_n} \quad n > 0 \]
\end{definicion}

Podemos extender el complejo estableciendo \( K_{-1} = A \), \( \partial_0 = \varepsilon: K_0 \to K_{-1} \) y \( s_{-1} = f \). Aplicando la \autoref{def:contr_homotopy}, \( s: \id_K \rightarrow 0 \) es una homotopía de cadenas. Si \( \varepsilon: K \to A \) tiene una homotopía contráctil, sus grupos de homología son isomorfos por \( \varepsilon_*: H_0(K) \rightarrow A \) para \( n = 0 \) y \( H_n(K) = 0 \) para \( n > 0 \).

Los complejos \( K \) de grupos abelianos libres surgen en topología. Si cada \( K_n \) es finitamente generado, entonces cada \( H_n(K) \) es un grupo abeliano finitamente generado. El teorema de estructura para tales grupos presenta \( H_n(K) \) como una suma directa
\[ \Z \oplus \Z \oplus \ldots \oplus \Z_{m_1} \oplus \ldots \oplus \Z_{m_k}, \]
donde el número \( b_n \), de sumandos cíclicos infinitos y los enteros \( m_1, \ldots, m_k \) (cada uno divisor del siguiente) dependen solo de \( H_n(K) \). El entero \( b_n \) se llama el \textbf{n-ésimo número de Betti} de \( K \), y los \( \{m_i\} \) los \textbf{n-ésimos coeficientes de torsión}.

\section{Sucesiones exactas}

\section{Sucesión de Mayer-Vietoris}

\endinput
%--------------------------------------------------------------------
% FIN DEL CAPÍTULO. 
%--------------------------------------------------------------------
