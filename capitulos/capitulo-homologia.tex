% !TeX root = ../tfg.tex
% !TeX encoding = utf8

\chapter{Homología}

Mediante el uso de estructuras algebraicas como módulos y complejos de cadenas,
la homología asigna a cada espacio topológico una serie de grupos de homología, que
reflejan características clave como agujeros y vacíos en diferentes dimensiones.
Estos grupos permiten no sólo discernir la estructura interna de los espacios, sino
también compararlos de manera abstracta. Usaremos de referencia
\cite{maclane2012homology}.

\section{Módulos diferenciales}

Comenzaremos definiendo lo que es un módulo de homología y estableceremos la terminología
que emplearemos cuando trabajemos con ellos.

\begin{definicion}
	Sea \(C\) un \(R\)-módulo junto a un endomorfismo \(d: C \rightarrow C\) tal que \(d^{2}
	= d \circ d = 0\). Diremos entonces que \(C\) es un \textbf{módulo diferencial} y
	llamaremos a \(d\) \textbf{operador borde} de \(C\).
\end{definicion}

Llamaremos a los elementos de \(C\) \textbf{cadenas}. El subgrupo de \textbf{ciclos}
será \(Z(C) = \ker d\), y el subgrupo de \textbf{bordes} \(B(C) = \im d\). Si nos
fijamos, el requisito \(d^{2} = 0\) es equivalente a exigir que \(\im{d}\subset \ker
{d}\).

\begin{definicion}
	Sea \(C\) un grupo diferencial. Definimos el \textbf{\(R\)-módulo de homología} de
	\(C\) como el módulo cociente \(H(C;R)\) tal que
	\[
		H(C;R) = \frac{Z(C)}{B(C)}
	\]
\end{definicion}

En particular, cuando \(C\) sea un \(\Z\)-módulo diferencial, lo llamaremos \textbf{grupo
diferencial} y notaremos \(H(C;\Z)\) simplemente por \(H(C)\).

Por tanto, el módulo de homología de un módulo diferencial \(C\) está formado por
las clases laterales \([c] = c + B(C)\) donde \(c\) es un ciclo de \(C\). A los elementos
de \(H(C;R)\) los llamaremos \textbf{clases de homología}. Dos ciclos \(c\) y \(c'\) diremos
que son \textbf{homólogos} si ambos pertenecen a la misma clase de homología, esto
es, \(c \sim c'\).

\begin{definicion}
	Sean \(C\) y \(C'\) dos módulos diferenciales y \(d, d'\) sus respectivos operadores
	borde. Diremos que \(f: C \rightarrow C'\) es un \textbf{homomorfismo de módulos
	diferenciales} si \(f\) es un homomorfismo de módulos y además \(d'f = fd\).
\end{definicion}

La anterior definición nos permite preservar la estructura algebraica del módulo
diferencial. De esta forma, si tomamos una cadena \(c \in C\) que sea un ciclo o un
borde y \(f:C \rightarrow C'\) es un homomorfismo de módulos diferenciales,
\(f(c) \in C'\) seguirá siendo un ciclo o un borde de manera correspondiente. En efecto,
pues si \(z \in Z(C)\), entonces \(d'f(z) = f(dz) = f(0) = 0\). Esto es, \(f(z) \in \ker
d'\). Ahora, si \(b \in B(C)\), entonces existe \(c \in C\) tal que \(dc = b\). En consecuencia,
\(d'f(c) = f(dc) = f(b)\) y por tanto, \(f(b) \in \im d'\).
%
%Los grupos diferenciales definen una categoría donde los objetos son los grupos diferenciales y los morfismos son los homomorfismos de grupos diferenciales. Tomamos como ley de composición interna la composición de dichos homomorfismos. Claramente es asociativa pues si \(C,C',\bar{C}, \tilde{C} \in \obj()\) y \(f: C \to C'\), \(g: C' \to \bar{C}\), \(h: \bar{C} \to \tilde{C}\), entonces \(h \circ (g \circ f)\) se cumple si, y sólo si,
%\begin{gather*}
%	\tilde{d}(h \circ (g \circ f))
%	= (\tilde{d}h) \circ (g \circ f)
%	=(h\bar{d}) \circ (g \circ f))
%	=h \circ (\bar{d} g) \circ f \\
%	=h \circ (g d') \circ f
%	=h \circ g \circ (d' f)
%	=h \circ g \circ (fd)
%	=(h \circ g) \circ fd \\
%	=((h \circ g) \circ f) d
%\end{gather*}
%y por tanto \(h \circ (g \circ f) = (h \circ g) \circ f\). La propiedad de identidad se sigue de existir el homomorfismo identidad de grupos.
\begin{definicion}
	Sean \(C, C'\) módulos diferenciales y \(f:C \rightarrow C'\) un homomorfismo de módulos
	diferenciales. Definimos la función \(f_{*} = H(f): H(C;R) \rightarrow H(C';R)\)
	tal que
	\[
		f_{*}([c]) = [f(c)]
	\]
	Diremos que \(H(f)\) es el \textbf{homomorfismo inducido} por \(f\).
\end{definicion}
%\begin{proposicion}
%	En estas condiciones, \(H\) es un funtor covariante de la categoría de grupos diferenciales a la categoría de grupos.
%\end{proposicion}
%\begin{proof}
%	Por la definición dada del grupo de homología, es claro que la función objeto \(H\) asigna a cada grupo diferencial \(C\) un grupo de homología \(H(C)\). En cuanto a la función de morfismos, la identidad de grupos diferenciales se preserva pues \(H(\id)([c]) = \id_*([c]) = [\id(c)] = [c]\) para todo \(c \in C\). Además, si \(f,g \in \hom(C)\), entonces \(H(g \circ f)([c]) = (g \circ f)_*([c]) = [(g \circ f)(c)] = [g (f (c))] = g_*([f(c)]) = g_* ( f_*([c])) = (H(g) \circ H(f)) ([c])\) para todo \(c \in C\), manteniendo la ley de composición.
%\end{proof}

\section{Complejos de cadenas}

\begin{definicion}
	Sea \(R\) un anillo. Un \textbf{complejo de cadenas} \(C_{\bullet}\) de \(R\)-módulos
	es una familia \(\{C_{n}, \partial_{n}\}\) donde \(C_{n}\) son \(R\)-módulos y \(\partial
	_{n} : C_{n} \rightarrow C_{n-1}\) homomorfismos de \(R\)-módulos tales que \(\partial
	_{n} \partial_{n+1}= 0\) para todo \(n \in \Z\).
\end{definicion}
\begin{nota}
	Usualmente notaremos directamente \(\partial\) al homomorfismo \(\partial_{n}\) independientemente
	del valor de \(n\) siempre y cuando se sobrentienda por el contexto.
\end{nota}
\begin{observacion}
	La última condición es equivalente a que \(\im{\partial_{n+1}}\subset \ker{\partial_n}\).
\end{observacion}
Un complejo \(C_{\bullet}\) es por tanto una sucesión doblemente infinita
\[
	C_{\bullet}: \cdots \rightarrow C_{1}\rightarrow C_{0} \rightarrow C_{-1}\rightarrow
	\cdots
\]
donde toda composición de homomorfismos de dicha familia es el homomorfismo nulo.
La \textbf{homología} \(H(C_{\bullet})\) es la familia de \(R\)-módulos
\[
	H_{n}(C_{\bullet}) = \frac{\ker \partial_{n}}{\im \partial_{n+1}}
\]
donde \(H_{n}(C_{\bullet})\) es el \textbf{\(n\)-ésimo módulo de homología} de \(C_{\bullet}\).

Luego \(H_{n}(C_{\bullet})=0\) implica que la sucesión \(C_{\bullet}\) es exacta en \(C
_{n}\). A los elementos de \(C_{n}\) los llamaremos \textbf{n-cadenas} o \textbf{cadenas
de dimensión n}. Un \textbf{n-ciclo} o \textbf{ciclo de dimensión n} de \(C_{\bullet}\)
es un elemento del submódulo \(Z_{n}(C_{\bullet}) = \ker \partial_{n}\). Un
\textbf{n-borde} o \textbf{borde de dimensión n} es un elemento de
\(B_{n}(C_{\bullet})= \im{\partial_{n+1}}\). Si la dimensión se sobrentiende en estos
casos, no es necesario indicarla de manera explícita. La clase lateral de un ciclo
\(c\) la notaremos por \([c] = c + \partial C_{n+1}\). Dos \(n\)-ciclos
\(c,c' \in C_{n}\) pertenecientes a la misma clase lateral \([c]\) decimos que son
\textbf{homólogos}, es decir, \(c \sim c'\).

\begin{definicion}
	Sea \(\{C_{\bullet}^{i}, \partial^{i}\}_{i \in I}\) una familia de complejos de cadenas.
	Su \textbf{suma directa} la definimos como el complejo de cadenas \(\bigoplus_{i
	\in I}C^{i}_{\bullet}\) cuyos operadores borde vienen dados por \(\bigoplus_{i
	\in I}{\partial}^{i}_{n} : \bigoplus_{i \in I}C^{i}_{n} \to \bigoplus_{i \in I}
	C^{i}_{n-1}\) para todo \(n \in \Z\).
\end{definicion}
\begin{proposicion}
	\label{prop:hom-free-commute} Sea \(\{C_{\bullet}^{i}, \partial^{i}\}_{i \in I}\)
	una familia de complejos de cadenas. Entonces su homología conmuta con la suma
	directa, esto es,
	\(H_{n}(\bigoplus_{i \in I}C_{\bullet}^{i}) \cong \bigoplus_{i \in I}H_{n}(C_{\bullet}
	^{i})\)
	para todo \(n \in \Z\).
\end{proposicion}
\begin{proof}
	Para demostrar que la homología conmuta con sumas directas, queremos mostrar
	que para una colección de complejos de cadenas
	\(\{C_{\bullet}^{i}, \partial^{i}\}_{i \in I}\), los homomorfismos
	\begin{align*}
		\phi: H_{n} \left( \bigoplus_{i \in I}C_{\bullet}^{i} \right) \rightarrow \bigoplus_{i \in I}H_{n} (C_{\bullet}^{i}) : [(c_{i})] \mapsto ([c_{i}]), \\
		\psi: \bigoplus_{i \in I}H_{n} (C_{\bullet}^{i}) \rightarrow H_{n} \left( \bigoplus_{i \in I}C_{\bullet}^{i} \right) : ([c_{i}]) \mapsto [(c_{i})],
	\end{align*}
	están bien definidos y son inversos mutuos.

	En primer lugar, para comprobar que dichas aplicaciones están bien definidas, observemos
	que \([(c_{i})] = [(c_{i}')]\) si, y sólo si, \([0] = [(c_{i} - c_{i}')]\). Esto
	ocurre si, y sólo si, existe un \(b_{i} \in C_{\bullet}^{i}\) tal que
	\(\partial_{i}(b_{i}) = (c_{i} - c_{i}')\), lo cual es equivalente a \(c_{i} + \partial
	_{i}(b_{i}) = c_{i}'\) para cada \(i \in I\). Por lo tanto, \([(c_{i})] = [(c_{i}')
	]\) si, y sólo si, \(\phi([(c_{i})]) = \phi([(c_{i}')]) = [(c_{i}' + \partial_{i}
	(b_{i}))] = [(c_{i}')]\). De manera análoga, \([(c_{i})] = [(c_{i}')]\) si, y
	sólo si, \(\psi([(c_{i})]) = \psi([(c_{i}')])\). Esto implica que tanto \(\phi\)
	como \(\psi\) están bien definidos.

	En segundo lugar, es claro que \(\phi\) y \(\psi\) son homomorfismos de \(R\)-módulos.
	Además, \(\phi\) lleva la clase de equivalencia \([(c_{i})]\) a \(([c_{i}])\),
	mientras que \(\psi\) lleva \(([c_{i}])\) a \([(c_{i})]\), lo que demuestra que son inversos
	el uno del otro.

	Por lo tanto, \(\phi\) es un isomorfismo con \(\psi\) como su inverso, estableciendo
	que \(\bigoplus_{i \in I}H_{n} (C_{\bullet}^{i}) \cong H_{n} \left( \bigoplus_{i
	\in I}C_{\bullet}^{i} \right)\).
\end{proof}
\begin{definicion}
	Sean \(C_{\bullet},C_{\bullet}'\) complejos de cadenas. Una \textbf{aplicación
	de cadenas} o \textbf{morfismo de cadenas}
	\(f: C_{\bullet}\rightarrow C_{\bullet}'\) es una familia de homomorfismos de
	\(R\)-módulos \(f_{n}: C_{n} \rightarrow C_{n}'\) tal que
	\(\partial_{n}'f_{n} = f_{n-1}\partial_{n}\) para todo \(n \in \Z\).
\end{definicion}

\begin{equation}
	\label{chaincomplex}\xymatrix{ \cdots & C_{n-1} \ar[l] \ar[d]^{f_{n-1}} \ar@{..>}[dr] & C_n \ar[l]_{\partial_{n}} \ar[d]^{f_n} \ar@{..>}[dr] & C_{n+1} \ar[l]_{\partial_{n+1}} \ar[d]^{f_{n+1}} & \cdots \ar[l] \\ \cdots & C'_{n-1} \ar[l] & C'_n \ar[l]^{\partial'_{n}} & C'_{n+1} \ar[l]^{\partial'_{n+1}} & \cdots \ar[l] }
\end{equation}

Cuando se sobrentienda del contexto, notaremos simplemente por \(\partial\) a los
correspondientes \(\partial_{n}\) y \(\partial_{n}'\).

La función \(H_{n}(f) = f_{*}\) definida por
\(f_{*}([c]) = f_{*}(c + \partial C_{n+1}) = fc + \partial C'_{n+1}\) es un
homomorfismo \(H_{n}(f): H_{n}(C_{\bullet}) \rightarrow H_{n}(C_{\bullet}')\).% Así mismo, cada \(H_n\) es un funtor covariante de la categoría de complejos de cadenas y morfismos de cadenas a la categoría de módulos.

\begin{definicion}
	\label{def:chain_homotopy} Sean \(C_{\bullet},C_{\bullet}'\) complejos de cadenas
	y \(f,g: C_{\bullet}\rightarrow C_{\bullet}'\) dos aplicaciones de cadenas entre
	ellos. Una \textbf{homotopía de cadenas} u \textbf{homotopía algebraica} \(s\)
	es una familia de homomorfismos de módulos \(s_{n}: C_{n} \rightarrow C_{n+1}'\)
	para cada \(n \in \Z\) tal que
	\begin{equation}
		\partial_{n+1}'s_{n} + s_{n-1}\partial_{n} = f_{n} - g_{n}
	\end{equation}
	Diremos entonces que \(f\) y \(g\) son \textbf{algebraicamente homotópicas} y escribiremos
	\(f \simeq g\).
\end{definicion}

\begin{teorema}
	\label{teo:homot-cad-misma-homologia} Si \(s\) es una homotopía de cadenas entre
	\(f,g: C_{\bullet}\rightarrow C_{\bullet}'\), entonces
	\[
		H_{n}(f) = H_{n}(g) : H_{n}(C_{\bullet}) \rightarrow H_{n}(C_{\bullet}')
	\]
\end{teorema}
\begin{proof}
	Si \(c\) es un ciclo de \(C_{n}\), tenemos que \(\partial_{n} c = 0\). Por la \autoref{def:chain_homotopy}
	se cumple que \(f_{n}c-g_{n}c = \partial s_{n} c\). Como consecuencia \(f_{n} c\)
	y \(g_{n} c\) son homólogos lo que implica que \([f_{n} c] = [g_{n} c]\) en
	\(H_{n}(C_{\bullet}')\), como queríamos demostrar.
\end{proof}

\begin{definicion}
	Una aplicación de cadenas \(f: C_{\bullet}\rightarrow C_{\bullet}'\) es una \textbf{equivalencia
	de cadenas} si existe otra aplicación
	\(h: C_{\bullet}' \rightarrow C_{\bullet}\) y homotopías
	\(s: h \circ f \rightarrow{\id_C}_{\bullet}\), \(t: f \circ h \rightarrow{\id_C'}_{\bullet}\)
	tales que \(h \circ f \simeq{\id_C}_{\bullet}\),
	\(f \circ h \simeq{\id_{C'}}_{\bullet}\).
\end{definicion}

Como \(H_{n}({\id_C}_{\bullet}) = \id_{H_n(C_{\bullet})}\), del anterior teorema se
deduce lo siguiente.

\begin{corolario}
	\label{cor:equiv-homot} Si \(f: C_{\bullet}\rightarrow C_{\bullet}'\) es una equivalencia
	de cadenas, la aplicación inducida
	\(H_{n}(f): H_{n}(C_{\bullet}) \rightarrow H_{n}(C_{\bullet}')\) es un
	isomorfismo para cada \(n \in \Z\).
\end{corolario}

\begin{proposicion}
	Sean \(f,g: C_{\bullet}\rightarrow C_{\bullet}'\) y \(f',g': C_{\bullet}' \rightarrow
	C_{\bullet}''\) aplicaciones de cadenas. Sean \(s: f \rightarrow g\), \(s': f' \rightarrow
	g'\) homotopías de cadenas entre ellas tales que \(f \simeq g\), \(f' \simeq g'\). Entonces
	la composición
	\[
		f' s + s' g: f' \circ f \rightarrow g' \circ g \qquad g' \circ g : C_{\bullet}
		\rightarrow C_{\bullet}''
	\]
	es una homotopía de cadenas.
\end{proposicion}
\begin{proof}
	Por ser \(s,s'\) homotopías de cadenas tenemos que \(\partial s + s\partial = f-g\)
	y \(\partial s' + s'\partial = f'-g'\). Aplicando \(f'\) a la izquierda de la
	primera expresión y \(g\) a la derecha de la segunda nos queda
	\begin{equation}
		\begin{cases}
			f'\partial s + f's\partial = f' \circ f-f' \circ g,   \\
			\partial s' g + s'\partial g = f' \circ g-g' \circ g.
		\end{cases}
	\end{equation}
	Sumando ambas igualdades
	\begin{align*}
		f'\partial s + f's\partial + \partial s' g + s'\partial g    & = f' \circ f-f' \circ g + f' \circ g-g' \circ g, \\
		f'\partial s + f's\partial + \partial s' g + s'\partial g    & = f' \circ f - g' \circ g,                       \\
		\partial f' s + f's \partial + \partial s' g + s' g \partial & = f' \circ f - g' \circ g,
	\end{align*}
	donde finalmente queda
	\begin{align*}
		\partial (f' s + s' g) + (f's + s' g) \partial & = f' \circ f - g' \circ g .
	\end{align*}
\end{proof}

\section{Subcomplejos y complejos cociente}

\begin{definicion}
	Un \textbf{subcomplejo} \(S_{\bullet}\) de \(C_{\bullet}\) es una familia de
	submódulos \(S_{n} \subset C_{n}\) tal que para cada \(n \in \Z\), \(\partial S_{n}
	\subset S_{n-1}\).
\end{definicion}

Por tanto, \(S_{\bullet}\) es un complejo en sí con el operador borde \(\partial\)
inducido de \(C_{\bullet}\) y la inclusión \(i: S_{\bullet}\rightarrow C_{\bullet}\)
es una aplicación de cadenas.

\begin{definicion}
	Sea \(S_{\bullet}\) un subcomplejo de \(C_{\bullet}\). El \textbf{complejo
	cociente} \(C_{\bullet}/S_{\bullet}\) es la familia \((C_{\bullet}/S_{\bullet})_{n}
	= C_{n}/S_{n}\) de módulos cocientes con operador borde \(\partial'_{n}: C_{n}/S_{n}
	\rightarrow C_{n-1}/S_{n-1}\) inducido por \({\partial_C}_{\bullet}\).
\end{definicion}

%La proyección es la aplicación de cadenas \(\pi: C_{\bullet} \rightarrow C_{\bullet}/S\) y la sucesión corta \(S_n\) FLECHARARA \(C_n FLECHA (C_{\bullet}/S)_n\) es exacta para cada \(n \in \Z\).
%
%Si \( f: C_{\bullet} \to C_{\bullet}' \) es una aplicación de cadenas, entonces \( \ker(f) = \{\ker(f_n)\} \) es un subcomplejo de \( C_{\bullet} \), \( \text{Im}(f) = \{f_n(C_n)\} \) es un subcomplejo de \( C_{\bullet}' \), mientras que \( C_{\bullet}'/\text{Im}(f) \) es el cociente de \( f \) y \( C_{\bullet}/\ker(f) \) el conúcleo.

\begin{definicion}
	Sean \(f: C_{\bullet}\rightarrow C_{\bullet}'\),
	\(g: C_{\bullet}' \rightarrow C_{\bullet}''\) aplicaciones de cadenas. La sucesión
	de complejos
	\[
		C_{\bullet}\xrightarrow{f}C_{\bullet}' \xrightarrow{g}C_{\bullet}''
	\]
	es \textbf{exacta} en \(C_{\bullet}'\) si \(\im(f) = \ker(g)\); es decir, si cada
	sucesión \(C_{n} \xrightarrow{f_n}{C'}_{n} \xrightarrow{g_n}{C''}_{n}\) de
	módulos es exacta en \({C'}_{n}\).
\end{definicion}

%Para cualquier \( f: C_{\bullet} \to C_{\bullet}', \)
%\[ 0 \to \ker(f) \to C_{\bullet} \xrightarrow{f} C_{\bullet}' \to \text{Coker}(f) \to 0 \]
%es una sucesión exacta de complejos.

\begin{definicion}
	Un complejo \(C_{\bullet}\) es \textbf{positivo} si \(C_{n} = 0\) para todo
	\(n < 0\) con \(n \in \Z\). Su \(n\)-ésimo módulo de homología es entonces positivo ya
	que \(H_{n}(C_{\bullet}) = 0\) para todo \(n < 0\). De manera análoga, un complejo
	\(C_{\bullet}\) es \textbf{negativo} si \(C_{n} = 0\) para todo \(n > 0\) con
	\(n \in \Z\).
\end{definicion}

Los complejos negativos suelen notarse con índices superiores positivos donde \(C_{-n}\)
se sustituye por \(C^{n}\) y \(\partial_{-n}: C_{-n}\rightarrow C_{-n-1}\) por \(\delta
^{n}: C^{n} \rightarrow C^{n+1}\) quedando así
\[
	0 \to C^{0} \xrightarrow{\delta^0}C^{1} \xrightarrow{\delta^1}C^{2} \to \cdots,
	\quad \delta \circ \delta = 0
\]
donde el \(n\)-ésimo módulo de homología \(H^{n}(C_{\bullet}) = \ker(\delta^{n})/\text{Im}
(\delta^{n-1})\) es positivo en los índices superiores. A dicho complejo lo
llamaremos \textbf{complejo de cocadenas}.

\begin{definicion}
	Sea \(C_{\bullet}\) un complejo positivo de \(R\)-módulos. Denominaremos \textbf{aumento
	de \(C_{\bullet}\)} al homomorfismo sobreyectivo \(\varepsilon : C_{0} \to R\) de forma
	que \(\varepsilon \circ \partial_{1} = 0\).
\end{definicion}

\begin{definicion}
	Sea \(C_{\bullet}\) un complejo de cadenas positivo, \(\varepsilon : C_{0} \to R\)
	un aumento de \(C_{\bullet}\) y sea \(n \in \Z\). Consideremos el complejo positivo
	\(\widetilde{C}_{\bullet}\) tal que \(\widetilde{C}_{n} = C_{n}\) para todo \(n \geq
	0\), \(\widetilde{C}_{n} = 0\) para todo \(n < -1\) y \(\widetilde{C}_{-1}=R\). Consideremos
	también \(\widetilde{\partial}_{n} = \partial_{n}\) para todo \(n \geq 1\) y \(\widetilde
	{\partial}_{0}= \epsilon\). Llamaremos a este complejo \textbf{complejo
	aumentado} de \(C_{\bullet}\).
\end{definicion}
%
%\begin{definicion}
%Sean \(f\) y \(g\) aplicaciones de cadenas de \(C_{\bullet}\) a \(C_{\bullet}'\) y sea \(s\) una homotopía de cadenas entre ellas. Diremos que \(s\) es una \textbf{homotopía de cocadenas} si  está escrita con índices superiores. Esto es, \( s^n: C^n \to C'^{n-1} \) con \( \delta s + s \delta = f - g \).
%\end{definicion}

\begin{definicion}
	Sea \(A\) un módulo. Definimos el siguiente complejo positivo donde \(A_{0} = A\),
	\(A_{n} = 0\) para \(n \neq 0\) y \(\partial = 0\). Un \textbf{complejo sobre \(A\)} es
	un complejo positivo \(C_{\bullet}\) junto con una aplicación de cadenas \(\varepsilon
	: C_{\bullet}\to A\) donde \(\varepsilon\) es un homomorfismo de módulos \(\varepsilon
	: C_{0} \to A\) tal que \(\varepsilon \partial = 0: C_{1} \to A\).
\end{definicion}

\begin{definicion}
	\label{def:contr_homotopy} Una \textbf{homotopía contráctil} para
	\(\varepsilon: C_{\bullet}\to A\) es una aplicación de cadenas
	\(f: A \to C_{\bullet}\) tal que \(\varepsilon f = \id_{A}\) junto con una
	homotopía \(s:{\id_C}_{\bullet}\rightarrow f \varepsilon\) donde
	\({\id_C}_{\bullet}\simeq f \varepsilon\). En otras palabras, una homotopía
	contráctil consiste en homomorfismos de módulos \(f: A \to C_{0}\) y
	\(s_{n}: C_{n} \to C_{n+1}, n = 0, 1, \ldots\), tal que

	\[
		\varepsilon f = \id_{A}, \quad \partial_{1}s_{0} + f \varepsilon = \id_{C_0},
		\quad \partial_{n+1}s_{n} + s_{n-1}\partial_{n} = \id_{C_n}\quad n > 0.
	\]
\end{definicion}

Podemos extender el complejo estableciendo \(C_{-1}= A\),
\(\partial_{0} = \varepsilon: C_{0} \to C_{-1}\) y \(s_{-1}= f\). Aplicando la \autoref{def:contr_homotopy},
\(s:{\id_C}_{\bullet}\rightarrow 0\) es una homotopía de cadenas. Si \(\varepsilon:
C_{\bullet}\to A\) tiene una homotopía contráctil, sus grupos de homología son
isomorfos por \(\varepsilon_{*}: H_{0}(C_{\bullet}) \rightarrow A\) para \(n = 0\) y
\(H_{n}(C_{\bullet}) = 0\) para \(n > 0\).

Considérese un complejo de cadenas
\(C_{\bullet}= \{ C_{n}, d_{n} \}_{n \in \mathbb{Z}}\), donde cada \(C_{n}\) es un \(\mathbb{Z}\)-módulo
libre y \(d_{n}: C_{n} \to C_{n-1}\) es el operador diferencial de \(C_{\bullet}\) que
cumple \(d_{n-1}\circ d_{n} = 0\) para todo \(n\). Este tipo de complejos aparece
frecuentemente en el estudio de espacios topológicos.

Supóngase además que cada \(C_{n}\) es finitamente generado. Entonces, el n-ésimo
grupo de homología de \(C_{\bullet}\), definido como
\[
	H_{n}(C_{\bullet}) = \frac{\ker(d_{n})}{\im(d_{n+1})},
\]
es un grupo abeliano finitamente generado. Este resultado se sigue del hecho de
que el núcleo y la imagen de los morfismos entre \(\mathbb{Z}\)-módulos libres
finitamente generados son también finitamente generados.

El teorema de estructura para grupos abelianos finitamente generados afirma que
cualquier grupo abeliano finitamente generado \(G\) puede expresarse como una suma
directa de grupos cíclicos de la forma
\[
	G \cong \mathbb{Z}^{\beta}\oplus \mathbb{Z}_{m_1}\oplus \ldots \oplus \mathbb{Z}
	_{m_k},
\]
donde \(\beta\) es el rango de \(G\) y cada \(\mathbb{Z}_{m_i}\) es un grupo cíclico
de orden \(m_{i}\), con \(m_{1} | m_{2} | \ldots | m_{k}\). Aplicando este teorema
al n-ésimo módulo de homología \(H_{n}(C_{\bullet})\), obtenemos que
\[
	H_{n}(C_{\bullet}) \cong \mathbb{Z}^{\beta_n}\oplus \mathbb{Z}_{m_1}\oplus \ldots
	\oplus \mathbb{Z}_{m_k},
\]
donde \(\beta_{n}\) es el rango de \(H_{n}(C_{\bullet})\), conocido como el \textbf{n-ésimo
número de Betti} de \(C_{\bullet}\), y los \(m_{i}\) son los \textbf{n-ésimos
coeficientes de torsión}, satisfaciendo \(m_{1} | m_{2} | \ldots | m_{k}\).
\begin{definicion}
	Sea \(C_{\bullet}\) un complejo de cadenas y \(k\) un entero no negativo. El
	\textbf{\(k\)-ésimo número de Betti}, \(\beta_{k}(C_{\bullet})\), se define como el
	rango del k-ésimo grupo de homología de \(C_{\bullet}\), \(H_{k}(C_{\bullet})\). Esto
	es, \(\beta_{k}(C_{\bullet}) = \rg(H_{k}(C_{\bullet}))\).
\end{definicion}
Los números de Betti \(\beta_{n}\) proporcionan una medida de la dimensionalidad de
la n-ésima homología, mientras que los coeficientes de torsión \(\{m_{i}\}\)
capturan información adicional sobre la estructura algebraica de
\(H_{n}(C_{\bullet})\). Ambos conjuntos de invariantes son cruciales para el estudio
de la topología algebraica, ya que ofrecen una perspectiva profunda sobre la
estructura de los espacios topológicos desde el punto de vista homológico.
%
%Los complejos \( C_{\bullet} \) de \(\Z\)-módulos libres aparecen en topología. Si cada \( C_n \) es finitamente generado, entonces cada \( H_n(C_{\bullet}) \) es un grupo abeliano finitamente generado. El teorema de estructura para tales grupos presenta \( H_n(C_{\bullet}) \) como una suma directa
%\[ \Z \oplus \ldots \oplus \Z \oplus \ldots \oplus \Z_{m_1} \oplus \ldots \oplus \Z_{m_k}, \]
%donde el número \( \beta_n \) de sumandos cíclicos infinitos y los enteros \( m_1, \ldots, m_k \) (cada uno divisor del siguiente) dependen solo de \( H_n(C_{\bullet}) \). El entero \( \beta_n \) lo llamaremos el \textbf{n-ésimo número de Betti} de \( C_{\bullet} \) y a los \( \{m_i\} \) los \textbf{n-ésimos coeficientes de torsión}.

\endinput
%--------------------------------------------------------------------
% FIN DEL CAPÍTULO.
%--------------------------------------------------------------------