% !TeX root = ../tfg.tex
% !TeX encoding = utf8

\chapter{Invarianza homotópica}
\section{Subdivisión baricéntrica}
\section{Teorema de aproximación simplicial}
\section{Invarianza homotópica de la homología simplicial}

\begin{definicion}
	\label{def:op_subdiv}
	Sea $\{C_n(K;R), \partial_n\}$ un complejo de cadenas simpliciales. Llamaremos \textbf{operador subdivisión} al homomorfismo ${\sd_*}_n : C_n(K;R) \to C_n(\sd K;R)$ tal que
	\[
	{\sd_*}_n(\sigma) = ?
	\]
\end{definicion}

\begin{lema}
	Sea $\{C_n(K;R), \partial_n\}$ un complejo de cadenas simpliciales y sean ${\sd_*}_n : C_n(K;R) \to C_n(\sd K;R)$ sus respectivos operadores subdivisión para todo $n \geq 0$. Entonces la familia $\{{\sd_*}_n\}$ es un homomorfismo de complejos de cadenas.
\end{lema}
\begin{proof}
	contenidos...
\end{proof}

\begin{lema}
	\label{lem:subdiv_cadenas}
	Sea $f: |K| \to |L|$ una aplicación simplicial y sea $f: |\sd K| \to |\sd L|$ la misma aplicación para sus subdivisiones baricéntricas. Entonces el siguiente diagrama es conmutativo
	
	DIAGRAMA
	
	donde $f_*$ denota $C_n(f)$.
\end{lema}
\begin{proof}
	contenidos...
\end{proof}

\begin{proposicion}
	Sea $\{C_n(K;R), \partial_n\}$ un complejo de cadenas simpliciales y  y sean $\{{\sd_*}_n\}$ sus respectivos operadores subdivisión para todo $n \geq 0$. Entonces dicha familia induce un homomorfismo $\sd_*: H_n(K;R) \to H_n(\sd K;R)$ que de la misma forma llamaremos \textbf{operador subdivisión}. Si $f: |K| \to |L|$ es una aplicación simplicial, entonces el siguiente diagrama es conmutativo
	
	DIAGRAMA
\end{proposicion}
\begin{proof}
	Es consecuencia inmediata de \autoref{lem:subdiv_cadenas}.
\end{proof}

\begin{definicion}
	Sea $\sd_*: H_n(K;R) \to H_n(\sd K;R)$ el operador subdivisión asociado al módulo de homología $H_n(\sd K;R)$. Entonces diremos que 
	\[
	\sd_*^m = \sd_* \circ \sd_* \circ \overset{m}{\cdots} \circ \sd_*: H_n(K;R) \to H_n(\sd K;R)
	\]
	es el \textbf{$m$-ésimo operador subdivisión} del complejo $K$.
\end{definicion}

\begin{proposicion}
	Sea $K$ un complejo simplicial finito. Se tiene que su $m$-ésimo operador subdivisión es un isomorfismo.
\end{proposicion}
\begin{proof}
	contenidos... IMPORTANTE: CONOS, MAYER-VIETORIS, LEMA DE LOS CINCO (III*1.7)
\end{proof}

\begin{proposicion}
	Sea $K$ un complejo simplicial. Por tanto existe una aproximación simplicial $\lambda: \sd^n K \to K$ de la identidad $\id: |K| \to |K|$ tal que $\lambda_* \circ \sd_* = \id_*$.
\end{proposicion}
\begin{proof}
	contenidos...
\end{proof}

\endinput
%--------------------------------------------------------------------
% FIN DEL CAPÍTULO. 
%--------------------------------------------------------------------
