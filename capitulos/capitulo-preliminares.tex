% !TeX root = ../tfg.tex
% !TeX encoding = utf8

\chapter{Preliminares algebraicos}

\section{Módulos}

\begin{definicion}
	Sea $R$ un anillo cuyo elemento identidad $1 \neq 0$. Un \textbf{$R$-módulo izquierdo} $A$ es un grupo abeliano aditivo junto con una función $p: R \times A \rightarrow A$ con $(r, a) \to ra$ tal que dados $r,r' \in R$, $a,a' \in A$ se tiene
	\begin{enumerate}
		\item $(r+r')+a = ra + r'a$
		\item $(rr')a = r(r'a)$
		\item $r(a+a') = ra + ra'$
		\item $1a = a$
	\end{enumerate}
\end{definicion}

De la definición anterior se sigue que $0a = 0$ y $(-1)a = -a$.

De manera análoga, definimos \textbf{$R$-módulo derecho} donde el anillo actúa por la izquierda en vez de por la derecha de forma que $p: A \times R \rightarrow A$. Si $R$ es un anillo conmutativo, los $R$-módulos izquierdos y derechos coinciden y les llamamos simplemente $R$-módulos. Como los resultados de $R$-módulos izquierdos y derechos son análogos, trabajaremos simplemente con los $R$-módulos izquierdos y nos referiremos a ellos como $R$-módulos a menos que se indique explícitamente lo contrario.

\begin{ejemplo}
	ej
\end{ejemplo}

\begin{definicion}
	Sea $A$ un $R$-módulo izquierdo y $S$ un subconjunto de $A$. Diremos que $S$ es un submódulo izquierdo de $A$, esto es, $S \subset A$, si $S$ es cerrado respecto a la suma y si $r,s \in S$ entonces $rs \in S$. Por tanto, $S$ es un $R$-módulo.
\end{definicion}

Si un submódulo de $R$ es un subconjunto $L \subset R$ cerrado respecto a la suma tal que $rL \subset L$ para todo $r \in R$, lo llamaremos \textbf{ideal izquierdo} de $R$. Tomando un ideal izquierdo $L$ de $R$ y $A$ un $R$-módulo izquierdo, definimos el producto del ideal $L$ por el módulo $A$
\[ LA = \{\text{todas las sumas finitas } \sum l_ia_i, \text{ para } l_i \in L,\ a_i \in A \} \]
donde $LA$ es un submódulo de $A$. En particular, el producto de dos ideales izquierdos $LL'$ es también un ideal izquierdo y $(LL')A = L(L'A)$.

\begin{definicion}
	Sean $A$, $B$ $R$-módulos. Definimos el \textbf{homomorfismo de $R$-módulos} de $A$ a $B$ como la aplicación $\alpha: A \rightarrow B$ tal que
	\begin{enumerate}
		\item $\alpha(a+a') = \alpha a + \alpha a'$
		\item $\alpha(ra) = r(\alpha a)$
	\end{enumerate}
	para todo $a,a' \in A$, $r \in R$.
\end{definicion}

También es frecuente escribir el homomorfismo de $R$-módulos $\alpha: A \rightarrow B$ como $A \xrightarrow{\alpha} B$.

Cuando $\alpha: A \rightarrow B$ sea un homomorfismo de $R$-módulos, diremos que $A$ es el \textbf{dominio} y $B$ el \textbf{rango}. La \textbf{imagen} de $\alpha$ es el conjunto $\im(\alpha) = \{ \alpha(a) : a \in A \}$. El \textbf{núcleo} será el conjunto de elementos que se anulan en su imagen, esto es, $\ker(\alpha) = \{ a \in A : \alpha(a) = 0 \}$. Diremos que $\alpha$ es un \textbf{epimorfismo} cuando $\alpha(A) = B$, un \textbf{monomorfismo} cuando $\ker(\alpha) = \{0\}$ y un \textbf{isomorfismo} si $\alpha$ es un epimorfismo y un monomorfismo a la vez. Si existe un isomorfismo entre $A$ y $B$ diremos que son \textbf{isomorfos} y lo notaremos $A \cong B$. Un homomorfismo $\omega: A \rightarrow A$ lo llamaremos \textbf{endomorfismo}.

Dados dos homomorfismos de $R$-módulos $\alpha_1, \alpha_2 : A \rightarrow B$, su \textbf{suma} $\alpha_1 + \alpha_2$ la definimos como $(\alpha_1 + \alpha_2)(a) = \alpha_1(a) + \alpha_2(a)$ para todo $a \in A$. Además, dados dos homomorfismos de $R$-módulos $\alpha: A \rightarrow B$, $\beta: B \rightarrow C$, su \textbf{composición} $\beta \circ \alpha: A \rightarrow C$ es también un homomorfismo de $R$-módulos. Nótese que para que la composición sea posible, el rango de $\alpha$ tiene que ser igual al dominio de $\beta$. En ocasiones usaremos la notación $\alpha\beta = \alpha \circ \beta$. Llamaremos \textbf{inversa} (por ambos lados) de $\alpha : A \rightarrow B$ al homomorfismo $\alpha^{-1} : B \rightarrow A$ tal que $\alpha^{-1} \circ \alpha = 1_A$ y $\alpha \circ \alpha^{-1} = 1_B$. Una \textbf{inversa izquierda} de $\alpha$ es una función $\gamma: A \rightarrow A$ tal que $\gamma \circ \alpha = 1_A$. No tiene por qué existir ni ser única.

\begin{definicion}
	Sea $\{A_i, \alpha_i\}$ una familia de $R$-módulos $A_i$ y homomorfismos entre ellos tal que $\alpha_i: A_i \rightarrow A_{i+1}$. Diremos que la secuencia
	\[ \cdots \xrightarrow{\alpha_{i-2}} A_{i-1} \xrightarrow{\alpha_{i-1}} A_i \xrightarrow{\alpha_{i}} A_{i+1} \xrightarrow{\alpha_{i+1}} \cdots \]
	es \textbf{exacta} cuando $\im \alpha_i = \ker \alpha_{i+1}$.
\end{definicion}

Para cada submódulo $T \subset B$, la inclusión de $T$ en $B$ es un monomorfismo $i: T \rightarrow B$. Las \textbf{clases laterales} de $T$ en $B$ son los conjuntos $b + T = \{b + t : t \in T\}$ donde $b \in B$. Dos clases laterales $b_1 + T$, $b_2 + T$ son iguales si $b_1 - b_2 \in T$. Como $T$ es un submódulo, el grupo abeliano $B/T$ se convierte en un $R$-módulo cuando $r(b+T) = rb + T$ para todo $r \in R$. A este $R$-módulo lo llamaremos el \textbf{módulo cociente} de $B$ sobre $T$. El homomorfismo $\pi: B \rightarrow B/T$ tal que $\pi(b) = b + T$ es un epimorfismo que llamaremos \textbf{proyección canónica} de $B$.

\begin{proposicion}\label{prop:first_iso}

Sea \( \beta: B \rightarrow B' \) un homomorfismo de módulos con \( T \subset \ker \beta \). Existe entonces un único homomorfismo de módulos \( \beta': B/T \rightarrow B' \) con \( \beta'\pi = \beta \); es decir, el siguiente diagrama con \( \beta(T) = 0 \)
\[
\begin{array}{ccc}
	B & \stackrel{\pi}{\longrightarrow} & B/T \\
	& \searrow \beta & \downarrow \beta' \\
	& & B'
\end{array}
\]
es conmutativo.
\end{proposicion}

\begin{proof}
Definamos \( \beta'(b + T) = \beta(b) \). Por estar $T$ contenida en el núcleo de $\beta$, la función está bien definida. En efecto, si $a,b \in B$ entonces $a+T = b+T \Rightarrow a-b \in T \subset \ker \beta \Rightarrow \beta(a-b) = 0 \Rightarrow \beta(a)=\beta(b)$. Como $\beta$ es un homomorfismo, $\beta'$ también lo es.
\end{proof}

En particular, si \( \beta: B \rightarrow B' \) es un epimorfismo con núcleo \( T \), \( \beta': B/T \rightarrow B' \) es un isomorfismo. Esta afirmación puede expresarse de la siguiente manera: cada \( \beta \) con \( \beta(T) = 0 \) \textit{factoriza de manera única} a través de la proyección \( \pi \). Esta propiedad caracteriza a \( \pi: B \rightarrow B/T \) hasta un isomorfismo de \( B/T \), de la siguiente manera:

\begin{proposicion}
	Si \( T \subset B \) y \( \eta: B \rightarrow D \) es tal que \( \eta(T) = 0 \) y cada \( \beta: B \rightarrow B' \) con \( \beta(T) = 0 \) factoriza de manera única a través de \( \eta \), entonces hay un isomorfismo \( \theta: B/T \cong D \) con \( \theta \pi = \eta \).
\end{proposicion}

\begin{proof}
	Factorizar \(\eta\) a través de \(\pi\) y \(\pi\) a través de \(\eta\), así que \(\eta = (\eta' \pi) \eta = 1_\eta\). Pero \(\eta\) factoriza \textit{únicamente} a través de \(\pi\), así que \(\eta' \pi = 1\). Simétricamente, \(\pi' \eta = 1\). Por lo tanto \(\pi' = (\eta')^{-1}\) y \(\eta'\) es el isomorfismo deseado \(\theta\).
\end{proof}

Para cualquier \(T \subseteq B\) la inyección \(i\) y la proyección \(\pi\) producen una secuencia exacta.
\[ 0 \rightarrow T \xrightarrow{i} B \xrightarrow{\pi} B/T \rightarrow 0. \]

\begin{definicion}
Sean $A,B$ y $C$ $R$-módulos y $\sigma: A \rightarrow B$, $\gamma: B \rightarrow C$ homomorfismos entre ellos. Diremos que la secuencia 
\[ (\sigma, \gamma): 0 \rightarrow A \xrightarrow{\sigma} B \xrightarrow{\gamma} C \rightarrow 0 \]
es una \textbf{secuencia exacta corta}. Es decir, una secuencia exacta de cinco \(R\)-módulos con los dos módulos exteriores siendo cero (y por lo tanto las dos funciones exteriores triviales).
\end{definicion}

La exactitud en \(A\) significa que \(\sigma\) es un monomorfismo, en \(B\) significa que \(\sigma A = \ker \gamma\) y en \(C\) que \(\gamma\) es un epimorfismo. Así la secuencia exacta corta puede escribirse como \(A \xrightarrow{\sigma} B \xrightarrow{\gamma} C\), con exactitud en \(B\). Ahora \(\sigma\) induce un isomorfismo \(\sigma': A \to A\) y \(\gamma\) un isomorfismo \(\gamma': B/\sigma A \to C\); juntos estos proveen un isomorfismo de secuencias exactas cortas, en la forma de un diagrama conmutativo
\[
\begin{tikzcd}
	0 \arrow[r] & A \arrow[r, "\sigma"] \arrow[d, "\sigma'"] & B \arrow[r, "\gamma"] \arrow[d, no head, Rightarrow] & C \arrow[r] \arrow[d, "(\gamma')^{-1}"] & 0 \\
	0 \arrow[r] & \sigma A \arrow[r, "i"]                    & B \arrow[r]                                          & B/\sigma A \arrow[r]                    & 0
\end{tikzcd}
\]

En resumen, una secuencia exacta corta es simplemente otro nombre para un submódulo y su cociente.

%Cada homomorfismo \(\alpha: A \rightarrow B\) determina dos módulos cociente
%\[\text{Coim } \alpha = A / \text{Ker } \alpha, \quad \text{Coker } \alpha = B / \text{Im } \alpha,\]
%llamados la \textbf{coimagen} y el \textbf{conúcleo} de \(\alpha\). Esta definición provee dos secuencias exactas cortas
%\[ \text{Ker } \alpha \hookrightarrow A \twoheadrightarrow \text{Coim } \alpha, \quad \text{Im } \alpha \hookrightarrow B \twoheadrightarrow \text{Coker } \alpha, \]
%un isomorfismo \(\text{Coim } \alpha \cong \text{Im } \alpha\) y una secuencia exacta más larga
%\[ 0 \rightarrow \text{Ker } \alpha \xrightarrow{i} A \xrightarrow{\alpha} B \rightarrow \text{Coker } \alpha \rightarrow 0. \]
%
%Por la \ref{prop:first_iso}, \(\beta \alpha = 0\) implica que \(\beta\) factoriza de manera única a través de \(\pi\) como \(\beta = \beta' \pi\). Dualmente, si algún \(\gamma': A' \rightarrow A\) tiene \(\alpha \gamma' = 0\), entonces \(\gamma'\) factoriza a través de \(i\) como \(\gamma' = i \gamma''\) para un único \(\gamma'': A' \rightarrow \ker \alpha\). Esta propiedad caracteriza \(i: \ker \alpha \rightarrow A\) como un isomorfismo de \(\text{Ker } \alpha\). Observa las afirmaciones duales: \(\alpha\) es un monomorfismo si y solo si \(\ker \alpha = 0\), y es un epimorfismo si y solo si \(\text{Coker } \alpha = 0\).

Si \(\alpha: A \rightarrow B\) y \(S \subseteq A\), el conjunto \(\alpha S\) de todos los elementos \(\alpha s\) para \(s \in S\) es un submódulo de \(B\) llamado la \textbf{imagen} de \(S\) bajo \(\alpha\). De manera similar, si \(T \subseteq B\), el conjunto \(\alpha^{-1}T\) de todos los \(s \in A\) con \(\alpha s \in T\) es un submódulo de \(A\), llamada la \textbf{imagen inversa} (completa) de \(T\). En particular, \(\ker \alpha = \alpha^{-1}0\), donde \(0\) denota el submódulo de \(B\) que consiste solo del elemento cero.

Para \(K \subseteq S \subseteq A\) el módulo \(S/K\) es llamado un \textbf{subcociente} de \(A\); es un módulo cociente del submódulo \(S\) de \(A\), y simultáneamente un submódulo del módulo cociente \(A/K\). Además, si \(K' \subseteq K \subseteq S' \subseteq S \subseteq A\), entonces \(K'/K\) es un submódulo de \(S'/K\) y la proyección compuesta \(S' \rightarrow (S'/K)/(K'/K)\) tiene núcleo \(K'\), por lo tanto el isomorfismo familiar \((S'/K)/(K'/K) \cong S'/K'\). Esto nos permite escribir cada subcociente \((S'/K)/(K'/K)\) de un subcociente \(S/K\) directamente como un subcociente de \(A\). Si \( \alpha: A \rightarrow A'\) tiene \(\alpha S \subseteq S'\) y \(\alpha K \subseteq K'\), entonces \(\alpha s + K'\) es una clase lateral de \(S'/K'\) determinada de manera única por la clase lateral \(s+K\) de \(S/K\). Por lo tanto \(\alpha_{\ast}(s+K) = \alpha s + K'\) define un homomorfismo
\[\alpha_{\ast}: S/K \rightarrow S'/K'\]
\[(\alpha S \subseteq S', \alpha K \subseteq K')\]
llamado el homomorfismo \textbf{inducido} por \(\alpha\) en los subcocientes dados.

Si \(S\) y \(T\) son submódulos de \(A\), su \textbf{intersección} \(S \cap T\) (como conjuntos) es también un submódulo, así como su \textbf{unión} \(S + T\), consistiendo de todas las sumas \(s + t\) para \(s \in S\), \(t \in T\). El \textbf{teorema del isomorfismo de Noether} afirma que \(1_{A}\) induce un isomorfismo

\[ 1_{\ast}: S/(S \cap T) \cong (S + T)/T. \]

\section{Categorías}

\begin{definicion}
	Una \textbf{categoría} $\mathcal{C}$ es una tripleta $(\mathcal{O}, \hom, \circ)$ formada por
	\begin{enumerate}
		\item Una clase $\mathcal{O}$, cuyos elementos denominamos \textbf{objetos} de $\mathcal{C}$ y notamos por $Obj(\mathcal{C})$.
		\item Por cada par de objetos $(A,B)$ de $\mathcal{C}$, un conjunto $\hom(A,B)$ cuyos elementos son llamados \textbf{morfismos} de $A$ a $B$. Si $f \in \hom(A,B)$, normalmente escribiremos $f: A \rightarrow B$ o $A \xrightarrow{f} B$.
		\item Una \textbf{ley de composición} que asocia a cada morfismo $f: A \rightarrow B$ y a cada morfismo $g: B \rightarrow C$ un morfismo $g \circ f : A \rightarrow C$ satisfaciendo
		\begin{itemize}
			\item \textbf{Asociatividad}. Si $f: A \rightarrow B$, $g: B \rightarrow C$ y $h : C \rightarrow D$ son morfismos de $\mathcal{C}$, entonces $h \circ (g \circ f) = (h \circ g) \circ f$.
			\item \textbf{Identidad}. A cada objeto $B$ le podemos asociar un morfismo identidad $1_B : B \rightarrow B$ tal que si $f: A \rightarrow B$ y $g: B \rightarrow C$ entonces $g \circ 1_B = g$ y $1_B \circ f = f$.
		\end{itemize}
		Llamaremos a este morfismo la \textbf{composición} de $f$ y $g$.
	\end{enumerate}
\end{definicion}

\begin{ejemplo}
	Como veremos a continuación, la definición anterior nos va a permitir trabajar con un gran número de espacios matemáticos que ya conocemos en el contexto de la teoría de categorías. Algunos de ellos son:
	\begin{itemize}
		\item \textbf{La categoría de espacios topológicos}, donde los objetos son todos los espacios topológicos y los morfismos todas las aplicaciones continuas entre espacios topológicos $f: X \rightarrow Y$.
		\item \textbf{La categoría de grupos}, donde los objetos son todos los grupos y los morfismos todos los homomorfismos de grupos.
		\item \textbf{La categoría de conjuntos}, cuyos objetos son todos los conjuntos y sus morfismos todas las aplicaciones entre conjuntos.
	\end{itemize}
\end{ejemplo}

\begin{definicion}
	Sea $f \in \hom(A,B)$ un morfismo en la categoría $\mathcal{C}$. Diremos que $f$ es una \textbf{equivalencia} en $\mathcal{C}$ si existe en $\mathcal{C}$ otro morfismo $g \in \hom(B,A)$ tal que $g \circ f = 1_A$ y $f \circ g = 1_B$.
\end{definicion}

Nótese que si $f \in \hom(A,B)$ es una equivalencia en $\mathcal{C}$, $g \in \hom(B,A)$ debe ser única. En efecto, si suponemos que existe $g' \in \hom(B,A)$ tal que $g' \circ f = 1_A$, entonces $g = g'\circ f \circ g = g' \circ 1_B = g'$.

\section{Funtores}

\begin{definicion}
	Sean $\mathcal{C}, \mathcal{D}$ dos categorías. Un \textbf{funtor covariante} de $\mathcal{C}$ a $\mathcal{D}$ es una pareja de funciones \textit{denotadas por la misma letra $T$} tal que:
	\begin{enumerate}
		\item Una \textbf{función objeto} que asigna a cada objeto $C \in \mathcal{C}$ un objeto $T(C) \in \mathcal{D}$.
		\item Una \textbf{función de morfismos} qu asigna a cada morfismo $\gamma: C \rightarrow C'$ de $\mathcal{C}$ un morfismo $T(\gamma): T(C) \rightarrow T(C')$ de $\mathcal{D}$. Este par de funciones satisfacen las siguientes condiciones:
		\begin{equation}
			T(1_C) = 1_{T(C)}, \qquad C \in \mathcal{C},
		\end{equation}
		\begin{equation}
			T(\beta \gamma) = T(\beta)T(\gamma), \qquad \beta \gamma \text{ definido en } \mathcal{C}.
		\end{equation}
	\end{enumerate}
\end{definicion}

Es decir, un funtor covariante $T: \mathcal{C} \rightarrow \mathcal{D}$ es una aplicación que preserva el rango, dominio, identidades y composiciones de $\mathcal{C}$ en $\mathcal{D}$.

\endinput
%--------------------------------------------------------------------
% FIN DEL CAPÍTULO. 
%--------------------------------------------------------------------
