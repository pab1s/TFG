% !TeX root = ../tfg.tex
% !TeX encoding = utf8

\chapter{Aprendizaje automático y visión artificial}

La \textbf{inteligencia artificial} (IA) se define en  \cite{russell2016artificial} como el campo de estudio de la informática que se centra en la creación de agentes inteligentes, es decir, sistemas que perciben su entorno y toman acciones que maximizan sus posibilidades de éxito en algún objetivo o tarea.

La IA abarca una amplia variedad de problemas y aplicaciones, entre los que destacan:

\begin{itemize}
	\item \textbf{Resolución de Problemas y Búsqueda}: La IA desarrolla algoritmos para encontrar soluciones óptimas o satisfactorias en espacios de búsqueda complejos. Esto incluye problemas clásicos como el ajedrez, el rompecabezas del cubo de Rubik o la planificación de rutas.
	
	\item \textbf{Representación del Conocimiento y Razonamiento}: Este área se centra en cómo representar información sobre el mundo de manera que un ordenador pueda utilizarla para resolver problemas complejos y tomar decisiones. Ejemplos incluyen sistemas expertos y razonamiento basado en casos.
	
	\item \textbf{Aprendizaje Automático}: Como veremos posteriormente, el aprendizaje automático se enfoca en el desarrollo de algoritmos que permitan a los ordenadores aprender a partir de los datos y mejorar con la experiencia.
	
	\item \textbf{Procesamiento del Lenguaje Natural (NLP)}: El NLP busca permitir que los ordenadores comprendan, interpreten y respondan al lenguaje humano de manera útil. Algunas aplicaciones de este campo incluyen la traducción automática, el análisis de sentimientos y la implementación de agentes conversacionales.
	
	\item \textbf{Visión Artificial}: Este campo se enfoca en que los ordenadores sean capaces de interpretar y comprender el contenido de imágenes y vídeos. Aplicaciones incluyen el reconocimiento facial, la conducción autónoma y la detección de objetos.
	
	\item \textbf{Robótica}: La IA aplicada a la robótica busca diseñar agentes que interactúen con el mundo físico. Esto incluye tareas como la navegación, la manipulación de objetos y la interacción humano-robot.
\end{itemize}

Es importante destacar que los problemas que aborda la IA pueden involucrar múltiples áreas. Por ejemplo, un sistema de conducción autónoma requiere la integración de visión artificial para interpretar el entorno, aprendizaje automático para mejorar su desempeño a partir de la experiencia y toma de decisiones, y algoritmos de planificación para determinar la mejor ruta a seguir. Esta interrelación de diferentes áreas y técnicas permite a los sistemas de IA abordar problemas complejos y dinámicos de manera más eficaz.

\section{Aprendizaje automático}

El \textbf{aprendizaje automático}  o \textbf{\textit{machine learning}} es la subdisciplina de la IA que se centra en el desarrollo de algoritmos y técnicas que permiten a los ordenadores aprender y hacer predicciones o tomar decisiones basadas en datos \cite{hastie2009elements}. Esta capacidad de aprendizaje se logra a través de la construcción de modelos que pueden mejorar su rendimiento en tareas específicas mediante la experiencia y los datos disponibles. Este enfoque supuso un cambio de paradigma respecto al diseño de algoritmos tradicional, donde su desempeño dependía explícitamente del conocimiento sobre el problema de los ingenieros que lo implementaban.

Existen varios paradigmas del aprendizaje automático en función del nivel de supervisión humana necesaria durante el proceso de aprendizaje. Los tres tipos principales son:

\begin{itemize}
	\item \textbf{Aprendizaje Supervisado}: En este tipo, el algoritmo aprende a partir de un conjunto de datos etiquetados. Cada ejemplo en el conjunto de datos incluye una entrada y una salida esperada. El objetivo es que el modelo aprenda a identificar las salidas correctas a partir de las entradas para poder hacer predicciones sobre datos nuevos. Aplicaciones comunes incluyen la clasificación y la regresión.
	
	\item \textbf{Aprendizaje No Supervisado}: A diferencia del aprendizaje supervisado, en el aprendizaje no supervisado los datos no están etiquetados. El objetivo es descubrir estructuras o patrones ocultos en los datos. Las técnicas de agrupamiento o \textit{clustering}, y la reducción de dimensionalidad son ejemplos comunes de aprendizaje no supervisado.
	
	\item \textbf{Aprendizaje por Refuerzo}: En este enfoque, un agente aprende a tomar decisiones mediante la interacción con un entorno dinámico. El agente recibe recompensas o penalizaciones en función de las acciones que realiza, y su objetivo es maximizar la recompensa acumulada a lo largo del tiempo. Este tipo de aprendizaje es especialmente útil en problemas de toma de decisiones secuenciales, como en juegos y robótica.
\end{itemize}

El aprendizaje automático se utiliza para abordar una amplia variedad de problemas en diferentes dominios. Algunos de los problemas más destacados incluyen:

\begin{itemize}
	\item \textbf{Clasificación}: El objetivo de la clasificación es asignar una etiqueta a una entrada entre un conjunto de etiquetas posibles. Ejemplos de problemas de clasificación incluyen el reconocimiento de imágenes, el filtrado de spam y el diagnóstico médico.
	
	\item \textbf{Regresión}: La regresión se centra en predecir un valor continuo a partir de una entrada. Un ejemplo clásico es la predicción de precios de viviendas basándose en características como el tamaño y la ubicación.
	
	\item \textbf{Agrupamiento (\textit{Clustering})}: Este problema implica agrupar un conjunto de datos en subconjuntos, o clusters, de tal manera que los datos en el mismo \textit{cluster} sean más similares entre sí que con los datos de otros \textit{clusters}. Aplicaciones incluyen segmentación de mercado y análisis de imágenes.
	
	\item \textbf{Reducción de Dimensionalidad}: La reducción de dimensionalidad se utiliza para simplificar modelos complejos reduciendo el número de variables en el análisis de datos. Técnicas como el Análisis de Componentes Principales (PCA) o \textit{Locally Linear Embedding} (LLE) son comunes en este campo.
	
	\item \textbf{Detección de Anomalías}: También conocido como detección de \textit{outliers}, este problema implica identificar datos que no se ajustan al comportamiento normal esperado. Es crucial en áreas como la detección de fraudes y el monitoreo de sistemas.
\end{itemize}

El avance en el aprendizaje automático ha llevado a desarrollos significativos en múltiples campos, transformando la manera en que se analizan los datos y se toman decisiones en el mundo real.

\section{Visión artificial}

Como hemos comentado anteriormente de manera breve, la \textbf{visión artificial} es la rama de la IA cuyo objetivo es comprender el mundo que percibimos a partir de una o más imágenes y reconstruir propiedades suyas como su forma, iluminación o distribución de colores \cite{szeliski2022computer}. Su objetivo final es desarrollar sistemas que puedan interpretar y comprender el entorno visual de manera efectiva.

La visión artificial es un campo muy diverso que integra métodos de varias áreas del conocimiento para lograr su objetivo principal. Como resultado, abarca una amplia gama de aplicaciones y tareas. Algunas de las más destacadas incluyen:

\begin{itemize} 
	\item \textbf{Clasificación de objetos}: identificar y categorizar objetos en una imagen, lo que es fundamental en aplicaciones como la detección de objetos en imágenes de seguridad o la clasificación de productos en una tienda en línea. 
	\item \textbf{Generación de imágenes}: crear imágenes a partir de modelos o datos, lo que tiene aplicaciones en la creación de gráficos 3D, la síntesis de imágenes para la publicidad o la creación de contenido en redes sociales. 
	\item \textbf{Reconstrucción de entornos 3D}: reconstruir entornos tridimensionales a partir de imágenes 2D, lo que es fundamental en aplicaciones como la creación de modelos 3D de edificios o la planificación de rutas en un entorno desconocido. 
	\item \textbf{Detección de objetos}: detectar la presencia de objetos en una imagen, lo que es fundamental en aplicaciones como la seguridad en espacios públicos o la detección de anomalías en la producción industrial. 
	\item \textbf{Segmentación de imágenes}: dividir una imagen en regiones significativas, lo que es fundamental en aplicaciones como la medicina, la agricultura o la minería. 
\end{itemize}

La visión artificial es un campo en constante evolución que se ha desarrollado significativamente en las últimas décadas, al punto de haber superado a los seres humanos en ciertas aplicaciones específicas, como el reconocimiento de imágenes \cite{5206848} y la detección de objetos \cite{wang2022yolov7trainablebagoffreebiessets}. Sin embargo, aún quedan numerosos desafíos por abordar, especialmente en contextos más complejos y variados, como la interpretación de escenas en tiempo real y la comprensión de imágenes en condiciones adversas.

\endinput
%--------------------------------------------------------------------
% FIN DEL CAPÍTULO. 
%--------------------------------------------------------------------
