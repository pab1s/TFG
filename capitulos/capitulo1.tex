% !TeX root = ../tfg.tex
% !TeX encoding = utf8

\chapter{Símplices y complejos simpliciales}

\section{Símplices}

INTRO

\begin{definicion}
	Sea $\{a_0, \dots, a_n\}$ un conjunto de puntos en $\mathbb{R}^N$. 
	Diremos que dicho conjunto es \textbf{afínmente independiente} si 
	para cualesquiera $t_i \in \mathbb{R}$, las ecuaciones
	\[ \sum_{i=0}^{n}t_i=0 \quad \text{y} \quad \sum_{i=0}^{n}t_ia_i=0 \]
	implican que $t_0 = t_1 = \dots = t_n$.
\end{definicion}

TEXTO

\begin{definicion}
	Sea $\{a_0, \dots, a_n\}$ un conjunto de puntos afínmente independiente. 
	Definimos el \textbf{n-plano} $P$ generado por $\{a_0, \dots, a_n\}$ como
	el conjunto de puntos $x \in \mathbb{R}^N$ tales que
	\[ x = a_0 + \sum_{i=1}^{n}t_i(a_i - a_0) \]
	para algunos $t_1, \dots, t_n \in \mathbb{R}$. Diremos entonces que $P$ es el 
	plano que pasa por $a_0$ paralelo a los vectores $a_i - a_0$, $i \in \{1, \dots, n\}$.
\end{definicion}

\begin{definicion}
	Def de aplicación afín.
\end{definicion}

\begin{definicion}
	Sea $\{a_0, \dots, a_n\}$ un conjunto de puntos afínmente independiente en 
	$\mathbb{R}^N$. Definimos el \textbf{símplice} $\sigma = [a_0, \dots, a_n]$ 
	generado por $a_0, 	\dots, a_n$ como el conjunto de todos los $x \in \mathbb{R}^N$ 
	tales que
	\[ x=\sum_{i=0}^{n}t_ia_i \quad \text{y} \quad \sum_{i=0}^{n}t_i=1 \]
	con $t_i \geq 0$, $i \in \{1, \dots, n\}$.
\end{definicion}
Los coeficientes $t_i$ están determinados de manera única por el punto $x$. Al conjunto 
$\{t_0, \dots, t_n \}$ lo llamamos las \textbf{coordenadas baricéntricas} de $\sigma$
con respecto a $a_0, \dots, a_n$.

Los puntos $a_0, \dots, a_n$ que generan $\sigma$ los llamaremos \textbf{vértices} de $\sigma$
y al número $n$ lo llamaremos la \textbf{dimensión} de $\sigma$.

\begin{definicion}
	Sea $\sigma=[a_0, \dots, a_n]$ un símplice. Una \textbf{cara} de $\sigma$ será cualquier
	símplice generado por un subconjunto de $\{a_0, \dots, a_n\}$.
\end{definicion}
En particular, la cara de $\sigma$ generada por $a_0, \dots, a_{i-1}, a_{i+1}, \dots, a_n$ la 
llamamos la \textbf{cara opuesta} de $a_i$, $i \in \{0, \dots, n\}$. Las caras de $\sigma$ 
diferentes de $\sigma$ diremos que son \textbf{caras propias} de $\sigma$ y la unión de todas ellas la 
llamaremos el \textbf{borde} de $\sigma$. Finalmente, definimos el \textbf{interior} de $\sigma$
como el conjunto de puntos de $\sigma$ que no pertenecen a su borde.

\section{Complejos simpliciales}

\endinput
%--------------------------------------------------------------------
% FIN DEL CAPÍTULO. 
%--------------------------------------------------------------------
