% !TeX root = ../tfg.tex
% !TeX encoding = utf8
%
%*******************************************************
% Summary
%*******************************************************

\selectlanguage{english}
\chapter{Abstract}

This Bachelor's Thesis explores the integration of Topological Data Analysis (TDA)
with convolutional neural networks (CNNs) to clarify and enhance our
understanding of how CNNs manipulate data. Given the common perception of CNNs as
\enquote{black boxes} due to the opacity on their internal decision-making
processes, this work adopts an alternative approach through the application of persistent
homology techniques, a key tool in TDA. This allows for a detailed analysis of the
data structure during CNN processing, facilitating greater transparency and
understanding of these networks' internal workings from a topological point of view.

The study focuses on the applicability of persistent homology to provide an additional
layer of explainability and optimization to deep learning models. The underlying
theory is explored, and implementations are carried out in practical cases using
advanced network architectures such as ResNet, EfficientNet and DenseNet. Through
controlled experiments, it is demonstrated that topological regulation not only
improves the performance of CNNs in image classification and transfer learning
tasks, but also offers new insights into the data structure throughout the learning
process.

Specific results indicate that different configurations in CNN models significantly
influence the topological characteristics of the data. The dynamics of the
architectures studied show an initial tendency to simplify the data, possibly to
remove noise and irrelevant details. However, as learning progresses, the
topological complexity of the data increases, suggesting a deliberate strategy to
develop richer and more detailed representations, thereby facilitating better
differentiation between classes.

The implementation of a topological regularizer in selected models such as
EfficientNet-B0 and DenseNet-121 has proven to be particularly promising. This approach
adjusts the topological complexity of the data in a way that reflects a significant
improvement in the accuracy and efficacy of classification. Moreover, it is observed
that knowledge transfer markedly improves when the data topology is appropriately
manipulated, suggesting that topological modifications could be an effective
strategy for optimizing CNNs.

In conclusion, this document not only enhances the understanding of CNNs'
internal mechanisms through TDA but also marks a step towards more transparent
and reliable artificial intelligence (AI) models. The findings shows the utility
of TDA in the field of deep learning, proposing a new alternatives for
explainability and efficiency that may be interesting to consider in the future
evolution of computer vision and data science.

\bigskip
\textbf{Keywords}: convolutional neural networks, topological data analysis,
persistent homology, AI explainability, optimization of deep learning models, transfer
learning.

% Al finalizar el resumen en inglés, volvemos a seleccionar el idioma español para el documento
\selectlanguage{spanish}
\chapter{Resumen}

Este Trabajo de Fin de Grado (TFG) explora la integración del Análisis de Datos
Topológico (TDA) con redes neuronales convolucionales (CNNs) para clarificar y
mejorar nuestra comprensión de cómo las CNNs manipulan los datos. Dada la percepción
común de las CNNs como \enquote{cajas negras} debido a la opacidad en sus
procesos de toma de decisiones internos, este trabajo adopta un enfoque alternativo
mediante la aplicación de técnicas de homología persistente, una herramienta
clave en TDA. Esto permite un análisis detallado de la estructura de datos durante
el procesamiento en las CNNs, facilitando una mayor transparencia y
entendimiento de los mecanismos internos de estas redes desde un punto de vista
topológico.

El estudio se centra en la aplicabilidad de la homología persistente para
proporcionar una mayor explicabilidad y optimización de los modelos de
aprendizaje profundo. Se explora la teoría matemática subyacente, para
posteriormente realizar implementaciones en casos prácticos utilizando
arquitecturas de CNNs avanzadas como ResNet, EfficientNet y DenseNet. A través
de experimentos controlados, se demuestra que la regulación topológica no solo mejora
el rendimiento de las CNNs en tareas de clasificación de imágenes y de transferencia
de conocimiento, sino que también ofrece nuevas perspectivas sobre la estructura
de los datos a lo largo del proceso de aprendizaje.

Los resultados indican que las diferentes configuraciones en los modelos de CNNs
influyen significativamente en las características topológicas de los datos. La dinámica
de las arquitecturas estudiadas muestra una tendencia inicial a simplificar los datos,
posiblemente para eliminar ruido y detalles irrelevantes. Sin embargo, a medida
que avanza el aprendizaje, la complejidad topológica de los datos aumenta,
sugiriendo una estrategia deliberada para obtener representaciones más ricas y
detalladas, facilitando así una mejor diferenciación entre clases.

La implementación de un regularizador topológico en modelos como EfficientNet-B0
y DenseNet-121 ha demostrado ser particularmente prometedora. Este enfoque
ajusta la complejidad topológica de los datos de manera que refleja una mejora
significativa en la precisión y eficacia de la clasificación. Además, se observa
que la transferencia de conocimiento mejora considerablemente cuando la
topología de los datos se manipula adecuadamente, sugiriendo que las modificaciones
topológicas podrían ser una estrategia efectiva para optimizar las CNNs.

En conclusión, este documento no solo mejora la comprensión de los mecanismos
internos de las CNNs a través de TDA, sino que también marca un paso hacia modelos
de inteligencia artificial (IA) más transparentes y confiables. Los hallazgos
destacan la utilidad del TDA en el campo del aprendizaje profundo, proponiendo nuevas
alternativas para la explicabilidad y eficiencia que pueden ser interesantes de considerar
en la evolución futura de la visión artificial y la ciencia de datos.

\bigskip
\textbf{Palabras clave}: redes neuronales convolucionales, análisis de datos
topológico, homología persistente, explicabilidad en IA, optimización de modelos
de aprendizaje profundo, transferencia de conocimiento.

\endinput