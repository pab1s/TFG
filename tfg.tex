% !TEX program = pdflatex
% !TEX encoding = UTF-8 Unicode

% Plantilla, basada en la clase `scrbook` del paquete KOMA-script,  para la elaboración de un TFG siguiendo las directrices del la comisión del Grado en Matemáticas de la Universidad de Granada.

% Francisco Torralbo Torralbo

\documentclass[print, color]{ugrTFG}
\usetikzlibrary{arrows,calc}

% Importamos paquetes
%\setlength{\parskip}{5pt}%
%\setlength{\parskip}{\baselineskip}
%\setlength{\parindent}{0pt}%
\newcommand{\Cat}[1]{\mathbf{#1}}
\DeclareMathOperator{\im}{Im}
\DeclareMathOperator{\id}{id}
\DeclareMathOperator{\ann}{Ann}
\DeclareMathOperator{\tor}{Tor}
\DeclareMathOperator{\rg}{rg}
\DeclareMathOperator{\interior}{Int}
\DeclareMathOperator{\sop}{sop}
\DeclareMathOperator{\sd}{sd}
\DeclareMathOperator{\bd}{Bd}
\DeclareMathOperator{\obj}{\textit{Obj}}
\DeclareMathOperator{\diam}{diam}
\DeclareMathOperator{\meas}{m}
\DeclareMathOperator{\st}{st}
\DeclareMathOperator{\sgn}{sgn}
% VERSIÓN ELECTRÓNICA PARA TABLETA
% Cambiando la opción "print" por "tablet" generaremos un pdf adaptado para leerlo en tabletas de 9 pulgadas.

% -------------------------------------------------------------------
% INFORMACIÓN DEL TFG Y EL AUTOR
% -------------------------------------------------------------------

\newcommand{\miTitulo}{Aplicación de la topología algebraica en redes neuronales\xspace}
\newcommand{\miNombre}{Pablo Olivares Martínez\xspace}
\newcommand{\miGrado}{Grado en Ingeniería Informática y Matemáticas}
\newcommand{\miFacultad}{Facultad de Ciencias \\ E.T.S. de Ingenierías Informática y de Telecomunicación}
\newcommand{\miUniversidad}{Universidad de Granada}

% Añadir tantos tutores como sea necesario separando cada uno de ellos mediante el comando `\medskip` y una línea en blanco
\newcommand{\miTutor}{
  Miguel Ortega Titos \\ \emph{Departamento de Geometría y Topología} 

  % Añadir tantos tutores como sea necesario. 

  \medskip
  Julián Luengo Martín \\ \emph{Departamento de Ciencias de la Computación e Inteligencia Artificial}
}
\newcommand{\miCurso}{2023-2024\xspace}

\hypersetup{
	pdftitle={\miTitulo},
	pdfauthor={\textcopyright\ \miNombre, \miFacultad, \miUniversidad}
}

\begin{document}

\maketitle

% -------------------------------------------------------------------
% FRONTMATTER
% -------------------------------------------------------------------
\frontmatter % Desactiva la numeración de capítulos y usa numeración romana para las páginas

\input{preliminares/declaracion-originalidad}   
% !TeX root = ../tfg.tex
% !TeX encoding = utf8

%*******************************************************
% Dedication
%*******************************************************
\thispagestyle{empty}
\phantomsection 
\pdfbookmark[1]{Dedicatoria}{Dedicatoria}

\hfill
\vfill

\begin{flushright}
\itshape
A mi familia y amigos, \\
y en especial a Inés.
\end{flushright}

\vfill

\cleardoublepage
\endinput
                % Opcional
\input{preliminares/tablacontenidos}            
% !TeX root = ../tfg.tex
% !TeX encoding = utf8

%*******************************************************
% Agradecimientos
%*******************************************************

\chapter{Agradecimientos}

Quisiera comenzar agradeciendo a mis tutores, Miguel y Julián, por su gran ayuda
y dedicación en todo momento de este trabajo.
\bigskip

A mis padres y hermanos, así como a mis abuelos, por apoyarme de manera
incondicional desde la distancia y no dejar que me rindiera.
\bigskip

También quisiera agradecer a todos aquellos compañeros y amigos que, de una forma
u otra, me han ayudado con este trabajo tanto con sus ideas y opiniones como con
su aliento.
\bigskip

Por último, quisiera agradecérselo a Inés. Por tu paciencia y tu apoyo continuo,
por estar a mi lado cuando más lo necesitaba, gracias.

\cleardoublepage
\endinput            % Opcional

% !TeX root = ../tfg.tex
% !TeX encoding = utf8
%
%*******************************************************
% Summary
%*******************************************************

\selectlanguage{english}
\chapter{Abstract}

This Bachelor's Thesis explores the integration of Topological Data Analysis (TDA)
with convolutional neural networks (CNNs) to clarify and enhance our
understanding of how CNNs manipulate data. Given the common perception of CNNs as
\enquote{black boxes} due to the opacity on their internal decision-making
processes, this work adopts an alternative approach through the application of persistent
homology techniques, a key tool in TDA. This allows for a detailed analysis of the
data structure during CNN processing, facilitating greater transparency and
understanding of these networks' internal workings from a topological point of view.

The study focuses on the applicability of persistent homology to provide an additional
layer of explainability and optimization to deep learning models. The underlying
theory is explored, and implementations are carried out in practical cases using
advanced network architectures such as ResNet, EfficientNet and DenseNet. Through
controlled experiments, it is demonstrated that topological regulation not only
improves the performance of CNNs in image classification and transfer learning
tasks, but also offers new insights into the data structure throughout the learning
process.

Specific results indicate that different configurations in CNN models significantly
influence the topological characteristics of the data. The dynamics of the
architectures studied show an initial tendency to simplify the data, possibly to
remove noise and irrelevant details. However, as learning progresses, the
topological complexity of the data increases, suggesting a deliberate strategy to
develop richer and more detailed representations, thereby facilitating better
differentiation between classes.

The implementation of a topological regularizer in selected models such as
EfficientNet-B0 and DenseNet-121 has proven to be particularly promising. This approach
adjusts the topological complexity of the data in a way that reflects a significant
improvement in the accuracy and efficacy of classification. Moreover, it is observed
that knowledge transfer markedly improves when the data topology is appropriately
manipulated, suggesting that topological modifications could be an effective
strategy for optimizing CNNs.

In conclusion, this document not only enhances the understanding of CNNs'
internal mechanisms through TDA but also marks a step towards more transparent
and reliable artificial intelligence (AI) models. The findings shows the utility
of TDA in the field of deep learning, proposing a new alternatives for
explainability and efficiency that may be interesting to consider in the future
evolution of computer vision and data science.

\bigskip
\textbf{Keywords}: convolutional neural networks, topological data analysis,
persistent homology, AI explainability, optimization of deep learning models, transfer
learning.

% Al finalizar el resumen en inglés, volvemos a seleccionar el idioma español para el documento
\selectlanguage{spanish}
\chapter{Resumen}

Este Trabajo de Fin de Grado (TFG) explora la integración del Análisis de Datos
Topológico (TDA) con redes neuronales convolucionales (CNNs) para clarificar y
mejorar nuestra comprensión de cómo las CNNs manipulan los datos. Dada la percepción
común de las CNNs como \enquote{cajas negras} debido a la opacidad en sus
procesos de toma de decisiones internos, este trabajo adopta un enfoque alternativo
mediante la aplicación de técnicas de homología persistente, una herramienta
clave en TDA. Esto permite un análisis detallado de la estructura de datos durante
el procesamiento en las CNNs, facilitando una mayor transparencia y
entendimiento de los mecanismos internos de estas redes desde un punto de vista
topológico.

El estudio se centra en la aplicabilidad de la homología persistente para
proporcionar una mayor explicabilidad y optimización de los modelos de
aprendizaje profundo. Se explora la teoría matemática subyacente, para
posteriormente realizar implementaciones en casos prácticos utilizando
arquitecturas de CNNs avanzadas como ResNet, EfficientNet y DenseNet. A través
de experimentos controlados, se demuestra que la regulación topológica no solo mejora
el rendimiento de las CNNs en tareas de clasificación de imágenes y de transferencia
de conocimiento, sino que también ofrece nuevas perspectivas sobre la estructura
de los datos a lo largo del proceso de aprendizaje.

Los resultados indican que las diferentes configuraciones en los modelos de CNNs
influyen significativamente en las características topológicas de los datos. La dinámica
de las arquitecturas estudiadas muestra una tendencia inicial a simplificar los datos,
posiblemente para eliminar ruido y detalles irrelevantes. Sin embargo, a medida
que avanza el aprendizaje, la complejidad topológica de los datos aumenta,
sugiriendo una estrategia deliberada para obtener representaciones más ricas y
detalladas, facilitando así una mejor diferenciación entre clases.

La implementación de un regularizador topológico en modelos como EfficientNet-B0
y DenseNet-121 ha demostrado ser particularmente prometedora. Este enfoque
ajusta la complejidad topológica de los datos de manera que refleja una mejora
significativa en la precisión y eficacia de la clasificación. Además, se observa
que la transferencia de conocimiento mejora considerablemente cuando la
topología de los datos se manipula adecuadamente, sugiriendo que las modificaciones
topológicas podrían ser una estrategia efectiva para optimizar las CNNs.

En conclusión, este documento no solo mejora la comprensión de los mecanismos
internos de las CNNs a través de TDA, sino que también marca un paso hacia modelos
de inteligencia artificial (IA) más transparentes y confiables. Los hallazgos
destacan la utilidad del TDA en el campo del aprendizaje profundo, proponiendo nuevas
alternativas para la explicabilidad y eficiencia que pueden ser interesantes de considerar
en la evolución futura de la visión artificial y la ciencia de datos.

\bigskip
\textbf{Palabras clave}: redes neuronales convolucionales, análisis de datos
topológico, homología persistente, explicabilidad en IA, optimización de modelos
de aprendizaje profundo, transferencia de conocimiento.

\endinput                    
% !TeX root = ../tfg.tex
% !TeX encoding = utf8
%
%*******************************************************
% Introducción
%*******************************************************

% \manualmark
% \markboth{\textsc{Introducción}}{\textsc{Introducción}}

\chapter{Introducción}

\section{Contexto}

El desarrollo humano ha sido constantemente impulsado por avances tecnológicos que
han redefinido nuestra comprensión e interacción con mundo. Desde la invención
de la imprenta hasta la revolución digital actual, cada era ha estado marcada por
innovaciones clave. En particular, la invención de los ordenadores y el avance de
las tecnologías de la información han convergido en la capacidad de generar,
almacenar y analizar grandes volúmenes de datos. Este volumen de información ha desencadenado
lo que ahora conocemos como la Era de la Información, que se caracteriza por el
desarrollo de algoritmos avanzados que extraen valor de estos datos de manera
automática y eficiente.

Dentro de esta revolución tecnológica, el aprendizaje profundo y, en particular,
las redes neuronales convolucionales (CNNs) \cite{bengio2017deep} han emergido como
útiles herramientas, especialmente en el ámbito de la visión artificial. Estos modelos
son capaces de identificar patrones complejos en datos visuales, superando a
menudo el rendimiento humano en tareas de reconocimiento de imágenes. Sin embargo,
las decisiones tomadas por las CNN a menudo son opacas y difíciles de
interpretar, lo que ha llevado a que se les describa como \enquote{cajas negras}.

Ante este problema, han surgido diversas técnicas para clarificar cómo estas redes
toman decisiones. Una de las más novedosas y prometedoras es el análisis de
datos topológico (TDA) \cite{dey2022computational}, que emplea herramientas de la
topología algebraica para ofrecer soluciones. El TDA busca comprender la
\enquote{forma} de los datos, ofreciendo así conocimiento sobre cómo las CNNs
estructuran y manipulan la información a nivel global, no solo basándose en instancias
individuales.

En este trabajo, aplicaremos técnicas de TDA para desentrañar cómo las CNNs
procesan y transforman conjuntos de datos, con el objetivo de comprender los mecanismos
subyacentes de estos modelos de aprendizaje profundo y así proporcionar nuevas claves
que nos permitan aprovecharlas mejor. Al enfocarnos en las estructuras globales
de los datos en lugar de cambios individuales, esperamos ofrecer una comprensión
más clara y detallada de cómo trabajan estas redes, contribuyendo así a una
mayor comprensión y confianza en los algoritmos de aprendizaje profundo.

El TDA es un campo relativamente reciente. A comienzo de los años 90, bajo la
premisa de que los datos tienen una \enquote{forma}, matemáticos como Patrizio
Frosini o Vanessa Robins estudiaron las propiedades que podían extraerse en el
estudio de la distancia entre variedades \cite{Frosini_1990} y estructuras
relacionadas por el homomorfismo de inclusión \cite{robins1999towards}. Estas ideas
cautivaron a Edelsbrunner, quien les dio forma en lo que hoy se conoce como
homología persistente, junto con un algoritmo para calcularla y visualizarla de manera
efectiva \cite{edelsbrunner2002topological}. La homología persistente es
actualmente la piedra angular del TDA, permitiendo el análisis de características
topológicas que persisten a través de diferentes escalas. La homología
persistente resuelve desafíos en la selección de parámetros al codificar
información de todos los valores posibles. En 2008, Gunnar Carlsson
\cite{carlsson2009topology} dio un paso adelante al reformular la homología
persistente dentro del ámbito del álgebra conmutativa, proporcionando lo que hoy
conocemos como código de barras, facilitando su comprensión y ampliando su
aplicabilidad en ciencia de datos y otras áreas tecnológicas. Desde entonces,
varios autores como Liwen Zhang, Gregory Naitzat y Lek-Heng Lim \cite{naitzat2020topology}
han aplicado estas técnicas con el fin de comprender mejor el funcionamiento de las
CNN. Su trabajo mostraba que, efectivamente, las CNNs simplificaban la \enquote{forma}
de los datos. Inspirado por los resultados, German Magai \cite{magai2023deep} profundizó
en estos hallazgos para confirmar dichas afirmaciones.

\section{Motivación}

La comprensión del funcionamiento de los modelos de aprendizaje profundo ha
emergido como una necesidad en el campo de la ciencia de datos. Las CNNs, aunque
pesar de ser muy efectivas en muchas tareas de visión artificial, a menudo actúan
como \enquote{cajas negras}. Esto puede ser un problema, ya que nos impide
comprender por qué un modelo ha tomado ciertas decisiones que pueden ser
erróneas y por tanto ser incapaces de ofrecer solución al problema.

La topología es la rama de las matemáticas que estudia las transformaciones
continuas, por lo que nos ofrece una perspectiva única para investigar cómo las CNNs
procesan los datos. Aunque la topología es un área bastante abstracta de las
matemáticas, su uso en el análisis de datos complejos es relativamente nuevo y prometedor.
Dentro de este campo, la homología persistente aplicada en el marco del TDA se presenta
como una herramienta innovadora para descifrar la manera en que las CNNs modifican
la \enquote{forma} de los datos durante su procesamiento.

La aplicación de métodos topológicos a los problemas del aprendizaje profundo no
es solo novedosa, sino que también tiene un gran potencial para transformar nuestra
comprensión de los modelos complejos. Al explorar cómo el TDA puede mejorar la
transparencia y eficacia de las CNNs, este estudio no solo busca aclarar el funcionamiento
interno de estos modelos, sino que también se adentra en un campo poco explorado
que cruza varias disciplinas con el objetivo final de abrir nuevas vías de
investigación y aplicaciones prácticas no solo para mejorar la manera en que
interactuamos con estas tecnologías, sino también para entender y confiar en las
decisiones que toman.

\section{Estructura del trabajo}

Con el fin de profundizar en la comprensión de los modelos de CNNs desde la
perspectiva de la homología persistente, este trabajo se ha estructurado en cuatro
partes.

La primera parte se centra en establecer las bases teóricas de la homología persistente,
para lo cual se explora detalladamente la teoría de la homología simplicial. Esta
área de la topología algebraica se desarrolló originalmente para estudiar el concepto
de \enquote{agujero} en diversas dimensiones mediante el uso de símplices, que son
estructuras que generalizan el concepto de triángulo a múltiples dimensiones.
Los símplices suelen agruparse en lo que se conoce como complejos simpliciales, que
son conjuntos de símplices que se combinan de manera que sus intersecciones cumplen
ciertas propiedades. Estas estructuras forman parte de una familia más general, conocida
como CW-complejos, que ofrecen un marco más flexible para la construcción de
espacios topológicos a través de la unión de piezas llamadas celdas. También exploraremos
la sucesión de Mayer-Vietoris, una herramienta poderosa en topología algebraica
que permite descomponer espacios topológicos complejos en uniones de subespacios
más simples, facilitando el cálculo de sus invariantes topológicos como los módulos
de homología y la relación de estos con conceptos como la conexión. Con todo
esto podemos finalmente introducir la homología persistente y el Teorema de
Correspondencia, el cual nos dará las herramientas necesarias que hoy emplea el TDA.

En la segunda parte de este trabajo, se realizará un estudio detallado sobre los
principios del aprendizaje profundo que fundamentan las CNNs, comenzando con un repaso
histórico desde las neuronas artificiales básicas hasta los modelos de CNN más avanzados
en la actualidad. Esta sección abordará tanto las propiedades fundamentales de
las CNNs como sus características específicas, mostrando cómo estas han evolucionado
para proporcionar mejores y más eficientes predicciones en el ámbito de la clasificación
de imágenes.

Finalmente introduciremos el TDA, explicando sus principios y técnicas, y exploraremos
su aplicación en el ámbito de las CNNs para comprender cómo la estructura de los
datos afecta el aprendizaje de la red. Esta exploración teórica establecerá la
base para los estudios y experimentos realizados en este trabajo. se expondrán los
resultados y la metodología empleada en un exhaustivo estudio de la homología persistente
y de las propuestas realizadas.

Para terminar, se presentan las conclusiones obtenidas y propuestas futuras para
el estudio de la manera en que las CNNs cambian la \enquote{forma} de los datos.

\section{Objetivos}

El presente trabajo propone una serie de objetivos con el fin de profundizar en
las bases de la homología persistente y realizar un estudio práctico en el
ámbito de las CNNs:

\begin{enumerate}
	\item En el ámbito de las \textbf{matemáticas} se propuso como objetivo
	principal profundizar en los contenidos de la topología algebraica y una
	introducción rigurosa a la homología persistente. Con dicho fin, la
	\autoref{part:math} cumple con los siguientes puntos:
	\begin{itemize}
		\item Se realiza una descripción de las principales herramientas algebraicas
		necesarias para el estudio de la homología simplicial y la homología persistente
		\autoref{chapter:alg-fundamentals}.
		
		\item Se estudian los complejos simpliciales, objetos de estudio de la
		homología simplicial. Además, se exploran otras estructuras topológicas
		relevantes como los CW-complejos y las variedades topológicas en el
		\autoref{chapter:complex}.
		
		\item Se introduce la homología simplicial y una de sus principales herramientas,
		la sucesión de Mayer-Vietoris, además de su relación con la conexión topológica
		en el \autoref{chapter:homology}.
		
		\item Se introduce el concepto de homología persistente y su principal resultado,
		el Teorema de correspondencia, descritos en el \autoref{chapter:persistent-homology}.
	\end{itemize}
	
	\item Posteriormente, desde el ámbito del \textbf{aprendizaje profundo} se plantea
	comprender los principales modelos de CNNs en el ámbito de la clasificación de
	imágenes. Para ello, la \autoref{part:deep-learning} explora los siguientes conceptos:
	\begin{itemize}
		\item Se repasan los principales conceptos de inteligencia artificial,
		aprendizaje automático y visión artificial en el
		\autoref{chapter:concepts}.
		
		\item Se realiza un repaso histórico y del estado del arte de las CNNs en los
		Capítulos \ref{chapter:ann} y \ref{chapter:cnn}.
	\end{itemize}
	
	\item Finalmente, se propone realizar un estudio de las CNNs mediante el uso
	de técnicas de TDA y formalizar una propuesta en base a las conclusiones
	obtenidas. Para cumplir con estos objetivos, se han superados los siguientes
	hitos en la \autoref{part:proposal}:
	\begin{itemize}
		\item Se analiza cómo transforman diferentes modelos y elementos de las CNNs
		la homología persistente de los datos en el \autoref{chapter:analisis}.
		
		\item Se propone un regularizador topológico con el objetivo de mejorar la
		tasa de clasificación y la transferibilidad de los modelos en los Capítulos
		\ref{chapter:tda} y \ref{chapter:analisis}.
	\end{itemize}
\end{enumerate}

\section{Presupuesto}

La primera consideración en el coste de la elaboración del estudio viene dada
por la mano de obra empleada. El equipo empleado por el trabajador se trata de un
ordenador portátil valorado en 600€ y una vida útil de 6 años. El proyecto se ha
realizado durante un periodo de 10 meses y medio, donde en promedio se le ha dedicado
4 horas diarias en los días de semana, lo que se traduce en 20 horas semanales. Por
otro lado, según el portal de transparencia empresarial Glassdoor\footnote{\href{https://www.glassdoor.es/Sueldos/data-scientist-sueldo-SRCH_KO0,14.htm}{https://www.glassdoor.es/Sueldos/data-scientist-sueldo-SRCH\_KO0,14.htm}},
el salario de un científico de datos promedio en España se comprende en el rango
de los 30.000€ a 45.000€. Dado que el perfil del empleado es de junior,
supondremos un salario de 30.000€ anuales, lo que implica un salario de 15€ por
hora.

El coste derivado del entrenamiento de modelos de aprendizaje profundo suele ser
elevado debido al consumo energético de las GPUs y el tiempo de entrenamiento
necesario. Además, ha de tenerse en cuenta gastos derivados como mantenimiento y
refrigeración. Dado que se han entrenado un total de 88 modelos con tiempos de entrenamiento
medios de una hora, más algunas pruebas y expermientos, llegamos a la conclusión
que el tiempo total de GPU empleado es de unas 100 horas. En particular, se han empleado
dos Quadro RTX 8000, con un consumo de 260W por hora. Sabiendo que el precio de la
luz en Granada ronda los 0,1€/kWh en promedio, nos queda el presupuesto total
que figura en la \autoref{tab:presupuesto}.

\begin{table}[h!]
	\centering
	\begin{tabular}{|l|r|}
		\hline
		\textbf{Categoría}                     & \textbf{Costo}      \\
		\hline
		Mano de obra                           & 13,640€             \\
		\hline
		Amortización del equipo                & 87,50€              \\
		\hline
		Coste de energía (GPU)                 & 26€                 \\
		\hline
		Coste de mantenimiento y refrigeración & 20€                 \\
		\hline
		\textbf{Total}                         & \textbf{13,773,50€} \\
		\hline
	\end{tabular}
	\caption{Presupuesto detallado del estudio.}
	\label{tab:presupuesto}
\end{table}

\section{Planificación}

La organización de este trabajo supuso un reto desde comienzos del curso 2023-2024.
El hecho de tener que compaginar un curso completo con un trabajo de tal calibre
requería de una organización minuciosa. Por ello, desde comienzos de septiembre
de 2023 se comenzó a investigar y profundizar en los conceptos teóricos
requeridos para tener una comprensión profunda de la homología persistente. Esta
fase requirió de más tiempo del esperado, tal y como puede observarse en la
\autoref{fig:plan}. La principal complicación surgió debido a la aparición de requisitos
no contemplados en demostraciones puntuales que, junto a los exámenes, retrasó la
finalización del marco teórico matemático.

La etapa final del desarrollo teórico se compaginó con reuniones con los tutores
para ir concretando el camino de los experimentos y el planteamiento de hipótesis.
Esto llevó a que la implementación del código necesario se realizara en su
mayoría durante los meses de abril y mayo, que posteriormente iría siendo modificado
en función de las necesidades de la investigación y los resultados obtenidos.

\begin{figure}[H]
	\centering
	\includegraphics[width=150mm]{img/planificacion.png}
	\caption{Planificación temporal prepuesta frente a la finalmente realizada.}
	\label{fig:plan}
\end{figure}

\endinput               

% -------------------------------------------------------------------
% MAINMATTER
% -------------------------------------------------------------------
\mainmatter % activa la numeración de capítulos, resetea la numeración de las páginas y usa números arábigos

\part{Fundamento matemático} % Dividir un TFG en partes OPCIONAL
% Añadir tantos capítulos como sea necesario
% !TeX root = ../tfg.tex
% !TeX encoding = utf8

\chapter{Fundamentos del álgebra homológica}
La teoría de homología es una rama de la topología que trata de resolver
problemas topológicos en el ámbito del álgebra. Por este motivo es importante conocer
muy bien algunas herramientas algebraicas que iremos utilizando con frecuencia.
En todo el capítulo usaremos como referencia principal
\cite{maclane2012homology}.

\section{Módulos}

La estructura de módulo surge con la idea de generalizar el concepto de espacio vectorial
sobre un cuerpo a un anillo. Nuestro interés en ellos radica en que la teoría de
homología se construye sobre módulos y por ello es necesario hacer una
introducción al campo. Esta sección recoge algunas definiciones y resultados de interés
vistos en la asginatura de Álgebra Moderna y complementada con los contenidos de
\cite{dummit2004abstract}.

\begin{definicion}
	Sea \(R\) un anillo con elemento identidad \(1 \neq 0\). Un \textbf{\(R\)-módulo
	izquierdo} \(A\) es un grupo abeliano aditivo junto con una función \(p: R \times
	A \rightarrow A\) con \((r, a) \to ra\) tal que dados \(r,r' \in R\), \(a,a' \in A\)
	se tiene
	\begin{enumerate}
		\item \((r+r') a = r a + r' a\),
		\item \((rr') a = r (r' a)\),
		\item \(r(a+a') = ra + ra'\),
		\item \(1a = a\).
	\end{enumerate}
\end{definicion}

De la definición anterior se sigue que \(0a = 0\) y \((-1)a = -a\).

De manera análoga, definimos \textbf{\(R\)-módulo derecho} donde el anillo actúa
por la derecha en vez de por la izquierda de forma que
\(p: A \times R \rightarrow A\). Si \(R\) es un anillo conmutativo, los \(R\)-módulos
izquierdos y derechos coinciden y les llamamos simplemente \(R\)-módulos. Como los
resultados de \(R\)-módulos izquierdos y derechos son análogos, trabajaremos con
los \(R\)-módulos izquierdos y nos referiremos a ellos como \textbf{\(R\)-módulos} o
\textbf{módulos} a menos que se indique explícitamente lo contrario.

\begin{ejemplo}
	El interés de los \(R\)-módulos subyace en la cantidad de estructuras conocidas que
	engloba. Si por ejemplo consideramos el \(K\)-módulo donde \(K\) es un cuerpo, éste
	adquiere la estructura de \textbf{espacio vectorial}. Ahora sea \(A\) un \(\mathbb{Z}\)-módulo.
	Definimos el producto \(p\) de forma que para \(n \in \mathbb{Z}\) y \(a \in A\) con
	\(n>0\), \(na = a + a + \ldots + a\) (\(n\) veces), \(0a=0\) y \((-n)a = -(na)\). Entonces
	\(A\) ha de tener estructura de \textbf{grupo abeliano}. En particular, si \(R\)
	es un anillo entonces es también un \(R\)-módulo.
\end{ejemplo}

\begin{definicion}
	Sea \(A\) un \(R\)-módulo izquierdo y \(S\) un subconjunto de \(A\). Diremos que \(S\)
	es un \textbf{submódulo} de \(A\), esto es, \(S \subset A\), si \(S\) es cerrado
	respecto a la suma y si \(r \in R\), \(s \in S\) entonces \(rs \in S\).
\end{definicion}
De la definición anterior se deduce que \(S\) es un \(R\)-módulo.
\begin{definicion}
	Sea \(R\) un \(R\)-módulo. Si un submódulo de \(R\) es un subconjunto \(L \subset R\)
	cerrado respecto a la suma tal que \(rL = \{rl \ : \ l \in L\} \subset L\) para
	todo \(r \in R\), lo llamaremos \textbf{ideal} de \(R\).
\end{definicion}

Tomando un ideal izquierdo \(L\) de \(R\) y un \(R\)-módulo izquierdo \(A\), definimos
el producto del ideal \(L\) por el módulo \(A\)
\[
	LA = \left\lbrace \sum_{i=0}^{n} l_{i}a_{i} \ : \ l_{i} \in L,\ a_{i} \in A,\ n
	\in \mathbb{N}\right\rbrace
\]
donde \(LA\) es un submódulo de \(A\). En particular, el producto de dos ideales
izquierdos \(LL'\) es también un ideal izquierdo y \((LL')A = L(L'A)\).
\begin{definicion}
	Sea \(A\) un \(R\)-módulo. Definimos el \textbf{anulador de \(A\)} como el submódulo
	\(\ann(A) = \{r \in R : ra=0 \ \forall a \in A\}\). De manera análoga, llamaremos
	\textbf{anulador de \(a \in A\)} al submódulo
	\(\ann(A) = \{r \in R : ra=0 \ \forall a \in A\}\).
\end{definicion}
\begin{definicion}
	Sea \(A\) un \(R\)-módulo y sea \(a \in A\). El conjunto
	\(Ra = \{r \cdot a : r \in R\}\) es un submódulo de \(A\) que denominaremos
	\textbf{submódulo cíclico generado por \(a\)}.
\end{definicion}
\begin{definicion}
	Sea \(A\) un \(R\)-módulo y sean \(a_{1}, \ldots, a_{n} \in A\). Entonces
	\(Ra_{1} + \ldots + Ra_{n} = \{r \cdot a_{1} + \ldots + r \cdot a_{n} : r \in R\}\)
	es un submódulo de \(A\) que llamaremos \textbf{submódulo generado por
	\(a_{1}, \ldots, a_{n}\)}.
\end{definicion}
\begin{definicion}
	Sea \(M\) un \(R\)-módulo. Un submódulo \(N\) de \(M\) es \textbf{finitamente generado}
	si existe un subconjunto finito \(A \subset M\) tal que \(N = RA\).
\end{definicion}
\begin{definicion}
	Sean \(A\), \(B\) \(R\)-módulos. Definimos el \textbf{homomorfismo de \(R\)-módulos}
	de \(A\) a \(B\) como la aplicación \(\alpha: A \rightarrow B\) tal que
	\begin{enumerate}
		\item \(\alpha(a+a') = \alpha(a) + \alpha(a')\),
		\item \(\alpha(ra) = r\alpha(a)\)
	\end{enumerate}
	para todo \(a,a' \in A\), \(r \in R\).
\end{definicion}

Cuando \(\alpha: A \rightarrow B\) sea un homomorfismo de \(R\)-módulos, diremos que
\(A\) es el \textbf{dominio} y \(B\) el \textbf{rango}. La \textbf{imagen} de \(\alpha\)
es el conjunto \(\im(\alpha) = \{ \alpha(a) : a \in A \}\). El \textbf{núcleo} será
el conjunto de elementos que se anulan en su imagen, esto es, \(\ker(\alpha) = \{
a \in A : \alpha(a) = 0 \}\). Diremos que \(\alpha\) es un \textbf{epimorfismo} cuando
\(\alpha\) sea sobreyectiva, un \textbf{monomorfismo} cuando \(\alpha\) sea inyectiva
y un \textbf{isomorfismo} si \(\alpha\) es un epimorfismo y un monomorfismo a la
vez. Si existe un isomorfismo entre \(A\) y \(B\) diremos que son \textbf{isomorfos}
y lo notaremos \(A \cong B\). Un homomorfismo \(\alpha: A \rightarrow A\) lo llamaremos
\textbf{endomorfismo}.

Dado que el núcleo y la imagen de un homomorfismo de \(R\)-módulos coincide con el
de los grupos abelianos subyacentes, la siguiente caracterización es inmediata de
la ya conocida para grupos:

\begin{proposicion}
	Sea \(\alpha: A \to B\) un homomorfismo de \(R\)-módulos. Entonces
	\begin{enumerate}
		\item \(\alpha\) es un monomorfismo si, y sólo si, \(\ker(\alpha) = 0\).
		\item \(\alpha\) es un epimorfismo si, y sólo si, \(\im(\alpha) = B\).
	\end{enumerate}
\end{proposicion}

Es frecuente escribir el homomorfismo de \(R\)-módulos \(\alpha: A \rightarrow B\)
como \(A \xrightarrow{\alpha}B\). Respecto a la notación de la imagen de un elemento
\(a \in A\) por \(\alpha\), pondremos \(\alpha(a)\) o simplemente \(\alpha a\). En cuanto
a la imagen de \(A\) por \(\alpha\), lo representaremos de manera análoga por
\(\alpha(A)\) o \(\alpha A\).

Dados dos homomorfismos de \(R\)-módulos
\(\alpha_{1}, \alpha_{2} : A \rightarrow B\), su \textbf{suma}
\(\alpha_{1} + \alpha_{2}\) la definimos como
\((\alpha_{1} + \alpha_{2})(a) = \alpha_{1}(a) + \alpha_{2}(a)\) para todo
\(a \in A\). Además, dados dos homomorfismos de \(R\)-módulos
\(\alpha: A \rightarrow B\), \(\beta: B \rightarrow C\), su \textbf{composición} \(\beta
\circ \alpha: A \rightarrow C\) es también un homomorfismo de \(R\)-módulos. Nótese
que para que la composición sea posible, el rango de \(\alpha\) tiene que ser igual
al dominio de \(\beta\). En ocasiones usaremos la notación por yuxtaposición
\(\alpha\beta = \alpha \circ \beta\). Llamaremos \textbf{inversa} (por ambos lados)
de \(\alpha : A \rightarrow B\) al homomorfismo \(\alpha^{-1}: B \rightarrow A\) tal
que \(\alpha^{-1}\circ \alpha = \id_{A}\) y \(\alpha \circ \alpha^{-1}= \id_{B}\). Una
\textbf{inversa izquierda} de \(\alpha\) es una función \(\gamma: B \rightarrow A\)
tal que \(\gamma \circ \alpha = \id_{A}\). De manera análoga, el homomorfismo
\(\theta : B \to A\) es \textbf{inversa derecha} de \(\alpha\) si \(\alpha \circ \theta
= \id_{B}\).

Si \(T \subseteq B\), el conjunto
\(\alpha^{-1}T = \{ s \in A \ : \ \alpha(s) \in T \}\) es un submódulo de \(A\), llamado
la \textbf{imagen inversa} (completa) de \(T\). En particular,
\(\ker \alpha = \alpha^{-1}0\), donde \(0\) denota el submódulo de \(B\) que consiste solo
del elemento cero.

Sea \(T \subseteq B\) donde \(B\) es un \(R\)-módulo, llamaremos \textbf{inclusión} o \textbf{inyección
canónica} al homomorfismo \(i: T \rightarrow B\) tal que \(i(t) = t\) para todo
\(t \in T\). En particular, \(i\) es un monomorfismo. Las \textbf{clases laterales}
de \(T\) en \(B\) son los conjuntos \(b + T = \{b + t : t \in T\}\) donde \(b \in B\). Dos
clases laterales \(b_{1} + T\), \(b_{2} + T\) son iguales si \(b_{1} - b_{2} \in T\). Como
\(T\) es un submódulo, el grupo abeliano \(B/T\) se convierte en un \(R\)-módulo
cuando \(r(b+T) = rb + T\) para todo \(r \in R\). A este \(R\)-módulo lo llamaremos el
\textbf{módulo cociente} de \(B\) sobre \(T\). El homomorfismo \(\pi: B \rightarrow B/
T\) tal que \(\pi(b) = b + T\) es un epimorfismo que llamaremos \textbf{proyección
canónica} de \(B\).

\begin{proposicion}
	[Teorema de factorización] \label{prop:teo-fact} Sea \(\beta: B \rightarrow B'\)
	un homomorfismo de módulos con \(T \subset \ker \beta\). Existe entonces un
	único homomorfismo de módulos \(\beta': B/T \rightarrow B'\) con
	\(\beta'\pi = \beta\); es decir, el siguiente diagrama con \(\beta(T) = 0\)
	\begin{equation}
		\label{firstisothm}\xymatrix{ B \ar[r]^\pi \ar[dr]^\beta & B/T \ar[d]^{\beta'} \\ & B' }
	\end{equation}
	es conmutativo. Al homomorfismo \(\beta'\) lo llamaremos \textbf{homomorfismo
	inducido} por \(\beta\).
\end{proposicion}
\begin{proof}
	Consideremos el \(R\)-módulo \(B\) y definamos \(\beta'(b + T) = \beta(b)\). Por
	estar \(T\) contenida en el núcleo de \(\beta\), la función está bien definida. En
	efecto, si \(a,b \in B\) entonces \(a+T = b+T\) lo que implica que \(a-b \in T \subset
	\ker \beta\). Esto es, \(\beta(a-b) = 0\) y por tanto, \(\beta(a)=\beta(b)\). Como \(\beta\)
	es un homomorfismo,
	\[
		\beta'((ra + sb) + T) = \beta(ra+sb) = r\beta(a) + s\beta(b) = r \beta'(a+T)
		+ s \beta'(b+T)
	\]
	para todo \(r,s \in R\), \(a,b \in B\). En consecuencia, \(\beta'\) también es un homomorfismo
	de \(R\)-módulos.
\end{proof}
%En particular, si \( \beta: B \rightarrow B' \) es un epimorfismo con núcleo \( T \), \( \beta': B/T \rightarrow B' \) es un isomorfismo. Esta afirmación puede expresarse de la siguiente manera: cada \( \beta \) con \( \beta(T) = 0 \) \textit{factoriza de manera única} a través de la proyección \( \pi \). Esta propiedad caracteriza a \( \pi: B \rightarrow B/T \) hasta un isomorfismo de \( B/T \), de la siguiente manera:
%
%\begin{proposicion}
%	Si \( T \subset B \) y \( \eta: B \rightarrow D \) es tal que \( \eta(T) = 0 \) y cada \( \beta: B \rightarrow B' \) con \( \beta(T) = 0 \) factoriza de manera única a través de \( \eta \), entonces hay un isomorfismo \( \theta: B/T \rightarrow D \) con \( \theta \pi = \eta \).
%\end{proposicion}
%
%\begin{proof}
%	Factorizamos \(\eta\) a través de \(\pi\) y \(\pi\) a través de \(\eta\), así que \(\eta = (\eta' \pi) \eta = 1_\eta\). Pero \(\eta\) factoriza \textit{únicamente} a través de \(\pi\), así que \(\eta' \pi = 1\). Simétricamente, \(\pi' \eta = 1\). Por lo tanto \(\pi' = (\eta')^{-1}\) y \(\eta'\) es el isomorfismo deseado \(\theta\).
%\end{proof}
%
\begin{teorema}
	[Primer teorema de isomorfía] \label{teo:first-iso} Sea \(\beta : B \to B'\) un homomorfismo
	de \(R\)-módulos. Entonces
	\[
		\frac{B}{\ker \beta}\cong \im \beta
	\]
\end{teorema}
\begin{proof}
	La demostración es análoga a la ya vista para grupos. Construimos un
	homomorfismo de \(R\)-módulos \(\beta: B \to \im \beta\) donde aplicando el
	\nameref{prop:teo-fact} con \(T = \ker \beta\), obtenemos el homomorfismo
	inducido \(\beta' : B/\ker \beta \to \im \beta\) tal que
	\(\beta'(b+\ker \beta) = \beta(b)\). Veamos que está bien definida y que es biyectiva.
	Para ello, consideremos que existe otro \(c \in B\) tal que \(b + \ker B = c + \ker
	B\) lo cual se cumple si, y sólo si, \((b + \ker B) - (c + \ker B) = (b - c) + \ker
	\beta = 0\). Esto es, que \(b - c \in \ker \beta\) y por tanto \(\beta(b-c) = \beta
	(b) - \beta(c) = 0\). Es decir, \(\beta(b) = \beta(c)\). En consecuencia, la aplicación
	\(\beta'\) está bien definida y además es inyectiva. Claramente
	\(\beta: B \to \im \beta\) es sobreyectiva.
\end{proof}
%Cada homomorfismo \(\alpha: A \rightarrow B\) determina dos módulos cociente
%\[\text{Coim } \alpha = A / \text{Ker } \alpha, \quad \text{Coker } \alpha = B / \text{Im } \alpha,\]
%llamados la \textbf{coimagen} y el \textbf{conúcleo} de \(\alpha\). Esta definición provee dos sucesiones exactas cortas
%\[ \text{Ker } \alpha \hookrightarrow A \twoheadrightarrow \text{Coim } \alpha, \quad \text{Im } \alpha \hookrightarrow B \twoheadrightarrow \text{Coker } \alpha, \]
%un isomorfismo \(\text{Coim } \alpha \cong \text{Im } \alpha\) y una sucesión exacta más larga
%\[ 0 \rightarrow \text{Ker } \alpha \xrightarrow{i} A \xrightarrow{\alpha} B \rightarrow \text{Coker } \alpha \rightarrow 0. \]
%Por la \autoref{prop:first_iso}, \(\beta \alpha = 0\) implica que \(\beta\) factoriza de manera única a través de \(\pi\) como \(\beta = \beta' \pi\). Dualmente, si algún \(\gamma': A' \rightarrow A\) tiene \(\alpha \gamma' = 0\), entonces \(\gamma'\) factoriza a través de \(i\) como \(\gamma' = i \gamma''\) para un único \(\gamma'': A' \rightarrow \ker \alpha\). Esta propiedad caracteriza \(i: \ker \alpha \rightarrow A\) como un isomorfismo de \(\text{Ker } \alpha\). Observa las afirmaciones duales: \(\alpha\) es un monomorfismo si y solo si \(\ker \alpha = 0\), y es un epimorfismo si y solo si \(\text{Coker } \alpha = 0\).
%
%Para \(K \subseteq S \subseteq A\) el módulo \(S/K\) es llamado un \textbf{subcociente} de \(A\). Este es, a su vez, un módulo cociente del submódulo \(S\) de \(A\) y un submódulo del módulo cociente \(A/K\). Además, si \(K' \subseteq K \subseteq S' \subseteq S \subseteq A\), entonces \(K'/K\) es un submódulo de \(S'/K\) y la proyección compuesta \(S' \rightarrow (S'/K)/(K'/K)\) tiene núcleo \(K'\), por lo tanto el isomorfismo familiar \((S'/K)/(K'/K) \cong S'/K'\). Esto nos permite escribir cada subcociente \((S'/K)/(K'/K)\) de un subcociente \(S/K\) directamente como un subcociente de \(A\). Si \( \alpha: A \rightarrow A'\) tiene \(\alpha S \subseteq S'\) y \(\alpha K \subseteq K'\), entonces \(\alpha s + K'\) es una clase lateral de \(S'/K'\) determinada de manera única por la clase lateral \(s+K\) de \(S/K\). Por lo tanto \(\alpha_{\ast}(s+K) = \alpha s + K'\) define un homomorfismo
%\[\alpha_{\ast}: S/K \rightarrow S'/K'\]
%\[(\alpha S \subseteq S', \alpha K \subseteq K')\]
%llamado el homomorfismo \textbf{inducido} por \(\alpha\) en los subcocientes dados.
%
%Si \(S\) y \(T\) son submódulos de \(A\), su \textbf{intersección} \(S \cap T\) (como conjuntos) es también un submódulo, así como su \textbf{unión} \(S + T\), consistiendo de todas las sumas \(s + t\) para \(s \in S\), \(t \in T\). El \textbf{teorema del isomorfismo de Noether} afirma que \(1_{A}\) induce un isomorfismo
%
%\[ 1_{\ast}: S/(S \cap T) \cong (S + T)/T. \]
%\subsection{Módulos libres}
%
%\begin{definicion}
%	Sea \(A\) un \(R\)-módulo y \(S\) un subconjunto de \(A\). Llamaremos \textbf{submódulo generado por \(S\)} a la intersección de todos los submódulo de \(A\) que contienen a \(S\) y lo notaremos por \(\langle S \rangle\).
%\end{definicion}
%
%\begin{definicion}
%	Sea \(A\) un \(R\)-módulo y \(S\) un subconjunto finito de \(A\). Si \(\langle S \rangle = A\), diremos que \(A\) es \textbf{finitamente generado} por \(S\).
%\end{definicion}
%
%\begin{definicion}
%	Sea \(A\) un \(R\)-módulo. Una familia de elementos \((e_i)_{i\in I}\) indexada por \(I\) diremos que es \textbf{linealmente independiente} si para cada combinación lineal \(\sum_{i \in I}r_ie_i=0\) con \(r_i \in R\), tenemos que \(r_i=0\) salvo un número finito de índices \(i\).
%\end{definicion}

\begin{definicion}
	Sea \(\{A_{i}\}_{i \in I}\) una familia de \(R\)-módulos indexada por \(I\).
	Definimos el \textbf{producto directo} o \textbf{producto directo externo} de
	\(\{A_{i}\}_{i \in I}\) como el producto cartesiano
	\[
		\prod_{i \in I}A_{i} = \{(x_{i})_{i \in I}: x_{i} \in A_{i} \}
	\]
	donde las operaciones se definen componente a componente:
	\begin{align*}
		(x_{i})_{i \in I}+ (y_{i})_{i \in I} & = (x_{i}+y_{i})_{i \in I} \\
		r(x_{i})_{i \in I}                   & = (rx_{i})_{i \in I}
	\end{align*}
	para todo \(r \in R\), \(x_{i},y_{i} \in A_{i}\), \(i \in I\).
\end{definicion}

\begin{definicion}
	Sea \(\{A_{i}\}_{i \in I}\) una familia de \(R\)-módulos indexada por \(I\).
	Definimos la \textbf{suma directa} o \textbf{suma directa interna} de
	\(\{A_{i}\}_{i \in I}\) como el submódulo de \(\prod_{i \in I}A_{i}\) tal que
	\[
		\bigoplus_{i \in I}A_{i} = \{(x_{i})_{i \in I}: x_{i} = 0 \ \text{p.c.t.}\ i
		\in I\}
	\]
\end{definicion}
\begin{nota}
	Recordemos que una condición se cumple "para casi todo" (p.c.t.) elemento de
	un conjunto si se cumple para todo elemento en él salvo en un subconjunto
	finito de elementos.
\end{nota}
\begin{definicion}
	Sea \(B\) un conjunto y sea \(R \langle B \rangle\) el \(R\)-módulo tal que \(R \langle
	B \rangle = \bigoplus_{b \in B}R_{b}\) donde \(R_{b}=R\) para todo \(b \in B\).
	Llamaremos a dicho \(R\)-módulo el \textbf{\(R\)-módulo libre de base \(B\)}. De
	esta forma cada \(x \in R \langle B \rangle\) se representa por
	\(x = \sum_{b\in B}\lambda_{b} \cdot b\) donde \(\lambda_{b} \in R\) son
	coeficientes no nulos en un número finito de posiciones \(b\).
\end{definicion}
\begin{definicion}
	Sea \(M\) un \(R\)-módulo libre. Si para toda base \(B\) de \(M\), \(B\) tiene la misma cardinalidad,
	entonces decimos que \(M\) tiene \textbf{rango} \(\rg \, M = |B|\), donde \(B\) es
	alguna base de \(M\).
\end{definicion}
\begin{definicion}
	Sea \(x\) un elemento de un \(R\)-módulo. Decimos que \(x\) es un\textbf{ elemento de
	torsión} si existe un \(r \in R \setminus \{0\}\) tal que \(rx = 0\). Por otro lado,
	\(x\) es un \textbf{elemento sin torsión} si el único elemento \(r \in R\) que satisface
	\(rx = 0\) es \(r = 0\). Un \(R\)-módulo se clasifica como \textbf{módulo de torsión}
	si cada uno de sus elementos es un elemento de torsión. Recíprocamente, un \textbf{módulo
	sin torsión} es aquel cuyos elementos no nulos son elementos sin torsión.
\end{definicion}
\begin{definicion}
	Definimos el \textbf{submódulo de torsión} de un \(R\)-módulo \(M\) como el
	conjunto \(\tor(M)= \{x \in M : \ann(x) \neq \{0\}\}\). Es decir, el conjunto de
	todos los elementos de torsión de \(M\).
\end{definicion}
\begin{teorema}
	[Descomposición cíclica primaria] Sea \(R\) un DIP y sea \(M\) un \(R\)-módulo finitamente
	generado. Entonces \(M\) se descompone como la suma directa
	\[
		M \cong R^{f} \oplus \bigoplus_{i = 1}^{k} \frac{R}{\langle a_{i} \rangle}
	\]
	donde \(R^{f}\) es un módulo libre de rango \(f\) y
	\(R/\langle a_{1} \rangle, \ldots, R/\langle a_{k} \rangle\) son módulos cíclicos
	con anuladores \(\langle a_{1} \rangle, \ldots, \langle a_{k} \rangle\). Además,
	\(f\) y los ideales \(\langle a_{1} \rangle, \ldots, \langle a_{k} \rangle\) de \(R\)
	generados por \(a_{1}, \ldots, a_{k} \in R\) están determinados de manera única
	salvo el orden por \(M\).
\end{teorema}
\begin{teorema}
	[Propiedad universal de los módulos libres] \label{teo:univ-prop-free-mod} Sean
	\(B\) un conjunto, \(M\) un \(R\)-módulo y \(\varphi : B \to M\) una aplicación entre
	conjuntos. Entonces existe un único homomorfismo de \(R\)-módulos \(\phi : R \langle
	B \rangle \to M\) de forma que \(\phi(b) = \varphi(b)\) para todo \(b \in B\). Es
	decir, el diagrama
	\[
		\xymatrix{ B \ar@{->}[r]^{\varphi} \ar@{->}[d]^{i} & M \\ R \langle B \rangle \ar@{->}[ru]^{\phi} & }
	\]
	conmuta.
\end{teorema}

\section{Sucesiones exactas}

\begin{definicion}
	Sea \(\{A_{i}, \alpha_{i}\}_{i \in \mathbb{Z}}\) una familia de \(R\)-módulos \(A_{i}\)
	y homomorfismos entre ellos tal que \(\alpha_{i}: A_{i} \rightarrow A_{i+1}\).
	Diremos que la sucesión
	\[
		\cdots \xrightarrow{\alpha_{i-2}}A_{i-1}\xrightarrow{\alpha_{i-1}}A_{i} \xrightarrow
		{\alpha_{i}}A_{i+1}\xrightarrow{\alpha_{i+1}}\cdots
	\]
	es \textbf{exacta} cuando \(\im \alpha_{i} = \ker \alpha_{i+1}\).
\end{definicion}

\begin{definicion}
	Sean \(A,B\) y \(C\) \(R\)-módulos y \(\sigma: A \rightarrow B\),
	\(\gamma: B \rightarrow C\) homomorfismos entre ellos. Diremos que la \textbf{sucesión
	exacta} es \textbf{corta} si
	\[
		(\sigma, \gamma): 0 \rightarrow A \xrightarrow{\sigma}B \xrightarrow{\gamma}C
		\rightarrow 0.
	\]
	Es decir, una sucesión exacta de cinco \(R\)-módulos con los dos módulos
	exteriores siendo cero (y por lo tanto las dos funciones exteriores triviales).
\end{definicion}

\begin{proposicion}
	Sean \(A,B\) y \(C\) \(R\)-módulos y \(\sigma: A \rightarrow B\),
	\(\gamma: B \rightarrow C\) homomorfismos entre ellos. Entonces
	\begin{enumerate}
		\item La sucesión \(0 \to A \overset{\sigma}{\to}B\) es exacta (en \(A\)) si, y
			sólo si, \(\sigma\) es inyectiva.
		\item La sucesión \(B \to C \overset{\gamma}{\to}0\) es exacta (en \(C\)) si, y
			sólo si, \(\gamma\) es sobreyectiva.
	\end{enumerate}
\end{proposicion}
\begin{proof}
	El único homomorfismo que cumple \(0 \to A\) tiene imagen \(0\) en \(A\) y por tanto,
	el núcleo de \(\sigma\) será este si, y sólo si, \(\sigma\) es inyectiva. De manera
	similar, el único homomorfismo \(C \to 0\) es el homomorfismo nulo para todo
	elemento de \(C\), que es la imagen de \(\gamma\) si, y sólo si, \(\gamma\) es sobreyectiva.
\end{proof}
\begin{corolario}
	La sucesión \(0 \rightarrow A \xrightarrow{\sigma}B \xrightarrow{\gamma}C \rightarrow
	0\) es exacta si, y sólo si, \(\sigma\) es inyectiva, \(\gamma\) es sobreyectiva y
	\(\im \sigma = \ker \gamma\).
\end{corolario}

% PUEDE MEJORARSE CON DUMMIT PROPOSICION 22 DE MODULOS
Como acabamos de probar, la exactitud en \(A\) significa que \(\sigma\) es un
monomorfismo, en \(B\) significa que \(\sigma A = \ker \gamma\) y en \(C\) que \(\gamma\)
es un epimorfismo. Así la sucesión exacta corta puede escribirse como
\(A \xrightarrow{\sigma}B \xrightarrow{\gamma}C\), con exactitud en \(B\). Ahora
\(\sigma\) induce un isomorfismo \(\sigma': A \to \sigma A\) y \(\gamma\) un
isomorfismo \(\gamma': B/\sigma A \to C\); juntos estos proveen un isomorfismo de sucesiones
exactas cortas, en la forma de un diagrama conmutativo
\begin{equation}
	\label{exactshortseq-equiv}\xymatrix{ 0 \ar[r] & A \ar[r]^{\sigma} \ar[d]^{\sigma'} & B \ar[r]^\gamma \ar@{=}[d] & C \ar[r] \ar[d]^{(\gamma')^{-1}} & 0 \\ 0 \ar[r] & \sigma A \ar[r]_i & B \ar[r] & B/\sigma A \ar[r] & 0. }
\end{equation}

En resumen, una sucesión exacta corta es simplemente otro nombre para un
submódulo y su cociente.

\begin{ejemplo}
	Respecto al \nameref{prop:teo-fact}, la inclusión \(i\) y la proyección \(\pi\) producen
	una sucesión exacta corta.
	\[
		0 \rightarrow T \xrightarrow{i}B \xrightarrow{\pi}B/T \rightarrow 0.
	\]
\end{ejemplo}
%
%\begin{definicion}
%	Sea \(0 \rightarrow A \xrightarrow{\sigma} B \xrightarrow{\gamma} C \rightarrow 0\) una sucesión exacta corta. Diremos que la sucesión es \textbf{escindible} si existe un homomorfismo \(\rho: C \to B\) tal que \(\gamma \circ \rho = \id_C\).
%\end{definicion}
%\begin{observacion}
%	En las condiciones anteriores, \(\rho\) tiene que ser inyectiva.
%\end{observacion}
%
%\begin{lema}[Lema de escisión]
%	\label{lem:escision}
%	Sea \(0 \rightarrow A \xrightarrow{\sigma} B \xrightarrow{\gamma} C \rightarrow 0\) una sucesión exacta corta. Son equivalentes las siguientes afirmaciones
%	\begin{enumerate}
%		\item La sucesión es escindible.
%		\item B = \(\im \sigma \oplus C'\) donde \(C'\) es un submódulo de \(B\) isomorfo a \(C\). En particular, \(B \cong A \oplus C\).
%		\item Existe \(\eta : B \to A\) tal que \(\eta \circ \sigma = \id_A\).
%	\end{enumerate}
%\end{lema}
%\begin{proof}
%	\((a) \implies (b)\). Por ser la sucesión escindible el diagrama
%	\[
%		\xymatrix{
%			C \ar@{->}[r]^{\rho} \ar@/_/@{->}[rr]_{\id} & B \ar@{->}[r]^{\gamma} & C
%		}
%	\]
%	implica que \(B = \im \rho \oplus \ker \gamma\) ya que \(x = \rho(\gamma(x)) + (x - \rho(\gamma(x)))\) para todo \(x \in B\). En consecuencia, \(B = C' \oplus \im \rho\). Además, por la exactitud de la sucesión deducimos que \(\ker \gamma = \sigma(A)\).
%
%	\((b) \implies (c)\). Es sufiente considerar \(B \overset{\pi}{\to} \sigma(A) \overset{\sigma^{-1}}{\to}\) donde \(\pi\) es la proyección canónica. Por ser la sucesión exacta, \(\sigma\) es inyectiva.
%
%	\((c) \implies (a)\). Siguiendo un razonamiento análogo al anterior, el diagrama
%	\[
%	\xymatrix{
%		A \ar@{->}[r]^{\sigma} \ar@/_/@{->}[rr]_{\id} & B \ar@{->}[r]^{\eta} & A
%	}
%	\]
%	implica que \(B = \sigma(A) \oplus \ker \eta\). Por el \nameref{teo:first-iso} y como \(\ker \gamma = \sigma(A)\), tenemos que \(\ker \eta \cong B/\ker(\gamma) \overset{\overline{\gamma}}{\to} \im \gamma = C\) donde \(\overline{\gamma}\) es el isomorfismo inducido por \(\gamma\). Finalmente, \(\rho = \overline{\gamma}^{-1}\).
%\end{proof}

Para realizar la prueba del siguiente resultado usaremos una estrategia típica en
álgebra homológica conocida como persecución de diagramas. Este método consiste
en buscar composiciones de aplicaciones equivalentes en diagramas conmutativos, y
en explotar las propiedades de los homomorfismos inyectivos, sobreyectivos y
biyectivos, así como de las sucesiones exactas.

\begin{lema}
	[Lema de los cinco] \label{lem:cinco} Consideremos el siguiente diagrama conmutativo
	de \(R\)-módulos
	\[
		\xymatrix{ A_1 \ar@{->}[r]^{\sigma_1} \ar@{->}[d]^{\phi_1} & A_2 \ar@{->}[r]^{\sigma_2} \ar@{->}[d]^{\phi_2} & A_3 \ar@{->}[r]^{\sigma_3} \ar@{->}[d]^{\phi_3} & A_4 \ar@{->}[r]^{\sigma_4} \ar@{->}[d]^{\phi_4} & A_5 \ar@{->}[d]^{\phi_5} \\ B_1 \ar@{->}[r]^{\gamma_1} & B_2 \ar@{->}[r]^{\gamma_2} & B_3 \ar@{->}[r]^{\gamma_3} & B_4 \ar@{->}[r]^{\gamma_4} & B_5 }
	\]
	cuyas filas son exactas. Si además \(\phi_{2}, \phi_{4}\) son isomorfismos, \(\phi
	_{1}\) es un epimorfismo y \(\phi_{5}\) es un monomorfismo, entonces \(\phi_{3}\)
	es un isomorfismo.
\end{lema}
\begin{proof}
	Comencemos viendo la inyectividad de \(\phi_{3}\). Sea \(x \in A_{3}\) tal que
	\(\phi_{3}(x) = 0\). Considerando la conmutatividad y la exactitud de las filas,
	obtenemos
	\[
		\phi_{4}(\sigma_{3}(x)) = \gamma_{3}(\phi_{3}(x)) = \gamma_{3}(0) = 0.
	\]
	Dado que \(\phi_{4}\) es inyectiva, concluimos que \(\sigma_{3}(x) = 0\). La
	exactitud en \(A_{3}\) implica la existencia de \(y \in A_{2}\) tal que
	\(\sigma_{2}(y) = x\). Aplicando \(\phi_{2}\) y utilizando la conmutatividad,
	tenemos que
	\[
		\gamma_{2}(\phi_{2}(y)) = \phi_{3}(\sigma_{2}(y)) = \phi_{3}(x) = 0.
	\]
	La exactitud en \(B_{2}\) implica que existe \(z \in B_{1}\) tal que
	\(\gamma_{1}(z) = \phi_{2}(y)\). Como \(\phi_{1}\) es sobreyectiva, existe
	\(w \in A_{1}\) con \(\phi_{1}(w) = z\). Nuevamente, la conmutatividad del
	diagrama y la inyectividad de \(\phi_{2}\) implican que
	\[
		\phi_{2}(\sigma_{1}(w)) = \gamma_{1}(\phi_{1}(w)) = \gamma_{1}(z) = \phi_{2}(
		y),
	\]
	y por tanto, \(\sigma_{1}(w) = y\). Por la exactitud en \(A_{2}\), concluimos \(x =
	\sigma_{2}(y) = \sigma_{2}(\sigma_{1}(w)) = 0\).

	Veamos ahora la sobreyectividad de \(\phi_{3}\). Dado cualquier \(x \in B_{3}\), como
	\(\phi_{4}\) es sobreyectiva, existe \(p \in A_{4}\) tal que
	\(\phi_{4}(p) = \gamma_{3}(x)\). La exactitud y la inyectividad de \(\phi_{5}\) nos
	llevan a
	\[
		\phi_{5}(\sigma_{4}(p)) = \gamma_{4}(\phi_{4}(p)) = \gamma_{4}(\gamma_{3}(x))
		= 0,
	\]
	implicando que \(\sigma_{4}(p) = 0\). Por exactitud, existe \(r \in A_{3}\) con \(\sigma
	_{3}(r) = p\), y utilizando la conmutatividad, obtenemos
	\[
		\gamma_{3}(\phi_{3}(r)) = \phi_{4}(\sigma_{3}(r)) = \phi_{4}(p) = \gamma_{3}(
		x).
	\]
	Entonces, \(\phi_{3}(r) - x\) pertenece a \(\ker \gamma_{3} = \im \gamma_{2}\). En
	consecuencia, existe \(s \in B_{2}\) tal que \(\gamma_{2}(s) = \phi_{3}(r) - x\). Utilizando
	la sobreyectividad de \(\phi_{2}\), encontramos \(t \in A_{2}\) con
	\(\phi_{2}(t) = s\), y por lo tanto,
	\[
		\phi_{3}(r - \sigma_{2}(t)) = \phi_{3}(r) - \gamma_{2}(\phi_{2}(t)) = \phi_{3}
		(r) - \gamma_{2}(s) = x.
	\]
\end{proof}

\section{Categorías y funtores}
\label{sec:cat}

La teoría de categorías fue introducida por primera vez por Samuel Eilenberg y Saunders
MacLane en \cite{eilenberg1945general}. En particular, las categorías son estructuras
algebraicas que capturan la noción de composición. Gracias a ellas podemos
analizar y comparar estructuras algebraicas, permitiendo sacar conclusiones comunes
y trasladar problemas complejos a otros espacios donde resolverlos es más sencillo.
En esta sección haré una breve introducción de las categorías apoyándome en
\cite{mac2013categories}.

\begin{definicion}
	Una \textbf{categoría} \(\mathcal{C}\) es una tripleta
	\((\mathcal{O}, \hom, \circ)\) formada por
	\begin{enumerate}
		\item Una clase \(\mathcal{O}\), cuyos elementos denominamos \textbf{objetos} de
			\(\mathcal{C}\) y notamos por \(Obj(\mathcal{C})\).
		\item Por cada par de objetos \((A,B)\) de \(\mathcal{C}\), un conjunto
			\(\hom(A,B)\) cuyos elementos son llamados \textbf{morfismos} de \(A\) a \(B\).
			Si \(f \in \hom(A,B)\), normalmente escribiremos \(f: A \rightarrow B\) o \(A \xrightarrow
			{f}B\).
		\item Una \textbf{ley de composición} que asocia a cada morfismo
			\(f: A \rightarrow B\) y a cada morfismo \(g: B \rightarrow C\) un morfismo
			\(g \circ f : A \rightarrow C\) que satisface
			\begin{itemize}
				\item \textbf{Asociatividad}. Si \(f: A \rightarrow B\),
					\(g: B \rightarrow C\) y \(h : C \rightarrow D\) son morfismos de
					\(\mathcal{C}\), entonces \(h \circ (g \circ f) = (h \circ g) \circ f\).
				\item \textbf{Identidad}. A cada objeto \(B\) le podemos asociar un morfismo
					identidad \(\id_{B} : B \rightarrow B\) tal que si \(f: A \rightarrow B\) y
					\(g: B \rightarrow C\) entonces \(g \circ \id_{B} = g\) y \(\id_{B} \circ f
					= f\).
			\end{itemize}
			Llamaremos a este morfismo la \textbf{composición} de \(f\) y \(g\).
	\end{enumerate}
\end{definicion}

\begin{ejemplo}
	Como veremos a continuación, la definición anterior nos va a permitir trabajar
	con un gran número de espacios matemáticos que ya conocemos en el contexto de
	la teoría de categorías. Algunos de ellos son:
	\begin{itemize}
		\item \textbf{La categoría de espacios topológicos}, donde los objetos son todos
			los espacios topológicos y los morfismos todas las aplicaciones continuas entre
			espacios topológicos \(f: X \rightarrow Y\).
		\item \textbf{La categoría de grupos}, donde los objetos son todos los grupos
			y los morfismos todos los homomorfismos de grupos.
		\item \textbf{La categoría de conjuntos}, cuyos objetos son todos los conjuntos
			y sus morfismos todas las aplicaciones entre conjuntos.
		\item \textbf{La categoría de sucesiones exactas de \(R\)-módulos de longitud
			n}. Los objetos son dichas sucesiones \(S: A_{1} \rightarrow \cdots \rightarrow
			A_{n}\). Para dos sucesiones \(S\) y \(S'\), los morfismos son de la forma \(\Gamma
			: S \rightarrow S'\) tal que \(\Gamma = (\gamma_{1}, \ldots, \gamma_{n})\) es una
			tupla donde los \(\gamma_{i}: A_{i} \rightarrow A_{i}'\) son homomorfismos de
			\(R\)-módulos tal que
			\begin{equation}
				\label{exactseqcat}\xymatrix{ A_1 \ar[r] \ar[d]_{\gamma_1} & A_2 \ar[r] \ar[d]_{\gamma_2} & \cdots \ar[r] & A_{n-1} \ar[r] \ar[d]_{\gamma_{n-1}} & A_n \ar[d]^{\gamma_n} \\ A'_1 \ar[r] & A'_2 \ar[r] & \cdots \ar[r] & A'_{n-1} \ar[r] & A'_n }
			\end{equation}
			conmuta para todo \(i \in \{ 1, \cdots, n \}\).
	\end{itemize}
\end{ejemplo}

\begin{definicion}
	Sea \(f \in \hom(A,B)\) un morfismo en la categoría \(\mathcal{C}\). Diremos que
	\(f\) es una \textbf{equivalencia} en \(\mathcal{C}\) si existe en \(\mathcal{C}\) otro
	morfismo \(g \in \hom(B,A)\) tal que \(g \circ f = \id_{A}\) y \(f \circ g = \id_{B}\).
\end{definicion}

Nótese que si \(f \in \hom(A,B)\) es una equivalencia en \(\mathcal{C}\),
\(g \in \hom(B,A)\) debe ser única. En efecto, si suponemos que existe
\(g' \in \hom(B,A)\) tal que \(g' \circ f = \id_{A}\), entonces \(g = g'\circ f \circ
g = g' \circ \id_{B} = g'\).

Dentro de la teoría de categorías los funtores tienen un papel principal, pues
nos va a permitir llevar objetos y morfismos de una categoría a otra preservando
identidades y composiciones.

\begin{definicion}
	Sean \(\mathcal{C}, \mathcal{D}\) dos categorías. Un \textbf{funtor covariante} de
	\(\mathcal{C}\) a \(\mathcal{D}\) es una pareja de funciones \textit{denotadas por
	la misma letra \(T\)} tal que:
	\begin{enumerate}
		\item Una \textbf{función objeto} que asigna a cada objeto
			\(C \in \mathcal{C}\) un objeto \(T(C) \in \mathcal{D}\).
		\item Una \textbf{función de morfismos} qu asigna a cada morfismo
			\(\gamma: C \rightarrow C'\) de \(\mathcal{C}\) un morfismo
			\(T(\gamma): T(C) \rightarrow T(C')\) de \(\mathcal{D}\). Este par de funciones
			satisfacen las siguientes condiciones:
			\begin{equation}
				T(1_{C}) = \id_{T(C)}, \qquad C \in \mathcal{C},
			\end{equation}
			\begin{equation}
				T(\beta \gamma) = T(\beta)T(\gamma), \qquad \beta \gamma \text{ definido
				en }\mathcal{C}.
			\end{equation}
	\end{enumerate}
\end{definicion}

Es decir, un funtor covariante \(T: \mathcal{C}\rightarrow \mathcal{D}\) es una
aplicación que preserva el rango, dominio, identidades y composiciones de
\(\mathcal{C}\) en \(\mathcal{D}\).

%\section{Transformación natural}

Mediante el uso de estructuras algebraicas como módulos y complejos de cadenas,
la homología asigna a cada espacio topológico una serie de grupos de homología, que
reflejan características clave como agujeros y vacíos en diferentes dimensiones.
Estos grupos permiten no sólo discernir la estructura interna de los espacios, sino
también compararlos de manera abstracta. Usaremos de referencia
\cite{maclane2012homology}.

\section{Módulos diferenciales}

Comenzaremos definiendo lo que es un módulo de homología y estableceremos la terminología
que emplearemos cuando trabajemos con ellos.

\begin{definicion}
	Sea \(C\) un \(R\)-módulo junto a un endomorfismo \(d: C \rightarrow C\) tal que \(d^{2}
	= d \circ d = 0\). Diremos entonces que \(C\) es un \textbf{módulo diferencial} y
	llamaremos a \(d\) \textbf{operador borde} de \(C\).
\end{definicion}

Llamaremos a los elementos de \(C\) \textbf{cadenas}. El subgrupo de \textbf{ciclos}
será \(Z(C) = \ker d\), y el subgrupo de \textbf{bordes} \(B(C) = \im d\). Si nos
fijamos, el requisito \(d^{2} = 0\) es equivalente a exigir que \(\im{d}\subset \ker
{d}\).

\begin{definicion}
	Sea \(C\) un grupo diferencial. Definimos el \textbf{\(R\)-módulo de homología} de
	\(C\) como el módulo cociente \(H(C;R)\) tal que
	\[
	H(C;R) = \frac{Z(C)}{B(C)}
	\]
\end{definicion}

En particular, cuando \(C\) sea un \(\Z\)-módulo diferencial, lo llamaremos \textbf{grupo
	diferencial} y notaremos \(H(C;\Z)\) simplemente por \(H(C)\).

Por tanto, el módulo de homología de un módulo diferencial \(C\) está formado por
las clases laterales \([c] = c + B(C)\) donde \(c\) es un ciclo de \(C\). A los elementos
de \(H(C;R)\) los llamaremos \textbf{clases de homología}. Dos ciclos \(c\) y \(c'\) diremos
que son \textbf{homólogos} si ambos pertenecen a la misma clase de homología, esto
es, \(c \sim c'\).

\begin{definicion}
	Sean \(C\) y \(C'\) dos módulos diferenciales y \(d, d'\) sus respectivos operadores
	borde. Diremos que \(f: C \rightarrow C'\) es un \textbf{homomorfismo de módulos
		diferenciales} si \(f\) es un homomorfismo de módulos y además \(d'f = fd\).
\end{definicion}

La anterior definición nos permite preservar la estructura algebraica del módulo
diferencial. De esta forma, si tomamos una cadena \(c \in C\) que sea un ciclo o un
borde y \(f:C \rightarrow C'\) es un homomorfismo de módulos diferenciales,
\(f(c) \in C'\) seguirá siendo un ciclo o un borde de manera correspondiente. En efecto,
pues si \(z \in Z(C)\), entonces \(d'f(z) = f(dz) = f(0) = 0\). Esto es, \(f(z) \in \ker
d'\). Ahora, si \(b \in B(C)\), entonces existe \(c \in C\) tal que \(dc = b\). En consecuencia,
\(d'f(c) = f(dc) = f(b)\) y por tanto, \(f(b) \in \im d'\).
%
%Los grupos diferenciales definen una categoría donde los objetos son los grupos diferenciales y los morfismos son los homomorfismos de grupos diferenciales. Tomamos como ley de composición interna la composición de dichos homomorfismos. Claramente es asociativa pues si \(C,C',\bar{C}, \tilde{C} \in \obj()\) y \(f: C \to C'\), \(g: C' \to \bar{C}\), \(h: \bar{C} \to \tilde{C}\), entonces \(h \circ (g \circ f)\) se cumple si, y sólo si,
%\begin{gather*}
%	\tilde{d}(h \circ (g \circ f))
%	= (\tilde{d}h) \circ (g \circ f)
%	=(h\bar{d}) \circ (g \circ f))
%	=h \circ (\bar{d} g) \circ f \\
%	=h \circ (g d') \circ f
%	=h \circ g \circ (d' f)
%	=h \circ g \circ (fd)
%	=(h \circ g) \circ fd \\
%	=((h \circ g) \circ f) d
%\end{gather*}
%y por tanto \(h \circ (g \circ f) = (h \circ g) \circ f\). La propiedad de identidad se sigue de existir el homomorfismo identidad de grupos.
\begin{definicion}
	Sean \(C, C'\) módulos diferenciales y \(f:C \rightarrow C'\) un homomorfismo de módulos
	diferenciales. Definimos la función \(f_{*} = H(f): H(C;R) \rightarrow H(C';R)\)
	tal que
	\[
	f_{*}([c]) = [f(c)]
	\]
	Diremos que \(H(f)\) es el \textbf{homomorfismo inducido} por \(f\).
\end{definicion}
%\begin{proposicion}
%	En estas condiciones, \(H\) es un funtor covariante de la categoría de grupos diferenciales a la categoría de grupos.
%\end{proposicion}
%\begin{proof}
%	Por la definición dada del grupo de homología, es claro que la función objeto \(H\) asigna a cada grupo diferencial \(C\) un grupo de homología \(H(C)\). En cuanto a la función de morfismos, la identidad de grupos diferenciales se preserva pues \(H(\id)([c]) = \id_*([c]) = [\id(c)] = [c]\) para todo \(c \in C\). Además, si \(f,g \in \hom(C)\), entonces \(H(g \circ f)([c]) = (g \circ f)_*([c]) = [(g \circ f)(c)] = [g (f (c))] = g_*([f(c)]) = g_* ( f_*([c])) = (H(g) \circ H(f)) ([c])\) para todo \(c \in C\), manteniendo la ley de composición.
%\end{proof}

\section{Complejos de cadenas}

\begin{definicion}
	Sea \(R\) un anillo. Un \textbf{complejo de cadenas} \(C_{\bullet}\) de \(R\)-módulos
	es una familia \(\{C_{n}, \partial_{n}\}\) donde \(C_{n}\) son \(R\)-módulos y \(\partial
	_{n} : C_{n} \rightarrow C_{n-1}\) homomorfismos de \(R\)-módulos tales que \(\partial
	_{n} \partial_{n+1}= 0\) para todo \(n \in \Z\).
\end{definicion}
\begin{nota}
	Usualmente notaremos directamente \(\partial\) al homomorfismo \(\partial_{n}\) independientemente
	del valor de \(n\) siempre y cuando se sobrentienda por el contexto.
\end{nota}
\begin{observacion}
	La última condición es equivalente a que \(\im{\partial_{n+1}}\subset \ker{\partial_n}\).
\end{observacion}
Un complejo \(C_{\bullet}\) es por tanto una sucesión doblemente infinita
\[
C_{\bullet}: \cdots \rightarrow C_{1}\rightarrow C_{0} \rightarrow C_{-1}\rightarrow
\cdots
\]
donde toda composición de homomorfismos de dicha familia es el homomorfismo nulo.
La \textbf{homología} \(H(C_{\bullet})\) es la familia de \(R\)-módulos
\[
H_{n}(C_{\bullet}) = \frac{\ker \partial_{n}}{\im \partial_{n+1}}
\]
donde \(H_{n}(C_{\bullet})\) es el \textbf{\(n\)-ésimo módulo de homología} de \(C_{\bullet}\).

Luego \(H_{n}(C_{\bullet})=0\) implica que la sucesión \(C_{\bullet}\) es exacta en \(C
_{n}\). A los elementos de \(C_{n}\) los llamaremos \textbf{n-cadenas} o \textbf{cadenas
	de dimensión n}. Un \textbf{n-ciclo} o \textbf{ciclo de dimensión n} de \(C_{\bullet}\)
es un elemento del submódulo \(Z_{n}(C_{\bullet}) = \ker \partial_{n}\). Un
\textbf{n-borde} o \textbf{borde de dimensión n} es un elemento de
\(B_{n}(C_{\bullet})= \im{\partial_{n+1}}\). Si la dimensión se sobrentiende en estos
casos, no es necesario indicarla de manera explícita. La clase lateral de un ciclo
\(c\) la notaremos por \([c] = c + \partial C_{n+1}\). Dos \(n\)-ciclos
\(c,c' \in C_{n}\) pertenecientes a la misma clase lateral \([c]\) decimos que son
\textbf{homólogos}, es decir, \(c \sim c'\).

\begin{definicion}
	Sea \(\{C_{\bullet}^{i}, \partial^{i}\}_{i \in I}\) una familia de complejos de cadenas.
	Su \textbf{suma directa} la definimos como el complejo de cadenas \(\bigoplus_{i
		\in I}C^{i}_{\bullet}\) cuyos operadores borde vienen dados por \(\bigoplus_{i
		\in I}{\partial}^{i}_{n} : \bigoplus_{i \in I}C^{i}_{n} \to \bigoplus_{i \in I}
	C^{i}_{n-1}\) para todo \(n \in \Z\).
\end{definicion}
\begin{proposicion}
	\label{prop:hom-free-commute} Sea \(\{C_{\bullet}^{i}, \partial^{i}\}_{i \in I}\)
	una familia de complejos de cadenas. Entonces su homología conmuta con la suma
	directa, esto es,
	\(H_{n}(\bigoplus_{i \in I}C_{\bullet}^{i}) \cong \bigoplus_{i \in I}H_{n}(C_{\bullet}
	^{i})\)
	para todo \(n \in \Z\).
\end{proposicion}
\begin{proof}
	Para demostrar que la homología conmuta con sumas directas, queremos mostrar
	que para una colección de complejos de cadenas
	\(\{C_{\bullet}^{i}, \partial^{i}\}_{i \in I}\), los homomorfismos
	\begin{align*}
		\phi: H_{n} \left( \bigoplus_{i \in I}C_{\bullet}^{i} \right) \rightarrow \bigoplus_{i \in I}H_{n} (C_{\bullet}^{i}) : [(c_{i})] \mapsto ([c_{i}]), \\
		\psi: \bigoplus_{i \in I}H_{n} (C_{\bullet}^{i}) \rightarrow H_{n} \left( \bigoplus_{i \in I}C_{\bullet}^{i} \right) : ([c_{i}]) \mapsto [(c_{i})],
	\end{align*}
	están bien definidos y son inversos mutuos.
	
	En primer lugar, para comprobar que dichas aplicaciones están bien definidas, observemos
	que \([(c_{i})] = [(c_{i}')]\) si, y sólo si, \([0] = [(c_{i} - c_{i}')]\). Esto
	ocurre si, y sólo si, existe un \(b_{i} \in C_{\bullet}^{i}\) tal que
	\(\partial_{i}(b_{i}) = (c_{i} - c_{i}')\), lo cual es equivalente a \(c_{i} + \partial
	_{i}(b_{i}) = c_{i}'\) para cada \(i \in I\). Por lo tanto, \([(c_{i})] = [(c_{i}')
	]\) si, y sólo si, \(\phi([(c_{i})]) = \phi([(c_{i}')]) = [(c_{i}' + \partial_{i}
	(b_{i}))] = [(c_{i}')]\). De manera análoga, \([(c_{i})] = [(c_{i}')]\) si, y
	sólo si, \(\psi([(c_{i})]) = \psi([(c_{i}')])\). Esto implica que tanto \(\phi\)
	como \(\psi\) están bien definidos.
	
	En segundo lugar, es claro que \(\phi\) y \(\psi\) son homomorfismos de \(R\)-módulos.
	Además, \(\phi\) lleva la clase de equivalencia \([(c_{i})]\) a \(([c_{i}])\),
	mientras que \(\psi\) lleva \(([c_{i}])\) a \([(c_{i})]\), lo que demuestra que son inversos
	el uno del otro.
	
	Por lo tanto, \(\phi\) es un isomorfismo con \(\psi\) como su inverso, estableciendo
	que \(\bigoplus_{i \in I}H_{n} (C_{\bullet}^{i}) \cong H_{n} \left( \bigoplus_{i
		\in I}C_{\bullet}^{i} \right)\).
\end{proof}
\begin{definicion}
	Sean \(C_{\bullet},C_{\bullet}'\) complejos de cadenas. Una \textbf{aplicación
		de cadenas} o \textbf{morfismo de cadenas}
	\(f: C_{\bullet}\rightarrow C_{\bullet}'\) es una familia de homomorfismos de
	\(R\)-módulos \(f_{n}: C_{n} \rightarrow C_{n}'\) tal que
	\(\partial_{n}'f_{n} = f_{n-1}\partial_{n}\) para todo \(n \in \Z\).
\end{definicion}

\begin{equation}
	\label{chaincomplex}\xymatrix{ \cdots & C_{n-1} \ar[l] \ar[d]^{f_{n-1}} \ar@{..>}[dr] & C_n \ar[l]_{\partial_{n}} \ar[d]^{f_n} \ar@{..>}[dr] & C_{n+1} \ar[l]_{\partial_{n+1}} \ar[d]^{f_{n+1}} & \cdots \ar[l] \\ \cdots & C'_{n-1} \ar[l] & C'_n \ar[l]^{\partial'_{n}} & C'_{n+1} \ar[l]^{\partial'_{n+1}} & \cdots \ar[l] }
\end{equation}

Cuando se sobrentienda del contexto, notaremos simplemente por \(\partial\) a los
correspondientes \(\partial_{n}\) y \(\partial_{n}'\).

La función \(H_{n}(f) = f_{*}\) definida por
\(f_{*}([c]) = f_{*}(c + \partial C_{n+1}) = fc + \partial C'_{n+1}\) es un
homomorfismo \(H_{n}(f): H_{n}(C_{\bullet}) \rightarrow H_{n}(C_{\bullet}')\).% Así mismo, cada \(H_n\) es un funtor covariante de la categoría de complejos de cadenas y morfismos de cadenas a la categoría de módulos.

\begin{definicion}
	\label{def:chain_homotopy} Sean \(C_{\bullet},C_{\bullet}'\) complejos de cadenas
	y \(f,g: C_{\bullet}\rightarrow C_{\bullet}'\) dos aplicaciones de cadenas entre
	ellos. Una \textbf{homotopía de cadenas} u \textbf{homotopía algebraica} \(s\)
	es una familia de homomorfismos de módulos \(s_{n}: C_{n} \rightarrow C_{n+1}'\)
	para cada \(n \in \Z\) tal que
	\begin{equation}
		\partial_{n+1}'s_{n} + s_{n-1}\partial_{n} = f_{n} - g_{n}
	\end{equation}
	Diremos entonces que \(f\) y \(g\) son \textbf{algebraicamente homotópicas} y escribiremos
	\(f \simeq g\).
\end{definicion}

\begin{teorema}
	\label{teo:homot-cad-misma-homologia} Si \(s\) es una homotopía de cadenas entre
	\(f,g: C_{\bullet}\rightarrow C_{\bullet}'\), entonces
	\[
	H_{n}(f) = H_{n}(g) : H_{n}(C_{\bullet}) \rightarrow H_{n}(C_{\bullet}')
	\]
\end{teorema}
\begin{proof}
	Si \(c\) es un ciclo de \(C_{n}\), tenemos que \(\partial_{n} c = 0\). Por la \autoref{def:chain_homotopy}
	se cumple que \(f_{n}c-g_{n}c = \partial s_{n} c\). Como consecuencia \(f_{n} c\)
	y \(g_{n} c\) son homólogos lo que implica que \([f_{n} c] = [g_{n} c]\) en
	\(H_{n}(C_{\bullet}')\), como queríamos demostrar.
\end{proof}

\begin{definicion}
	Una aplicación de cadenas \(f: C_{\bullet}\rightarrow C_{\bullet}'\) es una \textbf{equivalencia
		de cadenas} si existe otra aplicación
	\(h: C_{\bullet}' \rightarrow C_{\bullet}\) y homotopías
	\(s: h \circ f \rightarrow{\id_C}_{\bullet}\), \(t: f \circ h \rightarrow{\id_C'}_{\bullet}\)
	tales que \(h \circ f \simeq{\id_C}_{\bullet}\),
	\(f \circ h \simeq{\id_{C'}}_{\bullet}\).
\end{definicion}

Como \(H_{n}({\id_C}_{\bullet}) = \id_{H_n(C_{\bullet})}\), del anterior teorema se
deduce lo siguiente.

\begin{corolario}
	\label{cor:equiv-homot} Si \(f: C_{\bullet}\rightarrow C_{\bullet}'\) es una equivalencia
	de cadenas, la aplicación inducida
	\(H_{n}(f): H_{n}(C_{\bullet}) \rightarrow H_{n}(C_{\bullet}')\) es un
	isomorfismo para cada \(n \in \Z\).
\end{corolario}

\begin{proposicion}
	Sean \(f,g: C_{\bullet}\rightarrow C_{\bullet}'\) y \(f',g': C_{\bullet}' \rightarrow
	C_{\bullet}''\) aplicaciones de cadenas. Sean \(s: f \rightarrow g\), \(s': f' \rightarrow
	g'\) homotopías de cadenas entre ellas tales que \(f \simeq g\), \(f' \simeq g'\). Entonces
	la composición
	\[
	f' s + s' g: f' \circ f \rightarrow g' \circ g \qquad g' \circ g : C_{\bullet}
	\rightarrow C_{\bullet}''
	\]
	es una homotopía de cadenas.
\end{proposicion}
\begin{proof}
	Por ser \(s,s'\) homotopías de cadenas tenemos que \(\partial s + s\partial = f-g\)
	y \(\partial s' + s'\partial = f'-g'\). Aplicando \(f'\) a la izquierda de la
	primera expresión y \(g\) a la derecha de la segunda nos queda
	\begin{equation}
		\begin{cases}
			f'\partial s + f's\partial = f' \circ f-f' \circ g,   \\
			\partial s' g + s'\partial g = f' \circ g-g' \circ g.
		\end{cases}
	\end{equation}
	Sumando ambas igualdades
	\begin{align*}
		f'\partial s + f's\partial + \partial s' g + s'\partial g    & = f' \circ f-f' \circ g + f' \circ g-g' \circ g, \\
		f'\partial s + f's\partial + \partial s' g + s'\partial g    & = f' \circ f - g' \circ g,                       \\
		\partial f' s + f's \partial + \partial s' g + s' g \partial & = f' \circ f - g' \circ g,
	\end{align*}
	donde finalmente queda
	\begin{align*}
		\partial (f' s + s' g) + (f's + s' g) \partial & = f' \circ f - g' \circ g .
	\end{align*}
\end{proof}

\section{Subcomplejos y complejos cociente}

\begin{definicion}
	Un \textbf{subcomplejo} \(S_{\bullet}\) de \(C_{\bullet}\) es una familia de
	submódulos \(S_{n} \subset C_{n}\) tal que para cada \(n \in \Z\), \(\partial S_{n}
	\subset S_{n-1}\).
\end{definicion}

Por tanto, \(S_{\bullet}\) es un complejo en sí con el operador borde \(\partial\)
inducido de \(C_{\bullet}\) y la inclusión \(i: S_{\bullet}\rightarrow C_{\bullet}\)
es una aplicación de cadenas.

\begin{definicion}
	Sea \(S_{\bullet}\) un subcomplejo de \(C_{\bullet}\). El \textbf{complejo
		cociente} \(C_{\bullet}/S_{\bullet}\) es la familia \((C_{\bullet}/S_{\bullet})_{n}
	= C_{n}/S_{n}\) de módulos cocientes con operador borde \(\partial'_{n}: C_{n}/S_{n}
	\rightarrow C_{n-1}/S_{n-1}\) inducido por \({\partial_C}_{\bullet}\).
\end{definicion}

%La proyección es la aplicación de cadenas \(\pi: C_{\bullet} \rightarrow C_{\bullet}/S\) y la sucesión corta \(S_n\) FLECHARARA \(C_n FLECHA (C_{\bullet}/S)_n\) es exacta para cada \(n \in \Z\).
%
%Si \( f: C_{\bullet} \to C_{\bullet}' \) es una aplicación de cadenas, entonces \( \ker(f) = \{\ker(f_n)\} \) es un subcomplejo de \( C_{\bullet} \), \( \text{Im}(f) = \{f_n(C_n)\} \) es un subcomplejo de \( C_{\bullet}' \), mientras que \( C_{\bullet}'/\text{Im}(f) \) es el cociente de \( f \) y \( C_{\bullet}/\ker(f) \) el conúcleo.

\begin{definicion}
	Sean \(f: C_{\bullet}\rightarrow C_{\bullet}'\),
	\(g: C_{\bullet}' \rightarrow C_{\bullet}''\) aplicaciones de cadenas. La sucesión
	de complejos
	\[
	C_{\bullet}\xrightarrow{f}C_{\bullet}' \xrightarrow{g}C_{\bullet}''
	\]
	es \textbf{exacta} en \(C_{\bullet}'\) si \(\im(f) = \ker(g)\); es decir, si cada
	sucesión \(C_{n} \xrightarrow{f_n}{C'}_{n} \xrightarrow{g_n}{C''}_{n}\) de
	módulos es exacta en \({C'}_{n}\).
\end{definicion}

%Para cualquier \( f: C_{\bullet} \to C_{\bullet}', \)
%\[ 0 \to \ker(f) \to C_{\bullet} \xrightarrow{f} C_{\bullet}' \to \text{Coker}(f) \to 0 \]
%es una sucesión exacta de complejos.

\begin{definicion}
	Un complejo \(C_{\bullet}\) es \textbf{positivo} si \(C_{n} = 0\) para todo
	\(n < 0\) con \(n \in \Z\). Su \(n\)-ésimo módulo de homología es entonces positivo ya
	que \(H_{n}(C_{\bullet}) = 0\) para todo \(n < 0\). De manera análoga, un complejo
	\(C_{\bullet}\) es \textbf{negativo} si \(C_{n} = 0\) para todo \(n > 0\) con
	\(n \in \Z\).
\end{definicion}

Los complejos negativos suelen notarse con índices superiores positivos donde \(C_{-n}\)
se sustituye por \(C^{n}\) y \(\partial_{-n}: C_{-n}\rightarrow C_{-n-1}\) por \(\delta
^{n}: C^{n} \rightarrow C^{n+1}\) quedando así
\[
0 \to C^{0} \xrightarrow{\delta^0}C^{1} \xrightarrow{\delta^1}C^{2} \to \cdots,
\quad \delta \circ \delta = 0
\]
donde el \(n\)-ésimo módulo de homología \(H^{n}(C_{\bullet}) = \ker(\delta^{n})/\text{Im}
(\delta^{n-1})\) es positivo en los índices superiores. A dicho complejo lo
llamaremos \textbf{complejo de cocadenas}.

\begin{definicion}
	Sea \(C_{\bullet}\) un complejo positivo de \(R\)-módulos. Denominaremos \textbf{aumento
		de \(C_{\bullet}\)} al homomorfismo sobreyectivo \(\varepsilon : C_{0} \to R\) de forma
	que \(\varepsilon \circ \partial_{1} = 0\).
\end{definicion}

\begin{definicion}
	Sea \(C_{\bullet}\) un complejo de cadenas positivo, \(\varepsilon : C_{0} \to R\)
	un aumento de \(C_{\bullet}\) y sea \(n \in \Z\). Consideremos el complejo positivo
	\(\widetilde{C}_{\bullet}\) tal que \(\widetilde{C}_{n} = C_{n}\) para todo \(n \geq
	0\), \(\widetilde{C}_{n} = 0\) para todo \(n < -1\) y \(\widetilde{C}_{-1}=R\). Consideremos
	también \(\widetilde{\partial}_{n} = \partial_{n}\) para todo \(n \geq 1\) y \(\widetilde
	{\partial}_{0}= \epsilon\). Llamaremos a este complejo \textbf{complejo
		aumentado} de \(C_{\bullet}\).
\end{definicion}
%
%\begin{definicion}
%Sean \(f\) y \(g\) aplicaciones de cadenas de \(C_{\bullet}\) a \(C_{\bullet}'\) y sea \(s\) una homotopía de cadenas entre ellas. Diremos que \(s\) es una \textbf{homotopía de cocadenas} si  está escrita con índices superiores. Esto es, \( s^n: C^n \to C'^{n-1} \) con \( \delta s + s \delta = f - g \).
%\end{definicion}

\begin{definicion}
	Sea \(A\) un módulo. Definimos el siguiente complejo positivo donde \(A_{0} = A\),
	\(A_{n} = 0\) para \(n \neq 0\) y \(\partial = 0\). Un \textbf{complejo sobre \(A\)} es
	un complejo positivo \(C_{\bullet}\) junto con una aplicación de cadenas \(\varepsilon
	: C_{\bullet}\to A\) donde \(\varepsilon\) es un homomorfismo de módulos \(\varepsilon
	: C_{0} \to A\) tal que \(\varepsilon \partial = 0: C_{1} \to A\).
\end{definicion}

\begin{definicion}
	\label{def:contr_homotopy} Una \textbf{homotopía contráctil} para
	\(\varepsilon: C_{\bullet}\to A\) es una aplicación de cadenas
	\(f: A \to C_{\bullet}\) tal que \(\varepsilon f = \id_{A}\) junto con una
	homotopía \(s:{\id_C}_{\bullet}\rightarrow f \varepsilon\) donde
	\({\id_C}_{\bullet}\simeq f \varepsilon\). En otras palabras, una homotopía
	contráctil consiste en homomorfismos de módulos \(f: A \to C_{0}\) y
	\(s_{n}: C_{n} \to C_{n+1}, n = 0, 1, \ldots\), tal que
	
	\[
	\varepsilon f = \id_{A}, \quad \partial_{1}s_{0} + f \varepsilon = \id_{C_0},
	\quad \partial_{n+1}s_{n} + s_{n-1}\partial_{n} = \id_{C_n}\quad n > 0.
	\]
\end{definicion}

Podemos extender el complejo estableciendo \(C_{-1}= A\),
\(\partial_{0} = \varepsilon: C_{0} \to C_{-1}\) y \(s_{-1}= f\). Aplicando la \autoref{def:contr_homotopy},
\(s:{\id_C}_{\bullet}\rightarrow 0\) es una homotopía de cadenas. Si \(\varepsilon:
C_{\bullet}\to A\) tiene una homotopía contráctil, sus grupos de homología son
isomorfos por \(\varepsilon_{*}: H_{0}(C_{\bullet}) \rightarrow A\) para \(n = 0\) y
\(H_{n}(C_{\bullet}) = 0\) para \(n > 0\).

Considérese un complejo de cadenas
\(C_{\bullet}= \{ C_{n}, d_{n} \}_{n \in \mathbb{Z}}\), donde cada \(C_{n}\) es un \(\mathbb{Z}\)-módulo
libre y \(d_{n}: C_{n} \to C_{n-1}\) es el operador diferencial de \(C_{\bullet}\) que
cumple \(d_{n-1}\circ d_{n} = 0\) para todo \(n\). Este tipo de complejos aparece
frecuentemente en el estudio de espacios topológicos.

Supóngase además que cada \(C_{n}\) es finitamente generado. Entonces, el n-ésimo
grupo de homología de \(C_{\bullet}\), definido como
\[
H_{n}(C_{\bullet}) = \frac{\ker(d_{n})}{\im(d_{n+1})},
\]
es un grupo abeliano finitamente generado. Este resultado se sigue del hecho de
que el núcleo y la imagen de los morfismos entre \(\mathbb{Z}\)-módulos libres
finitamente generados son también finitamente generados.

El teorema de estructura para grupos abelianos finitamente generados afirma que
cualquier grupo abeliano finitamente generado \(G\) puede expresarse como una suma
directa de grupos cíclicos de la forma
\[
G \cong \mathbb{Z}^{\beta}\oplus \mathbb{Z}_{m_1}\oplus \ldots \oplus \mathbb{Z}
_{m_k},
\]
donde \(\beta\) es el rango de \(G\) y cada \(\mathbb{Z}_{m_i}\) es un grupo cíclico
de orden \(m_{i}\), con \(m_{1} | m_{2} | \ldots | m_{k}\). Aplicando este teorema
al n-ésimo módulo de homología \(H_{n}(C_{\bullet})\), obtenemos que
\[
H_{n}(C_{\bullet}) \cong \mathbb{Z}^{\beta_n}\oplus \mathbb{Z}_{m_1}\oplus \ldots
\oplus \mathbb{Z}_{m_k},
\]
donde \(\beta_{n}\) es el rango de \(H_{n}(C_{\bullet})\), conocido como el \textbf{n-ésimo
	número de Betti} de \(C_{\bullet}\), y los \(m_{i}\) son los \textbf{n-ésimos
	coeficientes de torsión}, satisfaciendo \(m_{1} | m_{2} | \ldots | m_{k}\).
\begin{definicion}
	Sea \(C_{\bullet}\) un complejo de cadenas y \(k\) un entero no negativo. El
	\textbf{\(k\)-ésimo número de Betti}, \(\beta_{k}(C_{\bullet})\), se define como el
	rango del k-ésimo grupo de homología de \(C_{\bullet}\), \(H_{k}(C_{\bullet})\). Esto
	es, \(\beta_{k}(C_{\bullet}) = \rg(H_{k}(C_{\bullet}))\).
\end{definicion}
Los números de Betti \(\beta_{n}\) proporcionan una medida de la dimensionalidad de
la n-ésima homología, mientras que los coeficientes de torsión \(\{m_{i}\}\)
capturan información adicional sobre la estructura algebraica de
\(H_{n}(C_{\bullet})\). Ambos conjuntos de invariantes son cruciales para el estudio
de la topología algebraica, ya que ofrecen una perspectiva profunda sobre la
estructura de los espacios topológicos desde el punto de vista homológico.
%
%Los complejos \( C_{\bullet} \) de \(\Z\)-módulos libres aparecen en topología. Si cada \( C_n \) es finitamente generado, entonces cada \( H_n(C_{\bullet}) \) es un grupo abeliano finitamente generado. El teorema de estructura para tales grupos presenta \( H_n(C_{\bullet}) \) como una suma directa
%\[ \Z \oplus \ldots \oplus \Z \oplus \ldots \oplus \Z_{m_1} \oplus \ldots \oplus \Z_{m_k}, \]
%donde el número \( \beta_n \) de sumandos cíclicos infinitos y los enteros \( m_1, \ldots, m_k \) (cada uno divisor del siguiente) dependen solo de \( H_n(C_{\bullet}) \). El entero \( \beta_n \) lo llamaremos el \textbf{n-ésimo número de Betti} de \( C_{\bullet} \) y a los \( \{m_i\} \) los \textbf{n-ésimos coeficientes de torsión}.

\endinput
%--------------------------------------------------------------------
% FIN DEL CAPÍTULO.
%--------------------------------------------------------------------
% !TeX root = ../tfg.tex
% !TeX encoding = utf8

\chapter{Símplices y complejos simpliciales}

Los espacios topológicos pueden llegar a ser complicados de estudiar. Los
complejos simpliciales tienen la ventaja de ser estructuras fáciles de estudiar.
Por este motivo, los dotaremos de cierta topología que nos permitirá construir
homeomorfismos a un gran número de espacios topológicos. En este capítulo nos centraremos
en la definición y el estudio de estos objetos en profundidad en la línea de \cite{munkres2018elements}
y lo complementaremos con algunas aportaciones de \cite{lee2010introduction}.

\section{Símplices}
Con la finalidad de generalizar estructuras como el triángulo y el tetraedro, a
finales del siglo XIX nace un nuevo concepto: el símplice. Su sencillez y
propiedades lo convirtieron en una herramienta muy versátil en el estudio de la
topología algebraica, dando lugar a lo que hoy conocemos como homología simplicial.
En esta sección definiremos lo que es un símplice y algunos conceptos asociados
a él que nos serán de gran utilidad en el estudio de dicho campo. Comenzamos recordando
algunos conceptos de la geometría afín.

Como tan sólo será necesario trabajar en el espacio afín usual $N$-dimensional,
lo notaremos simplemente por $\mathbb{R}^{N}$.

\begin{definicion}
	Sea $\{a_{0}, \ldots, a_{p}\}$ un conjunto de puntos en $\mathbb{R}^{N}$.
	Diremos que dicho conjunto es \textbf{afínmente independiente} si para cualesquiera
	$t_{i}\in \mathbb{R}$, las ecuaciones
	\[
	\sum_{i=0}^{p}t_{i}=0 \quad \text{y}\quad \sum_{i=0}^{p}t_{i}a_{i}=0
	\]
	implican que $t_{0}= t_{1}= \ldots = t_{p}$.
\end{definicion}

\begin{definicion}
	Sea $\{a_{0}, \ldots, a_{p}\}$ un conjunto de puntos afínmente independiente.
	Definimos el \textbf{plano afín} $P$ generado por $\{a_{0}, \ldots, a_{p}\}$ como
	el conjunto de puntos $x \in \mathbb{R}^{N}$ tales que
	\[
	x = \sum_{i=0}^{p}t_{i}a_{i}= a_{0}+ \sum_{i=1}^{p}t_{i}(a_{i}- a_{0})
	\]
	para algunos $t_{1}, \ldots, t_{p}\in \mathbb{R}$. Diremos entonces que $P$ es
	el plano que pasa por $a_{0}$ paralelo a los vectores $a_{i}- a_{0}$, $i \in \{
	1, \ldots, p\}$.
\end{definicion}

Nótese que la transformación afín $T$ de $\mathbb{R}^{N}$ tal que
$T(x) = x - a_{0}$ es una traslación que lleva el plano $P$ al subespacio
vectorial de $\mathbb{R}^{N}$ con base
$a_{1}-a_{0}, a_{2}-a_{0}, \ldots, a_{p}-a_{0}$. Si componemos dicha transformación
con una aplicación lineal que lleve cada vector $a_{1}-a_{0}, a_{2}-a_{0}, \ldots
, a_{p}-a_{0}$ a los primeros $N$ vectores de la base usual $\{e_1, \ldots, e_N\}$, obtenemos una
transformación afín $S: P \rightarrow \R^{N}\times \{0\}$ tal que
$S(a_{i}) = e_i$ para cada $i \in \{1, \ldots, p\}$.

\begin{definicion}
	\label{def:simplex} Sea $\{a_{0}, \ldots, a_{p}\}$ un conjunto de puntos afínmente
	independiente en $\mathbb{R}^{N}$. Definimos el \textbf{p-símplice} o \textbf{símplice}
	$\sigma = [a_{0}, \ldots, a_{p}]$ generado por $a_{0}, \ldots, a_{p}$ como el
	conjunto de todos los $x \in \mathbb{R}^{N}$ tales que
	\[
	x=\sum_{i=0}^{p}t_{i}a_{i}\quad \text{y}\quad \sum_{i=0}^{p}t_{i}=1
	\]
	con $t_{i}\geq 0$, $i \in \{0, 1, \ldots, p\}$. Diremos que $t_{i}$ es la
	\textbf{$i$-ésima coordenada baricéntrica} de $x$ respecto a $a_{0}, a_{1}, \ldots
	, a_{p}$.
\end{definicion}

\begin{figure}[h]
	\begin{subfigure}
		{.24\textwidth}
		\centering
		\begin{tikzpicture}
			% 3-símplice
			\node[fill, circle, inner sep=1.5pt] (n0) at (0.0, 0.0) {}; % root
			
			% DRAW NODES
			\draw (n0);
		\end{tikzpicture}
		\caption{0-símplice}
	\end{subfigure}
	\hfill
	\begin{subfigure}
		{.24\textwidth}
		\centering
		\begin{tikzpicture}
			% 3-símplice
			\node[fill, circle, inner sep=1.5pt] (n0) at (0.0, 0.0) {}; % root
			\node[fill, circle, inner sep=1.5pt] (n1) at (1.0, 0.0) {}; % extreme
			
			% DRAW TREE
			\path[draw] (n0)--(n1);
			
			% DRAW NODES
			\draw (n0); \draw (n1);
		\end{tikzpicture}
		\caption{1-símplice}
	\end{subfigure}
	\hfill
	\begin{subfigure}
		{.24\textwidth}
		\centering
		\begin{tikzpicture}
			% 3-símplice
			\node[fill, circle, inner sep=1.5pt] (n0) at (0.0, 0.0) {}; % root
			\node[fill, circle, inner sep=1.5pt] (n1) at (1.0, 0.0) {}; % extreme
			\node[fill, circle, inner sep=1.5pt] (n2) at (0.5, 0.87) {}; % extreme
			
			% DRAW TREE
			\fill[fill=gray!20] (n0.center)--(n1.center)--(n2.center); \path[draw] (n0)--(n1);
			\path[draw] (n1)--(n2); \path[draw] (n2)--(n0);
			
			% DRAW NODES
			\draw (n0); \draw (n1); \draw (n2);
		\end{tikzpicture}
		\subcaption{2-símplice}
	\end{subfigure}
	\hfill
	\begin{subfigure}
		{.24\textwidth}
		\centering
		\begin{tikzpicture}
			% 3-símplice
			\node[fill, circle, inner sep=1.5pt] (n0) at (0.0, 0.0) {}; % root
			\node[fill, circle, inner sep=1.5pt] (n1) at (1.0, 0.0) {}; % extreme
			\node[fill, circle, inner sep=1.5pt] (n2) at (0.5, 0.87) {}; % extreme
			\node[fill, circle, inner sep=1.5pt] (n3) at (1.0, 0.4) {}; % inside
			
			% DRAW TREE
			\fill[fill=gray!60] (n0.center)--(n1.center)--(n2.center); \fill[fill=gray!60]
			(n0.center)--(n2.center)--(n3.center); \fill[fill=gray!60] (n0.center)--(n1.center)--(n3.center);
			\fill[fill=gray!60] (n3.center)--(n1.center)--(n2.center); \path[draw] (n0)--(n1);
			\path[draw] (n1)--(n2); \path[draw] (n2)--(n0); \path[draw] (n0)--(n3);
			\path[draw] (n1)--(n3); \path[draw] (n2)--(n3);
			
			% DRAW NODES
			\draw (n0); \draw (n1); \draw (n2); \draw (n3);
		\end{tikzpicture}
		\caption{3-símplice}
	\end{subfigure}
	\caption{Símplices de dimensión $0$, $1$, $2$ y $3$.}
	\label{fig:simplex}
\end{figure}

\begin{proposicion}
	Sea $\sigma$ un $k$-símplice definido como en \ref{def:simplex}. Entonces, para
	cualquier $p \in \sigma$, las coordenadas baricéntricas $t_{0}, \ldots, t_{k}$
	de $p$ están determinadas de manera única.
\end{proposicion}
\begin{proof}
	Por definición, cualquier punto arbitrario $p \in \sigma$ puede escribirse
	como una combinación convexa de los puntos $a_{i}$. Esto garantiza la existencia
	de una solución (no negativa) al sistema lineal
	\[
	At= \left(
	\begin{array}{ccc}
		a_{01} & \cdots & a_{k1} \\
		\vdots & \ddots & \vdots \\
		a_{0N} & \cdots & a_{kN} \\
		1      & \cdots & 1
	\end{array}
	\right) \left(
	\begin{array}{c}
		t_0    \\
		\vdots \\
		t_k
	\end{array}
	\right) = \left(
	\begin{array}{c}
		p_1    \\
		\vdots \\
		p_N    \\
		1
	\end{array}
	\right) = p^{*},
	\]
	donde $A$ es la matriz que contiene a los $a_{i}$ como columnas, extendidos
	con un $1$ en la última fila para incorporar la condición de que la suma de
	$t_{i}$ sea igual a $1$, asegurando que estamos considerando combinaciones convexas.
	
	Para demostrar la unicidad, supongamos la existencia de otro vector $t'$ tal
	que $A t'= p^{*}$. Esto lleva a $A(t - t' ) = 0$. Supongamos que
	$A(t - t') = A v = 0$, donde $v = t - t'$. Esto implica que para
	$v_{i}= t_{i}- t_{i}'$ para todo $i \in \{0, \ldots, k\}$
	\[
	\sum_{i=0}^{k}v_{i}\cdot \left(
	\begin{array}{c}
		a_{0i} \\
		\vdots \\
		a_{ki} \\
		1
	\end{array}
	\right) = 0,
	\]
	lo que lleva a que $v_{0}= v_{1}= \cdots = v_{k}= 0$, debido a la independencia
	lineal de las columnas de $A$. En consecuencia, $t= t'$, demostrando así que
	las coordenadas baricéntricas son únicas para cualquier punto $p$ en $\sigma$.
\end{proof}

Los puntos $a_{0}, \ldots, a_{p}$ que generan $\sigma$ los llamaremos \textbf{vértices}
de $\sigma$ y al número $p$ lo llamaremos la \textbf{dimensión} de $\sigma$, que
notaremos por $\dim \sigma$.

\begin{definicion}
	Sea $\sigma=[a_{0}, \ldots, a_{p}]$ un símplice. Una \textbf{cara de dimensión
		$p$} de $\sigma$ será cualquier símplice generado por un subconjunto no vacío de
	$\{a_{0}, \ldots, a_{p}\}$.
\end{definicion}
En particular, la cara de $\sigma$ generada por
$a_{0}, \ldots, a_{i-1}, a_{i+1}, \ldots, a_{p}$ la llamamos la \textbf{cara
	opuesta} de $a_{i}$, $i \in \{0, \ldots, p\}$. Las caras de $\sigma$ diferentes de
$\sigma$ diremos que son \textbf{caras propias} de $\sigma$ y la unión de todas
ellas la llamaremos el \textbf{borde} de $\sigma$ y lo notaremos $\text{Bd}\ \sigma$.
Finalmente, definimos el \textbf{interior} de $\sigma$, $\interior \sigma$, como
el conjunto de puntos de $\sigma$ que no pertenecen a su borde.

En ocasiones, para dos símplices $\sigma$ y $\tau$, escribiremos
$\tau \preceq \sigma$ si $\tau$ es cara de $\sigma$. En caso de ser cara propia,
lo notaremos por $\tau \prec \sigma$.

\begin{proposicion}
	\label{prop:union-disjunta-simplices} Si $\sigma$ es un símplice, entonces es
	unión disjunta del interior de todas sus caras.
\end{proposicion}
\begin{proof}
	Sea $x$ un elemento del símplice $\sigma = [a_{0},\ldots,a_{p}]$ y sean $t_{0},
	\ldots ,t_{p}$ sus coordenadas baricéntricas. Consideremos ahora $\sigma_{k}$
	el símplice resultante de eliminar los vértices cuya coordenada tenía valor
	nulo. Esto es, tomamos el símplice $\sigma_{k}= [a_{i_1}, \ldots, a_{i_k}]$
	donde $t_{i_s}> 0$ para todo $s \in \{1, \ldots, k\}$. Por la construcción de $\sigma
	_{k}$, tenemos que $x$ pertenece a su interior.
	
	Ahora sabemos que todo punto de un símplice pertenece al interior de una cara.
	Finalmente, la unicidad de las coordenadas baricéntricas nos garantiza que la
	unión del interior de dos caras es disjunta.
\end{proof}

Dado un símplice $\sigma$ podemos definir un orden sobre sus vértices. Dos
órdenes de $\sigma$ los consideraremos equivalentes si podemos pasar de uno a
otro con un número par de permutaciones. Así, los ordenamientos posibles para
los vértices de $\sigma$ se pueden agrupar en dos clases de equivalencia
distintas, que definimos como las \textbf{orientaciones del símplice} $\sigma$.

\begin{definicion}
	Decimos que un símplice $\sigma = [a_{0}, a_{1}, \ldots, a_{p}]$ está \textbf{orientado}
	si se le ha asignado una de estas orientaciones. Utilizaremos $[a_{0}a_{1}\ldots
	a_{p}]$ para denotar la clase de equivalencia dada por la orientación $a_{0}< a
	_{1}< \cdots < a_{p}$ del símplice generado por los vértices $a_{0},a_{1},\ldots
	, a_{p}$.
\end{definicion}

\section{Complejos simpliciales}

La importancia de los complejos simpliciales reside en su capacidad para
descomponer espacios topológicos en componentes manejables, permitiendo un análisis
detallado de su estructura. Al considerar la forma en que estos símplices se
conectan y orientan entre sí, los complejos simpliciales facilitarán la definición
de cadenas y ciclos simpliciales que serán indispensables en el estudio de la homología
simplicial.

\begin{definicion}
	Un \textbf{complejo simplicial geométrico} (finito), o simplemente \textbf{complejo simplicial} (finito), \(K\) en \(\mathbb{R}^{N}\), es una
	colección (finita) de símplices en \(\mathbb{R}^{N}\) tal que:
	\begin{enumerate}
		\item Toda cara de un símplice de \(K\) está en \(K\).
		
		\item La intersección de cualesquiera dos símplices de \(K\) es o el vacío o
		una cara de ambos símplices.
	\end{enumerate}
\end{definicion}
\begin{nota}
	Si bien los complejos simpliciales se pueden formular sin la restricción de finitud,
	nosotros trabajaremos principalmente en el caso finito por simplicidad en
	algunos resultados.
\end{nota}

\begin{figure}
	\centering
	\begin{tikzpicture}
		% Vértices
		\node[fill, circle, inner sep=1.5pt] (A) at (0,1) {}; \node[fill, circle,
		inner sep=1.5pt] (B) at (1,0) {}; \node[fill, circle, inner sep=1.5pt] (C) at
		(2,0.5) {}; \node[fill, circle, inner sep=1.5pt] (D) at (1,2) {};
		
		\node[fill, circle, inner sep=1.5pt] (E) at (1,3.5) {}; \node[fill, circle, inner
		sep=1.5pt] (F) at (2.5,4) {}; \node[fill, circle, inner sep=1.5pt] (G) at (3,3)
		{};
		
		\node[fill, circle, inner sep=1.5pt] (H) at (0,3) {};
		
		\node[fill, circle, inner sep=1.5pt] (I) at (3,1) {}; \node[fill, circle, inner
		sep=1.5pt] (J) at (5,3) {}; \node[fill, circle, inner sep=1.5pt] (K) at (4,1)
		{};
		
		% Dibujar el 3-simplex
		\draw[thick, fill=blue!30, opacity=0.5] (A.center) -- (B.center) -- (C.center)
		-- (D.center) -- cycle; \draw[thick] (A.center) -- (D.center); \draw[thick]
		(B.center) -- (D.center); \draw[thick] (C.center) -- (D.center);
		
		% Línea punteada
		\draw[dotted, thick] (A.center) -- (C.center);
		
		% Dibujar el 1-simplex
		\draw[thick] (D) -- (E) -- (F) -- (G); \draw[thick] (D) -- (F);
		
		% Dibujar el 2-simplex
		\draw[thick, fill=blue!30, opacity=0.5] (I.center) -- (J.center) -- (K.center)
		-- cycle;
		
		% Dibujar 0-simplex
		\fill[black] (H) circle (2pt);
	\end{tikzpicture}
	\caption{Visualización de un complejo simplicial.}
\end{figure}

En ciertas ocasiones puede ser interesante saber si dada una colección cualquiera
de símplices, esta es un complejo simplicial o no. Para ello, el siguiente lema
nos puede ser de utilidad.

\begin{lema}
	Una colección $K$ de símplices es un complejo simplicial si, y sólo si, se cumplen
	las siguientes condiciones:
	\begin{enumerate}
		\item Toda cara de un símplice de $K$ está en $K$.
		
		\item La intersección dos a dos del interior de los símplices de $K$ es vacía.
	\end{enumerate}
\end{lema}
\begin{proof}
	Primero, asumamos que $K$ es un complejo simplicial. Dados dos símplices $\sigma
	, \tau \in K$ veamos que si el interior de ambos tiene un punto $x$ en común,
	entonces $\sigma = \tau$. Sea $s = \sigma \cap \tau$ y considero $x \in s$. Si
	$s$ fuera una cara propia de $\sigma$, entonces $x$ pertenecería a la frontera
	de $\sigma$, lo cual no se cumple ya que $x$ pertenece al interior de $\sigma$.
	Por tanto $s = \sigma$. De manera análoga, $s = \tau$, luego $\sigma = \tau$.
	
	Asumamos ahora que se cumplen \textit{(1)} y \textit{(2)}. Queremos ver que si
	el conjunto $\sigma \cap \tau \neq \emptyset$, dicha intersección es la cara
	$\sigma'$ de $\sigma$ generada por los vértices $b_{0},\ldots,b_{m}$ de
	$\sigma$ que están en $\tau$. Primero, $\sigma' \subset \sigma \cap \tau$ por
	ser $\sigma \cap \tau$ convexa y contener a $b_{0}, \ldots, b_{m}$. Para la otra
	inclusión supongamos que $x \in \sigma \cap \tau$. Esto implica que
	$x \in \interior s \cap \interior t$ para alguna cara $s$ de $\sigma$ y alguna
	cara $t$ de $\tau$. Se sigue de \textit{(2)} que $s = t$ por lo que los
	vértices de $s$ están en $\tau$ y por definición, son elementos del conjunto $\{
	b_{0}, \ldots, b_{m}\}$. Concluimos entonces que $s$ es una cara de $\sigma'$,
	lo que implica que $x \in \sigma'$, como queríamos ver.
\end{proof}

\begin{definicion}
	Si $L$ es una subcolección del complejo simplicial $K$ que contiene todas las caras
	de sus elementos, entonces $L$ es un complejo simplicial que llamaremos
	\textbf{subcomplejo} de $K$.
\end{definicion}
\begin{definicion}
	Sea $K$ un complejo simplicial. Diremos \textbf{p-esqueleto} de $K$ al subcomplejo
	formado por todas las caras de $K$ cuya dimensión sea menor o igual que $p$.
	Lo denotaremos por $K^{(p)}$. En particular, $K^{(0)}$ lo llamaremos el
	\textbf{conjunto de vértices} de $K$.
\end{definicion}
\begin{definicion}
	Sea $K$ un complejo simplicial de $\mathbb{R}^{N}$ y sea $|K|$ el subconjunto de
	$\mathbb{R}^{N}$ tal que $|K|$ es la unión de todos los símplices de $K$.
	Definimos el \textbf{politopo} o \textbf{espacio subyacente} de $K$ como el
	espacio topológico $(|K|, \mathcal{T})$ donde los abiertos de $\mathcal{T}$
	son aquellos $O \subseteq |K|$ tal que $O \cap \sigma$ es abierto en $\sigma$
	con la topología inducida de $\mathbb{R}^{N}$ para todo $\sigma \in K$.
\end{definicion}

Veamos que en efecto $(|K|, \mathcal{T})$ es un espacio topológico. $\emptyset, |
K| \in \mathcal{T}$ ya que son abiertos trivialmente en $\sigma$, pues
$\emptyset \cap \sigma = \emptyset$ y $|K| \cap \sigma = \sigma$ para todo
$\sigma \in K$. Si $O_{1}, O_{2}\in \mathcal{T}$, entonces $O_{1}\cap \sigma$, $O
_{2}\cap \sigma$ son abiertos en $\sigma$ luego $(O_{1}\cap O_{2}) \cap \sigma =
(O_{1}\cap \sigma) \cap (O_{2}\cap \sigma)$ es abierto en $\sigma$ para todo $\sigma
\in K$. Por tanto $O_{1}\cap O_{2}\in \mathcal{T}$. Finalmente, consideremos una
familia $\{O_{i}\}_{i \in I}\subset \mathcal{T}$ donde $I$ es un conjunto de
índices. Para cada $\sigma \in K$,
$(\cup_{i \in I}O_{i}) \cap \sigma = \cup_{i \in I}(O_{i}\cap \sigma)$ que
efectivamente es una unión arbitraria de abiertos de $\sigma$. En consecuencia,
$\cup_{i \in I}O_{i}\in \mathcal{T}$.

En general, la topología de $|K|$ es más fina que la inducida de la topología usual
de $\R^{N}$. Si $A$ es cerrado en $|K|$ con la topología inducida de la usual, $A
=B \cap |K|$ para algún cerrado $B$ de $\R^{N}$ y por tanto $B \cap \sigma$ sería
cerrado en $\sigma$ para cada símplice $\sigma$ de $K$. Como consecuencia, $B \cap
|K|=A$ es cerrado en $|K|$ con la topología $\mathcal{T}$ definida anteriormente.
No obstante, la otra inclusión no tiene por qué cumplirse:

\begin{ejemplo}
	Consideremos el complejo no finito $K$ en $\R$ cuyos símplices son todos los intervalos
	$[m,m+1]$ con $m \in \Z \backslash \{0\}$, todos los intervalos de la forma
	$[1/( n+1), 1/n]$ donde $n \in \N$ y todas sus respectivas caras. Como resultado
	tenemos que $|K| = \R$, donde $F = \{1/n : n \in \N\}$ es cerrado en nuestra
	topología $\mathcal{T}$ pero no en la inducida por la usual. Dicho de otra forma,
	$\R \backslash F$ es abierto en $\mathcal{T}$ pero no en la usual.
\end{ejemplo}

Si no hay lugar a confusión, simplemente notaremos al politopo de $K$ por $|K|$ y
lo llamaremos el \textbf{poliedro} $|K|$.

A continuación, mencionemos algunas propiedades relevantes de este espacio topológico.
Para ello fijaremos un complejo simplicial finito $K$ en $\R^{N}$.

\begin{proposicion}
	Sea $K$ un complejo simplicial finito. Entonces el poliedro $|K|$ es compacto.
\end{proposicion}
\begin{proof}
	Si $K$ es un complejo simplicial, sus símplices son conjuntos cerrados y
	acotados. En consecuencia, $|K|$ es unión finita de conjuntos cerrados y
	acotados, luego es cerrado y acotado en $\R^{N}$. Por lo tanto, es compacto.
\end{proof}

\begin{proposicion}
	Sea $K$ un complejo simplicial. \label{prop:simpl-soporte} Si $x \in |K|$, entonces
	existe un único símplice en $K$ tal que $x$ pertenece a su interior.
\end{proposicion}
\begin{proof}
	Si $x \in |K|$, entonces existe algún símplice $\sigma$ de $K$ tal que
	$x \in \sigma$. Por la \autoref{prop:union-disjunta-simplices}, $x$ pertenece
	al interior de alguna cara $\tau$ de $\sigma$. Supongamos ahora que existe otro
	símplice $\rho$ de $K$ tal que $x \in \interior \rho$. Por consiguiente, si $x
	\in \interior \rho \cap \interior \tau$, entonces $x$ pertenecería a una cara
	común $\mu$ de $\rho$ y $\tau$. Esto es, $\mu = \rho \cap \tau$. Ahora si $\rho
	\neq \mu$, el elemento $x$ debería tener alguna coordenada baricéntrica nula
	respecto a los vértices de $\rho$, en contradicción con que $x$ pertenece al interior
	de $\rho$. En consecuencia, $\rho = \mu$. De manera análoga obtenemos
	$\tau = \mu$ y por tanto, $\rho = \tau$.
\end{proof}

\begin{definicion}
	Sea $K$ un complejo simplicial y sea $x \in |K|$. Llamaremos \textbf{símplice
		soporte de $x$} al único símplice que contiene a $x$ en su interior y lo notaremos
	por $\sop(x)$.
\end{definicion}
%
%En consecuencia de la anterior definición, podemos definir las \textbf{coordenadas baricéntricas de un complejo simplicial} como aquellas que coinciden con las del símplice al que pertenece y son $0$ para el resto de vértices del complejo simplicial.
%
%\begin{lema}
%	El poliedro \(|K|\) es Hausdorff.
%\end{lema}
%\begin{proof}
%	Dados \(x_0 \neq x_1\) en $|K|$, existe al menos un vértice \(v\) tal que \(t_v x_0 \neq t_v x_1\), donde $t_v$ es la coordenada baricéntrica respecto a $v$ en $|K|$. Entonces podemos escoger un radio \(r\) entre estos dos números de forma que los conjuntos \(\{x : t_v x < r\}\) y \(\{x : t_v x > r\}\) son los conjuntos abiertos disjuntos requeridos.
%\end{proof}

\begin{corolario}
	\label{cor:simpl-soporte} Sean $\sigma, \tau$ símplices de $K$ tal que $\interior
	\sigma \cap \tau$ es no vacía. Entonces $\sigma$ es una cara de $\tau$.
\end{corolario}
\begin{proof}
	Consideremos $x \in \interior \sigma \cap \tau$. Por la
	\autoref{prop:union-disjunta-simplices} sabemos que $\tau$ es la unión de
	todas sus caras lo que implica que existe una cara $\mu$ de $\tau$ cuyo
	interior contiene a $x$. Por lo tanto,
	$x \in \interior \mu \cap \interior \sigma$ y como consecuencia de la
	\autoref{prop:simpl-soporte}, $\mu = \sigma$.
\end{proof}
\begin{lema}
	\label{lem:cont_poly} Sea $K$ un complejo simplicial y $X$ un espacio topológico.
	Una aplicación $f: |K| \rightarrow X$ es continua si, y sólo si, $f|_{\sigma}$
	es continua para cada $\sigma \in K$.
\end{lema}
\begin{proof}
	Si $f$ es continua, también lo es $f|_{\sigma}$ por ser $\sigma$ un subespacio
	de $K$. Supongamos ahora que $f|_{\sigma}$ es continua para cada $\sigma \in K$.
	Si $C$ es un cerrado de $X$, $f^{-1}(C) \cap \sigma = f|_{\sigma}^{-1}(C)$ es un
	cerrado en $\sigma$ por la continuidad de $f|_{\sigma}$. Concluimos que
	$f^{-1}(C)$ es cerrado en $|K|$ por definición.
\end{proof}

Para finalizar esta sección, comentaremos la principal utilidad del estudio de complejos simpliciales introduciendo el concepto de triangulación.

\begin{definicion}
	Un espacio topológico $X$ es \textbf{triangulable} si existe un complejo
	simplicial $K$ cuyo espacio subyacente es homeomorfo a $X$. Diremos entonces que
	el homeomorfismo $h: |K| \rightarrow X$ es una \textbf{triangulación}.
\end{definicion}

La triangulación nos permite representar un espacio topológico usando un complejo simplicial, lo que es útil porque muchos espacios topológicos importantes pueden ser triangulados. En el \autoref{chapter:homology}, veremos que estudiar ciertas propiedades topológicas en estos complejos es más sencillo y manejable.

El siguiente ejemplo muestra la triangulación de una esfera en $\R^3$ por el método de "proyección radial". Para ello, recordemos el siguiente resultado:

\begin{proposicion}
	\label{lem:closed-map}
	Sean $X,Y$ espacios topológicos tales que $X$ es compacto e $Y$ es Hausdorff. Si además $f: X\to Y$ es una aplicación continua, entonces $f$ es cerrada. En particular, si $f$ es biyectiva, entonces es un homeomorfismo.
\end{proposicion}

\begin{ejemplo}
	Consideremos el complejo simplicial \(K\), que es un tetraedro en \(\mathbb{R}^3\). Denotamos su poliedro asociado por \(|K|\). Supongamos que el origen de \(\mathbb{R}^3\) es el centro de \(|K|\) y que \(\mathbb{S}^2\) es la esfera unidad con radio \(1\). Definimos la función \(f: |K| \to \mathbb{S}^2\) por
	\[
	f(t) = \frac{t}{\|t\|}
	\]
	que es continua en todo \(|K|\), pues \(0 \not \in |K|\). Para cada \(s \in \mathbb{S}^2\), definimos \(g: \mathbb{S}^2 \to |K|\) tal que
	\[
	g(s) = t = s \cdot \sup\{\lambda : \lambda s \in \text{envolvente convexa de }|K|\}
	\]
	donde el supremo asegura que \(t\) está en la frontera de \(|K|\). De este modo, se verifica que \(g(f(t)) = t\) para todo \(t \in |K|\). Por lo tanto, \(f\) es una biyección continua desde un conjunto compacto hacia un espacio Hausdorff, luego es un homeomorfismo y en consecuencia, una triangulación de la esfera $\mathbb{S}^2$.
\end{ejemplo}

En particular, las esferas pertenecen a una familia de espacios topológicos conocidos como \textbf{variedades topológicas}. Estos espacios son el objeto central de estudio en la topología, siendo el objetivo su clasificación bajo homeomorfismo.

\begin{definicion}
	Sea \(X\) un espacio topológico Hausdorff no vacío. Diremos que \(X\) es una \textbf{\(m\)-variedad} o \textbf{variedad de dimensión $n$}
	si cada punto de \(X\) tiene un entorno homeomorfo a un subconjunto abierto de \(\R^m\) con la 
	topología usual.% Análogamente, diremos que \(X\) es una \textbf{\(m\)-variedad con borde} si 
	%cada punto tiene un entorno homeomorfo al semiespacio 
	%\(\mathbb{H}^m = \{ (x_1, x_2, \ldots,x_m) \in \R^m : x_i \geq 0,\ 1 \leq i \leq m \}\).
\end{definicion}
%\begin{observacion}
%	Nótese que toda \(m\)-variedad es una \(m\)-variedad con borde. Para verlo, consideremos \(x\) 
%	en la \(m\)-variedad y un entorno homeomorfo a un subconjunto abierto de \(\R^m\). Entonces, 
%	tiene un entorno abierto homeomorfo a cualquier bola abierta de \(\R^m\) y en particular, 
%	a una contenida en \(\mathbb{H}\).
%\end{observacion}

\begin{figure}[h]
	\centering
	\includegraphics[width=120mm]{img/manifolds.jpg}
	\caption{Ejemplos de variedades de dimensión 2: (a) Esfera, (b) Toro, (c) Banda de Möbius y (d) Botella de Klein. Fuente \cite{deng2020topology}.}
\end{figure}

Aunque existen un gran número de resultados fundamentales relacionados con el estudio de las variedades topológicas, este análisis escapa de los contenidos de este trabajo. Sin embargo, veremos que estos espacios serán necesarios en la aplicación experimental realizada para comprender cómo las redes neuronales transforman los datos desde el punto de vista de la topología.

\section{Celdas y CW-complejos}

A continuación presentamos una generalización del concepto de complejo
simplicial, propuesta por J.H.C. Whitehead en \cite{MR0030759}. Los CW-complejos
reemplazan la estructura simplicial tradicional por estructuras homeomorfas a bolas
abiertas, facilitando el estudio de una gama más amplia de espacios topológicos.

Iniciaremos esta sección estableciendo la notación que utilizaremos. Denotaremos
la \textbf{bola abierta} centrada en $x_{0}$ y de radio $r$ en el espacio
$\mathbb{R}^{N}$ con la topología usual por el conjunto
$B_{r}(x_{0}) = \{ x \in \R^{N} : \| x - x_{0} \| < r \}$. Además, para
cualquier subconjunto $U$ de $\R^{N}$, denotaremos su \textbf{clausura} como
$\overline{U}$ y su \textbf{frontera} como $\bd U$. Por último, la \textbf{esfera
	unidad} de dimensión $n$ será representada simplemente como $\sphere^{n-1}$.

\begin{definicion}
	Sea $X$ un espacio topológico. Diremos que $X$ es una \textbf{celda} abierta (cerrada)
	de dimensión $p$ o $p$-celda si $X$ es homeomorfo a la bola unidad abierta (cerrada)
	de dimensión $p$.
\end{definicion}

Sería interesante disponer de resultados que nos digan cuándo un subconjunto dado
puede ser una celda. La siguiente proposición será de gran utilidad para ver que los complejos
simpliciales no son más que un caso particular de los CW-complejos.

\begin{proposicion}
	\label{prop:compact-convex-closed-cell} Si $D \subseteq \mathbb{R}^{n}$ es un subconjunto
	convexo compacto con interior no vacío, entonces $D$ es una $n$-celda cerrada
	y su interior es una $n$-celda abierta. De hecho, dado cualquier punto
	$p \in \interior D$, existe un homeomorfismo $F : \overline{B}_{1}(0) \to D$ que
	envía $0$ a $p$, $B_{1}(0)$ a $\interior D$, y $\sphere^{n-1}$ a $\bd D$.
\end{proposicion}
\begin{proof}
	Sea $p$ un punto interior de $D$. Reemplazando $D$ por su imagen bajo la
	traslación $x \mapsto x - p$, podemos suponer que $p = 0 \in \interior D$.
	Entonces existe algún $\varepsilon > 0$ tal que la bola $B_{\varepsilon}(0)$
	está contenida en $D$. Utilizando la dilatación $x \mapsto x/\varepsilon$,
	podemos asumir $B_{1}(0) \subseteq D$.
	\begin{figure}
		\centering
		\includegraphics[width=10cm]{img/proof-cell.png}
		\caption{Esquema que muestra la idea de que cada rayo tan sólo tiene un
			punto en la frontera en la demostración de la
			\autoref{prop:compact-convex-closed-cell}. Fuente \cite{lee2010introduction}.}
	\end{figure}
	A continuación demostraremos que cada semirrecta cerrada que comienza en el origen
	interseca $\bd D$ en exactamente un punto. Sea $R$ tal semirrecta cerrada. Dado
	que $D$ es compacto, su intersección con $R$ es compacta; por lo tanto, hay un
	punto $x_{0}$ en esta intersección donde la distancia al origen alcanza su
	máximo. Este punto se identifica fácilmente como parte del borde de $D$. Para
	demostrar que solo puede haber un punto así, mostramos que el segmento de línea
	desde $0$ hasta $x_{0}$ consta enteramente de puntos interiores de $D$,
	excepto $x_{0}$ mismo. Cualquier punto en este segmento que no sea $x_{0}$ se puede
	escribir en la forma $\lambda x_{0}$ para $0 \leq \lambda < 1$. Supongamos que
	$z \in B_{1-\lambda}(\lambda x_{0})$, y sea $y = (z - \lambda x_{0})/(1 - \lambda)$.
	En consecuencia $|y| < 1$, por lo que $y \in B_{1}(0) \subseteq D$. Como $y$ y
	$x_{0}$ están ambos en $D$ y $z = \lambda x_{0} + (1 - \lambda)y$, se sigue de
	la convexidad que $z \in D$. Así, la bola abierta $B_{1-\lambda}(x_{0})$ está contenida
	en $D$, lo que implica que $\lambda x_{0}$ es un punto interior.
	
	Ahora definamos una aplicación $f : \bd D \to \sphere^{n-1}$ por
	\[
	f(x) = \frac{x}{\|x\|}.
	\]
	
	Básicamente, $f(x)$ es el punto donde el segmento de línea que parte del
	origen hasta $x$ interseca la esfera unidad. Puesto que $f$ es la restricción de
	una aplicación continua, es continua, por lo que la discusión del párrafo anterior
	muestra que es biyectiva. Dado que $\bd D$ es compacto, la \autoref{lem:closed-map} nos garantiza que $f$ es cerrada. Por ser $f$ biyectiva, continua y cerrada, entonces $f$ es un homeomorfismo.
	
	Finalmente, definamos $F : \overline{B}_{1}(0) \to D$ de forma que
	\[
	F(x) =
	\begin{cases}
		\|x\| f^{-1}\left( \frac{x}{\|x\|}\right), & x \neq 0; \\
		0,                                     & x = 0.
	\end{cases}
	\]
	Entonces $F$ es continua lejos del origen porque $f^{-1}$ lo es, y también lo es
	en el origen pues la acotación de $f^{-1}$ implica que $F(x) \to 0$ conforme $x
	\to 0$. Geométricamente, $F$ lleva cada segmento de línea radial que conecta $0$
	con un punto $\omega \in \sphere^{n-1}$ linealmente sobre el segmento radial de
	$0$ al punto $f^{-1}(\omega) \in \bd D$. Por convexidad, $F$ toma sus valores en
	$D$. La aplicación $F$ es inyectiva, ya que puntos en rayos distintos van a rayos
	distintos, y cada segmento radial se lleva linealmente a su imagen. Es sobreyectiva
	porque cada punto $y \in D$ está en algún rayo desde $0$. Por la
	\autoref{lem:closed-map}, $F$ es un homeomorfismo.
\end{proof}

\begin{definicion}
	Sea $(X,\mathcal{E})$, donde $X$ es un espacio topológico Hausdorff y
	$\mathcal{E}$ una colección de celdas abiertas. Diremos entonces que que $(X,\mathcal{E}
	)$ es un \textbf{CW-complejo} si se cumple que:
	\begin{enumerate}[font=\bfseries]
		\item[(C)] Consideremos la bola unidad $B_1(0)$ de dimensión $p$. Para cada $p$-celda $e \in \mathcal{E}$, existe una aplicación
		continua $f_{e} : \overline{B}_1(0) \to X$ de forma que la restricción a $B_1(0)$ es
		homeomorfa a la celda $e$ y la restricción a la frontera $\bd \overline{B}_1(0)$ está contenida en una unión
		finita de celdas de dimensión menor a $p$. A dicha función la llamaremos \textbf{función
			característica}.
		
		\item[(W)] Un subconjunto $F$ de $X$ es cerrado si, y sólo si, $F \cap \overline{e}$ es cerrado en $\overline{e}$ con la topología inducida de $X$ para todo $e \in \mathcal{E}$.
	\end{enumerate}
	Normalmente denotaremos al CW-complejo $(X,\mathcal{E})$ simplemente por $X$.
\end{definicion}

\begin{figure}[h!]
	\centering
	\begin{tikzpicture}
		% Sphere with a dot (0-cell)
		\begin{scope}
			[shift={(0,0)}] \shade[ball color = gray!40, opacity = 0.4] (0,0) circle (1cm);
			\fill (0,1) circle (2pt); \node at (0,-1.5)
			{(a) $\sphere^{2}$ como unión de sus celdas.};
		\end{scope}
		
		% Sphere with a hole
		\begin{scope}
			[shift={(4,0)}] \shade[ball color = gray!40, opacity = 0.4] (0,0) circle (1cm);
			\fill[white] (0,1) circle (2pt); \draw (0,1) circle (2pt); \node at (0,-1.5)
			{(b) $1$-celda de $\sphere^{2}$.};
		\end{scope}
		
		% Dot aligned with the hole
		\begin{scope}
			[shift={(8,0)}] \fill (0,1) circle (2pt); \node at (0,-1.5) {(c) $0$-celda de $\sphere^{2}$.};
		\end{scope}
	\end{tikzpicture}
	\caption{Visualización de un CW-complejo para $\sphere^{2}$.}
\end{figure}

Una propiedad importante de los CW-complejos es que mantienen su estructura en
subconjuntos bajo ciertas condiciones razonables.

\begin{definicion}
	Sea $X$ un CW-complejo. Diremos que $Y \subseteq X$ es un \textbf{subcomplejo}
	de $X$ si es unión de celdas de $X$ de forma que si $Y$ contiene una celda,
	entonces también contiene su clausura.
\end{definicion}

\begin{teorema}
	Sea $X$ un CW-complejo y sea $Y$ un subcomplejo de $X$. Entonces $Y$ es
	cerrado en $X$ y, además, es un CW-complejo con la topología y la colección de
	celdas inducidas.
\end{teorema}
\begin{proof}
	Es claro que $Y$ es Hausdorff. Además, por definición tenemos que $Y$ es la unión
	disjunta de sus celdas. Sea $e \subseteq Y$ una celda abierta de $Y$. Como su clausura
	también está contenida en $Y$, entonces existe un número finito de celdas de
	$X$ con intersección no vacía con $\overline{e}$ que, a su vez, son celdas de
	$Y$. En consecuencia, la condición $(C)$ se cumple. Es más, cualquier
	aplicación característica $f_{e} : \to X$ de $X$ lo es también de $Y$ para
	cualquier celda $e \subseteq Y$.
	
	En cuanto a la condición $(W)$, supongamos que $S$ es un subconjunto de $Y$ tal
	que $S \cap \overline{e}$ es cerrado en $\overline{e}$ con la topología inducida de $Y$ para toda celda en $Y$.
	Sea ahora $e$ una celda de $X$ que no esté contenida en $Y$. Sabemos que $\overline
	{e}\backslash e$ está contenido en la unión de un número finito de celdas de $X$,
	de las cuales un subconjunto de ellas están contenidas en $Y$. Llamemos a dichas
	celdas $e_{1}, \ldots, e_{n}$. Por consiguiente, $\overline{e}_{1} \cup \cdots
	\cup \overline{e}_{n} \subseteq Y$ y además,
	\[
	S \cap \overline{e}= S \cap (\overline{e}_{1} \cup \cdots \cup \overline{e}_{n}
	) \cap \overline{e}= \left( (S \cap \overline{e}_{1}) \cup \cdots \cap (S \cap
	\overline{e}_{n}) \right) \cap \overline{e},
	\]
	luego $S \cap \overline{e}$ es cerrado en $\overline{e}$ con la topología inducida de $Y$. Es decir, $S$ es
	cerrado en $X$ y por tanto en $Y$. Finalmente, concluimos que $Y$ es cerrado
	en $X$ tomando $S = Y$.
\end{proof}

\begin{definicion}
	Sea $X$ un CW-complejo. Diremos que el subespacio $X^{(p)}$ de $X$ es el
	\textbf{$p$-esqueleto} de $X$ si es igual a la unión de todas las celdas de dimensión
	menor o igual que $p$. En particular, es un subcomplejo de dimensión $p$ de $X$.
\end{definicion}

\begin{teorema}
	\label{teo:cw-conexion}
	Sea $X$ un CW-complejo. Entonces las siguientes propiedades son equivalentes:
	\begin{enumerate}
		\item $X$ es conexo por arcos.
		
		\item $X$ es conexo.
		
		\item El $1$-esqueleto de $X$ es conexo.
		
		\item Algún $n$-esqueleto de $X$ es conexo para algún $n$.
	\end{enumerate}
\end{teorema}
\begin{proof}
	Obviamente, $(1) \Rightarrow (2)$ y $(3) \Rightarrow (4)$, por lo que basta con
	demostrar que $(2) \Rightarrow (3)$ y $(4) \Rightarrow (1)$.
	
	Para probar $(2) \Rightarrow (3)$ razonaremos por contrarrecíproco. Supongamos
	que $X^{(1)}= X'^{(1)}\cup X''^{(1)}$ es una unión no conexa del $1$-esqueleto
	de $X$. Veamos por inducción en $n$ que para cada $n > 1$, el $n$-esqueleto
	$X^{(n)}$ puede expresarse como unión no conexa
	$X^{(n)}= X'^{(n)}\cup X''^{(n)}$ tal que $X'^{(n)}\subseteq X'^{(n-1)}$ y
	$X''^{(n)}\subseteq X''^{(n-1)}$ para cada $n$. Supongamos $X^{(n-1)}= X'^{(n-1)}
	\cup X''^{(n-1)}$ es una unión no conexa de $X^{(n-1)}$ para algún $n > 1$.
	Para cada celda $n$-dimensional $e$, la restricción de su aplicación función característica
	$f_{e} \colon D^{n} \to X^{(n)}$ a $\partial D^{n}$ es continua en $X^{(n-1)}$.
	Dado que $\partial D^{n} \cong \sphere^{n-1}$ es conexo, su imagen debe estar contenida
	en uno de los conjuntos $X'^{(n)}$ o $X''^{(n)}$. Por lo tanto, $\overline{f_e(D)}$
	tiene una intersección no trivial con $X'^{(n)}$ o $X''^{(n)}$, pero no con
	ambos. Dividimos las $n$-celdas en dos colecciones disjuntas $\mathcal{E}'$ y $\mathcal{E}
	''$, según si sus clausuras intersecan $X'^{(n-1)}$ o $X''^{(n-1)}$, respectivamente,
	y definimos
	\[
	X'^{(n)}= X'^{(n-1)}\cup \left(\bigcup_{e \in \mathcal{E}'}\overline{f_e(e)}\right
	),\quad X''^{(n)}= X''^{(n-1)}\cup \left(\bigcup_{e \in \mathcal{E}''}\overline
	{f_e(e)}\right).
	\]
	Claramente, $X^{(n)}$ es la unión disjunta de $X'^{(n)}$ y $X''^{(n)}$, y
	ambos conjuntos son no vacíos debido a la hipótesis de inducción.
	
	Ahora, definamos $X' = \bigcup_{n} X'^{(n)}$ y $X'' = \bigcup_{n} X''^{(n)}$.
	Como antes, $X = X' \cup X''$, y ambos conjuntos son no vacíos. Por el mismo argumento
	que arriba, si $e$ es cualquier celda de $X$ de cualquier dimensión, su clausura
	debe estar contenida en uno de estos conjuntos. Así, $X'$ y $X''$ son ambos abiertos
	y cerrados en $X$, lo que implica que $X$ no es conexo.
	
	Para demostrar $(4) \Rightarrow (1)$, supongamos que $X$ es un CW-complejo cuyo
	$n$-esqueleto es conexo para algún $n \geq 0$. Mostremos por inducción en $k$
	que $X^{(k)}$ es conexo por arcos para cada $k \geq n$. Primero, necesitamos
	mostrar que $X^{(n)}$ en sí mismo es conexo por arcos. Si $n = 0$, entonces
	$X^{(n)}$ es discreto y conexo, así que es un conjunto unitario y por lo tanto
	conexo por arcos. En caso contrario, elijamos cualquier punto $x_{0} \in X^{(n)}$
	y consideremos $S_{n}$ la componente arcoconexa de $X^{(n)}$ que contiene a $x_{0}$.
	Para cada celda $e$ de $X^{(n)}$, notemos que $\overline{f_e(e)}$ es la imagen
	continua de un espacio conexo por arcos, así que es conexo por arcos. Por lo
	tanto, si $\overline{f_e(e)}$ tiene una intersección no trivial con la
	componente arcoconexa $S_{n}$, debe estar contenida en $S_{n}$. En consecuencia,
	$S_{n}$ es cerrado y abierto en $X^{(n)}$. Como estamos asumiendo que
	$X^{(n)}$ es conexo, entonces $S_{n} = X^{(n)}$.
	
	Ahora, supongamos que hemos demostrado que $X^{(k-1)}$ es conexo por arcos
	para algún $k > n$ y sea $S_{k}$ la componente arcoconexa de $X^{(k)}$ que
	contiene a $X^{(k-1)}$. Para cada $k$-celda $e$, su clausura
	$\overline{f_e(e)}$ es un subconjunto de $X^{(k)}$ conexo por arcos que
	tiene intersección no trivial con $X^{(k-1)}$ y, por lo tanto, está contenido
	en $S_{k}$. Se sigue que $X^{(k)}= S_{k}$, completando la inducción.
\end{proof}

\begin{lema}
	\label{lem:cw-cl-finite-subcomplex} Sea $X$ un CW-complejo. Entonces la
	clausura de cada celda está contenida en un subcomplejo finito.
\end{lema}
\begin{proof}
	Consideremos cualquier $n$-celda $e \in X$ y probemos el lema por inducción.
	Para el caso $n=0$, $\overline{e}= e$ es trivialmente un subcomplejo finito.
	Supongamos ahora el lema cierto para las celdas de dimensión menor o igual que
	$n$ y veámoslo para $n+1$. Por la condición $(C)$, $\overline{e}\backslash e$
	está contenido en la unión de un número finito de celdas de dimensión menor
	que $n+1$. Dichas celdas están contenidas en subcomplejo finitos por hipótesis
	de inducción. Sin embargo, la unión de dichos subcomplejos finitos con $e$ es de
	hecho un subcomplejo finito que contiene a $\overline{e}$.
\end{proof}

\begin{lema}
	Sea $X$ un CW-complejo. Un subconjunto de $X$ es discreto si, y sólo si, su
	intersección con cada celda es finita.
\end{lema}
\begin{proof}
	Sea $S$ un subconjunto discreto de $X$. Entonces, la intersección de la clausura
	de cada celda $e$ de $X$ con $S$ es un subconjunto discreto de un conjunto compacto,
	luego es finito. En consecuencia, $S \cap e$ también lo es.
	
	Para la otra implicación supongamos que $S$ es un subconjunto cuya
	intersección con cualquier celda es finita. Como la clausura de cada celda está
	contenida en un subcomplejo finito, entonces por hipótesis tenemos que
	$S \cap \overline{e}$ es finito para cada celda $e$ de $X$. Esto significa que
	$S \cap \overline{e}$ es cerrado en $\overline{e}$ y por la condición $(W)$,
	$S$ es cerrado en $X$. Sin embargo, este argumento podemos aplicarlo a
	cualquier subconjunto de $S$, luego todo subconjunto de $S$ es cerrado en $X$.
	Por lo tanto, la topología inducida en $S$ es discreta.
\end{proof}

\begin{teorema}
	Sea $X$ un CW-complejo. Un subconjunto de $X$ es compacto si, y sólo si, es
	cerrado en $X$ y está contenido en un subcomplejo finito.
\end{teorema}
\begin{proof}
	Todo subcomplejo finito de $X$ es compacto pues es unión finita de clausuras de
	celdas, las cuales son compactas. En consecuencia, si $K$ es un subconjunto
	cerrado de $X$ contenido en un subcomplejo finito, entonces es compacto.
	
	Supongamos ahora que $K \subseteq X$ es compacto. Si $K$ intersecara una
	cantidad infinita de celdas, podríamos tomar un punto de cada intersección de forma
	que tuviéramos un subconjunto infinito discreto de $K$, lo cual es imposible. Es
	decir, $K$ está contenido en la unión de un número finito de celdas y por el
	\autoref{lem:cw-cl-finite-subcomplex}, está contenido en un subcomplejo finito.
\end{proof}

\begin{corolario}
	Un CW-complejo es compacto si, y sólo si, es un complejo finito.
\end{corolario}

\begin{proposicion}
	Todo $p$-símplice es una celda cerrada de dimensión $p$.
\end{proposicion}
\begin{proof}
	Inmediato por la \autoref{prop:compact-convex-closed-cell}.
\end{proof}

Una vez discutidas algunas propiedades básicas de los CW-complejos, ya estamos
en condiciones de verificar que efectivamente los complejos simpliciales son CW-complejos.

\begin{proposicion}
	Si $K$ es un complejo simplicial finito, entonces el poliedro $|K|$ junto con
	la colección $\mathcal{E}$ de interiores de los símplices de $K$ forman un CW-complejo.
\end{proposicion}
\begin{proof}
	Supongamos que $K$ es un complejo simplicial finito en $\mathbb{R}^{N}$. La
	condición $(C)$ se obtiene de manera directa a partir de la
	\autoref{prop:compact-convex-closed-cell}.
	
	En cuanto a la propiedad $(W)$, consideremos $F$ como un subconjunto de $|K|$.
	Sea $\{x_{n}\}_{n \in \N}$ una sucesión que converge a $x$ en $|K|$ y sea $U$
	un entorno de $x$. Por la compacidad de $|K|$ y el hecho de que $\mathcal{E}$ es
	un recubrimiento por abiertos de $|K|$, podemos escoger un subrecubrimiento
	finito $e_{1}, \ldots, e_{k}$ tal que
	$x \in U \subseteq \overline{e_1}\cup \ldots \cup \overline{e_k}$.
	
	Fijemos $n_{0} \in \N$ tal que $x_{n_i}\in U$ para todo $n_{i} \geq n_{0}$. Como
	hay un número finito de $e_{j}$ e infinitos $x_{n_i}$, existe una parcial
	convergente $\{x_{n_i}\}_{i \in \mathbb{N}}$ que converge a $x$ contenido en
	algún $\overline{e}_{j}$ para cierto $j \in \{1, \ldots, k\}$. Esto muestra que
	$x \in \overline{e}_{j}$ y, puesto que $x_{n_i}\in F \cap \overline{e_j}$ para
	todo $n_{i} \geq n_{0}$, y $F \cap \overline{e_j}$ es cerrado en $\overline{e_j}$,
	concluimos que $x \in F \cap \overline{e_j}$.
\end{proof}

\section{Aplicaciones simpliciales}

Cuando trabajemos con complejos simpliciales, será interesante tener en cuenta cuándo
las transformaciones entre ellos pueden ser continuas o incluso homeomorfismos.

\begin{lema}
	\label{lem:app-simpl} Sean $K$ y $L$ dos complejos simpliciales y sea $f: K^{(0)}
	\rightarrow L^{(0)}$ una aplicación entre los conjuntos de vértices de $K$ y $L$.
	Supongamos que siempre que los vértices $v_{0}, \ldots, v_{n}$ de $K$ generen
	un símplice en $K$, los puntos $f(v_{0}), \ldots, f(v_{n})$ son vértices de un
	símplice de $L$. Entonces podemos extender $f$ a una aplicación continua
	$|f| :|K| \rightarrow |L|$ tal que
	\[
	x = \sum_{i=0}^{n}t_{i}v_{i}\quad \implies \quad |f|(x) = \sum_{i=0}^{n}t_{i}
	f (v_{i})
	\]
	Llamaremos a $|f|$ la \textbf{aplicación simplicial} (lineal) inducida por $f$.
\end{lema}
\begin{proof}
	Por hipótesis, los vértices $f(v_{0}), \ldots, f(v_{n})$ generan un símplice
	$\tau$ en $L$. Por ser $K$ un complejo simplicial, la suma de sus coeficientes
	$t_{i}$, con $i \in \{0, \ldots, n\}$, es igual a uno, luego $|f|(x) = \sum_{i=0}
	^{n}t_{i}f(v_{i})$ es un punto de $\tau$. Es decir, $|f|$ es una aplicación lineal
	del símplice $\sigma$ generado por $v_{0}, \ldots, v_{n}$ al símplice $\tau$ generado
	por $f(v_{0}), \ldots, f(v_{n})$. Por ser $|f| : \sigma \rightarrow \tau$
	lineal en un espacio de dimensión finita, entonces es continua.
	
	Ahora tan solo nos queda ver que $|f| :|K| \rightarrow |L|$ es continua. Bien,
	pues por ser $|f| : \sigma \rightarrow \tau$ continua, también lo es
	$|f| : \sigma \rightarrow |L|$. Finalmente por el \autoref{lem:cont_poly},
	$|f| :|K| \rightarrow |L|$ es continua.
\end{proof}

Consideremos las funciones de la forma de $f$ descrita en \ref{lem:app-simpl}. Para
cualquier complejo $K$, existe una aplicación identidad $\id_{K} \colon K \to K$
que corresponde a la aplicación identidad en los vértices. Dadas tres aplicaciones
$f \colon K \to L$, $g \colon L \to M$ y $h : M \to N$, la aplicación compuesta $h
\circ (g \circ f) = (h \circ g) \circ f$, pues es una composición de
aplicaciones de conjuntos que preserva símplices. Por lo tanto, existe una
categoría de complejos simpliciales y estas funciones que denotaremos por
$\Cat{Csim}$.

Por otro lado, veamos que el \autoref{lem:app-simpl} nos garantiza la existencia
de un funtor covariante entre esta categoría y los espacios topológicos.

\begin{proposicion}
	Existe un funtor covariante $|\cdot| : \Cat{CSim}\to \Cat{Top}$ de la categoría
	de aplicaciones simpliciales a la categoría de espacios topológicos.
\end{proposicion}
\begin{proof}
	Para cada complejo simplicial $K$, la identidad en $\Cat{CSim}$ es la función
	identidad $\id_{K} : K \to K$. La aplicación simplicial inducida $|\id_{K}| : |
	K| \to |K|$ es tal que
	\[
	|\id_{K}|\left(\sum_{i=0}^{n}t_{i} v_{i}\right) = \sum_{i=0}^{n}t_{i} i_{K}(v
	_{i}) = \sum_{i=0}^{n}t_{i} v_{i},
	\]
	lo cual es precisamente la identidad en el espacio topológico $|K|$. Esto
	muestra que $|\cdot|$ preserva las identidades.
	
	Sean ahora $f: K \to L$ y $g: L \to M$ dos morfismos en $\Cat{CSim}$. La composición
	en $\Cat{CSim}$ es $g \circ f: K \to M$, y necesitamos demostrar que
	$|(g \circ f)| = |g| \circ |f|$. Para cualquier punto $x = \sum_{i=0}^{n}t_{i}
	v_{i}$ en $|K|$,
	\[
	|(g \circ f)|(x) = \sum_{i=0}^{n}t_{i} (g \circ f)(v_{i}) = \sum_{i=0}^{n}t_{i}
	g(f(v_{i})).
	\]
	Por otro lado,
	\[
	(|g| \circ |f|)(x) = |g|\left(|f|\left(\sum_{i=0}^{n}t_{i} v_{i}\right)\right
	) = |g|\left(\sum_{i=0}^{n}t_{i} f(v_{i})\right) = \sum_{i=0}^{n}t_{i} g(f(v_{i}
	)).
	\]
	Ambas expresiones son iguales y por tanto, $|\cdot|$ preserva la composición
	de morfismos.
\end{proof}

Normalmente abusaremos de la notación de forma que escribiremos la aplicación simplicial inducida $|f| : |K| \to |L|$ simplemente por $f : |K| \to |L|$.

\begin{lema}
	\label{lem:homeo_complex} Supongamos que $f:K^{(0)}\rightarrow L^{(0)}$ es una
	aplicación biyectiva tal que los vértices $v_{0}, \ldots, v_{n}$ de $K$ generan
	un símplice de $K$ si, y sólo si, $f(v_{0}), \ldots, f(v_{n})$ generan un símplice
	de $L$. Entonces la aplicación simplicial inducida $g:|K| \rightarrow |L|$ es
	un homeomorfismo. Diremos entonces que $g$ es un \textbf{homeomorfismo
		simplicial} de $K$ con $L$.
\end{lema}
\begin{proof}
	Por hipótesis, cada símplice $\sigma \in K$ se identifica con otro símplice
	$\tau \in L$. Por tanto, debemos comprobar que la aplicación lineal
	$h: \tau \rightarrow \sigma$ inducida por la correspondencia de vértices
	$f^{-1}$ es la inversa de $g: \sigma \rightarrow \tau$. Si consideramos $x = \sum
	_{i=0}^{n}t_{i}v_{i}$, entonces por definición
	$g(x) = \sum_{i=0}^{n}t_{i}f(v_{i})$. Luego
	\[
	h(g(x)) = h(\sum_{i=0}^{n}t_{i}f(v_{i})) = \sum_{i=0}^{n}t_{i}f^{-1}(v_{i}) =
	\sum_{i=0}^{n}t_{i}v_{i}= x
	\]
\end{proof}

\section{Complejos simpliciales abstractos}

Si bien la definición actual de los complejos simpliciales puede llegar a ser de
gran utilidad, en la práctica muchas veces no es necesario usar las herramientas
que nos proporciona la geometría afín. Es por ello que vamos a introducir una
descripción puramente combinatoria de los complejos simpliciales que, aun siendo
más simple, nos serán de gran utilidad a la hora de trabajar con espacios
topológicos.

\begin{definicion}
	Un \textbf{complejo simplicial abstracto} (o simplemente complejo abstracto)
	es una colección $\mathcal{S}$ de conjuntos finitos no vacíos tal que si
	$A \in \mathcal{S}$, entonces para todo $B \subset A$ con $B$ no vacío,
	$B \in \mathcal{S}$. Además, diremos que el complejo abstracto es \textbf{finito}
	si dicha colección es finita.
\end{definicion}

Al elemento $A$ de $\mathcal{S}$ lo llamaremos \textbf{símplice} de
$A \in \mathcal{S}$. La \textbf{dimensión} de $A$ es una menos que el número de
elementos que le pertenecen. Todo subconjunto de $A$ lo llamaremos \textbf{cara}
de $A$. En cuanto a la \textbf{dimensión} de $\mathcal{S}$, diremos que es igual
al máximo de las dimensiones de sus elementos o en caso de no haberlo, diremos
que la dimensión de $\mathcal{S}$ es infinita. El \textbf{conjunto de vértices}
$V$ de $\mathcal{S}$ diremos que es la unión de elementos de $\mathcal{S}$ que
contienen un único punto. Llamaremos \textbf{subcomplejo} de $\mathcal{S}$ a
cualquier subcolección de $\mathcal{S}$ que sea un complejo simplicial abstracto
en sí.

Sean $V_{S}$, $V_{T}$ los conjuntos de vértices de los complejos abstractos $\mathcal{S}$,
$\mathcal{T}$ respectivamente. Dos complejos abstractos $\mathcal{S}$ y $\mathcal{T}$
diremos que son \textbf{isomorfos} si existe una aplicación biyectiva
$f: V_{S}\rightarrow V_{T}$ tal que $\{a_{0}, \ldots, a_{p}\} \in \mathcal{S}$
si, y sólo si, $\{f(a_{0}), \ldots, f(a_{p})\} \in \mathcal{T}$.

\begin{definicion}
	Sean $K$ un complejo simplicial y $V$ su conjunto de vértices. Sea
	$\mathcal{K}$ la colección de todos los subconjuntos
	$\{a_{0}, \ldots, a_{p}\} \subset V$ tales que los vértices
	$a_{0}, \ldots, a_{p}$ generan un símplice de $K$. Entonces llamaremos a la colección
	$\mathcal{K}$ el \textbf{esquema de vértices} de $K$.
\end{definicion}
\begin{definicion}
	Si el complejo simplicial abstracto $\mathcal{S}$ es isomorfo al esquema de vértices
	del complejo simplicial $K$, diremos que $K$ es una \textbf{realización
		geométrica} de $\mathcal{S}$.
\end{definicion}
\begin{proposicion}
	Sea $\mathcal{S}$ un complejo simplicial abstracto finito de dimensión $N$.
	Entonces existe una realización geométrica de $\mathcal{S}$ en $\R^{2N+1}$.
\end{proposicion}
\begin{proof}
	Consideremos un conjunto de puntos $p_{i}\in \R^{2N+1}$ de forma que sus componentes
	son potencias de su índice $i$. Veamos que cualquier conjunto de $2N+2$ de
	estos puntos es afínmente independiente. Es decir, que los vectores formados
	por las diferencias entre estos puntos son linealmente independientes.
	
	Para demostrarlo, consideremos un subconjunto de puntos $\{p_{j_k}: 1 \leq k \leq
	2N+2\}$ de esta forma y analicemos el determinante de la matriz formada por los
	vectores correspondientes,
	
	\[
	\begin{vmatrix}
		j_{2}- j_{1}               & j_{3}- j_{1}               & \cdots & j_{2n+2}- j_{1}               \\
		j_{2}^{2}- j_{1}^{2}       & j_{3}^{2}- j_{1}^{2}       & \cdots & j_{2n+2}^{2}- j_{1}^{2}       \\
		\vdots                     & \vdots                     & \ddots & \vdots                        \\
		j_{2}^{2n+1}- j_{1}^{2n+1} & j_{3}^{2n+1}- j_{1}^{2n+1} & \cdots & j_{2n+2}^{2n+1}- j_{1}^{2n+1} \\
	\end{vmatrix}
	.
	\]
	
	Simplificando mediante operaciones elementales de fila, este determinante se transforma
	en el determinante de Vandermonde, cuyo valor es conocido y se calcula como el
	producto de las diferencias entre los términos seleccionados,
	\[
	\prod_{1 \leq k < l \leq 2N+2}(j_{k}- j_{l}).
	\]
	Este resultado no es cero siempre que todos los $j_{k}$ sean distintos,
	asegurando así la independencia lineal.
	
	Respecto a la construcción del complejo simplicial, tomemos un símplice abstracto
	$A$ en $\mathcal{S}$ con vértices $\{v_{i_0}, v_{i_1}, \ldots, v_{i_m}\}$ y consideremos
	el símplice geométrico $\sigma_{A}= [p_{i_0}, p_{i_1}, \ldots, p_{i_m}]$ en $\mathbb{R}
	^{2N+1}$. Dado que $m+1 \leq 2N + 2$, el símplice $\sigma_{A}$ tiene dimensión
	$m$. Definimos $K$ como el conjunto que contiene todos los símplices
	$\sigma_{A}$ para cada $A \in \mathcal{S}$. Veamos que la intersección de dos símplices
	$\sigma_{A}$ y $\sigma_{B}$ en $K$ es igual a $\sigma_{A \cap B}$ con $A,B \in
	\mathcal{S}$. Consideremos $\tau$ como el símplice en $\mathbb{R}^{2N+1}$
	cuyos vértices son la unión de los vértices pertenecientes a $\sigma_{A}$ y a
	$\sigma_{B}$, lo cual es posible ya que la suma de sus dimensiones no supera $2
	N$. De esta manera, la intersección $\sigma_{A}\cap \sigma_{B}$ resulta ser la
	cara de $\tau$ determinada por los vértices que $\sigma_{A}$ y $\sigma_{B}$ comparten,
	es decir, aquellos asociados a $A \cap B$. Concluimos entonces que
	$\sigma_{A}\cap \sigma_{B}= \sigma_{A \cap B}$.
\end{proof}

Como consecuencia inmediata de la proposición anterior y del \autoref{lem:homeo_complex},
tenemos el siguiente corolario.

\begin{corolario}
	Las siguientes afirmaciones son ciertas:
	\begin{enumerate}[label=(\alph{*})]
		\item Todo complejo abstracto finito $\mathcal{S}$ es isomorfo al esquema de
		vértices de algún complejo simplicial $K$.
		
		\item Dos complejos simpliciales son afínmente isomorfos si, y sólo si, sus esquemas
		de vértices son isomorfos como complejos simpliciales abstractos.
	\end{enumerate}
\end{corolario}
%\begin{ejemplo}
%	Supongamos que queremos encontrar un complejo simplicial \(K\) que sea homeomorfo al cilindro \(\sphere^1 \times [0,1]\). Una forma de hacerlo sería definiendo \(K\) como una colección de 6 \(2-\)símplices y sus caras tal y como se puede apreciar en la FIGURA ??. Otra forma sería definir un complejo simplicial \(L\) cuyo espacio subyacente sea un rectángulo dotado de una serie de vértices
%\end{ejemplo}

\endinput
%--------------------------------------------------------------------
% FIN DEL CAPÍTULO.
%--------------------------------------------------------------------
% !TeX root = ../tfg.tex
% !TeX encoding = utf8

\chapter{Homología simplicial}
\label{chapter:homology}

Este capítulo se centra en la homología simplicial, una rama de estudio crucial
de la topología algebraica que utiliza complejos simpliciales para analizar y
comprender la estructura de espacios topológicos triangulables. Tras explorar los
fundamentos del álgebra homológica y la teoría de complejos simpliciales, ahora
profundizamos en las propiedades teóricas y aplicaciones prácticas de la
homología simplicial siguiendo los contenidos de \cite{rafael2003elementos}.

\section{Homología simplicial orientada}

Dado un símplice $\sigma$, podemos definir un orden sobre sus vértices. Dos
órdenes de $\sigma$ los consideraremos equivalentes si podemos pasar de uno a
otro con un número par de permutaciones. 
Además, en el caso donde $\sigma$ sea un $0$-símplice, claramente existe una única orientación.
Así, los ordenamientos posibles para
los vértices de $\sigma$ se pueden agrupar en dos clases de equivalencia
distintas, que definimos como las \textbf{orientaciones del símplice} $\sigma$.

\begin{definicion}
	Decimos que un símplice $\sigma = [a_{0}, a_{1}, \ldots, a_{p}]$ está \textbf{orientado}
	si se le ha asignado una de estas orientaciones. Utilizaremos $[a_{0}a_{1}\ldots
	a_{p}]$ para denotar la clase de equivalencia dada por la orientación $a_{0}< a
	_{1}< \cdots < a_{p}$ del símplice generado por los vértices $a_{0},a_{1},\ldots
	, a_{p}$.
\end{definicion}

Consideremos \(\Sigma_{p}\) el conjunto de todos los símplices de dimensión \(p\) de
un complejo simplicial geométrico \(K\) y el conjunto de sus clases de equivalencia por la relación de orientación. Para cada \(\sigma \in \Sigma_{p}\), definimos
\(\Sigma_{p}^{+}\) y \(\Sigma_{p}^{-}\) como los conjuntos que contienen, respectivamente,
un símplice orientado \(\sigma^{+}\) y el símplice con orientación opuesta
\(\sigma^{-}\). En lo que sigue, \(R\) siempre será un \textbf{anillo unitario conmutativo}, a menos que se indique de manera explícita lo contrario.

\begin{definicion}
	Sea \(K\) un complejo simplicial y sea \(R\) un anillo. Consideremos los conjuntos definidos anteriormente. Definimos el \textbf{\(R\)-módulo de las \(p\)-cadenas simpliciales orientadas}
	de \(K\), \(C_{p}(K;R)\), como el cociente del \(R\)-módulo libre generado por
	\(\Sigma_{p}^{+}\cup \Sigma_{p}^{-}\) sobre el submódulo generado por el
	conjunto \(\{\sigma^{+}+ \sigma^{-}: \sigma \in \Sigma_{p}\}\). Esto es,
	\[
	C_{p}(K;R) = \frac{R\langle \Sigma_{p}^{+}\cup \Sigma_{p}^{-}\rangle}{\langle
		\sigma^{+}+ \sigma^{-}: \sigma \in \Sigma_{p}\rangle}.
	\]
	Para \(p < 0\) o \(p > \dim(K)\), definimos \(C_{p}(K;R)\) como el \(R\)-módulo trivial.
\end{definicion}
El interés de definir el \(R\)-módulo de \(p\)-cadenas simpliciales orientadas radica
tanto en la identificación de los elementos que contiene como en las operaciones
algebraicas aplicables sobre ellos. Esta construcción nos permite manejar un
símplice orientado y su opuesto como opuestos algebraicos en un marco formal. Veámoslo.

Nuestro objetivo es demostrar que efectivamente
\[
\frac{R\langle \Sigma_{p}^{+}\cup \Sigma_{p}^{-}\rangle}{\langle \sigma^{+}+
	\sigma^{-}: \sigma \in \Sigma_{p}\rangle}\cong R \langle \tilde{\Sigma}_{p}\rangle
,
\]
donde \(\tilde{\Sigma}_{p}\) representa el conjunto de \(p\)-símplices en \(\Sigma_{p}\)
con una orientación arbitrariamente fija para cada uno.

Para ello, definamos la aplicación
\(f : \Sigma^{+}_{p}\cup \Sigma^{-}_{p}\to R \langle \tilde{\Sigma}_{p}\rangle\). Esta
aplicación asigna a cada símplice orientado \(\sigma^{+}\) en \(\Sigma_{p}^{+}\), un
representante \(\sigma\) en \(R \langle \tilde{\Sigma}_{p}\rangle\) con una
orientación fija elegida arbitrariamente, y a cada \(\sigma^{-}\) en \(\Sigma_{p}^{-}\),
le asigna \(-\sigma\) en \(R \langle \tilde{\Sigma}_{p}\rangle\), donde \(-\sigma\) refleja
el elemento opuesto de \(\sigma\).

La aplicación \(f\) respeta las relaciones de orientación al asignar a símplices con
orientaciones opuestas a elementos que son opuestos algebraicos en \(R \langle \tilde
{\Sigma}_{p}\rangle\). Por la \nameref{teo:univ-prop-free-mod}, esta aplicación
induce un homomorfismo
\(\tilde{f}: R\langle \Sigma_{p}^{+}\cup \Sigma_{p}^{-}\rangle \to R \langle \tilde
{\Sigma}_{p}\rangle\)
que resulta ser sobreyectivo, ya que cada elemento en \(R \langle \tilde{\Sigma}_{p}
\rangle\) tiene al menos una preimagen en \(R\langle \Sigma_{p}^{+}\cup \Sigma_{p}^{-}
\rangle\).

Por definición de \(f\), para cada elemento de la forma \(\sigma^{+}+ \sigma^{-}\)
en \(\langle \sigma^{+}+ \sigma^{-}: \sigma \in \Sigma_{p}\rangle\), tenemos que \(\tilde
{f}(\sigma^{+}+ \sigma^{-}) = f(\sigma^{+}) + f(\sigma^{-}) = \sigma - \sigma = 0\),
demostrando que todo el submódulo
\(\langle \sigma^{+}+ \sigma^{-}: \sigma \in \Sigma_{p}\rangle\) tiene imagen cero
por \(\tilde{f}\) y, por ende, está contenido en el núcleo de \(\tilde{f}\).

Además, si consideramos un elemento \(x\) en \(R\langle \Sigma_{p}^{+}\cup \Sigma_{p}
^{-}\rangle\) tal que \(\tilde{f}(x) = 0\), este elemento puede expresarse como una
combinación lineal de elementos en \(\Sigma_{p}^{+}\) y \(\Sigma_{p}^{-}\). La condición
\(\tilde{f}(x) = 0\) implica que la suma de las imágenes bajo \(f\) de los términos en
esta combinación lineal debe ser cero en \(R \langle \tilde{\Sigma}_{p}\rangle\).
Esto solo ocurre si para cada \(\sigma\), la suma total de los coeficientes correspondientes
a \(\sigma^{+}\) y \(\sigma^{-}\) es cero, lo que significa que cada término en \(x\)
que contribuye a esta suma cero debe ser de la forma \(\sigma^{+}+ \sigma^{-}\) o
un múltiplo de este, luego \(\tilde{f}(x) = 0\) implica que \(x \in \langle \sigma^{+}
+ \sigma^{-}: \sigma \in \Sigma_{p}\rangle\).

Por tanto, el núcleo de \(\tilde{f}\) coincide precisamente con
\(\langle \sigma^{+}+ \sigma^{-}: \sigma \in \Sigma_{p}\rangle\), y aplicando el \nameref{teo:first-iso},
concluimos que
\[
\frac{R\langle \Sigma_{p}^{+}\cup \Sigma_{p}^{-}\rangle}{\langle \sigma^{+}+
	\sigma^{-}: \sigma \in \Sigma_{p}\rangle}\cong R \langle \tilde{\Sigma}_{p}\rangle
,
\]
estableciendo la estructura algebraica deseada y completando la prueba.

\begin{observacion}
	En particular, la anterior construcción asigna a cada símplice orientado una
	cadena cuyo coeficiente del anillo es \(1\), \(0\) o \(-1\). A estas cadenas las
	llamaremos \textbf{\(p\)-cadenas elementales}. En ocasiones abusaremos de la
	notación para designar por \(\sigma\) a la cadena elemental respectiva del
	símplice orientado \(\sigma\).
\end{observacion}

\begin{definicion}
	Sea \(K\) un complejo simplicial y sean \(C_{p}(K;R), C_{p-1}(K;R)\) \(R\)-módulos de
	\(p\)-cadenas. Definimos el \textbf{operador borde de \(p\)-cadenas} como el homomorfismo
	\(\partial_{p}: C_{p}(K;R) \to C_{p-1}(K;R)\) tal que
	\[
	\partial_{p}(\sigma) = \partial_{p}([v_{0}, v_{1}, \ldots, v_{p}]) = \sum_{i=0}
	^{p}(-1)^{i}[v_{0}, \ldots, \hat{v}_{i}, \ldots, v_{p}] .
	\]
	donde \(\hat{v}_{i}\) denota el vértice a eliminar.
\end{definicion}
%Debemos verificar que \(\partial_p\) esté bien definido y que \(\partial_p(-\sigma) = -\partial_p \sigma\). Para este propósito, basta con mostrar que el lado derecho de (*) cambia de signo si intercambiamos dos vértices adyacentes en el arreglo \([v_0, \ldots, v_p]\). Así que comparemos las expresiones para
%\[
%\partial_p[v_0, \ldots, v_j, v_{j+1}, \ldots, v_p]
%\]
%y
%\[
%\partial_p[v_0, \ldots, v_{j+1}, v_j, \ldots, v_p].
%\]
%Para \(i \neq j, j+1\), los términos \(i\)-ésimos en estas dos expresiones difieren precisamente por un signo; los términos son idénticos excepto que \(v_j\) y \(v_{j+1}\) han sido intercambiados.
%
%¿Qué pasa con los términos \(i\)-ésimos para \(i = j\) y \(i = j + 1\)? En la primera expresión, uno tiene
%\[
%(-1)^j[\ldots, v_j, \hat{v}_{j}, v_{j+1}, \ldots] + (-1)^{j+1}[\ldots, v_j, v_{j+1}, \hat{v}_{j+1}, \ldots].
%\]
%En la segunda expresión, uno tiene
%\[
%(-1)^j[\ldots, v_{j+1}, \hat{v}_{j+1}, v_j, \ldots] + (-1)^{j+1}[\ldots, v_{j+1}, v_j, \hat{v}_j, \ldots].
%\]
%Comparando, se ve que estas dos expresiones difieren por un signo.
\begin{lema}
	El operador borde \(\partial_{p}: C_{p}(K;R) \to C_{p-1}(K;R)\) está bien definido.
	En particular, si \(\sigma^{+}\) y \(\sigma^{-}\) son las dos orientaciones del \(p\)-símplice
	\(\sigma\), tenemos que
	\[
	\partial_{p}(\sigma^{+}+\sigma^{-}) = 0
	\]
\end{lema}
\begin{proof}
	Probaremos que la suma de la imagen por el operador borde de \(\sigma^{+}= [v_{0}
	v_{1}\ldots v_{p}]\) y \(\sigma^{-}= [v_{1}v_{0}\ldots v_{p}]\) es igual a \(0\).
	Para ello, observamos que
	\begin{align*}
		\partial_{p}\sigma^{+} & = [v_{1}v_{2}\ldots] - [v_{0}v_{2}\ldots] + \sum_{i\ne0,1}(-1)^{i}[v_{0}v_{1}\ldots \hat{v}_{i}\ldots v_{p}], \\
		\partial_{p}\sigma^{-} & = [v_{0}v_{2}\ldots] - [v_{1}v_{2}\ldots] + \sum_{i\ne0,1}(-1)^{i}[v_{1}v_{0}\ldots \hat{v}_{i}\ldots v_{p}].
	\end{align*}
	Al sumar ambas expresiones, los dos primeros términos de
	\(\partial_{p}\sigma^{+}\) y \(\partial_{p}\sigma^{-}\) se cancelan entre sí. Como
	consecuencia de la definición de \(C_{p-1}(K;R)\), los términos restantes
	definen orientaciones opuestas del mismo símplice por lo que se cancelan y
	\(\partial_{p}(\sigma^{+}+\sigma^{-})=0\).
\end{proof}

\begin{lema}
	Sean \(\partial_{p}: C_{p+1}(K;R) \to C_{p}(K;R)\),
	\(\partial_{p}: C_{p}(K;R) \to C_{p-1}(K;R)\) operadores borde. Entonces \(\partial
	_{p}\circ \partial_{p+1}= 0\).
\end{lema}
\begin{proof}
	\begin{gather*}
		\partial_{p}\partial_{p+1}[v_{0}, \ldots, v_{p+1}] = \partial_{p}\left( \sum_{i=0}
		^{p+1}(-1)^{i}[v_{0}\ldots \hat{v}_{i}\ldots v_{p+1}] \right) \\ = \sum_{i=0}
		^{p+1}(-1)^{i}\left[ \sum_{j>i}^{p+1}(-1)^{j}[v_{0}\ldots, \hat{v}_{i}\ldots
		\hat{v}_{j}\ldots v_{p+1}] + \sum_{j=0}^{j<i}(-1)^{j}[v_{0}\ldots \hat{v}_{j}
		\ldots \hat{v}_{i}\ldots v_{p+1}] \right].
	\end{gather*}
	Es decir, el símplice
	\([v_{0}\ldots,\hat{v}_{k}\ldots,\hat{v}_{t}\ldots, v_{p+1}]\) aparece dos veces
	en la anterior expresión con signos opuestos, donde \(k,t \in \{0, \ldots, p+1\}\).
	Esto nos lleva a discutir los siguientes casos. Supongamos sin pérdida de generalidad
	que \(k < t\). En el primer caso, \(i = k < j = t\) donde el coeficiente es \((-1)^{k}
	(-1)^{t-1}\). En el segundo caso, \(i = t > j = k\) con coeficiente \((-1)^{t}(-1)^{k}\).
	Concluimos por tanto que todo símplice de la expresión se anula y al anularse
	sobre los generadores, \(\partial_{p-1}\partial_{p}\) es el homomorfismo nulo.
\end{proof}

\begin{definicion}
	El complejo de cadenas positivo \(C_{\bullet}(K;R) = \{C_{p}(K;R), \partial_{p}\}\)
	lo llamaremos \textbf{complejo de cadenas simpliciales} de \(K\). La homología de
	dicho complejo la notaremos por \(H_{p}(K;R)\) y lo llamaremos \textbf{\(p\)-ésimo
		\(R\)-módulo de homología} de \(K\).
\end{definicion}
Si \(R=\Z\), el módulo \(H_{p}(K;\Z)\) lo notaremos simplemente por \(H_{p}(K)\) y diremos
que es el \textbf{\(p\)-ésimo grupo de homología} de \(K\).

\begin{proposicion}
	\label{prop:aumento} Sea \(K\) un complejo simplicial no vacío. Entonces el
	complejo de cadenas positivo \(\{ C_{p}(K;R), \partial_{p}\}\) admite un aumento.
\end{proposicion}
\begin{proof}
	Sea \(\varepsilon: C_{0}(K;R) \to R\) el homomorfismo que extiende linealmente \(\varepsilon
	(v) = 1\) para todo vértice \(v \in K\). Veamos que
	\(\varepsilon \circ \partial_{1}: C_{1}(K;R) \to R\) es nulo. Tomando \([v_{0},v_{1}
	] \in C_{1}(K;R)\) obtenemos que \(\varepsilon (\partial_{1}[v_{0},v_{1}]) = \varepsilon
	(v_{1}- v_{0}) = 1-1 = 0\), como queríamos ver.
\end{proof}

\begin{definicion}
	Sea \(\widetilde{C}_{\bullet}(K;R)\) el complejo aumentado del complejo de cadenas
	simpliciales \(C_{\bullet}(K;R)\). Denominaremos \textbf{\(p\)-ésimo módulo de
		homología reducida} de \(K\) al módulo de homología \(H_{p}(\widetilde{C}_{\bullet}
	;R)\) y lo denotaremos por \(\widetilde{H}(K;R)\).
\end{definicion}

\begin{proposicion}
	\label{prop:simpl_app_hom} Sean \(K\) y \(L\) dos complejos simpliciales junto con
	una aplicación simplicial \(f: |K| \to |L|\). Esta aplicación induce un
	homomorfismo entre los complejos de cadenas, \(C(f)\), el cual se define
	extendiendo linealmente la función
	\[
	C(f)([v_{0}\ldots v_{p}]) =
	\begin{cases}
		[f(v_{0}) \ldots f(v_{p})] & \text{si los vértices son distintos entre sí}, \\
		0                          & \text{en caso contrario}.
	\end{cases}
	\]
	En particular, si \(f\) es la identidad, entonces \(C(f)\) es simplemente la
	identidad también. Además, si \(g: |L| \longrightarrow |M|\) es otra aplicación
	simplicial, se cumple que \(C(g \circ f) = C(g) \circ C(f)\).
\end{proposicion}
\begin{proof}
	Para demostrar esto, primero observamos que la definición de \(C(f)\) es
	independiente de la orientación de los símplices. Luego, verificamos la
	igualdad \(\partial_{p}\circ C(f) = C(f) \circ \partial_{p}\). Si no hay vértices
	repetidos, se tiene que:
	\begin{gather*}
		C(f) \partial_{p}([v_{0}\ldots v_{p}]) = C(f) \left( \sum_{i=0}^{p}(-1)^{i}[v
		_{0}\ldots \hat{v}_{i}\ldots v_{p}] \right) = \\ \sum_{i=0}^{p}(-1)^{i}[f(v_{0}
		) \ldots \widehat{f(v_i)}\ldots f(v_{p})] = \partial_{p}C(f)([v_{0}\ldots v_{p}
		]).
	\end{gather*}
	Si hay vértices repetidos, digamos \(f(v_{i}) = f(v_{j})\), entonces \(\partial_{p}
	C(f)([v_{0}\ldots v_{p}]) = 0\). Por otro lado,
	\[
	\sum_{i=0}^{p}(-1)^{i}C(f)([v_{0}\ldots \hat{v_i}\ldots v_{p}]) = 0
	\]
	debido a que \(C(f)([v_{0}\ldots \hat{v}_{k}\ldots v_{p}]) = 0\) para \(k \neq i,j\)
	y cuando \(i < j\),
	\[
	(-1)^{i}[f(v_{0}) \ldots \widehat{f(v_i)}\ldots f(v_{j}) \ldots f(v_{p})] + (
	-1)^{j}[f(v_{0}) \ldots f(v_{i}) \ldots \widehat{f(v_j)}\ldots f(v_{p})] = 0
	\]
	también se anula. Esto se debe a que si no hay más vértices repetidos, como
	\(f(v_{i}) = f(v_{j})\), el número de trasposiciones necesarias para cambiar de un
	símplice orientado al otro es \(j-i-1\), dado que \(f(v_{j})\) ocupa el lugar
	\(j-1\) en el primer símplice. La fórmula \(C(g \circ f)=C(g)C(f)\) se sigue directamente
	de la definición de \(C(f)\).
\end{proof}
\begin{observacion}
	El resultado anterior nos garantiza que \(C: \Cat{Csim}\to R \textbf{-}\Cat{Ch_\bullet}\)
	es un funtor covariante entre la categoría de complejos simpliciales y la categoría
	de complejos de cadenas.
\end{observacion}

\begin{definicion}
	\label{def:chain-map-ind} Sea \(f : |K| \to |L|\) una aplicación simplicial y sea
	\(C(f): C_{\bullet}(K;R) \to C_{\bullet}(L;R)\) una aplicación de cadenas definida
	como en la \autoref{prop:simpl_app_hom}. Llamaremos a \(C(f)\) la \textbf{aplicación
		de cadenas inducida por} \(f\) y la notaremos por \(f_{\#}\).
\end{definicion}

\begin{corolario}
	Toda aplicación simplicial inducida \(f: |K| \to |L|\) induce un homomorfismo de
	\(R\)-módulos
	\[
	H_{p}(f) : H_{p}(K;R) \to H_{p}(L;R)
	\]
	que notaremos por \(f_{*}\) y que cumple que si \(g: |L| \to |M|\) es otra aplicación
	simplicial, entonces \((g \circ f)_{*}= g_{*}\circ f_{*}\) e \(\id_{*}= \id\).
\end{corolario}
\begin{observacion}
	La última implicación del corolario se traduce en que tenemos un funtor covariante
	que va de la categoría de complejos simpliciales con los homeomorfismos simpliciales
	a la categoría de \(R\)-módulos con sus homomorfismos.
\end{observacion}

\begin{lema}
	La aplicación de cadenas \(f_{\#}: C_{\bullet}(K;R) \to C_{\bullet}(L;R)\) preserva
	el homomorfismo de aumento y como resultado, induce un homomorfismo \(f_{*}\) de
	módulos de homología reducida.
\end{lema}
\begin{proof}
	Sea \(f : |K| \to |L|\) una aplicación simplicial, \(f_{\#}\) su aplicación de
	cadenas inducida y sean
	\(\varepsilon : C_{0}(K;R) \to R,\ \varepsilon : C_{0}(L;R) \to R\) aumentos de
	\(C_{\bullet}(K;R), C_{\bullet}(L;R)\) respectivamente. Llamemos indistintamente
	\(\varepsilon\) a ambos aumentos en función del dominio en el que nos encontremos.
	Ahora definamos \(\varepsilon (f_{\#}(v)) = 1\) y \(\varepsilon(v) = 1\) para todo
	vértice de \(K\) y extendamos por linealidad. Por consiguiente \(\varepsilon \circ
	f_{\#}= \varepsilon\). Esta ecuación implica que \(f_{\#}\) lleva el núcleo de
	\(\varepsilon_{K}: C_{0}(K;R) \to R\) al núcleo de
	\(\varepsilon_{L}: C_{0}(L;R) \to R\), lo que induce un homomorfismo \(f_{*}: \widetilde
	{H}_{0}(K;R) \to \widetilde{H}_{0}(L;R)\).
\end{proof}
\begin{teorema}
	Sean \(f, g\) aplicaciones simpliciales de \(K\) a \(L\); \(f_{\#}, g_{\#}\) sus
	aplicaciones de cadenas inducidas y sea \(s: f_{\#}\to g_{\#}\) una homotopía de
	cadenas entre ellas. Entonces los homomorfismos inducidos \(f_{*}, g_{*}\) para sus
	módulos de homología son iguales.
\end{teorema}
\begin{proof}
	Sea \(z\) un \(p\)-ciclo de \(K\). Entonces
	\[
	g_{*}(z) - f_{*}(z) = \partial sz + s\partial z = \partial sz + 0
	\]
	por lo que \(f(z)\) y \(g(z)\) tienen la misma clase de homología. Por tanto,
	\(f_{*}([z]) = g_{*}([z])\) como se quería.
\end{proof}
%
%El siguiente teorema muestra una propiedad esencial de los módulos de homología: el módulo \(H_0(K;R)\) indica el número de componentes conexas de \(K\).
%\begin{teorema}
%	Dado un complejo simplicial \( K \), se tiene que \( H_0(K) \) es un grupo abeliano libre. Además, si \( \{v_\alpha\} \) es una colección de vértices con un elemento \( v_\alpha \) por cada componente conexa de \( |K| \), entonces las clases de homología de los \( v_\alpha \) forman una base de \( H_0(K) \).
%\end{teorema}
%\begin{proof}
%	Seguiremos en un principio la demostración propuesta en el [10, Teorema 7.1], dividiendo la prueba en varios pasos:
%
%	\textbf{PASO 1.} Comenzamos estableciendo una relación de equivalencia entre vértices \(u\) y \(v\) de \(K\), definida por la existencia de una secuencia finita de vértices \(u = a_0, a_1, \ldots, a_n = v\), donde cada par \((a_{i-1}, a_i)\) forma un 1-símplice en \(K\). Denotemos por \(C_v\) la unión de las estrellas de todos los vértices \(u\) que están relacionados con \(v\), esto es,
%	\[ C_v = \bigcup_{u \sim v} St(u), \]
%	donde \(St(u)\) es la unión de los interiores de todas las caras de \(K\) que contienen a \(u\). Esta construcción garantiza que cada \(C_v\) es abierto, ya que cada \(St(u)\) es un abierto en \(|K|\), y además cada \(C_v\) es conexo por caminos, lo cual se deriva de la posibilidad de trazar caminos a través de las secuencias de 1-símplices que definen la relación de equivalencia. Así, cada \(C_v\) corresponde a una componente conexa de \(|K|\).
%
%	Veamos que las \( C_v \) se corresponden con las componentes conexas de \( |K| \):
%
%	\begin{itemize}
	%		\item Los \( C_v \) son abiertos en \( K \) porque \( St(u) \) es abierto en \( |K| \).
	%		\item Cada \( C_v \) es conexo por caminos y por consiguiente conexo ([20, Teorema 2.7]).
	%	\end{itemize}
%
%	Dado un vértice \( u \), sea \( u \sim v \) y \( x \in St(u) \). Por definición de la relación de equivalencia, existen vértices \( u = a_0, a_1, \ldots, a_n = v \) que cumple la condición antes enunciada. Son precisamente los 1-símplices \( [a_0, a_1], \ldots, [a_{n-1}, a_n] \) los que garantizan la existencia de un camino entre \( u \) y \( v \). Por su parte, como camino entre \( u \) y \( z \) basta tomar la recta que une ambos puntos, la cual estará contenida en \( K \) debido al carácter afín de un símplice presentado en la Definición 1.2.
%
%	Si \( C_u \neq C_v \) entonces \( C_u \cap C_v \) será vacío. Supongamos que existe un \( x \in C_u \cap C_v \). Entonces
%	\[ x \in St(u) \cap St(u'), \]
%	con \( u \sim v \) y \( u' \sim v' \). Si \( u = u' \), entonces \( u \sim v \) es \( u' \sim v' \), lo cual resulta una contradicción. Por el contrario, si \( u \neq u' \), tenemos que existe una cara \( A \) con \( u \) como vértice que contiene a \( x \) en su interior y que existe una cara \( B \) con \( u' \) como vértice que contiene a \( x \) en su interior. Supongamos que la dimensión de \( A \) es menor o igual que la de \( B \). Entonces, como la intersección de dos caras es una cara y \( x \) pertenece al interior de \( A \), tenemos que \( A \subseteq B \), y por tanto el 1-símplice \( [u,u'] \) está en \( K \). En consecuencia, \( u \sim v \) o \( u' \sim v' \), lo cual nos lleva a una contradicción.
%
%	\textbf{PASO 2.} Sea \( w \) una colección de vértices de \( K \) con \( v_\alpha \in C_\alpha \). Sea \( w \) un vértice de \( K \). Se tendrá \( w \in C_\alpha \) para algún \( \alpha \). Por definición, existe una sucesión de vértices \( v_\alpha = a_0, a_1, \ldots, a_n = w \) tal que los 1-símplices \( [a_i, a_{i+1}] \) están en \( K \). Consideremos la 1-cadena
%	\[ c = [a_0, a_1] + [a_1, a_2] + \ldots + [a_{n-1}, a_n] . \]
%
%	Se tiene
%	\[ \partial c = a_n - a_0 = w - v_\alpha , \]
%	con lo cual la cadena \( w \) es homóloga a la cadena \( v_\alpha \). De este modo, cualquier 0-cadena en \( K \) es homóloga a una cadena de la forma \( \sum n_\alpha v_\alpha \).
%
%	Llegados a este punto, para ver que las clases de homología de los \( v_\alpha \) forman una base de \( H_0(K) \), tenemos que ver el siguiente y último paso de la demostración.
%
%	\textbf{PASO 3.} Veamos que una 0-cadena de la forma \( c = \sum n_\alpha v_\alpha \) es borde si y solo si \( n_\alpha = 0 \) para todo \( \alpha \).
%
%	Supongamos para ello que \( c = \partial d \). Podemos expresar la 2-cadena \( d \) como una sumatoria de 1-cadenas donde cada sumando tiene soporte en una componente conexa de \( K \). Es decir, \( d = \sum d_\alpha \), donde cada \( d_\alpha \) tiene soporte en \( C_\alpha \). De este modo, \( \partial d = \sum \partial d_\alpha \), donde \( \partial d_\alpha \) es una 0-cadena soportada por \( C_\alpha \). Por la definición de \( C_0(K) \) como módulo libre, esto implica que \( \partial d_\alpha = n_\alpha v_\alpha \) para todo \( \alpha \).
%
%	Veamos ahora que esto implica que \( n_\alpha = 0 \):
%
%	Definimos la aplicación
%	\[ \varepsilon : C_0(K) \rightarrow \mathbb{Z}, \quad c = \sum m_\alpha v_\alpha \mapsto \sum m_\alpha . \]
%	(Si construimos nuestras cadenas sobre otro módulo la definición de esta aplicación no cambia, solo variará el rango de la misma). En particular tenemos que
%	\[ \varepsilon(\partial[w, v]) = \varepsilon(w - v) = \varepsilon(w) - \varepsilon(v) = 0 , \]
%	de donde se deduce que \( \varepsilon(\partial d_\alpha) = 0 \), lo cual nos lleva a que
%	\[ n_\alpha = \varepsilon(n_\alpha v_\alpha) = \varepsilon(\partial d_\alpha) = 0 , \]
%	tal y como queríamos demostrar.
%\end{proof}

\section{Homología del complejo cono}
A continuación, exploraremos un nuevo complejo simplicial que construiremos a partir
de otro dado. El complejo cono nos facilitará la obtención de algunos resultados
relevantes en homología.
\begin{definicion}
	Sea \(K\) un complejo simplicial de \(\R^{N}\) y sea \(w \in \R^{N}\) tal que cada semirrecta
	con origen \(w\) corta a \(|K|\) a lo sumo en un punto. Definimos el \textbf{cono
		sobre \(K\) con vértice \(w\)} como el conjunto cuyos elementos son los símplices de
	\(K\) o símplices de la forma \([w,v_{0},\ldots,v_{p}]\), donde
	\([v_{0}, \ldots, v_{p}] \in K\). Lo denotaremos por \(w \ast K\).
\end{definicion}

\begin{figure}[h]
	\centering % This centers the figure
	\begin{tikzpicture}
		% Define the style for the nodes
		\tikzstyle{every node}
		=[circle, draw, fill=black, inner sep=0pt, minimum width=4pt]
		
		% Nodes definition with labels
		\node (a) at (-2, 0) [label=left:a] {}; \node (b) at (2, 0) [label=right:b] {};
		\node (w) at (0, 3.5) [label=above:w] {}; \node (c) at (2, 1.5) [label=below right:c]
		{}; \node (d) at (4, 1.5) [label=right:d] {};
		
		% Draw the gray-filled area
		\fill[gray] (a.center) -- (b.center) -- (c.center) -- cycle;
		
		% Draw the translucent light blue-filled areas
		\filldraw[fill=blue, opacity=0.1] (a.center) -- (w.center) -- (c.center) --
		cycle; \filldraw[fill=blue, opacity=0.1] (b.center) -- (w.center) -- (c.center)
		-- cycle; \filldraw[fill=blue, opacity=0.1] (a.center) -- (w.center) -- (b.center)
		-- cycle; \filldraw[fill=blue, opacity=0.1] (c.center) -- (d.center) -- (w.center)
		-- cycle;
		
		% Draw the tetrahedron edges
		\draw (a.center) -- (b.center); \draw (b.center) -- (w.center); \draw (w.center)
		-- (a.center); \draw (a.center) -- (c.center); \draw (b.center) -- (c.center);
		\draw (w.center) -- (c.center); \draw (c.center) -- (d.center); \draw (w.center) -- (d.center); 
	\end{tikzpicture}
	\caption{Cono sobre el complejo formado por el \(2\)-símplice \([a,b,c]\), el $1$-símplice $[c,d]$ y todas sus caras con vértice \(w\).}
\end{figure}

\begin{lema}
	El cono \(w \ast K\) es un complejo simplicial.
\end{lema}
\begin{proof}
	Sea \(\sigma = [v_{0},\ldots,v_{p}]\) un símplice de \(K\). Primero veamos que el
	conjunto \(\{w,v_{0},\ldots,v_{p}\}\) es afínmente independiente. Si \(w\) perteneciera
	al plano \(P\) generado por los puntos \(v_{0},\ldots,v_{p}\), podríamos
	considerar el segmento que une \(w\) con un punto de \(x \in \interior \sigma\). Dicho
	conjunto, por ser abierto en \(P\), contendría un intervalo de puntos en el segmento,
	contradiciendo la hipótesis de que las semirrectas que parten de \(w\) cortan a
	lo sumo en un punto a \(|K|\).
	
	Veamos ahora que \(w \ast K\) es un complejo simplicial. Los símplices de \(w \ast
	K\) pueden ser de tres tipos:
	\begin{enumerate}
		\item Símplices \([v_{0},\ldots, v_{p}]\) pertenecientes a \(K\).
		
		\item Símplices de la forma \([w,v_{0},\ldots, v_{p}]\).
		
		\item El \(0\)-símplice \([w]\).
	\end{enumerate}
	Si \(\sigma,\tau\) son símplices del primer tipo, entonces \(\interior \sigma \cap
	\interior \tau = \emptyset\) puesto que \(K\) es un complejo simplicial. El
	símplice \(\interior [w,v_{0},\ldots,v_{p}]\) es la unión de todos los segmentos
	abiertos que unen \(w\) con \(v_{0},\ldots, v_{p}\), luego dos símplices de esta forma
	tienen intersección vacía pues las semirrectas que parten de \(w\) cortan a \(K\) a
	lo sumo en un punto. Finalmente, si \(\sigma\) es del primer tipo y \(\tau\) del segundo,
	\(\interior \sigma \cap \interior \tau = \emptyset\) por el mismo argumento
	recién dado.
\end{proof}

\begin{proposicion}
	\label{prop:char-homol-cono} Sea \(K\) un complejo simplicial y sea \(w \ast K\) el
	cono sobre \(K\) de vértice \(w\). Entonces la homología orientada de \(w \ast K\)
	es \(H_{p}(w \ast K;R) = 0\) para todo \(p \neq 0\) y \(H_{0}(w \ast K;R) \cong R\).
	En el caso de la homología reducida, \(\widetilde{H}_{0}(w \ast K;R) = 0\) para
	todo \(p \in \Z\).
\end{proposicion}
\begin{proof}
	Sea \(D_{\bullet}= \{D_{p}, \partial_{p}\}\) un complejo de cadenas tal que \(D_{p}
	= 0\) para todo \(p \neq 0\) y \(D_{0}= R\). Definimos la aplicación de cadenas
	\(f: D_{\bullet}\to C_{\bullet}(w \ast K;R)\) de forma que \(f_{p}= 0\) para todo
	\(p \neq 0\) y \(f_{0}(r)=rw\). Por otro lado, por la \autoref{prop:aumento}
	podemos definir el aumento
	\(\varepsilon: C_{\bullet}(w \ast K;R) \to D_{\bullet}\) dado por
	\(\varepsilon_{p}=0\) para todo \(p \neq 0\) y \(\varepsilon_{0}(v) = 1\) para todo
	vértice \(v\) del cono. Nuestro objetivo es ver que efectivamente \(f\) es una equivalencia
	de cadenas junto a \(\varepsilon\). De manera directa tenemos que
	\(\varepsilon \circ f = \id_{D}\), luego \(\varepsilon \circ f \simeq \id_{D}\).
	Veamos ahora que \(f \circ \varepsilon\) es homotópica a la identidad. Para ello
	vamos a definir \(s\) como la familia \(\{s_{p}\}\) de homomorfismos \(s_{p}: C_{p}(
	w \ast K;R) \to C_{p+1}(w \ast K;R)\) tal que
	\[
	s_{p}([v_{0}\ldots v_{p}]) =
	\begin{cases}
		[wv_{0}\ldots v_{p}] \  & \text{si}\ v_{i}\neq w \quad 0 \leq i \leq p,\quad p \geq 0, \\
		0 \                     & \text{en caso contrario},
	\end{cases}
	\]
	induce una extensión lineal. Dicha familia está bien definida para
	\(C_{p}(w \ast K;R)\). Veamos que \(\partial_{p+1}s_{p}+ s_{p-1}\partial_{p}= \id_{C_p(w
		\ast K;R)}- f_{p}\varepsilon_{p}\) se cumple, por lo que \(s\) es una homotopía
	de cadenas. Para el caso en que \(p \in \Z\) es menor que \(0\), se cumple de manera
	trivial. Si \(p = 0\), distinguimos dos casos. Cuando \(v \neq w\) tenemos que \((\partial
	_{1}s_{0}+s_{-1}\partial_{0})(v) = \partial_{1}[w,v] = v-w = (\id_{0}- f_{0}\varepsilon
	_{0})(v)\). Por el contrario si \(v = w\), \((\partial_{1}s_{0}+s_{-1}\partial_{0})
	(v) = 0\) y también \((\id_{0}- f_{0}\varepsilon_{0})(v) = \id_{0}(w) - (f_{0}\varepsilon
	_{0})(w) = w - w = 0\). Por último, veamos que sucede cuando \(p > 0\).
	Supongamos primero que \(w \neq v_{i}\). Entonces
	\begin{gather*}
		(\partial_{p+1}s_{p}+ s_{p-1}\partial_{p})[v_{0}\ldots v_{p}] =\partial_{p+1}
		[wv_{0}\ldots v_{p}]+s_{p-1}\left(\sum_{i=0}^{p}(-1)^{i}[v_{0}\ldots\hat{v}_{i}
		\ldots v_{p}]\right) \\ =[v_{0}\ldots v_{p}]+\sum_{i=0}^{p}(-1)^{i+1}[wv_{0}\ldots
		\hat{v}_{i}\ldots v_{p}]+\sum_{i=0}^{p}(-1)^{i}[wv_{0}\ldots\hat{v}_{i}\ldots
		v_{p}] \\ =[v_{0}\ldots v_{p}]=(id_{C_{p}}-f_{p}\varepsilon_{p})[v_{0}\ldots
		v_{p}].
	\end{gather*}
	Finalmente si \(w = v_{i_0}\) para algún \(i_{0}\) entonces
	\begin{gather*}
		(\partial_{p+1}s_{p}+s_{p-1}\partial_{p})[v_{0}\ldots v_{p}]=s_{p-1}\partial_{p}
		[v_{0}\ldots v_{p}] =s_{p-1}\left( \sum_{i=0}^{p-1}(-1)^{i}[v_{0}\ldots \hat{v}
		_{i}\ldots v_{p}] \right) \\ =(-1)^{i_0}s_{p-1}[v_{0}\ldots \hat{v}_{i_0}\ldots
		v_{p}] =(-1)^{i_0}[wv_{0}\ldots \hat{v}_{i_0}\ldots v_{p}] \\ =(-1)^{i_0}[v_{i_0}
		v_{0}\ldots \hat{v}_{i_0}\ldots v_{p}] =[v_{0}\ldots v_{p}].
	\end{gather*}
	Es decir, \(f \circ \varepsilon \simeq \id_{C(w \ast K;R)}\) y por el
	\autoref{cor:equiv-homot} induce un isomorfismo
	\(\varepsilon_{*}: H_{p}(w \ast K;R) \to H_{p}(D;R)\).
	
	Para el caso reducido consideremos el complejo aumentado \(D_{\bullet}\) dado
	por el aumento \(\id_{R}: D_{0}\to R\). Como consecuencia, la homología de
	\(\widetilde{D}\) es trivial. Además, podemos extender los homomorfismos
	\(\varepsilon\) y \(f\) a homomorfismos \(\widetilde{\varepsilon}\) y
	\(\widetilde{f}\) para los complejos aumentados de forma que
	\(\widetilde{\varepsilon}_{-1}= \widetilde{f}_{-1}= \id_{R}\). Por la misma homotopía
	\(s\) obtenemos que \(\widetilde{\varepsilon}\) y \(\widetilde{f}\) son equivalencias
	homotópicas entre los complejos aumentados y por tanto,
	\(\widetilde{H}_{p}(w \ast K;R) = 0\) para todo \(p \in \Z\).
\end{proof}
\begin{corolario}
	\label{cor:cono-nulo} La homología simplicial reducida de cualquier símplice es
	nula.
\end{corolario}
\begin{corolario}
	Sea \(\sigma\) un \(n\)-símplice y sea \(\bd \sigma\) su borde. Entonces
	\(\widetilde{H}_{p}(\bd \sigma;R) = 0\) es trivial si \(p \neq n-1\) y
	\(\widetilde{H}_{n-1}(\bd \sigma; R) \cong R\). Además, para el caso no trivial,
	un generador es la clase de la cadena \(\partial(\sigma)\).
\end{corolario}
\begin{proof}
	Dado el símplice anterior, los complejos de cadenas aumentados de \(\sigma\) y
	su borde coinciden hasta dimensión \(p \leq n-1\). Por el \autoref{cor:cono-nulo}
	deducimos que \(\widetilde{H}_{p}(\bd \sigma; R) = 0\) para \(p \leq n-2\). Además,
	\(C_{p}(\bd \sigma; R) = 0\) para \(p \geq n\). Por lo tanto, \(\widetilde{H}_{n-1}(
	\bd\sigma;R)=\ker \partial_{n-1}\). Aquí, \(\partial_{n-1}\) representa el operador
	borde en ambos complejos aumentados (es decir, \(\partial_{0}=\varepsilon\) indica
	el aumento). Dado que el complejo aumentado de \(\sigma\) tiene homología trivial,
	entonces \(\ker \partial_{n-1}=\im, \partial_{n}\), y además \(\partial_{n}\) es inyectivo
	donde el operador borde \(\partial_{n}: C_{n}(\sigma; R) \rightarrow C_{n-1}(\sigma
	; R) = C_{n-1}(\bd \sigma; R)\) aparece en el complejo de \(\sigma\). Puesto que
	\(C_{n}(\sigma;R)\) es isomorfo a \(R\) generado por \(\sigma\), se sigue que \(\im \partial
	_{n}\), y por tanto \(\widetilde{H}_{n-1}(\bd\sigma;R)\), es isomorfo a \(R\) generado
	por \(\partial(\sigma)\).
\end{proof}

\section{Sucesión de Mayer-Vietoris}
Nombrada en honor a los matemáticos austriacos Walther Mayer y Leopold Vietoris,
la sucesión de Mayer-Vietoris es una herramienta esencial en la topología algebraica
y la teoría de homología. Esta sucesión permite analizar la homología de un
complejo simplicial a partir de la homología de sus subcomplejos, de manera análoga
a como el teorema de Seifert-van Kampen describe el grupo fundamental de un espacio
topológico a partir de subespacios abiertos y conexos por arcos.
\begin{lema}
	[Lema de la serpiente] \label{lem:zig-zag} Sean \(A_{\bullet}= \{A_{n},\partial_{A}
	\}, B_{\bullet}= \{B_{n},\partial_{A}\}\) y \(C_{\bullet}= \{C_{n},\partial_{C}\}\)
	complejos de cadenas y sean \(f,g\) aplicaciones de cadenas tales que la sucesión
	\[
	0 \to A_{\bullet}\overset{f}{\to}B_{\bullet}\overset{g}{\to}C_{\bullet}\to 0
	\]
	es exacta. Existe entonces una sucesión exacta de homología
%	\begin{equation}
%		\label{eq:long-exact-hom}\cdots \to H_{p}(A_{\bullet};R) \overset{f_*}{\to}H_{p}
%		(B_{\bullet};R) \overset{g_*}{\to}H_{p}(C_{\bullet};R) \overset{\partial_*}{\to}
%		H_{p-1}(A_{\bullet};R) \overset{f_*}{\to}H_{p-1}(B_{\bullet};R) \to \cdots
%	\end{equation}
	\begin{center}
	\begin{tikzpicture}[>=triangle 60]
		\matrix[matrix of math nodes,column sep={80pt,between origins},row sep={40pt,between origins},nodes={asymmetrical rectangle}] (s)
		{
			|[name=02]| \cdots &|[name=A]| H_{p}(A_{\bullet};R) &|[name=B]| H_{p}(B_{\bullet};R) &|[name=C]| H_{p}(C_{\bullet};R) \\
			%
			&|[name=A']| H_{p-1}(A_{\bullet};R) &|[name=B']| H_{p-1}(B_{\bullet};R) &|[name=C']| H_{p-1}(C_{\bullet};R)  &|[name=01]| \cdots \\
		};
		\draw[->]
		(A) edge node[auto] {\(f_*\)} (B)
		(B) edge node[auto] {\(g_*\)} (C)
		(02) edge (A)
		(C') edge (01)
		(A') edge node[auto] {\(f_*\)} (B')
		(B') edge node[auto] {\(g_*\)} (C')
		;
		\draw[->,gray,rounded corners] (C) -- ++(2,0) -- ++(0,-0.7) -- ($(B)!.5!(B')$) node[auto,text=black,pos=1.0]
		{\(\partial_*\)} -|
		($(A'.west)+(-.5,0)$) -- (A');
	\end{tikzpicture}
	\end{center}
	donde \(\partial_{*}\) es el operador borde inducido en \(B_{\bullet}\) que
	llamaremos \textbf{operador conector}.
\end{lema}
\begin{proof}
	Para realizar esta prueba usaremos una persecución de diagramas. Usaremos el
	siguiente diagrama como guía:
	\[
	\xymatrix{ 0 \ar@{->}[r] & A_{p+1} \ar@{->}[r]^{f} \ar@{->}[d]_{\partial_A} & B_{p+1} \ar@{->}[r]^{g} \ar@{->}[d]_{\partial_B} & C_{p+1} \ar@{->}[r] \ar@{->}[d]_{\partial_C} & 0 \\ 0 \ar@{->}[r] & A_p \ar@{->}[r]^{f} \ar@{->}[d]_{\partial_A} & B_p \ar@{->}[r]^{g} \ar@{->}[d]_{\partial_B} & C_p \ar@{->}[r] \ar@{->}[d]_{\partial_C} & 0 \\ 0 \ar@{->}[r] & A_{p-1} \ar@{->}[r]^{f} & B_{p-1} \ar@{->}[r]^{g} & C_{p-1} \ar@{->}[r] & 0 }
	\]
	\textit{Paso 1}. Para definir el operador conector \(\partial_{*}\), primero
	tenemos que comprobar que si tenemos un ciclo de \(C_{p}\), entonces podemos asignarle
	un único ciclo en \(A_{p-1}\). Por tanto, sea \(c_{p}\) un ciclo de \(C_{p}\) (esto
	es, \(c_{p}\in \ker \partial_{C}\)) y escojamos \(b_{p}\in B_{p}\) tal que
	\(g(b_{p}) = c_{p}\) (recordemos que \(g\) es sobreyectiva por ser la sucesión exacta
	corta). El elemento \(\partial_{B}b_{p}\) de \(B_{p-1}\) pertenece al núcleo de \(g\)
	pues \(g(\partial_{B}b_{p})=\partial_{C}g(b_{p})=\partial_{C}c_{p}=0\). Por
	tanto, existe un elemento \(a_{p-1}\in A_{p-1}\) tal que \(f(a_{p-1})=\partial_{B}
	b_{p}\), pues \(\ker g = \im f\). Tenemos que dicho elemento es único por ser \(f\)
	inyectiva. Además, \(a_{p-1}\) es un ciclo. Como
	\(f(\partial_{A}a_{p-1}) = \partial_{B}f(a_{p-1}) = \partial_{B}\partial_{B}b_{p}
	= 0\), entonces \(\partial_{A}a_{p-1}= 0\) por ser \(f\) inyectiva. Definimos
	\(\partial_{*}[c_{p}] = [a_{p-1}]\) donde los corchetes denotan la clase de
	homología.
	
	\textit{Paso 2}. Queremos probar ahora que \(\partial_{*}\) es un homomorfismo
	de módulos bien definido. Sean \(c_{p}, c_{p}'\) dos elementos del núcleo de \(\partial
	_{C}: C_{p}\to C_{p-1}\). Sean \(b_{p}, b_{p}'\) elementos de \(B_{p}\) tal que
	\(g(b_{p}) = c_{p}\) y \(g(b_{p}')=c_{p}'\). Escojamos ahora \(a_{p-1}\) y \(a_{p-1}'\)
	tal que \(f(a_{p-1}) = \partial_{B}b_{p}\) y \(f(a_{p-1}') = \partial_{B}b_{p}'\).
	
	Para probar que \(\partial_{*}\) está bien definido, veamos que no depende del
	\(b_{p}\) y \(c_{p}\) escogido. Supongamos que \(c_{p}\sim c_{p}'\) y veamos entonces
	que \(a_{p-1}\) y \(a_{p-1}'\) también lo son. Por tanto, supongamos que \(c_{p}- c_{p}
	' = \partial_{C}c_{p+1}\). Escogemos \(b_{p+1}\) tal que \(g(b_{p+1}) = c_{p+1}\). Esto
	implica que
	\[
	f(b_{p}- b_{p}' - \partial_{B}b_{p+1}) = c_{p}- c_{p}' - \partial_{C}g(b_{p+1}
	) = c_{p}- c_{p}' - \partial_{C}c_{p+1}= 0.
	\]
	En consecuencia, podemos tomar \(a_{p}\) tal que
	\(f(a_{p}) = b_{p}- b_{p}' - \partial_{B}b_{p+1}\) luego
	\[
	f(\partial_{A}a_{p}) = \partial_{B}f(a_{p}) = \partial_{B}(b_{p}- b_{p}') - 0
	= f(a_{p-1}- a_{p-1}').
	\]
	Por ser \(f\) inyectiva, \(\partial_{A}a_{p}= a_{p-1}- a_{p-1}'\), como buscábamos.
	
	Ya sabemos que \(\partial_{*}\) está bien definido, veamos que es un homomorfismo
	de módulos. Para ello basta fijarnos en que \(g(b_{p}+ b_{p}') = c_{p}+ c_{p}'\)
	y que \(f(a_{p-1}+ a_{p-1}') = \partial_{B}(b_{p}+ b_{p}')\). Por tanto \(\partial
	_{*}[ c_{p}+ c_{p}'] = [a_{p-1}+ a_{p-1}']\) por definición y en consecuencia,
	\(\partial_{*}[c_{p}+ c_{p}'] = \partial_{*}[c_{p}] + \partial_{*}[c_{p}']\). Ahora
	si \(\lambda \in R\), de manera análoga obtenemos que
	\(\lambda \partial_{*}[b_{p}] = \lambda [c_{p}] = [ \lambda c_{p}] = \partial_{*}
	[ \lambda b_{p}]\).
	
	\textit{Paso 3}. Probaremos la exactitud de \(H_{p}(B_{\bullet};R)\) por doble
	inclusión. Como \(g \circ f = 0\) tenemos que \(g_{*}\circ f_{*}= 0\). Esto
	implica que si \(\gamma \in \im f_{*}\), entonces \(g_{*}(\gamma) = 0\).
	
	Para probar la otra inclusión, consideremos \(\gamma = [b_{p}]\) y supongamos
	que \(g_{*}(\gamma) = 0\). Entonces \(g(b_{p}) = \partial_{C}c_{p+1}\) para algún \(c
	_{p+1}\in C_{p}\). Escojamos \(b_{p+1}\) de manera que \(g(b_{p+1}) = c_{p+1}\). Entonces
	\[
	g(b_{p}- \partial_{B}b_{p+1}) = g(b_{p}) - \partial_{C}g(b_{p+1}) = g(b_{p})
	- \partial_{C}c_{p+1}= 0,
	\]
	luego \(b_{p}- \partial_{B}b_{p+1}= f(a_{p})\) para algún \(a_{p}\). Ahora, \(a_{p}\)
	es un ciclo pues
	\[
	f(\partial_{A}a_{p}) = \partial_{B}f(a_{p}) = \partial_{B}b_{p}- 0 = 0
	\]
	y \(f\) es inyectiva. Es más, \(f_{*}[a_{p}] = [f(a_{p})] = [b_{p}- \partial_{B}b_{p+1}
	] = [b_{p}]\) y por tanto \([b_{p}] \in \im f_{*}\) como queríamos.
	
	\textit{Paso 4}. Probemos la exactitud en \(H_{p}(C_{\bullet};R)\). Sea
	\(\alpha = [c_{p}]\) un elemento de \(H_{p}(C_{\bullet};R)\). Escojamos \(b_{p}\) tal
	que \(g(b_{p}) = c_{p}\) y ahora tomemos \(a_{p-1}\) tal que \(f(a_{p-1}) = \partial
	_{B}b_{p}\). En consecuencia, \(\partial_{*}\alpha = [a_{p-1}]\) por definición.
	
	Procederemos de nuevo por doble inclusión. Consideremos primero que
	\(\alpha \in \im g_{*}\). Entonces \(\alpha = [g(b_{p})]\) donde \(b_{p}\) es un ciclo
	en \(B\). Esto implica que \(f(a_{p-1}) = 0\) de donde \(a_{p-1}= 0\) y por tanto
	\(\partial_{*}\alpha = 0\).
	
	Supongamos ahora que \(\partial_{*}\alpha = 0\). Entonces \(a_{p-1}= \partial_{A}a
	_{p}\) para algún \(a_{p}\). Deducimos entonces que \(b_{p}- f(a_{p})\) es un ciclo
	y que \(\alpha = g_{*}[b_{p}- f(a_{p})]\) luego \(\alpha \in \im g_{*}\). Realizando
	los cálculos obtenemos que
	\[
	\partial_{B}(b_{p}- f(a_{p})) = \partial_{B}(b_{p}) - \partial_{B}(f(a_{p}))
	= \partial_{B}(b_{p}) - f(a_{p-1}) = 0,
	\]
	\[
	g_{*}[b_{p}- f(a_{p})] = [g(b_{p}) - 0] = [c_{p}] = \alpha.
	\]
	
	\textit{Paso 5}. Finalmente obtengamos la exactitud para
	\(H_{p-1}(A_{\bullet};R)\). Si \(\beta \in \im \partial_{*}\), entonces
	\(\beta = [a_{p-1}]\) donde \(f(a_{p-1}) = \partial_{B}b_{p}\) para algún \(b_{p}\)
	por definición. En consecuencia,
	\[
	f_{*}(\beta) = [f(a_{p-1})] = [\partial_{B}b_{p}] = 0.
	\]
	
	Consideremos ahora el caso donde \(f_{*}(\beta) = 0\). Sea \(\beta = [a_{p-1}]\). Entonces
	\([f(a_{p-1})] = 0\) por lo que \(f(a_{p-1}) = \partial_{B}b_{p}\) para algún \(b_{p}\).
	Definimos \(c_{p}= g(b_{p})\). En consecuencia, \(c_{p}\) es un ciclo ya que
	\(\partial_{c}c_{p}= g(\partial_{B}b_{p}) = g(f(a_{p-1})) = 0\) y
	\(\beta = \partial_{*}[c_{p}]\) por definición. Esto es,
	\(\beta \in \im \partial_{*}\).
\end{proof}
\begin{definicion}
	En las condiciones del anterior lema, llamaremos a la sucesión obtenida
	\textbf{sucesión exacta larga de homología}.
\end{definicion}
Una consecuencia importante del resultado anterior es su naturalidad, un concepto
de gran interés en teoría de categorías.
\begin{teorema}
	Sean \(\alpha : A_\bullet \to A_\bullet'\), \(\beta : B_\bullet \to B_\bullet'\) y \(\gamma : C_\bullet \to C_\bullet'\) aplicaciones de cadenas. Consideremos el siguiente diagrama conmutativo
	\[
	\xymatrix{ 0 \ar@{->}[r] & A_{\bullet} \ar@{->}[r]^{f} \ar@{->}[d]^{\alpha} & B_{\bullet} \ar@{->}[r]^{g} \ar@{->}[d]^{\beta} & C_{\bullet} \ar@{->}[r] \ar@{->}[d]^{\gamma} & 0 \\ 0 \ar@{->}[r] & A'_{\bullet} \ar@{->}[r]^{f'} & B'_{\bullet} \ar@{->}[r]^{g'} & C'_{\bullet} \ar@{->}[r] & 0 }
	\]
	donde las sucesiones horizontales son sucesiones exactas de complejos de cadenas.
	Entonces el diagrama
	\[
	\xymatrix{ {} \ar@{->}[r] & H_p(A_{\bullet};R) \ar@{->}[r]^{f_*} \ar@{->}[d]^{\alpha_*} & H_p(B_{\bullet};R) \ar@{->}[r]^{g_*} \ar@{->}[d]^{\beta_*} & H_p(C_{\bullet};R) \ar@{->}[r]^{\partial_*} \ar@{->}[d]^{\gamma_*} & H_{p-1}(A_{\bullet};R) \ar@{->}[r] \ar@{->}[d]^{\alpha_*} & {} \\ {} \ar@{->}[r] & H_p(A'_{\bullet};R) \ar@{->}[r]^{f'_*} & H_p(B'_{\bullet};R) \ar@{->}[r]^{g'_*} & H_p(C'_{\bullet};R) \ar@{->}[r]^{\partial'_*} & H_{p-1}(A'_{\bullet};R) \ar@{->}[r] & }
	\]
	es conmutativo.
\end{teorema}
\begin{proof}
	Es claro que el diagrama
	\[
	\xymatrix{ H_p(A_{\bullet};R) \ar@{->}[r]^{f_*} \ar@{->}[d]^{\alpha_*} & H_p(B_{\bullet};R) \ar@{->}[r]^{g_*} \ar@{->}[d]^{\beta_*} & H_p(C_{\bullet};R) \ar@{->}[d]^{\gamma_*} \\ H_p(A'_{\bullet};R) \ar@{->}[r]^{f'_*} & H_p(B'_{\bullet};R) \ar@{->}[r]^{g'_*} & H_p(C'_{\bullet};R) }
	\]
	es conmutativo, pues los homomorfismos inducidos de las aplicaciones de
	cadenas conservan la conmutatividad. Por tanto, basta estudiar la
	conmutatividad en
	\[
	\xymatrix{ H_p(C_{\bullet};R) \ar@{->}[r]^{\partial_*} \ar@{->}[d]^{\gamma_*} & H_{p-1}(A_{\bullet};R) \ar@{->}[d]^{\alpha_*} \\ H_p(C'_{\bullet};R) \ar@{->}[r]^{\partial'_*} & H_{p-1}(A'_{\bullet};R)\ . }
	\]
	Sea \([a] \in H_{p}(A_{\bullet};R)\) y tomemos \(b_{p}\) de manera que
	\(g(b_{p})=c_{p}\). Además tomemos \(a_{p-1}\in A_{p}\) de forma que \(f(a_{p-1}) =
	\partial_{B}b_{p}\). En consecuencia, \(\partial'_{*}[c_{p}] = [a_{p-1}]\) por definición.
	Consideremos ahora \(c'_{p}= \gamma(c_{p})\). Nuestro objetivo es ver que \(\partial
	'_{*}[c'_{p}] = \alpha_{*}[a_{p-1}]\). Está claro que \(\beta(b_{p})\) es
	preimagen de \(c_{p}\) por \(g'\), pues \(g'\beta(b_{p}) = \gamma g(b_{p}) = \gamma(
	c_{p}) = e'_{p}\). Así mismo, \(\alpha(c_{p-1})\) lo es de \(\partial'_{D}\beta(b_{p}
	)\), pues
	\(f'\alpha(a_{p-1}) = \beta f (a_{p-1}) = \beta(\partial_{B}b_{p}) = \partial'_{D}
	\beta(b_{p})\). Esto es, \(\partial'_{*}[c_{p}] = [\alpha(a_{p-1})]\) por
	definición.
\end{proof}
\begin{proposicion}
	[Sucesión de Mayer-Vietoris] 
	\label{prop:mayer-vietoris}
	Sea \(K\) un complejo simplicial y sean \(K_{1},K_{2}\)
	subcomplejos de \(K\) tales que \(K = K_{1}\cup K_{2}\). Entonces existe una
	sucesión exacta
	\[
	\cdots \to H_{p}(K_{1}\cap K_{2};R) \overset{f}{\to}H_{p}(K_{1};R) \oplus H_{p}
	(K_{2};R) \overset{g}{\to}H_{p}(K;R) \to H_{p-1}(K_{1}\cap K_{2};R) \to \cdots
	\]
	tal que \(f(c) = ({i_1}_{\#}(c),-{i_2}_{\#}(c))\), \(g(d,e) ={j_1}_{\#}(d)+{j_2}_{\#}
	(e)\) donde \(i_{t}: K_{1}\cap K_{2}\to K_{t}\) y \(j_{t}: K_{t}\to K_{1}\cup K_{2}\)
	para \(t \in \{1,2\}\) son las respectivas inclusiones.
\end{proposicion}
\begin{proof}
	La demostración consiste en construir la sucesión exacta corta de complejos de
	cadena
	\[
	0 \to C_{\bullet}(K_{1}\cap K_{2};R) \overset{f}{\to}C_{\bullet}(K_{1};R) \oplus
	C_{\bullet}(K_{2};R) \overset{g}{\to}C_{\bullet}(K;R) \to 0
	\]
	y aplicar el \nameref{lem:zig-zag}.
	
	Para ello comencemos describiendo el complejo de cadenas
	\(C_{\bullet}(K_{1};R) \oplus C_{\bullet}(K_{2};R)\). Recordemos que la suma directa
	de un complejo de cadenas se definía como la suma directa de los \(R\)-módulos de
	dimensión \(p\) \(C_{p}(K_{1};R) \oplus C_{p}(K_{2};R)\), cuyo operador borde
	\(\partial'(d,e) = (\partial_{1}d, \partial_{2}e)\) donde
	\(\partial_{1}, \partial_{2}\) corresponden a los operadores borde de \(K_{1}\) y
	\(K_{2}\) respectivamente.
	
	Para comprobar la exactitud de la sucesión, comencemos estudiando la exactitud
	en los extremos de ésta. Es claro que \(f\) es inyectiva por ser una inclusión. En
	cuanto a la sobreyectividad de \(g\), tomemos \(d \in C_{p}(K;R)\) donde \(d\) sea
	la suma de símplices orientados. Notemos por \(d_{1}\) a los elementos de dicha suma
	provenientes de \(K_{1}\). Entonces \(d - d_{1}\in K_{2}\) y
	\(g(d_{1}, d-d_{1}) = d\).
	
	Para estudiar la exactitud en
	\(C_{\bullet}(K_{1};R) \oplus C_{\bullet}(K_{2};R)\), consideremos la inclusión \(k
	: K_{1}\cap K_{2}\to K\) y la respectiva inclusión de cadenas inducida \(k_{\#}:
	C_{\bullet}(K_{1}\cap K_{2};R) \to C_{\bullet}(K;R)\). Nótese que
	\(g(f(c)) = k_{\#}(c) - k_{\#}(c) = 0\). Sea ahora \(g(d,e) = 0\), entonces
	\(d = -e\) si las consideramos como cadenas de \(K\). Como \(d\) proviene de \(K_{1}\)
	y \(e\) de \(K_{2}\), ambas deben de provenir de \(K_{1}\cap K_{2}\) y en
	consecuencia, \((d,e) = (d,-d) = f(d)\), como queríamos.
	
	La homología de \(C_{\bullet}(K_{1};R) \oplus C_{\bullet}(K_{2};R)\) de
	dimensión \(p\), que notaremos por \(H_{p}(K_{1}\oplus K_{2};R)\), es entonces
	\[
	H_{p}(K_{1}\oplus K_{2};R) \cong H_{p}(K_{1};R) \oplus H_{p}(K_{2};R)
	\]
	por la \autoref{prop:hom-free-commute}. Finalmente aplicamos el \nameref{lem:zig-zag}
	y en consecuencia tenemos la sucesión deseada.
	
	Para obtener la sucesión de Mayer-Vietoris de homología reducida, reemplazaremos
	los complejos de cadenas anteriores por sus correspondientes complejos de cadenas
	aumentados. Consideremos para ello el siguiente diagrama
	\[
	\xymatrix{ 0 \ar@{->}[r] & C_0(K_1 \cap K_2;R) \ar@{->}[d]^{\varepsilon_{K_1 \cap K_2}} \ar@{->}[r] & C_0(K_1;R) \oplus C_0(K_2;R) \ar@{->}[r] \ar@{->}[d]^{\varepsilon_1 \oplus \varepsilon_2} & C_0(K;R) \ar@{->}[d]^{\varepsilon} \ar@{->}[r] & 0 \\ 0 \ar@{->}[r] & R \ar@{->}[r]_{\widetilde{f}} & R \oplus R \ar@{->}[r]_{\widetilde{g}} & R \ar@{->}[r] & 0 }
	\]
	La conmutatividad y la exactitud se mantienen en la parte inferior del
	diagrama si definimos \(\widetilde{f}(r) = (r,r)\) y
	\(\widetilde{g}(r',r) = r'+r\). Las aplicaciones \(\varepsilon_{K_1 \cap K_2}, \varepsilon
	_{1}\oplus \varepsilon_{2}\) y \(\varepsilon\) son sobreyectivas pues la intersección
	de \(K_{1}\) y \(K_{2}\) es no vacía. De este modo, la homología de sus respectivos
	complejos de cadenas es nula en dimensión \(-1\) y en dimensión \(0\) es igual a la
	de sus respectivos módulos de homología reducida \(\widetilde{H}_{0}(K_{1}\cap K
	_{2};R)\), \(\widetilde{H}_{0}(K_{1};R) \oplus \widetilde{H}_{0}(K_{2};R)\) y
	\(\widetilde{H}_{0}(K;R)\). Para finalizar, aplicamos de nuevo el \nameref{lem:zig-zag}.
\end{proof}

\section{Conexión y el módulo de homología \(H_0(K;R)\)}

Uno de los resultados más destacados en la teoría de homología simplicial es su capacidad para identificar y clasificar las componentes conexas de un complejo simplicial. Utilizando el módulo de homología de dimensión cero, \( H_0 \), veremos que es posible determinar directamente el número de componentes conexas en el complejo.

\begin{proposicion}
	Sea \( K \) un complejo simplicial. Entonces \( K \) se puede partir en subcomplejos disjuntos \( K_1, K_2, \ldots, K_s \) cuyos poliedros son las componentes conexas del poliedro \( |K| \).
\end{proposicion}
\begin{proof}
	Consideremos las componentes conexas \( X_1, X_2, \ldots, X_s \) del politopo de \( K \). Para cada \( j \), consideremos \( K_j \) como la colección de todos los símplices \( \sigma \) de \( K \) tales que \( \sigma \subset X_j \). Si un símplice pertenece a \( K_j \) para algún \( j \), entonces todas sus caras también pertenecen a \( K_j \). Por lo tanto, \( K_1, K_2, \ldots, K_s \) son subcomplejos de \( K \). Estos subcomplejos son disjuntos entre sí, debido a que las componentes conexas \( X_1, X_2, \ldots, X_s \) del poliedro \( |K| \) son disjuntos. Además, si \( \sigma \in K \) entonces \( \sigma \subset X_j \) para algún \( j \), ya que \( \sigma \) es un subconjunto conexo del espacio topológico \( |K| \), y todo subconjunto conexo de un espacio topológico se encuentra contenido en alguna componente conexa. Por consiguiente, \( \sigma \) pertenece a \( K_j \). En consecuencia, \( K = K_1 \cup K_2 \cup \ldots \cup K_s \) y \( |K| = |K_1| \cup |K_2| \cup \ldots \cup |K_s| \).
\end{proof}

\begin{definicion}
	Sea \(K\) un complejo simplicial y sean \(v,w\) dos vértices de \(K\). Diremos que \(v,w\) pueden unirse por un \textbf{camino de aristas} si existen vértices \(v_0, \ldots, v_k\) en \(K\) de forma que \(v_0 = v\), \(v_k = w\) y \([v_i, v_{i+1}]\) es un \(1\)-símplice para todo \(i \in \{0, \ldots, k-1\}\).
\end{definicion}

\begin{lema}
	\label{lem:edge-path-conexion}
	El poliedro \(|K|\) de un complejo simplicial \(K\) es un espacio topológico conexo si, y sólo si, cualesquiera dos vértices de \(K\) pueden ser unidos por un camino de aristas.
\end{lema}
\begin{proof}
	Consideremos un par de vértices cualesquiera del camino de aristas \(v_{i_0}, v_{j_0}\) de \(K\). Claramente si \(i_0 = j_0\) entonces es trivialmente conexo. Supongamos entonces sin pérdida de generalidad \(i_0 < j_0\). Entonces podemos definir una aplicación lineal y continua \(\alpha_i : \left[\frac{i-i_0}{j_0-i_0}, \frac{i+1-i_0}{j_0-i_0}\right] \to [v_i, v_{i+1}]\) tal que 
	\[
	\alpha_i(\lambda) = (1-\lambda)v_{i_0} + \lambda v_{j_0+1}
	\]
	 para todo \(i \in \{ i_0, \ldots, j_0-1\}\). Por tanto, la función \(\alpha : [0,1] \to [v_{i_0} v_{j_0}]\) tal que \(\alpha(\lambda) = \alpha_i(\lambda)\) si \(\lambda \in [i, i+1]\) es un arco que conecta ambos vértices. En consecuencia, \([v,w]\) es arco conexo. Por ser cada símplice convexo, y por tanto arco conexo, \(|K|\) es arco conexo. Concluimos aplicando el \autoref{teo:cw-conexion}.
\end{proof}

\begin{teorema}
	\label{teo:comp-conex-iso-R}
	Sea \( K \) un complejo simplicial y \( R \) un anillo unitario conmutativo. Supongamos que el poliedro \( |K| \) de \( K \) es conexo. Entonces \( H_0(K; R) \cong R \).
\end{teorema}
\begin{proof}
	Consideremos el complejo de cadenas aumentado \( \widetilde{C}(K; R) \) y su respectiva homología reducida \( \widetilde{H}(K; R) \). Es claro que el submódulo de bordes del complejo aumentado \( B_0(K; R) \) está contenido en \( \ker \widetilde{\partial}_0 \), dado que \( \widetilde{\partial}_0 \circ \widetilde{\partial}_1 = 0 \).
	
	Para la otra inclusión, consideremos \( w_0, w_1, \ldots, w_m \) vértices de \( K \) que determinan un camino de aristas. Cada \( w_j - w_{j-1} \) es una arista de \( K \) para \( j = 1, 2, \ldots, m \), y se sigue que:
	\[
	[w_m] - [w_0] = \sum_{j=1}^m ([w_j] - [w_{j-1}]) = \widetilde{\partial}_1 \left( \sum_{j=1}^m [w_j, w_{j-1}] \right) \in B_0(K; R).
	\]
	Dado que \( |K| \) es conexo, por el \autoref{lem:edge-path-conexion} sabemos que cualquier par de vértices de \( K \) puede ser unido por un camino de aristas. Por lo tanto, \( v - u \in B_0(K; R) \) para cualquier par de vértices \( u \) y \( v \) de \( K \).
	
	Escojamos un vértice \( u \in K \). Entonces, para cualquier conjunto de coeficientes \( r_1, r_2, \ldots, r_s \in R \) y vértices \( v_1, v_2, \ldots, v_s \) de \( K \), tenemos que
	\[
	\sum_{j=1}^s r_j[v_j] = \sum_{j=1}^s r_j([v_j] - [u]) + \left( \sum_{j=1}^s r_j \right) [u],
	\]
	y, por lo tanto,
	\[
	\sum_{j=1}^s r_j([v_j] - [u]) \in B_0(K; R).
	\]
	En consecuencia,
	\[
	z - \widetilde{\partial}_0([u]) \in B_0(K; R)
	\]
	para todo \( z \in \widetilde{C}_0(K; R) \). Esto muestra que \( \ker \widetilde{\partial}_0 \subseteq B_0(K; R) \).
	Finalmente, el homomorfismo \( \widetilde{\partial_0} : \widetilde{C}_0(K; R) \rightarrow R \) es sobreyectivo y su núcleo es \( B_0(K; R) \). Además, sabemos que \(Z_0(K; R) = C_0(K; R)\), pues \( \widetilde{\partial_0} \) es el homomorfismo nulo. Entonces
	\[
	H_0(K; R) =\frac{ Z_0(K; R)}{B_0(K; R)} = \frac{C_0(K; R)}{B_0(K; R)}.
	\]
	Por el \nameref{teo:first-iso}, el homomorfismo \( \widetilde{\partial_0} \) induce un isomorfismo de \( H_0(K; R) \) a \( R \), y por lo tanto \( H_0(K; R) \cong R \), como se requería.
\end{proof}

\begin{corolario}
	Sea \(K\) un complejo simplicial y sea \(R\) un anillo unitario conmutativo. Entonces \(H_0(K; R) \cong R^s\), donde \(s\) es el número de componentes conexas del poliedro \(|K|\).
\end{corolario}
\begin{proof}
	Procederemos por inducción sobre el número de componentes conexas de \(|K|\). Si \(|K|\) es conexo, entonces el resultado se sigue del \autoref{teo:comp-conex-iso-R}. Supongamos ahora que podemos descomponer \(K\) en subcomplejos \(K_1, \ldots, K_{s}\) disjuntos dos a dos. Por la \nameref{prop:mayer-vietoris}, tenemos que la sucesión
	\begin{align}
	\cdots \to H_{0}\left(K_{1}\cap K \backslash K_1;R\right) &\to  H_{0}(K_{1};R) \oplus H_0\left( K \backslash K_1;R\right) \to \\ &\to H_{0}(K;R) \to H_{-1}\left(K_{1}\cap K \backslash K_1;R\right) \to \cdots
	\end{align}
	es exacta, donde \(K \backslash K_1 = \bigcup_{i=2}^{n+1} K_{i}\). Sin embargo, la intersección \(K_{1}\cap K \backslash K_1\) es vacía luego su módulo de homología de dimensión \(0\) y \(-1\) es el trivial. Por hipótesis de inducción, \( H_0(K_1) \oplus H_0(K \backslash K_1;R) \cong R \oplus R^{s-1} = R^s\). Finalmente, por ser la secuencia exacta en \(H_0(K;R)\) y ser \(H_{-1}\left(K_{1}\cap K \backslash K_1;R\right)\) el módulo trivial, el núcleo del operador conector es todo \(H_0(K;R)\) y por tanto, \(H_0(K;R) \cong H_{0}(K_{1};R) \oplus H_0\left( K \backslash K_1;R\right)\).
\end{proof}

%\section{Computabilidad de la homología}
%
%La Forma Normal de Smith (SNF, por sus siglas en inglés) de una matriz y su algoritmo de computación se pueden describir formalmente de la siguiente manera:
%
%\textbf{Definición}
%
%Sea \(A\) una matriz \(m \times n\) no nula sobre un dominio de ideales principales \(R\). Existen matrices invertibles \(m \times m\) y \(n \times n\), \(S\) y \(T\) respectivamente (con coeficientes en \(R\)), tales que el producto \(SAT\) es de la forma:
%
%\[
%SAT = \begin{pmatrix}
	%	\alpha_1 & 0 & 0 & \cdots & 0 & \cdots & 0 \\
	%	0 & \alpha_2 & 0 & & & & \\
	%	0 & 0 & \ddots & & \vdots & & \vdots\\
	%	\vdots & & & \alpha_r & & & \\
	%	0 & & \cdots & & 0 & \cdots & 0 \\
	%	\vdots & & & & \vdots & & \vdots \\
	%	0 & & \cdots & & 0 & \cdots & 0
	%\end{pmatrix}
	%\]
	%
	%y los elementos diagonales \(\alpha_i\) satisfacen \(\alpha_i \mid \alpha_{i+1}\) para todos \(1 \leq i < r\). Esta es la Forma Normal de Smith de la matriz \(A\). Los elementos \(\alpha_i\) son únicos hasta la multiplicación por una unidad y se denominan "divisores elementales", "invariantes" o "factores invariantes". Se pueden calcular (hasta la multiplicación por una unidad) como:
	%
	%\[
	%\alpha_i = \frac{d_i(A)}{d_{i-1}(A)},
	%\]
	%
	%donde \(d_i(A)\) (llamado el \(i\)-ésimo "divisor determinante") es igual al máximo común divisor de los determinantes de todos los menores de \(i \times i\) de la matriz \(A\) y \(d_0(A) := 1\).
	%
	%\textbf{Ejemplo}
	%
	%Para una matriz de \(2 \times 2\), se tiene que:
	%
	%\[
	%\text{SNF} \begin{pmatrix} a & b \\ c & d \end{pmatrix} = \text{diag}(d_1, \frac{d_2}{d_1})
	%\]
	%
	%con \(d_1 = \gcd(a, b, c, d)\) y \(d_2 = ad - bc\).
	%
	%\textbf{Algoritmo}
	%
	%El primer objetivo es encontrar matrices cuadradas invertibles \(S\) y \(T\) tal que el producto \(SAT\) sea diagonal. Este es la parte más difícil del algoritmo. Una vez se logra la diagonalidad, se facilita la conversión de la matriz a la Forma Normal de Smith. Expresado de manera más abstracta, el objetivo es demostrar que, considerando \(A\) como una aplicación de \(R^n\) (el módulo \(R\)-libre de rango \(n\)) a \(R^m\) (el módulo \(R\)-libre de rango \(m\)), existen isomorfismos \(S: R^m \to R^m\) y \(T: R^n \to R^n\) tales que \(S \cdot A \cdot T\) tiene la forma simple de una matriz diagonal. Las matrices \(S\) y \(T\) se pueden encontrar comenzando con matrices identidad del tamaño apropiado y modificándolas cada vez que se realiza una operación de fila sobre \(A\) en el algoritmo por la operación de columna correspondiente (por ejemplo, si la fila \(i\) se suma a la fila \(j\) de \(A\), entonces la columna \(j\) debe ser restada de la columna \(i\) de \(S\) para mantener el producto invariante), y modificando \(T\) de manera similar para cada operación de columna realizada. Dado que las operaciones de fila son multiplicaciones a la izquierda y las operaciones de columna son multiplicaciones a la derecha, esto preserva el invariante \(A' = S' \cdot A \cdot T'\) donde \(A'\), \(S'\), \(T'\) denotan los valores actuales y \(A\) denota la matriz original; eventualmente, las matrices en este invariante se convierten en diagonales. Solo se realizan operaciones de fila y columna invertibles, lo que asegura que \(S\) y \(T\) permanezcan como matrices invertibles.
	%
	%Para \(a \in R\setminus \{0\}\), escribimos \(\delta(a)\) para el número de factores primos de \(a\) (estos existen y son únicos ya que cualquier PID también es un dominio de factorización única). En particular, \(R\) también es un dominio de Bézout, por lo que es un dominio de gcd y el gcd de cualquier dos elementos satisface una identidad de Bézout.
	%
	%Para poner una matriz en la Forma Normal de Smith, se puede aplicar repetidamente el siguiente procedimiento, donde \(t\) varía de 1 a \(m\).
	%
	%\textbf{Paso I: Elegir un pivote}
	%
	%Sea \(A\) una matriz \(m \times n\) con coeficientes en \(R\). Para un paso dado \(t\), elegimos \(j_t\) como el índice de la columna más pequeña de \(A\) con una entrada no nula, iniciando la búsqueda en el índice de columna \(j_{t-1} + 1\) si \(t > 1\).
	%
	%Deseamos que \(a_{t,j_t} \neq 0\). Si este es el caso, este paso está completo. De lo contrario, por suposición, existe algún \(k\) tal que \(a_{k,j_t} \neq 0\), y podemos intercambiar las filas \(t\) y \(k\), obteniendo así \(a_{t,j_t} \neq 0\).
	%
	%Nuestro pivote elegido ahora está en la posición \((t, j_t)\).
	%
	%\textbf{Paso II: Mejorar el pivote}
	%
	%Si existe una entrada en la posición \((k, j_t)\) tal que \(a_{t,j_t} \nmid a_{k,j_t}\), entonces, dejando \(\beta = \gcd(a_{t,j_t}, a_{k,j_t})\), sabemos por la propiedad de Bézout que existen \(\sigma, \tau \in R\) tales que
	%\[
	%a_{t,j_t} \cdot \sigma + a_{k,j_t} \cdot \tau = \beta.
	%\]
	%Mediante la multiplicación a la izquierda con una matriz invertible adecuada \(L\), se puede lograr que la fila \(t\) del producto de la matriz sea la suma de \(\sigma\) veces la fila original \(t\) y \(\tau\) veces la fila original \(k\), que la fila \(k\) del producto sea otra combinación lineal de esas filas originales, y que todas las otras filas permanezcan sin cambios. Explícitamente, si \(\sigma\) y \(\tau\) satisfacen la ecuación anterior, entonces para \(\alpha = a_{t,j_t} / \beta\) y \(\gamma = a_{k,j_t} / \beta\) (divisiones posibles por la definición de \(\beta\)) se tiene
	%\[
	%\sigma \cdot \alpha + \tau \cdot \gamma = 1,
	%\]
	%de modo que la matriz
	%\[
	%L_0 = \begin{pmatrix} \sigma & \tau \\ -\gamma & \alpha \end{pmatrix}
	%\]
	%es invertible, con inversa
	%\[
	%\begin{pmatrix} \alpha & -\tau \\ \gamma & \sigma \end{pmatrix}.
	%\]
	%Ahora, \(L\) se puede obtener ajustando \(L_0\) en las filas y columnas \(t\) y \(k\) de la matriz identidad. Por construcción, la matriz obtenida después de multiplicar a la izquierda por \(L\) tiene la entrada \(\beta\) en la posición \((t, j_t)\) (y debido a nuestra elección de \(\alpha\) y \(\gamma\) también tiene una entrada \(0\) en la posición \((k, j_t)\), lo cual es útil aunque no esencial para el algoritmo). Esta nueva entrada \(\beta\) divide a la entrada \(a_{t,j_t}\) que estaba allí antes, y por lo tanto, en particular, \(\delta(\beta) < \delta(a_{t,j_t})\); por lo tanto, repetir estos pasos debe terminar eventualmente. Se termina con una matriz que tiene una entrada en la posición \((t, j_t)\) que divide todas las entradas en la columna \(j_t\).
	%
	%\textbf{Paso III: Eliminación de entradas}
	%
	%Finalmente, añadiendo múltiplos apropiados de la fila \(t\), se puede lograr que todas las entradas en la columna \(j_t\) excepto la que está en la posición \((t, j_t)\) sean cero. Esto se puede lograr mediante la multiplicación a la izquierda con una matriz apropiada. Sin embargo, para hacer la matriz completamente diagonal, también necesitamos eliminar las entradas no nulas en la fila de la posición \((t, j_t)\). Esto se puede lograr repitiendo los pasos en el Paso II para las columnas en lugar de las filas, y utilizando la multiplicación a la derecha por la transposición de la matriz \(L\) obtenida. En general, esto resultará en que las entradas cero de la aplicación anterior del Paso III se vuelvan no nulas de nuevo.
	%
	%Sin embargo, observe que cada aplicación del Paso II, ya sea para filas o columnas, debe continuar reduciendo el valor de \(\delta(a_{t,j_t})\), y por lo tanto, el proceso debe detenerse eventualmente después de cierto número de iteraciones, llevando a una matriz donde la entrada en la posición \((t, j_t)\) es la única entrada no nula tanto en su fila como en su columna.
	%
	%En este punto, solo el bloque de \(A\) a la derecha inferior de \((t, j_t)\) necesita ser diagonalizado, y conceptualmente el algoritmo puede aplicarse recursivamente, tratando este bloque como una matriz separada. En otras palabras, podemos incrementar \(t\) en uno y volver al Paso I.
	%
	%
	%\textbf{Paso final}
	%
	%La aplicación de los pasos descritos anteriormente a las columnas restantes no nulas de la matriz resultante (si las hay), nos da una matriz \(m \times n\) con índices de columna \(j_1 < \ldots < j_r\) donde \(r \leq \min(m, n)\). Las entradas de la matriz \((l, j_l)\) son no nulas, y todas las demás entradas son cero.
	%
	%Ahora podemos mover las columnas nulas de esta matriz a la derecha, de modo que las entradas no nulas estén en posiciones \((i, i)\) para \(1 \leq i \leq r\). En resumen, establecemos \(\alpha_i\) para el elemento en la posición \((i, i)\).
	%
	%La condición de divisibilidad de las entradas diagonales puede no estar satisfecha inicialmente. Para cualquier índice \(i < r\) para el cual \(\alpha_i \nmid \alpha_{i+1}\), se puede reparar esta deficiencia mediante operaciones en las filas y columnas \(i\) y \(i+1\) solamente: primero se agrega la columna \(i+1\) a la columna \(i\) para obtener una entrada \(\alpha_{i+1}\) en la columna \(i\) sin alterar la entrada \(\alpha_i\) en la posición \((i, i)\), y luego se aplica una operación de fila para hacer que la entrada en la posición \((i, i)\) sea igual a \(\beta = \gcd(\alpha_i, \alpha_{i+1})\) como en el Paso II; finalmente, se procede como en el Paso III para hacer que la matriz sea diagonal nuevamente. Dado que la nueva entrada en la posición \((i+1, i+1)\) es una combinación lineal de \(\alpha_i, \alpha_{i+1}\) originales, es divisible por \(\beta\).
	%
	%El valor \(\delta(\alpha_1) + \cdots + \delta(\alpha_r)\) no cambia por la operación anterior (es \(\delta\) del determinante de la submatriz superior \(r \times r\)), por lo que esa operación disminuye (moviendo factores primos hacia la derecha) el valor de
	%\[
	%\sum_{j=1}^r(r-j)\delta(\alpha_j).
	%\]
	%Después de aplicar finitamente muchas veces esta operación, no se puede aplicar más, lo que significa que hemos obtenido \(\alpha_1 \mid \alpha_2 \mid \cdots \mid \alpha_r\) como se deseaba.
	%
	%Dado que todas las manipulaciones de filas y columnas involucradas en el proceso son invertibles, esto demuestra que existen matrices invertibles \(m \times m\) y \(n \times n\), \(S\) y \(T\), de modo que el producto \(SAT\) satisface la definición de una Forma Normal de Smith. En particular, esto muestra que la Forma Normal de Smith existe, lo cual se asumió sin prueba en la definición.
	
	\endinput
	%--------------------------------------------------------------------
	% FIN DEL CAPÍTULO.
	%--------------------------------------------------------------------
%\input{capitulos/capitulo-invariancia-homotopica}
% !TeX root = ../tfg.tex
% !TeX encoding = utf8

\chapter{Homología persistente}
\label{chapter:persistent-homology}

Este capítulo se dedica a la homología persistente, un concepto de gran
relevancia en la topología computacional que proporciona herramientas poderosas
para analizar y entender la estructura subyacente de los datos a través de
múltiples escalas. Originada en los trabajos iniciales de matemáticos como Edelsbrunner,
Letscher y Zomorodian \cite{edelsbrunner2002topological}, la homología persistente
permite la identificación y el análisis de características topológicas que persisten
a lo largo de variaciones en la escala en la que se observa.

Este capítulo se centrará en detallar los principios teóricos detrás de la homología
persistente. En particular, estudiaremos en profundidad el Teorema de correspondencia,
un resultado central en la teoría que muestra cómo transformar el nacimiento y
muerte de las clases de homología en objetos matemáticos más sencillos de
analizar y manipular computacionalmente. Seguiremos como esquema principal los resultados
de \cite{zomorodian2004computing} y \cite{dey2022computational}.

\section{Complejos de \u Cech y Vietoris-Rips}

La homología persistente es comúnmente empleada para analizar conjuntos de datos
representados como nubes de puntos. Aunque la homología en sí de estos conjuntos
puede no ser de gran interés debido a su simplicidad o falta de estructura topológica
relevante, la homología persistente permite revelar información de interés
mediante la construcción de estructuras topológicas construidas a partir de los
datos.

En este contexto, los complejos de \u Cech y Vietoris-Rips se emplean frecuentemente
para capturar la estructura topológica subyacente de las nubes de puntos. Estos
complejos dotan de estructura de complejo simplicial a los datos, facilitando su
representación, el estudio de su forma y sus características a múltiples escalas.

\begin{definicion}
	Sea $X$ un espacio topológico y sea $\mathcal{U}= \{U_{v}\}_{v \in V}$ un recubrimiento
	de $X$ por puntos $v \in V$. Llamaremos \textbf{nervio} de $\mathcal{U}$ al complejo
	simplicial abstracto con conjunto de vértices $V$ tal que la familia $v_{0}, \dots
	, v_{p} \in V$ genera un $p$-símplice si, y sólo si, $U_{v_0}\cap \dots \cap U_{v_p}
	\neq \emptyset$. Lo notaremos por $N(\mathcal{U})$.
\end{definicion}

\begin{definicion}
	Sea $(X,d)$ un espacio métrico y sea $V$ un subconjunto de puntos finito de $X$.
	Dado $\varepsilon > 0$, definimos el \textbf{complejo de \u Cech}
	$C(V, \varepsilon)$ como el nervio $N(\mathcal{B}_{\varepsilon})$, donde
	\[
	\mathcal{B}_{\varepsilon} = \{ B_{\varepsilon}(v) : v \in V \},
	\]
	siendo $B_{\varepsilon}(v)$ la bola abierta de centro $x$ y radio $\varepsilon
	> 0$.
\end{definicion}

\begin{proposicion}
	El complejo de \u Cech $C(V, \varepsilon)$ es un complejo simplicial abstracto.
\end{proposicion}
\begin{proof}
	Sea $V = \{v_{i}\}_{i=1}^{M}$ un subconjunto de puntos en el espacio métrico $X$.
	Fijado cierto $\varepsilon > 0$, definimos $\mathcal{B}_{\varepsilon} = \{B_{\varepsilon}
	(v) : v \in V \}$ como un recubrimiento por abiertos por puntos de $V$ del subespacio
	topológico $Y = \bigcup_{v \in V}B_{\varepsilon}(v)$. Veamos que el nervio
	$N(\mathcal{B}_{\varepsilon})$ de $\mathcal{B}_{\varepsilon}$ es el complejo
	abstracto cuyos vértices son los conjuntos $B_{\varepsilon}(x)$ y los
	símplices se forman por colecciones de estos conjuntos que tienen
	intersecciones no vacías.
	
	Supongamos que tenemos un símplice $\sigma = \{v_{i_1}, \ldots, v_{i_k}\}$ en
	$N(\mathcal{B}_{\varepsilon})$. Esto implica que
	\[
	\bigcap_{j=1}^{k} B_{\varepsilon}(v_{i_j}) \neq \emptyset.
	\]
	Consideremos ahora cualquier subconjunto de bolas de la forma $\{B_{\varepsilon}
	(v_{i_j})\}_{j \in J}$ donde $J \subseteq \{1, \ldots, k\}$. Es claro que por
	ser $\sigma$ un símplice,
	\[
	\bigcap_{j \in \{1, \ldots, k\} \setminus J}B_{\varepsilon}(v_{i_j}) \neq \emptyset
	,
	\]
	por lo que el conjunto de vértices restantes también forma un símplice en $N(\mathcal{B}
	_{\varepsilon})$. Por lo tanto, todas las caras de cualquier símplice $\sigma$
	son también símplices en $N(\mathcal{B}_{\varepsilon})$. Es decir, el nervio $N
	(\mathcal{B}_{\varepsilon})$ es cerrado bajo inclusiones.
	
	Dado que $V$ es finito y cada símplice se define como un subconjunto de $V$,
	entonces cada símplice sólo puede tener un número finito de subconjuntos. En consecuencia,
	el número de caras de cada símplice también es finito.
\end{proof}

El siguiente teorema será de utilidad para comprender y estudiar el espacio topológico
que definen los complejos de \u Cech. Por motivos de espacio, no trataremos la prueba
y simplemente enunciaremos el resultado. Una demostración del teorema enunciado
de forma más general puede encontrarse en \cite{hatcher2002algebraic}.

\begin{teorema}
	[Teorema del nervio] \label{teo:nerve} Sea $X$ un espacio métrico y sea $\mathcal{U}
	= \{U_{v}\}_{v \in V}$ un recubrimiento por abiertos finito de $X$. Supongamos
	además que para todo subconjunto no vacío de vértices $S \subseteq V$ tenemos
	que $\bigcap_{s \in S}U_{s}$ es contráctil o vacío. Entonces, el politopo de
	la realización geométrica del nervio de $\mathcal{U}$, $|N(\mathcal{U})|$, es
	homotópicamente equivalente a $X$.
\end{teorema}

Claramente, la intersección de bolas abiertas es vacía o contráctil (pues es convexa).
Por el \nameref{teo:nerve}, tenemos que el poliedro de la realización geométrica
del complejo de \u Cech es homotópicamente equivalente al subespacio métrico formado
por la unión de dichas bolas. En consecuencia, el estudio de la topología del complejo
de \u Cech se resume al estudio de las bolas que recubren sus vértices. Sin
embargo, el complejo de \u Cech es costoso de obtener mediante métodos computacionales.
Por ello, se propone el complejo de Vietoris-Rips para resolver este problema.

\begin{definicion}
	Sea $(X,d)$ un espacio métrico y sea $V$ un subconjunto de puntos de $X$. Dado
	$\varepsilon > 0$, definimos el \textbf{complejo de Vietoris-Rips}
	$VR(V,\varepsilon)$ como el complejo abstracto cuyo conjunto de vértices es
	$V$, de forma que $\{v_{0}, v_{1}, \dots v_{p}\} \subseteq V$ genera un $p$-símplice
	si, y solo si, $d(v_{i},v_{j}) \leq \varepsilon$ para todo $0 \leq i$,
	$j \leq p$.
\end{definicion}

\begin{proposicion}
	El complejo de Vietoris-Rips $VR(V,\varepsilon)$ es un complejo simplicial abstracto.
\end{proposicion}

\begin{proof}
	Primero veamos que el conjunto es cerrado bajo inclusiones. Supongamos que
	$\sigma = \{v_{0}, v_{1}, \dots, v_{p}\}$ es un $p$-símplice en
	$VR(V, \varepsilon)$. Por definición, esto significa que para todo $i, j$ tal
	que $0 \leq i, j \leq p$, se cumple que $d(v_{i}, v_{j}) \leq \varepsilon$.
	Consideremos ahora un subconjunto no vacío
	$\tau = \{v_{i_1}, v_{i_2}, \dots, v_{i_k}\} \subseteq \sigma$. Para cualquier
	par de índices $a, b$ con $1 \leq a, b \leq k$, los vértices $v_{i_a}$ y
	$v_{i_b}$ también cumplen que $d(v_{i_a}, v_{i_b}) \leq \varepsilon$, pues $\tau
	\subseteq \sigma$. Por lo tanto, $\tau$ es un $(k-1)$-símplice en $VR(V, \varepsilon
	)$.
	
	Por otro lado, cada $p$-símplice $\sigma$ en $VR(V, \varepsilon)$ es un
	subconjunto finito de $V$. El número de subconjuntos de cualquier conjunto finito
	es finito, y en particular, el número de subconjuntos de $\sigma$ es $2^{|\sigma|}$,
	donde $|\sigma|$ es el número de vértices en $\sigma$. En consecuencia, cada
	símplice en $VR(V, \varepsilon)$ tiene un número finito de caras.
\end{proof}

\begin{figure}
	\centering
	\includegraphics[width=110mm]{img/cech-vr.png}
	\caption{Representación de complejos simpliciales \u Cech (izquierda) y
		Vietoris-Rips (derecha) para un conjunto de puntos en $\mathbb{R}^{2}$. En el complejo
		de \u Cech, los símplices se forman por la intersección no vacía de círculos
		de radio $\varepsilon$ centrados en los puntos. El complejo de Vietoris-Rips conecta
		puntos que distan hasta $2\varepsilon$, independientemente de las
		intersecciones de los círculos. Los símplices de mayor dimensión están coloreados
		en azul, resaltando las diferencias en la estructura simplicial generada por
		cada método. Fuente \cite{chazal2021introduction}.}
\end{figure}

El complejo de Vietoris-Rips es interesante en el estudio de complejos de \u Cech.
Este hecho se debe a que estos últimos pueden ser aproximados en cierto sentido
por complejos de Vietoris-Rips:

\begin{proposicion}
	Sea $(X,d)$ un espacio métrico y sea $V$ un subconjunto de puntos de $X$.
	Entonces
	\[
	C(V, \varepsilon) \subseteq VR(V, 2\varepsilon) \subseteq C(V, 2\varepsilon).
	\]
\end{proposicion}
\begin{proof}
	La primera inclusión es inmediata pues si un punto $x$ pertenece a la intersección
	$\bigcap_{v \in V}B(v, \varepsilon)$, entonces la distancia para cada par de
	puntos de $V$ es, a lo sumo, $2 \varepsilon$. En consecuencia, cualquier
	símplice de $C(V,\varepsilon)$ se encuentra en $VR(V, 2\varepsilon)$.
	
	Para la segunda inclusión, consideremos ahora un símplice $\sigma = \{v_{0}, \dots
	, v_{p}\}$ de $VR(V, 2\varepsilon)$. Por la definición de complejo de Vietoris-Rips,
	tenemos que $d(v_{i}, v_{j}) \leq 2\varepsilon$ para todo $i,j \in \{0, \dots,
	p\}$. Considerando las bolas abiertas de radio $2\varepsilon$ centradas en
	$v_{i}$ y en $v_{j}$, tenemos que su intersección es no vacía, pues
	$v_{i} \in \overline{B}_{2\varepsilon}(v_{j})$ y
	$v_{j} \in \overline{B}_{2\varepsilon}(v_{i})$. En el supuesto de que los puntos
	pertenecieran a la frontera de las bolas, la intersección de las bolas
	abiertas también sería no vacía pues $\varepsilon > 0$. En consecuencia,
	tenemos que $\sigma \in C(V,2\varepsilon)$.
\end{proof}

\section{Módulos de homología persistente}

El módulo de homología persistente es el objeto central de estudio en este
capítulo. A partir de filtraciones de complejos simpliciales, esta estructura
nos va a permitir realizar un estudio de la homología simplicial de dicha
filtración.

\begin{definicion}
	Sea $K$ un complejo simplicial. Una \textbf{filtración} $\mathcal{F}$ de $K$ es
	una familia totalmente ordenada de subcomplejos $\{K^{n}\}_{n \in \N}$ tal que
	$\emptyset, K \in \mathcal{F}$ y si $i \leq j$, entonces
	$K^{i} \subseteq K^{j}$. En particular, llamaremos a dicho orden \textbf{filtro}.
\end{definicion}
A partir de la definición anterior, podemos construir los complejos de cadenas asociados
$C(K^{i};R)$ para todo $i \in \N$. Así mismo, podemos obtener sus respectivos submódulos
de ciclos $Z_{p}(K^{i})$ y bordes $B_{p}(K^{i})$ para cada $R$-módulo de cadenas
orientadas $C_{p}(K^{i};R)$.
\begin{definicion}
	Sea $\mathcal{F}$ una filtración, sea $p$ un número natural y sean
	$i,j \in \{0, \dots, n\}$ tales que $i \leq j$. Definimos el \textbf{$(i,j)$-ésimo
		$R$-módulo de homología persistente de dimensión $p$} asociado a $\mathcal{F}$
	como
	\[
	H_{p}^{i \to j}(\mathcal{F};R) := \im f_{p}^{i \to j},
	\]
	donde $f_{p}^{i \to j}$ es el homomorfismo inducido entre las clases de
	homología de la inclusión que va de $K^{i}$ a $K^{j}$. El rango de $H_{p}^{i
		\to j}(\mathcal{F};R)$ diremos que es el \textbf{$(i,j)$-ésimo número de Betti
		de persistencia de dimensión $p$} y lo notaremos por $\beta_{p}^{i \to j}$.
\end{definicion}
\begin{proposicion}
	Sea $\mathcal{F}$ una filtración del complejo simplicial $K$. Entonces
	\[
	H_{p}^{i \to j}(\mathcal{F};R) \cong \frac{Z_{p}(K^{i})}{B_{p}(K^{j}) \cap Z_{p}(K^{i})}
	\]
	es un isomorfismo de $R$-módulos.
\end{proposicion}
\begin{proof}
	Sabemos que el cociente anterior está bien definido, pues
	$Z_{p}(K^{i}) \cap B_{p}(K^{j})$ es un submódulo de $Z_{p}(K^{i})$. Para ver que
	en efecto existe un isomorfismo, consideraremos la proyección
	$\pi_{i} : Z_{p}(K^{i}) \to H_{p}(K^{j};R)$. Aplicando el \nameref{teo:first-iso},
	tenemos que
	\[
	\frac{Z_{p}(K^{i})}{\ker \pi_{i}}\cong \im \pi_{i}
	\]
	es un isomorfismo. Sin embargo, nótese que
	\begin{align*}
		\ker \pi_{i} & = \{z \in Z_{p}(K^{i}) : \pi_{i}(z) = [0] \} = \{z \in Z_{p}(K^{i}) : [z] = [0] \} \\
		& = \{z \in Z_{p}(K^{i}) : z \in B_{p}(K^{j}) \} = B_{p}(K^{j}) \cap Z_{p}(K^{i}).
	\end{align*}
	Además,
	\begin{align*}
		H_{p}^{i \to j}(\mathcal{F};R) & = \im f_{p}^{i \to j}= \{f_{p}^{i \to j}([z]) : [z] \in H_{p}(K^{i};R) \}                                  \\
		& = \{[({i_{i,j}}_{*})_{p}(z)] : z \in Z_{p}(K^{i}) \} = \{\pi_{i}(z) : z \in Z_{p}(K^{i}) \} = \im \pi_{i}.
	\end{align*}
\end{proof}
%
%\tikzset{every picture/.style={line width=0.75pt}} %set default line width to 0.75pt
%
%\begin{tikzpicture}[x=0.75pt,y=0.75pt,yscale=-1,xscale=1]
%	%uncomment if require: \path (0,235); %set diagram left start at 0, and has height of 235
%
%	%Shape: Ellipse [id:dp28793519162441483]
%	\draw  [fill={rgb, 255:red, 74; green, 144; blue, 226 }  ,fill opacity=1 ] (81,72.52) .. controls (81,21.96) and (104.41,-19.03) .. (133.29,-19.03) .. controls (162.18,-19.03) and (185.59,21.96) .. (185.59,72.52) .. controls (185.59,123.08) and (162.18,164.07) .. (133.29,164.07) .. controls (104.41,164.07) and (81,123.08) .. (81,72.52) -- cycle ;
%	%Shape: Ellipse [id:dp5681689316325549]
%	\draw  [fill={rgb, 255:red, 169; green, 208; blue, 255 }  ,fill opacity=1 ] (213.98,71.12) .. controls (213.98,20.55) and (237.39,-20.44) .. (266.27,-20.44) .. controls (295.15,-20.44) and (318.56,20.55) .. (318.56,71.12) .. controls (318.56,121.68) and (295.15,162.67) .. (266.27,162.67) .. controls (237.39,162.67) and (213.98,121.68) .. (213.98,71.12) -- cycle ;
%	%Shape: Ellipse [id:dp8309047921029127]
%	\draw  [fill={rgb, 255:red, 169; green, 208; blue, 255 }  ,fill opacity=1 ] (349.94,72.52) .. controls (349.94,21.96) and (373.35,-19.03) .. (402.24,-19.03) .. controls (431.12,-19.03) and (454.53,21.96) .. (454.53,72.52) .. controls (454.53,123.08) and (431.12,164.07) .. (402.24,164.07) .. controls (373.35,164.07) and (349.94,123.08) .. (349.94,72.52) -- cycle ;
%	%Shape: Ellipse [id:dp6471754723664047]
%	\draw  [fill={rgb, 255:red, 169; green, 208; blue, 255 }  ,fill opacity=1 ] (484.41,72.52) .. controls (484.41,21.96) and (507.82,-19.03) .. (536.71,-19.03) .. controls (565.59,-19.03) and (589,21.96) .. (589,72.52) .. controls (589,123.08) and (565.59,164.07) .. (536.71,164.07) .. controls (507.82,164.07) and (484.41,123.08) .. (484.41,72.52) -- cycle ;
%	%Shape: Ellipse [id:dp2887350161913229]
%	\draw  [fill={rgb, 255:red, 74; green, 144; blue, 226 }  ,fill opacity=1 ] (226.68,99.38) .. controls (226.68,64.42) and (244.4,36.09) .. (266.27,36.09) .. controls (288.14,36.09) and (305.86,64.42) .. (305.86,99.38) .. controls (305.86,134.33) and (288.14,162.67) .. (266.27,162.67) .. controls (244.4,162.67) and (226.68,134.33) .. (226.68,99.38) -- cycle ;
%	%Shape: Ellipse [id:dp5997979826300346]
%	\draw  [fill={rgb, 255:red, 74; green, 144; blue, 226 }  ,fill opacity=1 ] (362.64,100.78) .. controls (362.64,65.83) and (380.37,37.49) .. (402.24,37.49) .. controls (424.1,37.49) and (441.83,65.83) .. (441.83,100.78) .. controls (441.83,135.73) and (424.1,164.07) .. (402.24,164.07) .. controls (380.37,164.07) and (362.64,135.73) .. (362.64,100.78) -- cycle ;
%	%Shape: Ellipse [id:dp7175957985808079]
%	\draw  [fill={rgb, 255:red, 74; green, 144; blue, 226 }  ,fill opacity=1 ] (505.7,114.81) .. controls (505.7,87.61) and (519.58,65.55) .. (536.71,65.55) .. controls (553.83,65.55) and (567.71,87.61) .. (567.71,114.81) .. controls (567.71,142.02) and (553.83,164.07) .. (536.71,164.07) .. controls (519.58,164.07) and (505.7,142.02) .. (505.7,114.81) -- cycle ;
%	%Straight Lines [id:da9382562350971808]
%	\draw    (133.29,-19.03) -- (266.27,36.09) ;
%	%Straight Lines [id:da8538877694649378]
%	\draw    (266.27,36.09) -- (402.24,37.49) ;
%	%Straight Lines [id:da415521229926751]
%	\draw    (402.24,37.49) -- (536.71,65.55) ;
%	%Straight Lines [id:da5906511807851325]
%	\draw [fill={rgb, 255:red, 74; green, 144; blue, 226 }  ,fill opacity=1 ] [dash pattern={on 0.84pt off 2.51pt}]  (266.27,16.65) -- (399.25,16.65) ;
%	%Straight Lines [id:da7445210416359846]
%	\draw  [dash pattern={on 0.84pt off 2.51pt}]  (399.25,16.65) -- (536.71,114.81) ;
%
%	% Text Node
%	\draw (260.27,-0.47) node [anchor=north west][inner sep=0.75pt]    {\(\alpha \)};
%	% Text Node
%	\draw (375.75,-4.04) node [anchor=north west][inner sep=0.75pt]    {\(f_{p}^{i,j-1}( \alpha )\)};
%	% Text Node
%	\draw (515.12,113.56) node [anchor=north west][inner sep=0.75pt]    {\(f_{p}^{i,j}( \alpha )\)};
%	% Text Node
%	\draw (100.29,-38.98) node [anchor=north west][inner sep=0.75pt]    {\(H_{p}( K_{i-1})\)};
%	% Text Node
%	\draw (242.52,-40.98) node [anchor=north west][inner sep=0.75pt]    {\(H_{p}( K_{i})\)};
%	% Text Node
%	\draw (370.99,-39.38) node [anchor=north west][inner sep=0.75pt]    {\(H_{p}( K_{j-1})\)};
%	% Text Node
%	\draw (510.2,-39.98) node [anchor=north west][inner sep=0.75pt]    {\(H_{p}( K_{j})\)};
%	% Text Node
%	\draw (260,12.4) node [anchor=north west][inner sep=0.75pt]    {\(\bullet \)};
%	% Text Node
%	\draw (393,12.4) node [anchor=north west][inner sep=0.75pt]    {\(\bullet \)};
%	% Text Node
%	\draw (528,108.4) node [anchor=north west][inner sep=0.75pt]    {\(\bullet \)};
%\end{tikzpicture}

La homología persistente facilita la interpretación de la homología en los distintos
niveles de la filtración, permitiendo un análisis cuantitativo de su evolución. Al
observar el nacimiento y desaparición de las clases de homología mediante los módulos
de homología persistente, se obtiene información detallada de la estructura
topológica general de la filtración.

\begin{figure}[h]
	\centering
	\includegraphics[width=\textwidth]{img/persistence-diagram.png}
	\caption{Diagrama de persistencia para la homología de dimensión 1 de la
		filtración de Vietoris-Rips de 2000 puntos i.i.d. en un toro. El gráfico
		destaca dos puntos significativos que representan las clases de equivalencia
		de los ciclos unidimensionales del toro, destacando su persistencia topológica.
		Fuente \cite{divol2019}.}
\end{figure}

\begin{definicion}
	Dada una filtración $\mathcal{F}$, decimos que un elemento $\alpha \neq 0$ en
	$H_{p}(K^{i};R)$ \textbf{nace} en $K^{i}$ si $\alpha \not\in H_{p}^{i-1 \to i}(
	\mathcal{F};R)$. Además, decimos que $\alpha$ \textbf{muere} en $K^{j}$ si se
	fusiona con una clase proveniente de una dimensión anterior cuando se desplaza
	de $K^{j-1}$ a $K^{j}$. Esto es, si
	$f^{i \to j-1}_{p}(\alpha) \not\in H_{p}^{i-1 \to j-1}(\mathcal{F};R)$ pero
	$f^{i \to j}_{p}(\alpha) \in H_{p}^{i-1 \to j}(\mathcal{F};R)$.
\end{definicion}

\begin{ejemplo}
	\label{ex:pers-homology} Consideremos un conjunto de vértices $\{0, e_{1}, e_{2}
	, e_{3}\} = V$ formado por los vectores de la base usual y el vector nulo en $\mathbb{R}
	^{3}$ con la topología usual. Definamos el complejo simplicial $K$ a partir
	del 2-símplice formado por dichos vectores y todas sus caras. A continuación,
	vamos a estudiar la filtración inducida por inclusiones $\mathcal{K}$:
	\[
	\emptyset = K^{0} \to K^{1} \to K^{2} \to K^{3} \to K^{4} = K,
	\]
	donde
	\begin{align*}
		K^{1} & = \{[0], [e_{1}], [e_{2}], [e_{3}]\},                                     \\
		K^{2} & = K^{1} \cup \{[0, e_{1}], [0, e_{2}], [0, e_{3}]\},                      \\
		K^{3} & = K^{2} \cup \{[0, e_{1}, e_{2}], [0, e_{1}, e_{3}], [0, e_{2}, e_{3}]\}.
	\end{align*}
	
	Consideremos además las cadenas de $\mathbb{R}$-módulos asociadas, donde cada $C
	_{p}(K^{j};\mathbb{R})$ es un espacio vectorial con cuerpo $\mathbb{R}$ para cada
	dimensión $p \in \mathbb{Z}$.
	
	En $K^{1}$, el módulo de cadenas $C_{0}(K^{1}; \mathbb{R})$ es generado por
	los 0-símplices $\{ [0], [e_{1}], [e_{2}], [e_{3}] \}$, lo que resulta en un espacio
	vectorial $\mathbb{R}^{4}$, mientras que los módulos
	$C_{1}(K^{1}; \mathbb{R})$ y $C_{2}(K^{1}; \mathbb{R})$ son ambos cero, debido
	a la ausencia de 1-símplices y 2-símplices.
	
	En cuanto a $K^{2}$, el módulo $C_{0}(K^{2}; \mathbb{R})$ conserva la misma
	estructura que $C_{0}(K^{1}; \mathbb{R})$, pues no se añaden nuevos 0-símplices.
	Sin embargo, $C_{1}(K^{2}; \mathbb{R})$ se genera ahora a partir de las aristas
	$\{ [0, e_{1}], [0, e_{2}], [0, e_{3}] \}$, formando un espacio vectorial
	isomorfo a $\mathbb{R}^{3}$. El módulo $C_{2}(K^{2}; \mathbb{R})$ sigue siendo
	nulo debido a la falta de 2-símplices.
	
	Llegando a $K^{3}$, observamos que $C_{2}(K^{3}; \mathbb{R})$ ahora se genera
	por $\{ [0, e_{1}, e_{2}], [0, e_{1}, e_{3}], [0, e_{2}, e_{3}] \}$, generando
	otro espacio isomorfo a $\mathbb{R}^{3}$. Los módulos
	$C_{0}(K^{3}; \mathbb{R})$ y $C_{1}(K^{3}; \mathbb{R})$ son idénticos a sus
	pares en $K^{2}$, pues tienen los mismos símplices de dimensión 1 y 2.
	
	Finalmente, en $K^{4} = K$, todos los módulos $C_{p}(K^{4}; \mathbb{R})$ para
	$p \in \{ 0, 1, 2 \}$ son idénticos a los de $K^{3}$. Sin embargo, $K^{4}$
	tiene un 3-símplice formado por todos sus vértices, de forma que $C_{3}(K^{4};
	\R) = \R$.
	
	En $K^{2}$, todos los 1-símplices son ciclos porque no existen 2-símplices que
	puedan generar bordes, de modo que $Z_{1}(K^{2}) = C_{1}(K^{2})$ y $B_{1}(K^{2}
	) = 0$. En $K^{3}$, todos los 2-símplices son ciclos, ya que no hay 3-símplices
	en esta etapa; así, $Z_{2}(K^{3}) = C_{2}(K^{3})$ y los bordes de
	$C_{1}(K^{3})$ son generados por las aristas de los 2-símplices, indicando cierta
	interacción entre ellos.
	
	En $K^{4}$, el único 3-símplice forma un ciclo por sí mismo, puesto que $C_{4}(
	K^{4}) = 0$ y, por lo tanto, $Z_{3}(K^{4}) = C_{3}(K^{4})$. Los bordes de los
	2-símplices en $C_{2}(K^{4})$ resultan de aplicar $\partial_{3}$ al 3-símplice,
	produciendo una combinación de todos los 2-símplices, por lo tanto, $B_{2}(K^{4}
	)$ se compone de estos bordes.
	
	Para $K^{1}$, tenemos $H_{0}(K^{1}; \R) = \mathbb{R}^{4}$, mostrando cuatro componentes
	conexas. En $K^{2}$, la homología se simplifica a $H_{0}(K^{2}; \R) = \mathbb{R}$,
	indicando que todos los vértices forman una única componente conexa, mientras que
	$H_{1}(K^{2}; \R) = 0$ y $H_{2}(K^{2}; \R) = 0$ dado que no hay 2-símplices.
	En $K^{3}$, la homología permanece sin cambios para $H_{0}$ y $H_{2}$, pero
	todos los posibles 1-ciclos se convierten en bordes de los 2-símplices, lo que
	resulta en $H_{1}(K^{3}; \R) = 0$.
	
	Finalmente, en $K^{4}$, observamos que $H_{0}(K^{4}; \R) = \mathbb{R}$,
	$H_{1}(K^{4}; \R) = 0$, $H_{2}(K^{4}; \R) = 0$ y $H_{3}(K^{4}; \R) = \mathbb{R}$,
	indicando la presencia de una cavidad tridimensional que no es el borde de
	ningún símplice de mayor dimensión.
	
	Tras este exhaustivo estudio de la homología de la filtración, por fin estamos
	en condiciones de estudiar el nacimiento o muerte de las clases de homología.
	Por ejemplo, podemos observar que en el instante 1 de la filtración nacen cuatro
	clases de homología de dimensión 0, pues $H_{0}(K^{1};\R) = \R^{4}$. Sin embargo,
	en el segundo instante de la filtración, $H_{0}(K^{2}; \R) = \R$, lo que
	implica la muerte de tres clases de homología de dimensión 0. De esta forma, tendríamos
	que el módulo de homología persistente $H_{0}^{1 \to 2}(\mathcal{K}; \R) = \im{i_*}
	^{1 \to 2}= \R$, siendo
	$i^{1 \to 2}_{*} : H_{0}(K^{1};\R) \to H_{0}(K^{2};\R)$ el homomorfismo de
	inclusión inducido. En consecuencia, es claro que $(1,2)$-ésimo número de
	Betti de persistencia de dimensión 0 es 1.
\end{ejemplo}

Como hemos visto en el \autoref{ex:pers-homology}, el estudio de la homología
persistente requiere de un amplio estudio de la homología de la filtración a
todos sus niveles. Esta información a menudo es numerosa y difícil de interpretar,
por lo que en la siguiente sección mostraremos una representación de la
homología persistente que facilite su estudio e interpretación.

\section{Representación de la homología persistente}
\label{sec:barcode}

Nuestro objetivo ahora es obtener una representación de la homología persistente
que nos permita interpretarla. Originalmente, esta representación se ha realizado
mediante los \textbf{diagramas de persistencia}. En estos diagramas, cada clase
de homología que nace en un complejo simplicial $K^{i}$ y muere en $K^{j}$ se
representa por un punto $(i, j)$ en el plano cartesiano, para cada dimensión $p$.

Aunque intuitivos, los diagramas de persistencia presentan ciertas limitaciones.
El principal problema radica en que no logran capturar de manera efectiva la multiplicidad
de la persistencia; es decir, múltiples clases que nacen y mueren simultáneamente
se representan mediante un mismo punto, lo que puede conducir a ambigüedades en
la interpretación de los datos.

Para abordar estos problemas, Zomorodian y Carlsson introdujeron un enfoque innovador
mediante el uso del \textbf{código de barras} \cite{zomorodian2004computing}. Este
método no solo ofrece una visualización clara y detallada de la vida de cada clase
de homología, sino que también facilita la interpretación de su multiplicidad y
relevancia en el contexto de los datos analizados. Esta interpretación fue posible
ya que demostraron que todo módulo de persistencia puede identificarse de manera
biyectiva con una estructura algebraica más general, conocida como \textbf{módulos
	graduados}, bajo ciertas hipótesis.

A continuación profundizaremos en la aplicación y las implicaciones de estos
métodos, explorando en detalle el Teorema de correspondencia como se discute en \cite{Corbet_2018}.

\begin{definicion}
	Sea $\mathcal{M}= \{\{M_{i}\}_{i \in \N}, \{f_{i,j}\}_{i \leq j \in \N}\}$ una
	familia de $R$-módulos. Diremos que dicha familia es un \textbf{módulo de
		persistencia discreto} sobre $R$ si para cada $i \leq j$ existe un homomorfismo
	de $R$-módulos $f_{i,j}: M_{i} \to M_{j}$ tal que:
	\begin{enumerate}
		\item El homomorfismo $f_{i} = f_{i,i}= \mathrm{id}_{M_i}$ para todo $i \in \mathbb{N}$.
		
		\item La composición $f_{i,k}\circ f_{i,j}= f_{i,k}$ para todo $i \leq j \leq
		k$.
	\end{enumerate}
\end{definicion}
\begin{observacion}
	Nótese que con esta definición, todo módulo de homología persistente es de
	hecho un módulo de persistencia discreto. Dada una dimensión $p \in \Z$ fija,
	los $R$-módulos los componen los módulos de homología y sus homomorfismos
	inducidos por la inclusión cumplen dichas propiedades.
\end{observacion}

\begin{definicion}
	Sean $\mathcal{M}= \{\{M_{i}\}_{i \in \N}, \{f_{i,j}\}_{i \leq j \in \N}\}, \mathcal{N}
	= \{\{N_{i}\}_{i \in \N}, \{g_{i,j}\}_{i \leq j \in \N}\}$ dos módulos de persistencia
	discretos. Diremos que la familia de homomorfismos
	$\varphi_{\bullet} = \{\varphi_{i}\}_{i \in \N}$ tales que
	$\varphi_{i} : M_{i} \to N_{i}$ es un \textbf{homomorfismo de módulos de
		persistencia discreto} si $g_{i,j}\circ \varphi_{i} = \varphi_{j} \circ f_{i,j}$.
\end{definicion}
La anterior definición es equivalente a decir que el diagrama
\[
\xymatrix{ M_0 \ar@{->}[r]^{f_0} \ar@{->}[d]^{\varphi_0} & M_1 \ar@{->}[r]^{f_1} \ar@{->}[d]^{\varphi_1} & \cdots \ar@{->}[r]^{f_{i-1}} & M_i \ar@{->}[r]^{f_i} \ar@{->}[d]^{\varphi_i} & M_{i+1} \ar@{->}[r]^{f_{i+1}} \ar@{->}[d]^{\varphi_{i+1}} & \cdots \\ N_0 \ar@{->}[r]^{g_0} & N_1 \ar@{->}[r]^{g_1} & \cdots \ar@{->}[r]^{g_{i-1}} & N_i \ar@{->}[r]^{g_i} & N_{i+1} \ar@{->}[r]^{g_{i+1}} & \cdots }
\]
conmuta. En las condiciones anteriores, los módulos de persistencia discretos
junto a sus homomorfismos forman una categoría que notaremos por $R$-$\Cat{PersMod}$.
Claramente se tiene que existe el homomorfismo identidad $\id_{\mathcal{M}}: \mathcal{M}
\to \mathcal{M}$, donde a cada módulo y homomorfismo de módulos se le asocia él
mismo. Además, la identidad conmuta con la familia de homomorfismos de módulos, pues
es la composición con la identidad de homomorfismos de módulos componente a componente.
Veamos ahora la asociatividad en la composición de los homomorfismos de módulos
de persistencia discretos. Consideremos cuatro módulos de persistencia discretos
$\mathcal{M}= \{\{M_{i}\}_{i \in \mathbb{N}}, \{f_{i,j}\}_{i \leq j \in
	\mathbb{N}}\}$,
$\mathcal{N}= \{\{N_{i}\}_{i \in \mathbb{N}}, \{g_{i,j}\}_{i \leq j \in
	\mathbb{N}}\}$, $\mathcal{P}= \{\{P_{i}\}_{i \in \mathbb{N}}, \{h_{i,j}\}_{i
	\leq j \in \mathbb{N}}\}$ y $\mathcal{Q}= \{\{Q_{i}\}_{i \in \mathbb{N}}, \{p_{i,j}
\}_{i leq j \in \mathbb{N}}\}$; y tres homomorfismos
$\varphi_{\bullet} : \mathcal{M}\to \mathcal{N}$, $\psi_{\bullet} : \mathcal{N}\to
\mathcal{P}$, y $\theta_{\bullet} : \mathcal{P}\to \mathcal{Q}$. Para demostrar la
asociatividad, necesitamos verificar que
\[
(\theta \circ (\psi \circ \varphi))_{i} = ((\theta \circ \psi) \circ \varphi)_{i}
\quad \forall i \in \mathbb{N}.
\]
Sin embargo, dicha composición es la asociatividad de la composición de homomorfismos
de módulos componente a componente, por lo que tenemos que:
\[
\theta_{i} \circ (\psi_{i} \circ \varphi_{i}) = (\theta_{i} \circ \psi_{i}) \circ
\varphi_{i}.
\]

Además, para cualquier $i \leq j$ en $\mathbb{N}$, necesitamos mostrar que:
\[
h_{i,j}\circ (\theta \circ (\psi \circ \varphi))_{i} = (\theta \circ (\psi \circ
\varphi))_{j} \circ f_{i,j}.
\]
Sustituyendo la definición de composición, esto se reduce a verificar que
\[
h_{i,j}\circ (\theta_{i} \circ (\psi_{i} \circ \varphi_{i})) = (\theta_{j} \circ
(\psi_{j} \circ \varphi_{j})) \circ f_{i,j}\quad i \leq j \text{ en }\N.
\]
O equivalentemente, que $((h_{i,j}\circ \theta_{i}) \circ \psi_{i}) \circ \varphi
_{i} = \theta_{j} \circ (\psi_{j} \circ (\varphi_{j} \circ f_{i,j}))$. Debido a
que cada homomorfismo respeta las estructuras de los módulos de persistencia, tenemos:
\[
h_{i,j}\circ \theta_{i} = \theta_{j} \circ p_{i,j},
\]
\[
p_{i,j}\circ \psi_{i} = \psi_{j} \circ g_{i,j},
\]
\[
g_{i,j}\circ \varphi_{i} = \varphi_{j} \circ f_{i,j}.
\]
Aplicando estas igualdades y la asociatividad de homomorfismos de módulos, concluimos
que:
\begin{align*}
	h_{i,j}\circ (\theta_{i} \circ (\psi_{i} \circ \varphi_{i})) & = ((h_{i,j}\circ \theta_{i}) \circ \psi_{i}) \circ \varphi_{i} = ((\theta_{j} \circ p_{i,j}) \circ \psi_{i}) \circ \varphi_{i}                                                                  \\
	& = (\theta_{j} \circ (p_{i,j}\circ \psi_{i})) \circ \varphi_{i} = (\theta_{j} \circ (\psi_{j} \circ g_{i,j})) \circ \varphi_{i}                                                                  \\
	& = \theta_{j} \circ ((\psi_{j} \circ g_{i,j}) \circ \varphi_{i}) = \theta_{j} \circ (\psi_{j} \circ (g_{i,j}\circ \varphi_{i})) = \theta_{j} \circ (\psi_{j} \circ (\varphi_{j} \circ f_{i,j})).
\end{align*}

\begin{definicion}
	Sea $R$ un anillo. Diremos que $R$ es un \textbf{anillo graduado} si puede descomponerse
	como una suma directa
	\[
	R = \bigoplus_{n=0}^{\infty}R_{n},
	\]
	donde $R_{m} R_{n} \subseteq R_{m+n}$ para todo $m, n \in \mathbb{Z}$, siendo
	el anillo $R_{m} R_{n} = \{ r_{m} r_{n} : r_{m} \in R_{m}, r_{n} \in R_{n} \}$.
	Los elementos de $R_{n}$ distintos de cero se denominan \textbf{homogéneos de
		grado $n$}.
\end{definicion}

\begin{definicion}
	Sea $R$ un anillo graduado y sea $M$ un $R$-módulo. Diremos que $M$ es un
	\textbf{módulo graduado} si puede escribirse como
	\[
	M = \bigoplus_{n=0}^{\infty}M_{n},
	\]
	donde $M_{n}$ son grupos abelianos y $R_{m} M_{n} \subseteq M_{m+n}$ para todo
	$m, n \in \mathbb{Z}$, siendo $R_{m} M_{n}$ el $R_{m}$-módulo con grupo
	abeliano $M_{n}$. Un elemento de $M_{n}$ distinto de cero se llama \textbf{homogéneo
		de grado $n$}.
\end{definicion}

Los módulos graduados no son más que un tipo particular de módulos. Para cada
módulo graduado $M$, el morfismo identidad $\mathrm{id}_{M} : M \to M$ es simplemente
el morfismo identidad de $R$-módulos. Este morfismo es claramente un
homomorfismo de módulos graduados que preserva la graduación. Consideremos ahora
dos módulos graduados $M$, $N$ y un homomorfismo de módulos graduados
$f: M \to N$. Si tenemos otro módulo graduado $P$ y un homomorfismo $g: N \to P$,
su composición $g \circ f$ es la composición de homomorfismos de $R$-módulos. Por
lo tanto, la asociatividad de la composición se sigue directamente de la
asociatividad de la composición de homomorfismos de $R$-módulos. En consecuencia,
los módulos graduados sobre $R$ junto con los homomorfismos de módulos graduados
forman una categoría, que denotaremos por $R\text{-}\mathbf{GrMod}$.

Los módulos de persistencia discretos sobre un anillo $R$ y los $R[t]$-módulos graduados
son conceptos íntimamente relacionados. Si $\mathcal{M}$ es un módulo de
persistencia discreto, podemos definir un $R[t]$-módulo graduado $\alpha(\mathcal{M}
)$ como
\[
\alpha(\mathcal{M}) = \bigoplus_{i \in \N}M_{i},
\]
donde el producto por $t$ lo definimos como $t \cdot m_{i} = f_{i,i+1}(m_{i})$ para
todo $m_{i} \in M_{i}$. Análogamente, podemos definir un módulo de persistencia discreto
a partir de un $R[t]$-módulo $\bigoplus_{i \in \N}M_{i}$, de forma que
\[
\beta \left( \bigoplus_{i \in \N}M_{i} \right) = \mathcal{M}.
\]
Aquí, los morfismos los obtenemos a partir del producto por $t$, esto es, $f_{i,
	i+1}(m_{i}) = t \cdot m_{i}$ para todo $m_{i} \in M_{i}$. El siguiente resultado
nos proporciona formalmente cómo de íntima es esta relación.

\begin{lema}
	\label{lem:repr-carlsson-no-finito} Las aplicaciones $\alpha$ y $\beta$ definidas
	anteriormente forman una pareja isomorfa de funtores covariantes entre $R$-$\Cat
	{PersMod}$ y $R[t]$-$\Cat{GrMod}$. En particular, ambas categorías son isomorfas.
\end{lema}
\begin{proof}
	Sea $\varphi_{\bullet} : \mathcal{M}\to \mathcal{N}$ un morfismo de módulos de
	persistencia discretos. Definamos
	\[
	\alpha(\varphi_{\bullet}) : \bigoplus_{i \in \N}M_{i} \to \bigoplus_{i \in
		\N}N_{i}
	\]
	donde a cada $m_{i} \in M_{i}$ le asignamos $\varphi_{i}(m_{i})$ para cada
	$i \in \N$. Veamos que $\alpha : R$-$\Cat{PersMod}\to R[t]$-$\Cat{GrMod}$ es un
	funtor covariante. Primero veamos que $\alpha(\varphi_{\bullet})$ es un
	morfismo de módulos graduados. Tenemos que $\alpha(\varphi_{\bullet})$ es un homomorfismo
	de grupos pues cada $\varphi_{i}$ lo es, cumple que
	$\alpha(\varphi_{i})(M_{i}) \subseteq N_{i}$ y además, si $m = (m_{0}, m_{1}, \ldots
	)$ es un elemento de $\mathcal{M}$, entonces
	\begin{align*}
		\alpha(\varphi_{\bullet})(tm) & = \alpha(\varphi_{\bullet})(0, tm_{0}, tm_{1}, \ldots) = (0, \varphi_{0}(tm_{0}), \varphi_{1}(tm_{1}), \ldots) \\
		& = (0, t\varphi_{0}(m_{0}), t\varphi_{1}(m_{1}), \ldots) = t\alpha(\varphi_{\bullet})(m),
	\end{align*}
	donde la última igualdad es consecuencia de la segunda propiedad de los
	morfismos de módulos de persistencia discretos. En cuanto a las propiedades funtoriales,
	es evidente que $\alpha$ lleva identidades en identidades. Además, si
	$\psi_{\bullet}$ es otro morfismo de módulos de persistencia discretos, tenemos
	que
	\begin{align*}
		(\alpha(\psi_{\bullet} \circ \varphi_{\bullet}))(m) = (\psi_{i}(\varphi_{i}(m_{i})))_{i \in \N}= \alpha (\psi_{\bullet})(\varphi_{i}(m_{i}))_{i \in \N}= (\alpha(\psi_{\bullet}) \circ \alpha(\varphi_{\bullet}))(m).
	\end{align*}
	
	Consideremos ahora el homomorfismo de $R[t]$-módulos graduados
	\[
	\eta : \bigoplus_{i \in \N}M_{i} \to \bigoplus_{i \in \N}N_{i},
	\]
	que para cada $i \in \N$ induce un homomorfismo $\eta_{i} : M_{i} \to N_{i}$
	compatible con el producto por $t$. En consecuencia, el diagrama
	\[
	\xymatrix{ M_0 \ar@{->}[r]^{t} \ar@{->}[d]^{\eta_0} & M_1 \ar@{->}[r]^{t} \ar@{->}[d]^{\eta_1} & \cdots \ar@{->}[r]^{t} & M_i \ar@{->}[r]^{t} \ar@{->}[d]^{\eta_i} & M_{i+1} \ar@{->}[r]^{t} \ar@{->}[d]^{\eta_{i+1}} & \cdots \\ N_0 \ar@{->}[r]^{t} & N_1 \ar@{->}[r]^{t} & \cdots \ar@{->}[r]^{t} & N_i \ar@{->}[r]^{t} & N_{i+1} \ar@{->}[r]^{t} & \cdots }
	\]
	es conmutativo. Definamos ahora $\beta(\eta) = (\eta_{0}, \eta_{1}, \ldots)$ y
	veamos que es un homomorfismo de módulos de persistencia discretos entre $\mathcal{M}$
	y $\mathcal{N}$. En consecuencia, $\beta$ nos da homomorfismos de grupos $\eta_{i}
	: M_{i} \to N_{i}$ que, a su vez, son homomorfismos de $R$-módulos. Para
	comprobarlo, basta tomar cualquier $r \in R$ y $m_{i} \in M_{i}$ y vemos que $\eta
	_{i}(rm_{i}) = \eta(rm_{i}) = r \eta(m_{i}) = r \eta_{i}(m_{i})$. Como los
	homomorfismos de $R$-módulos de $\mathcal{M}$ y $\mathcal{N}$ se obtienen
	mediante la multiplicación por $t$, entonces para todo $m_{i} \in M_{i}$ tenemos
	que
	\[
	\eta_{i+1}(tm_{i}) = \eta(tm_{i}) = t\eta(m_{i}) = t \eta_{i}(m_{i}),
	\]
	por lo que $\beta(\eta)$ es un homomorfismo de módulos de persistencia discretos.
	Claramente $\beta$ conserva la identidad. Luego para otro $\theta : \mathcal{M}
	\to \mathcal{N}$ y cualquier $m=(m_{i})_{i \in \N}\in \mathcal{M}$,
	\[
	(\beta(\theta \circ \eta))(m) = (\theta(\eta(m_{i})))_{i \in \N}= \beta(\theta
	)(\eta(m_{i}))_{i \in \N}= (\beta(\theta) \circ \beta(\eta))(m).
	\]
	Esto es, $\beta$ es un funtor covariante. Finalmente, por la construcción de $\alpha$
	y $\beta$ tenemos que $\beta \circ \alpha$ es el funtor identidad en $R[t]$-$\Cat
	{GrMod}$ y que $\alpha \circ \beta$ es el funtor identidad en $R$-$\Cat{PersMod}$.
\end{proof}

En la práctica generalmente trabajaremos con módulos de persistencia que cumplen
ciertas condiciones de finitud. Por ello, resulta de gran interés conocer si la correspondencia
recién realizada se sigue cumpliendo bajo estos casos.

\begin{definicion}
	Diremos que un módulo de persistencia discreto $\mathcal{M}$ es de \textbf{tipo
		finito} si existe $N \in \N$ de forma que para todo $i,j \in \N$ tal que
	$N \leq i \leq j$ el homomorfismo $f_{i,j}$ es un isomorfismo.
\end{definicion}

\begin{definicion}
	Diremos que un módulo de persistencia discreto $\mathcal{M}$ es de \textbf{tipo
		finitamente presentado (generado)} si es de tipo finito y además, $M_{i}$ es finitamente
	presentado (generado) para todo $i \in \N$.
\end{definicion}

Nuestro objetivo será probar que el \autoref{lem:repr-carlsson-no-finito} sigue siendo
cierto para los módulos de persistencia discretos de tipo finitamente presentados.

\begin{lema}
	\label{lem:alpha-finito-presentado} Sea $\mathcal{M}$ un módulo de persistencia
	discreto. Si $\mathcal{M}$ es de tipo finitamente presentado, entonces $\alpha(
	\mathcal{M})$ es finitamente presentado.
\end{lema}
\begin{proof}
	Consideremos $N \in \N$ de forma que $f_{i,j}: M_{i} \to M_{j}$ es un isomorfismo
	para todo $N \leq i \leq j$. Sea $G$ un conjunto de generadores de $M_{i}$. Queremos
	ver que $G = \bigcup_{i=1}^{N} G_{i}$ es un sistema de generadores también para
	$\alpha(\mathcal{M})$. Para ello, veamos que todo elemento homogéneo de $\alpha
	(\mathcal{M})$ está generado por la unión de los $G_{i}$. Fijemos $k \in \N$ y
	sea $m_{k} \in \alpha(\mathcal{M})$ un elemento homogéneo de grado $k$. Si $k \leq
	N$, entonces $m_{k}$ está generado por los elementos de $G_{k}$ por
	construcción. Si $k > N$, veamos que $m_{k}$ está generado por $G_{N}$. Por ser
	$f_{N,k}$ un isomorfismo, podemos tomar $m_{N} = f^{-1}_{N,k}(m_{k})$. Pero como
	$m_{D}$ está generado por $G_{N}$, entonces $m_{k}$ está generado por
	$f_{N,k}(G_{N})$. Por como hemos construido $\alpha$,
	$f_{N,k}(G_{N}) = t^{k-N}G_{N}$ y como $t^{k-N}\in R[t]$, entonces $m_{k}$ está
	generado por $G_{N}$. En consecuencia, $\alpha(\mathcal{M})$ es finitamente generado.
	
	Para ver que $\alpha(\mathcal{M})$ es finitamente presentado, consideremos el
	epimorfismo $\mu_{i} : R^{n_i}\to M_{i}$ que genera $M_{i}$ por extensión
	lineal sobre $G_{i}$. Considerando $n = \sum_{i=1}^{N} n_{i}$, existe una
	aplicación $\mu : R[t]^{N} \to \alpha(\mathcal{M})$ que corresponde al sistema
	de generadores $G$. Para cada $g_{i} \in G$, denotemos por $e_{i}$ a su
	correspondiente elemento en el sistema de generadores de $R[t]^{N}$.
	
	A continuación definamos un conjunto finito de elementos del núcleo de $\mu$. sea
	$H_{i}$ el sistema de generadores de $\ker \mu_{i}$ para cada $0 \leq i \leq N$.
	Es claro que $H_{i} \subseteq \ker \mu_{i}$. Es más, para cualquier
	$0 \leq i < j \leq N$ y cualquier $g_{i} \in G_{i}$ tal que
	$f_{i,j}(g_{i}) \neq 0$, tenemos que
	\[
	f_{i,j}(g_{i}) = \sum_{k=0}^{n_j}\lambda_{k}{g_j}_{k}
	\]
	donde $\lambda_{k} \in R$ y $G_{i} = \{{g_j}_{0},{g_j}_{1}, \ldots,{g_j}_{k} \}$.
	Por tanto, el correspondiente elemento
	\[
	t^{j-i}e_{i} - \sum_{k=0}^{n_j}\lambda_{k}{e_j}_{k}
	\]
	pertenece al $\ker \mu$. Denotemos ahora por $H_{i,j}$ al conjunto finito
	obtenido tomando los elementos de la forma de la expresión anterior para cada
	$g_{i} \in G_{i}$ tal que $f_{i,j}(g_{i}) \neq 0$. Sea $H = \bigcup_{i=0}^{N} H
	_{i} \cup \bigcup_{0 < i \leq j \leq N}H_{i,j}$.
	
	A continuación, fijemos un elemento $x$ del núcleo de $\mu$ de la forma
	\[
	x = \sum_{l} \lambda_{l} e_{l}
	\]
	de forma que $\lambda_{l} \in R[t]$ y $e_{l}$ es un generador de $R[t]^{n}$. Podemos
	suponer sin pérdida de generalidad que $x$ es homogéneo de algún grado $k$.
	Veamos por casos que $x$ es finitamente generado por los elementos de $H_{k}$.
	
	Supongamos que $k \leq N$ y que todos los escalares $\lambda_{l}$ son de grado
	$0$. Entonces, todos los $e_{l}$ que aparecen en $x$ son del mismo grado y por
	tanto, sus imágenes por $\mu$ son generadores de $M_{k}$. Es decir, $x$ está generado
	por $H_{k}$.
	
	Supongamos ahora qe $k \leq N$ y que algún $\lambda_{l}$ es de grado positivo.
	Por ser $x$ homogéneo, entonces $\lambda_{l}$ es de la forma $r_{l} t^{d_l}$,
	donde $r_{l} \in R$ y $d_{l} > 0$. Como el grado de $e_{l}$ es $k - d_{l}$,
	entonces existe un elemento $h_{l} \in H_{k-d_l,k}$ de la forma
	\[
	h_{l} = t^{d_l}e_{l} - \sum_{m=0}^{n_l}\tilde{\lambda}_{m}{e_l}_{m},
	\]
	donde todos los ${e_l}_{m}$ son de grado $k$ y $\tilde{\lambda}_{m} \in R$. Por
	consiguiente, en $x - r_{l} h_{l}$ el coeficiente de $e_{l}$ en $x$ es $0$ en
	$t$ y por tanto, sólo estamos introduciendo sumandos de grado $0$ en $t$.
	
	Iterando esta construcción para cada sumando de con coeficiente de grado positivo,
	obtenemos un elemento $x' = x - \sum_{w} r_{w} h_{w}$, donde $r_{w} \in R$,
	$h_{w} \in H$ y $x'$ tiene solamente coeficientes de grado $0$ en $t$. Esto es,
	$x = x' \sum_{w} r_{w} h_{w}$. Finalmente, aplicando la primera parte de la
	demostración tenemos que $x$ es generado por $H$.
	
	Para concluir, consideremos $k > N$. En dicho caso, cada $\lambda_{l}$ es de grado
	al menos $k - N$, pues el grado maximal de $e_{l}$ es $N$. Luego $x = t^{k-N}x'$,
	donde $x'$ es homogéneo de grado $N$. Como $0 = \mu(x) = t^{k-N}\mu(x')$,
	entonces $x' \in \ker \mu$. Por la segunda parte de la demostración,
	concluimos que $x'$ es generado por $H$ y por tanto, $x$ también.
\end{proof}

Para los siguientes dos lemas, fijaremos el $R[t]$-módulo graduado finitamente presentado
$\mathbf{M}= \oplus_{i \in \N}M_{i}$ junto con la aplicación $\mu : R[t]^{n} \to
\mathbf{M}$ cuyo núcleo es finitamente generado. Consideremos además el sistema
de generadores $G = \{g_{1}, \ldots, g_{n}\}$ de $\mathbf{M}$ y sea
$H = \{h_{1}, \ldots, h_{m}\}$ un sistema de generadores de $\ker \mu$. Además,
consideremos que tanto los elementos de $G$ como de $H$ son homogéneos del grado
del respectivo módulo. Finalmente, vamos a asumir que dichos elementos están
ordenados por grado en orden no decreciente.

\begin{lema}
	\label{lem:finit-pres} Cada $M_{i}$ de $\mathbf{M}$ es finitamente presentado como
	un $R$-módulo.
\end{lema}
\begin{proof}
	Veamos primero que $M_{i}$ es finitamente generado. Sea $d_{j}$ el grado de
	$g_{j}$ para $1 \leq j \leq n$. Sea $n_{i}$ el número de elementos de $G$ con grado
	menor o igual que $i$. Definamos $\mu_{i} : R^{n_i}\to M_{i}$ de forma que
	$\mu_{i}$ asigne al $j$-ésimo generador ${e_i}_{j} \in R^{n_i}$ el elemento
	$t^{i-d_j}g_{j}$. Cada elemento $x \in M_{i}$ es un componente homogéneo de grado
	$i$ de una combinación lineal de los generadores de $\mathbf{M}$. Dado que
	solo los generadores $g_{j}$ cuyo grado $d_{j}$ es menor o igual que $i$
	pueden contribuir a esta combinación para formar un elemento de grado $i$, $x$
	puede expresarse como:
	\[
	x = \sum_{d_j = 1}^{i} r_{j} t^{i-d_j}g_{j},
	\]
	donde $r_{j} \in R$. Considerando el generador en $R^{n_i}$ cuyas entradas son
	los coeficientes $r_{j}$, entonces
	\[
	\mu_{i}((r_{1}, \ldots, r_{n_i})) = \sum_{d_j = 1}^{i} r_{j} t^{i-d_j}g_{j} =
	x
	\]
	y en consecuencia, $\mu_{i}$ es sobreyectivo. Esto es, $M_{i}$ es finitamente generado.
	
	A continuación veamos que $\ker \mu_{i}$ también es finitamente generado. Sean
	$e_{1}, \ldots, e_{n}$ los generadores de $R[t]^{n}$ con imagen
	$g_{1}, \ldots, g_{n}$ por $\mu$ respectivamente. Sea $m_{i}$ el número de elementos
	$h_{j}$ de $H$ cuyo grado $d'_{j}$ es menor o igual que $i$. Para cada $h_{j}$
	tal que $1 \leq j \leq m_{i}$, consideremos $t^{i-d'_j}h_{j}$ que podemos reescribir
	como
	\[
	t^{i-d'_j}h_{j} = \sum_{k=1}^{m_i}r_{k} t^{i-d_k}e_{k}
	\]
	para ciertos $r_{k} \in R$. Definamos ahora
	\[
	{h_j}_{i} = \sum_{k=1}^{n_i}r_{k}{e_k}_{i}
	\]
	y definamos $H_{i} = \{{h_j}_{i} : 1 \leq i \leq m_{i} \}$. Veamos que $H_{i}$
	genera el núcleo de $\mu$. Es claro que $\mu_{i}({h_j}_{i}) = \mu(h_{j}) = 0$.
	Fijemos ahora un elemento arbitrario $x$ de $\ker \mu_{i}$. Tenemos entonces
	que $x$ es combinación lineal de $\{{e_1}_{i}, \ldots,{e_{n_i}}_{i}\}$ con
	coeficientes en $R$. Reemplazando ${e_j}_{i}$ por $t^{i-d_j}e_{j}$, obtenemos
	un elemento homogéneo $x' \in R[t]^{n}$ de grado $i$. Por hipótesis, podemos
	escribir $x'$ como combinación de elementos de $H$ de forma que
	\[
	x' = \sum_{k=1}^{m_i}r'_{k} t^{i-d'_k}{h_k}_{i}
	\]
	donde $r'_{k} \in R$. En consecuencia, veamos que
	\[
	x = \sum_{k=1}^{m_i}r'_{k}{h_k}_{i}.
	\]
	Para ello, procederemos comparando coeficientes. Consideremos
	$j \in \{1, \ldots, n_{i}\}$ y sea $c_{j} \in R$ el coeficiente de ${e_j}_{i}$
	en $x$. Sea $c'_{j}$ el coeficiente de ${e_j}_{i}$ en la suma de la expresión anterior,
	escribiendo cada ${h_k}_{i}$ como combinación lineal de los ${e_j}_{i}$. Por la
	construcción realizada, $c_{j}$ es el coeficiente de $t^{i-d_j}e_{j}$ en $x'$
	y $c'_{j}$ es el coeficiente de $t^{i-d_j}e_{j}$ en la suma
	$\sum_{k=1}^{m_i}r'_{k} t^{i-d'_k}{h_k}_{i}$. Esto es, $c_{j} = c'_{j}$. Como $x$
	se escogió de manera arbitraria de $\ker \mu_{i}$, entonces $H_{i}$ lo genera.
\end{proof}

\begin{lema}
	\label{lem:beta-finito-presentado} $\beta(\mathbf{M})$ es de tipo finito. En
	particular, es de tipo finitamente presentado.
\end{lema}
\begin{proof}
	Sea $N$ el grado máximo de los $g_{j} \in G_{j}$, $h_{k} \in H_{k}$ de forma
	que $1 \leq j \leq n$, $1 \leq k \leq m$. Veamos que la multiplicación por $t$
	induce un isomorfismo entre $M_{i}$ y $M_{i+1}$ para todo $i \geq N$.
	
	Si $y \in M_{i+1}$, entonces existen $\lambda_{j} \in R[t]$ de grado al menos $1$
	de forma que $y = \sum_{j=1}^{n} \lambda_{j} g_{j}$. Por tanto, $y = ty'$ donde
	$y' \in M_{i}$ mostrando que la multiplicación por $t$ es sobreyectiva.
	
	Para ver que es inyectiva, consideremos $y \in M_{i}$ de forma que $ty = 0$. Sea
	$x \in R[t]^{n}$ tal que $\mu(x) = y$. Entonces $\mu(tx) = ty = 0$ y por tanto,
	veamos $tx$ se puede escribir como
	\[
	tx = \sum_{j=0}^{m} \tilde{\lambda}_{j} h_{j},
	\]
	donde cada $\lambda_{j}$ no trivial es un polinomio de grado al menos $1$. Es
	inmediato, pues cada $h_{j}$ es de grado menor o igual que $N$ y $tx$ es de grado
	mayor o igual que $N+1$. En consecuencia, también podemos descomponer $tx$ como
	\[
	tx = \sum_{j=0}^{m} t \lambda_{j} h_{j} = t \sum_{j=0}^{m} \lambda_{j} h_{j}.
	\]
	Por ser $R[t]^{n}$ un módulo libre, tenemos que
	$x = \sum_{j=0}^{m} \lambda_{j} h_{j}$ y por tanto, $x \in \ker \mu$ lo que implica
	que $y = 0$.
\end{proof}

\begin{teorema}
	[Teorema de correspondencia] \label{teo:correspondence} Sea $R$ un anillo unitario.
	Entonces, existe un isomorfismo entre la categoría de $R[t]$-módulos graduados
	finitamente presentados y la categoría de módulos de persistencia discretos de
	tipo finitamente presentados.
\end{teorema}
\begin{proof}
	Estas categorías son subcategorías de $R[t]$-$\Cat{GrMod}$ y $R$-$\Cat{PersMod}$
	respectivamente. Restringiendo $\alpha$ y $\beta$ a dichas subcategorías, por el
	\autoref{lem:alpha-finito-presentado} tenemos que $\alpha$ es un funtor covariante
	de los módulos de persistencia discretos de tipo finitamente presentados a los
	$R[t]$-módulos graduados finitamente presentados. Así mismo, por los lemas \ref{lem:finit-pres}
	y \ref{lem:beta-finito-presentado}, $\beta$ es un funtor covariante de los
	$R[t]$-módulos graduados finitamente presentados a los módulos de persistencia
	discretos de tipo finitamente presentados. En consecuencia, estas
	subcategorías son isomorfas.
\end{proof}

El Teorema de descomposición de módulos graduados nos dará la clave para representar
los módulos de persistencia. La idea general de la prueba consiste en aplicar el
\nameref{teo:desc-cicl-prim} y ajustar la graduación mediante las potencias de $t$
para garantizar el isomorfismo. Dicha demostración hace uso de algunos
resultados que, por motivo de extensión, no trataremos en el presente trabajo. Para
el lector interesado en los detalles de la prueba véase \cite{webb1985decomposition}.

\begin{teorema}
	[Teorema de descomposición de módulos graduados] \label{teo:desc-mod-grad} Sea
	$F$ un cuerpo y sea $M$ un $F[t]$-módulo graduado finitamente generado.
	Entonces $M$ se descompone de manera única, salvo isomorfismos, como
	\[
	M \cong \left( \bigoplus_{i=1}^{n-m}t^{a_i}\cdot R[t] \right) \oplus \left( \bigoplus
	_{j=1}^{m}t^{b_j}\cdot \frac{R[t]}{\langle t^{c_j}\rangle}\right),
	\]
	donde $a_{i}, b_{j}, c_{j} \in \mathbb{N}$, y para cada $j$, $t^{c_j}$ es un
	elemento homogéneo tal que divide a $t^{c_{j+1}}$.
\end{teorema}

De manera intuitiva, primero hemos construido una única estructura algebraica que
contiene todos los complejos de la filtración. Después, hemos seguido calculando
una suma directa a partir de los complejos, llegando así a un espacio mucho más
grande que está graduado según el orden inducido de la filtración. A continuación,
recordamos el momento en que cada símplice entra utilizando un coeficiente
polinomial, codificando así el orden de filtración en el anillo de coeficientes polinomiales.
A partir de aquí, el \nameref{teo:desc-mod-grad} nos proporciona una
factorización simple cuando el anillo base es un cuerpo $F$. Aquí, el anillo
graduado $F[t]$ es un DIP y sus únicos ideales graduados son homogéneos de la
forma $\langle t^{n} \rangle = t^{n} \cdot R[t]$, donde $n \geq 0$. Ahora que
hemos transformado los módulos de persistencia discretos en objetos más
manejables, queremos parametrizar las clases de isomorfismo de los $F[t]$-módulos
por objetos más sencillos de interpretar.

\begin{definicion}
	Definimos un \textbf{$\mathcal{P}$-intervalo} como un par ordenado $(i, j)$
	tal que $0 \leq i \leq j$ donde
	$i,j \in \mathbb{Z}^{\infty}= \mathbb{Z}\cup \{+\infty\}$.
\end{definicion}

Nuestro objetivo ahora es asociar un $F[t]$-módulo graduado a un conjunto $S$ de
$\mathcal{P}$-intervalos mediante una biyección $Q$. Para ello, definimos
$Q(i, j) = t^{i} F[t]/\langle t^{j-i}\rangle$ para el $\mathcal{P}$-intervalo
$(i, j)$. Es claro que $Q(i, +\infty) = t^{i} F[t]$. Además, para un conjunto de
$\mathcal{P}$-intervalos $S = \{(i_{1}, j_{1}), (i_{2}, j_{2}), \ldots, (i_{n}, j
_{n})\}$, definimos
\[
Q(S) = \bigoplus_{l=1}^{n} Q(i_{l}, j_{l}).
\]
Nuestra correspondencia puede ahora redefinirse como sigue.

\begin{corolario}
	La correspondencia $S \mapsto Q(S)$ define una biyección entre los conjuntos finitos
	de $\mathcal{P}$-intervalos y los módulos graduados finitamente generados sobre
	el anillo graduado $F[t]$. Consecuentemente, las clases de isomorfismo de los módulos
	de persistencia de tipo finito sobre $F$ están en correspondencia biyectiva con
	los conjuntos finitos de $\mathcal{P}$-intervalos.
\end{corolario}

En el contexto de filtraciones de complejos simpliciales, este resultado da
lugar a la descripción de la homología persistente conocida como \textbf{código
	de barras}: cada $\mathcal{P}$-intervalo $(i, j)$ describe un ciclo que nace en
la filtración $i$ y especifica su clase de homología hasta que se convierte en
borde (es decir, hasta que muere) en el instante $j$.% Finalmente, condensamos el desarrollo realizado en el siguiente corolario:

%\begin{corolario}
%	Sea $F$ un cuerpo, $K$ un complejo simplicial y $\mathcal{F}$ una filtración discreta finita asociada. Entonces, el $(i,j)$-ésimo número de Betti de persistencia de dimensión $k$, \(\beta_k^{i \to j}\), es igual al número de $\mathcal{P}$-intervalos $(i,j)$ asociados al módulo de homología persistente $k$-dimensional  \(H_k^{i \to j}(\mathcal{F};F)\).
%\end{corolario}
%\begin{proof}
%	Consideremos la filtración $\mathcal{F}$ del complejo simplicial $K$, donde cada subcomplejo $K^i$ está incluido en $K^{i+1}$ para cada $i \in \{0, \ldots, N\}$, siendo $K^0=\emptyset$ y $K^N = K$. Asociamos a cada $K^i$ el complejo de cadenas orientado $C_\bullet(K^i; F)$ y, así mismo, su módulo de homología $H_k(K^i; F)$ de dimensión $k$.
%
%	Esta sucesión de homologías induce un módulo de homología persistente $H_k^\bullet(\mathcal{F}; F)$, definido por los homomorfismos inducidos por las inclusiones:
%	\[
%	H_k(K^0; F) \to H_k(K^1; F) \to \dots \to H_k(K^N; F).
%	\]
%	Dado que estamos trabajando con módulos sobre un cuerpo $F$, tenemos que los módulos de homología tienen estructura de espacio vectorial. En consecuencia, escogemos generadores para cada $H_k(K^i; F)$. Para ello, seleccionamos una base inicial en $H_k(K^0; F)$ y la adaptamos adecuadamente extendiéndola o reduciéndola a lo largo de la filtración para reflejar la evolución de las clases de homología.
%
%	Aplicando el \nameref{teo:correspondence} y el \nameref{teo:desc-mod-grad} sobre el anillo de polinomios $F[t]$, podemos descomponer el módulo de persistencia como sigue:
%	\[
%	H_k^\bullet(\mathcal{F}; F) \cong \left( \bigoplus_{i=1}^{n-m} t^{a_i} \cdot F[t] \right) \oplus \left( \bigoplus_{j=1}^m t^{b_j} \cdot \frac{F[t]}{ \langle t^{c_j} \rangle} \right)
%	\]
%	donde los números \(a_i\) y \(b_j\) indican en qué parte del complejo aparece un ciclo de dimensión \(k\), mientras que \(c_j\) nos dice cuándo desaparece un ciclo que había surgido en \(b_j\). La duración de una clase de homología, que surge en \(b_j\) y termina en \(c_j\), se mide simplemente como \(b_j - c_j\). Por otro lado, los elementos libres nos dan información sobre los ciclos que comienzan en \(a_i\) y no terminan. Aplicando la imagen inversa por $\mathcal{Q}$ de la descomposición obtenemos los $\mathcal{P}$-intervalos requeridos.
%%
%%	Es decir, hemos demostrado que el número de Betti de persistencia $\beta_k^{i \to j}$, definido como el rango de $H_k^\bullet(\mathcal{F};F)$ desde el tiempo $i$ hasta justo antes de $j$, es igual al número de $\mathcal{P}$-intervalos $(i, j)$ que muestra la descomposición.
%\end{proof}
%
%\begin{ejemplo}
%	La \autoref{fig:barcode} muestra un ejemplo de representación en código de barras. En este caso, los números de Betti de persistencia nos darían $\beta_0^{0 \to 1} = 6$, $\beta_1^{1 \to \sqrt{3}} = 1$ y $\beta_2^{\sqrt{3} \to 2} = 1$. Es decir, la persistencia nos estaría indicando que, al comienzo de la filtración aparecen $6$ componentes conexas que en el instante $1$ de tiempo se fusionan en una sola con un agujero unidimensional. A continuación, dicho agujero desaparece en el instante $3$ para dar lugar a un vacío bidimensional con un breve periodo de vida.
%\end{ejemplo}
%
Los diagramas de código de barras derivados de la homología persistente han
demostrado ser herramientas muy útiles para la extracción de información
relevante en campos como el análisis de datos. Este hecho es debido a ciertas propiedades
de la homología persistente que la hacen relevante para un uso práctico. Además
de las ya comentadas, también destaca su \textbf{estabilidad}, lo que significa que
pequeñas perturbaciones en los datos no producen grandes cambios en los diagramas
de código de barras. Esto es fundamental porque asegura que las características
topológicas que se detectan son inherentes a la estructura de los datos y no
consecuencia del ruido o de variaciones menores
\cite{cohen2005stability, chazal2016structure}. Dicha propiedad contrasta con la
homología simplicial, pues dos complejos de \u Cech o Vietoris-Rips con distinto
radio pueden tener invariantes topológicos completamente diferentes. Estos métodos
no solo ofrecen una representación visual intuitiva de las características de los
datos, sino que también proporcionan un marco robusto para la cuantificación y
comparación de estructuras topológicas.

\begin{figure}[h]
	\label{fig:barcode}
	\centering
	\begin{tikzpicture}[scale=3]
		% Vertical dashed lines
		\draw[dashed, lightgray] (0,0) -- (0,.8); \draw[dashed, lightgray] (1,0) --
		(1,.8); \draw[dashed, lightgray] ({sqrt(3)},0) -- ({sqrt(3)},.9); \draw[dashed,
		lightgray] (2,0) -- (2,.9);
		% Axis
		\draw [-latex,shorten >=-3pt] (-.2,0) -- (3,0) node [below] {$\phantom{\sqrt{3}}t\phantom{\sqrt{3}}$};
		\foreach \x in {0,1,2} \node at (\x,0) {$\bullet$} node at (\x,0) [below]
		{$\phantom{\sqrt{3}}\x\phantom{\sqrt{3}}$}; \node at ({sqrt(3)},0) {$\bullet$}
		node at ({sqrt(3)},0) [below] {$\sqrt{3}$};
		% Horizontal lines
		\draw [{*[fill=white]}-,shorten <=-2.4pt] (0,.2) -- (3,.2); \foreach \x in {.3,.4,...,.7}
		\draw [{*[fill=white]}-*,shorten >=-2.4pt,shorten <=-2.4pt] (0,\x) -- (1,\x)
		; \draw [{*[fill=white]}-*,shorten >=-2.4pt,shorten <=-2.4pt] (1,.8) -- node
		[above] {$H_{1}$} ({sqrt(3)},.8) ; \draw [{*[fill=white]}-*,shorten >=-2.4pt,shorten
		<=-2.4pt] ({sqrt(3)},.9) -- node [above] {$H_{2}$} (2,.9) ; \draw (0,.1) -- (-.1,.1)
		-- node[left]{$H_{0}$} (-.1,.8) -- (0,.8);
	\end{tikzpicture}
	\caption{Ejemplo de código de barras mostrando la persistencia de clases de
		homología a través del tiempo. Cada línea horizontal representa una clase de homología
		que persiste a través de un intervalo específico, indicando la escala a la que
		dichas características topológicas son detectables.}
\end{figure}

\endinput
%--------------------------------------------------------------------
% FIN DEL CAPÍTULO.
%--------------------------------------------------------------------

\cleardoublepage\part{Informática}
% !TeX root = ../tfg.tex
% !TeX encoding = utf8

\chapter{Redes neuronales convolucionales}

Las \textbf{redes neuronales convolucionales} (CNNs) \cite{krizhevsky2012imagenet} son un tipo de modelo de ANN fundamental en el campo de la visión artificial y el procesamiento de imágenes. Fue introducido por Yann LeCun et al. en 1998 \cite{lecun1998gradient} y desde entonces ha sido ampliamente utilizado en una gran variedad de aplicaciones, desde problemas como la detección de objetos hasta la segmentación de imágenes.

\section{La corteza visual y el neocognitrón}

La comprensión del funcionamiento de la corteza visual en los seres humanos y otros animales ha sido una fuente de inspiración significativa para el desarrollo de algoritmos en el campo del aprendizaje profundo, especialmente en el diseño de CNNs. La corteza visual, ubicada en el lóbulo occipital del cerebro, es fundamental para el procesamiento de información visual. Estudios realizados por Hubel y Wiesel en la década de 1960 demostraron que ciertas neuronas en la corteza visual responden preferentemente a bordes específicos y orientaciones espaciales dentro de una región visual limitada \cite{hubel1962receptive}. Estas neuronas, conocidas como \textbf{células de orientación selectiva}, muestran una organización jerárquica que permite la percepción compleja a partir de la combinación de respuestas simples.

Inspirado en estas observaciones, Kunihiko Fukushima desarrolló el \textbf{neocognitrón} en 1980, una red neuronal que es considerada uno de los precursores de las modernas CNNs \cite{fukushima1980neocognitron}. El neocognitrón fue diseñado para reconocer patrones visuales complejos de manera robusta frente a traslaciones y otras pequeñas distorsiones de la imagen. Esta red consta de múltiples capas que alternan entre capas convolucionales, que detectan características locales y capas que agregan las respuestas de los detectores de características sobre áreas locales (\autoref{fig:neocognitron}). La estructura de esta red capta de manera efectiva la forma en que la corteza visual procesa la información visual, implementando una forma primitiva de invarianza a la traslación y la capacidad de extraer características jerárquicas.

\begin{figure}[h]
	\centering
	\includegraphics[width=110mm]{img/neocognitron.png}
	\caption{Diagrama del neocognitrón mostrando el flujo de procesamiento de una imagen a través de varias capas. Las capas \(U_{S}\) (simple) y \(U_{C}\) (compleja) procesan la información de forma jerárquica, comenzando con la imagen original \(U_0\) y extrayendo características cada vez más complejas en \(U_{S1}\), \(U_{C1}\), \(U_{S2}\), \(U_{C2}\), \(U_{S3}\) y \(U_{C3}\). Fuente \cite{fukushima1980neocognitron}.}
	\label{fig:neocognitron}
\end{figure}

La influencia de la corteza visual y el neocognitrón en el diseño de las CNNs es una muestra de la utilidad de estudios interdisciplinarios entre neurociencia y aprendizaje automático. Estos estudios no solo han facilitado avances tecnológicos en visión artificial sino que también han ofrecido nuevas perspectivas sobre cómo los seres humanos procesamos la información visual, proponiendo un puente entre la inteligencia artificial y la biológica \cite{serre2007feedforward}.


\section{Arquitectura de una CNN}

Las CNNs, introducidas por Yann LeCun, supusieron un avance significativo respecto al neocognitrón. Mientras que el neocognitrón sentó las bases al proponer una arquitectura inspirada en el procesamiento visual del cerebro, LeCun implementó un método de entrenamiento supervisado utilizando retropropagación, lo que permitió mejorar notablemente el rendimiento y la capacidad de generalización de las redes neuronales \cite{lecun1998gradient}. Además, LeCun sentó las bases de la arquitectura de CNNs, optimizando la eficiencia y escalabilidad de los modelos, incorporando capas como las capas convolucionales y de muestreo \cite{lecun1989backpropagation}.

\begin{figure}
% DEEP CONVOLUTIONAL NEURAL NETWORK
\centering
\begin{tikzpicture}[x=1.3cm,y=0.8cm]
	\large
	\message{^^JDeep convolution neural network}
	\readlist\Nnod{5,5,4,3,2,4,4,3} % array of number of nodes per layer
	\def\NC{6} % number of convolutional layers
	\def\nstyle{int(\lay<\Nnodlen?(\lay<\NC?min(2,\lay):3):4)} % map layer number on 1, 2, or 3
	\tikzset{ % node styles, numbered for easy mapping with \nstyle
		node 1/.style={node in},
		node 2/.style={node convol},
		node 3/.style={node hidden},
		node 4/.style={node out},
	}
	
	% TRAPEZIA
	\draw[myorange!40,fill=myorange,fill opacity=0.02,rounded corners=2]
	%(1.6,-2.5) rectangle (4.4,2.5);
	(1.6,-2.7) --++ (0,5.4) --++ (3.8,-1.9) --++ (0,-1.6) -- cycle;
	\draw[myblue!40,fill=myblue,fill opacity=0.02,rounded corners=2]
	(5.6,-2.0) rectangle++ (1.8,4.0);
	\node[font=\small,right=19,above=10,align=center,myorange!60!black] at (3.1,1.8) {Bloques\\[-0.2em]convolucionales};
	\node[font=\small,above=10,align=center,myblue!60!black] at (6.5,1.9) {Capas ocultas\\[-0.2em]totalmente conectadas};
	
	\message{^^J  Layer}
	\foreachitem \N \in \Nnod{ % loop over layers
		\def\lay{\Ncnt} % alias of index of current layer
		\pgfmathsetmacro\prev{int(\Ncnt-1)} % number of previous layer
		%\pgfmathsetmacro\Nprev{\Nnod[\prev]} % array of number of nodes in previous layer
		\message{\lay,}
		\foreach \i [evaluate={\y=\N/2-\i+0.5; \x=\lay; \n=\nstyle;}] in {1,...,\N}{ % loop over nodes
			%\message{^^J  Layer \lay, node \i}
			
			% NODES
			\node[node \n,outer sep=0.6] (N\lay-\i) at (\x,\y) {};
			
			% CONNECTIONS
			\ifnum\lay>1 % connect to previous layer
			\ifnum\lay<\NC % convolutional layers
			\foreach \j [evaluate={\jprev=int(\i-\j); \cconv=int(\Nnod[\prev]>\N); \ctwo=(\cconv&&\j>0);
				\c=int((\jprev<1||\jprev>\Nnod[\prev]||\ctwo)?0:1);}]
			in {-1,0,1}{
				\ifnum\c=1
				\ifnum\cconv=0
				\draw[connect,white,line width=1.2] (N\prev-\jprev) -- (N\lay-\i);
				\fi
				\draw[connect] (N\prev-\jprev) -- (N\lay-\i);
				\fi
			}
			
			\else % fully connected layers
			\foreach \j in {1,...,\Nnod[\prev]}{ % loop over nodes in previous layer
				\draw[connect,white,line width=1.2] (N\prev-\j) -- (N\lay-\i);
				\draw[connect] (N\prev-\j) -- (N\lay-\i);
			}
			\fi
			\fi % else: nothing to connect first layer
			
		}
	}
	
	% LABELS
	\node[font=\small,above=3,align=center,mygreen!60!black] at (N1-1.90) {Capa de\\[-0.2em]entrada};
	\node[font=\small, above=0,align=center,myred!60!black] at (N\Nnodlen-1.90) {Capa de\\[-0.2em]salida};
	
\end{tikzpicture}
\caption{Arquitectura de una CNN. Muestra la secuencia de capas en la red, iniciando con la capa de entrada, seguida por múltiples bloques convolucionales (resaltados en naranja), continuando con capas ocultas totalmente conectadas (resaltadas en azul) y finalmente la capa de salida. }
\end{figure}

Una CNN típicamente consiste en una secuencia de capas que transforman la entrada de imagen bruta en representaciones cada vez más abstractas y útiles para la tarea en cuestión. Cada tipo de capa dentro de una CNN tiene un propósito específico y contribuye de manera distinta al proceso de aprendizaje. Generalmente, las distintas capas de una CNN suelen agruparse para cumplir distintas funciones. Las principales agrupaciones que componen una CNN son:

\begin{itemize}
	\item \textbf{Capa de entrada}: Es la primera capa de la red, donde se introduce la imagen original o preprocesada. Su función principal es preparar y escalar la imagen para las operaciones de las capas siguientes.
	
	\item \textbf{Bloques convolucionales}: Estos bloques contienen una o más capas convolucionales seguidas frecuentemente por capas de normalización y funciones de activación. Cada capa convolucional aplica diferentes filtros a la entrada para crear mapas de características que resalten aspectos específicos de la imagen. Estos bloques suelen ir consecutivos intercalando capas de muestreo. 
	
	\item \textbf{Capas totalmente conectadas}: Después de varias capas convolucionales y de muestreo, la información en forma de matrices y tensores se aplana en vectores y se pasa a través de capas FC. Estas capas integran la información aprendida por las capas anteriores para realizar la clasificación final.
	
	\item \textbf{Capa de salida}: La última capa de una CNN, donde se obtiene el resultado final. En tareas de clasificación, esta capa suele usar una función de activación como la Softmax para asignar probabilidades a las distintas clases posibles.
\end{itemize}

A continuación, vamos a explorar las diferentes capas que componen estos bloques y cómo trabajan juntas para detectar y aprender patrones importantes en los datos.

\subsection{Capa convolucional}

La \textbf{capa convolucional} es el bloque de construcción fundamental de una CNN. Utiliza un conjunto de filtros que se aplican a la entrada mediante el operador de \textbf{convolución discreta}. Cada filtro detecta características específicas en una región local de la entrada. El operador de convolución discreta se puede expresar como:
\begin{equation}
	S(i, j) = (I \ast K)(i, j) = \sum_m \sum_n I(m, n) K(i-m, j-n),
\end{equation}
donde $m,n$ son las coordenadas en la imagen o matriz de entrada \(I\) de la región a convolucionar, $i,j$ son las coordenadas del centro del \textbf{kernel} o \textbf{filtro} \(K\) y \(S\) es la matriz de salida o \textbf{mapa de características}. La convolución discreta es un \textbf{operador lineal}, lo que implica que satisface propiedades como la \textbf{conmutatividad}, \textbf{asociatividad} y \textbf{distributividad}.

Estos filtros se desplazan sobre toda la superficie de la entrada, generando un mapa de características que resume la presencia de las particularidades de dicha entrada.

\usetikzlibrary{positioning, calc, decorations.pathreplacing}
\begin{figure}[H]
	\centering
\begin{tikzpicture}[
	2d-arr/.style={matrix of nodes, row sep=-\pgflinewidth, column sep=-\pgflinewidth, nodes={draw, text height=1.0ex, text depth=0.25ex}}
	]
	
	\matrix (mtr) [2d-arr] {
		0 & 1 & 1 & |[fill=orange!30]| 1 & |[fill=orange!30]| 0 & |[fill=orange!30]| 0 & 0\\
		0 & 0 & 1 & |[fill=orange!30]| 1 & |[fill=orange!30]| 1 & |[fill=orange!30]| 0 & 0\\
		0 & 0 & 0 & |[fill=orange!30]| 1 & |[fill=orange!30]| 1 & |[fill=orange!30]| 1 & 0\\
		0 & 0 & 0 & 1 & 1 & 0 & 0\\
		0 & 0 & 1 & 1 & 0 & 0 & 0\\
		0 & 1 & 1 & 0 & 0 & 0 & 0\\
		1 & 1 & 0 & 0 & 0 & 0 & 0\\
	};
	
	\node[below=of mtr-5-4] {$\mathbf I$};
	
	\node[right=0.2em of mtr] (str) {$*$};
	
	\matrix (K) [2d-arr, right=0.2em of str, nodes={draw, fill=teal!30}] {
		1 & 0 & 1 \\
		0 & 1 & 0 \\
		1 & 0 & 1 \\
	};
	\node[below=of K-3-2] {$\mathbf K$};
	
	\node[right=0.2em of K] (eq) {$=$};
	
	\matrix (ret) [2d-arr, right=0.2em of eq] {
		1 & 4 & 3 & |[fill=blue!80!black!30]| 4 & 1\\
		1 & 2 & 4 & 3 & 3\\
		1 & 2 & 3 & 4 & 1\\
		1 & 3 & 3 & 1 & 1\\
		3 & 3 & 1 & 1 & 0\\
	};
	\node[below=of ret-4-3] {$\mathbf{I * K}$};
	
	\draw[dashed, teal] (mtr-1-6.north east) -- (K-1-1.north west);
	\draw[dashed, teal] (mtr-3-6.south east) -- (K-3-1.south west);
	
	\draw[dashed, blue!80!black] (K-1-3.north east) -- (ret-1-4.north west);
	\draw[dashed, blue!80!black] (K-3-3.south east) -- (ret-1-4.south west);
	
	\foreach \i in {1,2,3} {
		\foreach \j in {4,5,6} {
			\node[font=\tiny, scale=0.6, shift={(-1.2ex,-2ex)}] at (mtr-\i-\j) {$\times \pgfmathparse{int(mod(\i+\j,2))}\pgfmathresult$};
		}
	}
	
\end{tikzpicture}
\caption{Ejemplo de convolución discreta. La matriz de entrada $\mathbf{I}$ se muestra con un segmento resaltado en naranja, mostrando la región afectada por el filtro $\mathbf{K}$, en color verde. La matriz resultante $\mathbf{I * K}$ destaca los valores resultantes de la convolución, con el resultado de la convolución en dicha región en azul.}
\end{figure}

Además de las propiedades del operador de convolución, las capas convolucionales muestran varias propiedades más que las hacen especialmente adecuadas para tareas de procesamiento de imágenes y visión artificial. Entre estas propiedades destacan:

\begin{itemize}
	\item \textbf{Conectividad local:}
	Cada neurona en una capa convolucional está conectada solo a un pequeño número de neuronas cercanas en la capa anterior. Esta estructura imita la manera en que los campos receptivos en el sistema visual humano se organizan, concentrándose en pequeñas regiones del espacio visual. La conectividad local permite a la red detectar características locales de la entrada sin la influencia de la estructura global, reduciendo la complejidad y el número total de parámetros necesarios.
	
	\item \textbf{Compartición de parámetros:}
	En una CNN, el mismo filtro se utiliza para cada posición de la entrada, a diferencia de una FNN donde cada peso es único para cada conexión. Esta compartición de parámetros permite que la red sea más eficiente en términos de memoria y computación. Además, también implica que las características aprendidas por un filtro son útiles en toda la imagen, lo que mejora la eficiencia del aprendizaje y ayuda a la red a generalizar mejor.
	
	\item \textbf{Equivarianza frente a traslaciones:}
	Debido al uso de la misma función de convolución a lo largo de toda la entrada, las CNNs son naturalmente equivariantes a las traslaciones. Esto significa que si la entrada se traslada, las características detectadas por la red también se trasladarán de manera correspondiente. Esta propiedad es particularmente interesante en tareas de visión artificial, donde la relevancia de una característica no suele depender de su posición específica en el espacio de entrada.
\end{itemize}

Los hiperparámetros de \textbf{paso} o \textbf{\textit{stride}}, y \textbf{relleno} o \textbf{\textit{padding}} son especialmente relevantes en la manipulación dimensional durante la convolución en CNNs. El \textit{stride} define el paso con el que el filtro se desplaza sobre la imagen o mapa de características, afectando la reducción dimensional del mapa resultante y permitiendo la captura de características a diversas escalas. Por otro lado, el \textit{padding} consiste en añadir píxeles artificiales alrededor de la imagen de entrada, lo que permite que el filtro acceda completamente a los bordes y mantiene el tamaño del volumen de salida, preservando la información en los bordes cruciales para la interpretación completa de la imagen.

\begin{figure}[H]
	\centering
	\includegraphics[width=100mm]{img/stride.png}
	\caption{Convolución de un filtro de 3x3 sobre una entrada de 5x5. La operación se realiza utilizando \textit{strides} de 2x2 y sin agregar relleno (\textit{padding} = 0), lo que resulta en una matriz de salida más pequeña y eficientemente espaciada. Fuente \cite{Dumoulin2016AGT}.}
\end{figure}
\begin{figure}[H]
	\centering
	\includegraphics[width=100mm]{img/padding.png}
	\caption{Convolución de un filtro de 3x3 sobre una entrada de 5x5. Se emplea un relleno de 2 unidades y pasos unitarios (\textit{stride} = 1), asegurando que la matriz de salida aumente las dimensiones de la matriz de entrada, al expandir el borde de la imagen original. Fuente \cite{Dumoulin2016AGT}.}
\end{figure}

La elección de dichos hiperparámetros influye directamente en las dimensiones del mapa de características de salida, cuya altura \(H_{\text{out}}\) y anchura \(W_{\text{out}}\) vienen dadas de la siguiente manera:
\[
H_{\text{out}} = \left\lfloor \frac{H_{\text{in}} + 2P - H_{\text{f}}}{S} + 1 \right\rfloor, \quad
W_{\text{out}} = \left\lfloor \frac{W_{\text{in}} + 2P - W_{\text{f}}}{S} + 1 \right\rfloor,
\]
donde \(H_{\text{in}}\) y \(W_{\text{in}}\) son las dimensiones de entrada, \(H_{\text{f}}\) y \(W_{\text{f}}\) son las del filtro, \(P\) es el \textit{padding}, \(S\) es el \textit{stride} y $\lfloor \cdot \rfloor$ denota la función suelo. La profundidad del mapa de salida, \(D_{\text{out}}\), está determinada por el número de filtros aplicados, donde \( D_{\text{out}} = \text{número de filtros} \).

Estas convoluciones, aunque predominantemente asociadas con dimensiones bidimensionales, pueden extenderse a contextos tridimensionales (3D) donde se aplican filtros 3D sobre varios canales al mismo tiempo. Por ejemplo, un filtro de tamaño $1 \times 1 \times 1$ puede transformar linealmente los mapas de características en cada ubicación del volumen de entrada. Las dimensiones de salida se mantienen como:
\[
H_{\text{out}} = H_{\text{in}}, \quad W_{\text{out}} = W_{\text{in}}, \quad D_{\text{out}} = \text{número de filtros}.
\]
Nótese que estas convoluciones son importantes para ajustar la dimensionalidad de los canales dentro de redes profundas, pues con una sola \textbf{convolución $1 \times 1$} podemos colapsar todos los canales de entrada en uno solo. Esto supone una reducción efectiva de parámetros y de la complejidad computacional en datos tridimensionales.

Otro tipo de convoluciones que surge en este contexto son las \textbf{convoluciones en profundidad} o \textbf{\textit{depthwise convolutions}}. En ellas, cada filtro se aplica de manera independiente a cada canal del volumen de entrada, permitiendo un procesamiento separado de las dimensiones espaciales y de profundidad. La fórmula para las dimensiones de salida se mantiene, pero esta vez manteniendo el número de canales:
\[
H_{\text{out}} = \left\lfloor \frac{H_{\text{in}} + 2P - H_{\text{f}}}{S} + 1 \right\rfloor, \quad W_{\text{out}} = \left\lfloor \frac{W_{\text{in}} + 2P - W_{\text{f}}}{S} + 1 \right\rfloor, \quad D_{\text{out}} = D_{\text{in}}.
\]
Estas convoluciones son eficaces para optimizar el rendimiento computacional en aplicaciones donde el manejo eficiente de los recursos es esencial.

\subsection{Capa de muestreo}

Las capas de \textbf{muestreo}, conocidas generalmente como capas de \textbf{\textit{pooling}}, buscan reducir la dimensionalidad espacial de los mapas de características, lo que permite disminuir la cantidad de parámetros y de cómputo en la red al mismo tiempo que se mantiene la información más relevante. Las capas de \textit{pooling} más comunes son las de \textbf{máximo} o \textbf{\textit{max pooling}}, y las de \textbf{promedio} o \textbf{\textit{average pooling}}. El \textit{max pooling} toma la mayor activación en la ventana del filtro como:
\begin{equation}
	P(i, j) = \max_{m, n \in W} I(i+m, j+n),
\end{equation}
donde \(W\) es la ventana del filtro de \textit{pooling}. Esta operación ayuda a hacer la representación obtenida invariante a pequeñas traslaciones y distorsiones. Por otro lado, el \textit{average pooling} calcula el promedio de las activaciones dentro de la ventana del filtro, proporcionando una representación que suaviza las características de entrada:
\begin{equation}
	P(i, j) = \frac{1}{|W|} \sum_{a, b \in W} I(i+a, j+b),
\end{equation}
donde \(|W|\) es el número de elementos en la ventana del filtro. Ambas técnicas de pooling tienen sus aplicaciones específicas dependiendo de la naturaleza del problema y de la arquitectura de la red. Mientras que el \textit{max pooling} es generalmente preferido para tareas relacionadas con la detección de características debido a su capacidad para preservar las activaciones más fuertes, el \textit{average pooling} puede ser más adecuado para tareas donde la uniformidad de las características es más importante \cite{krizhevsky2012imagenet}.

\newcommand{\matrixA}{%     
	\setlength{\tabcolsep}{0pt}
	\begin{NiceTabular}{*{4}{c}}[hvlines,rules/width=2pt, cell-space-limits=1.0ex,columns-width=4ex]
		\CodeBefore % color the blocks
		\rectanglecolor{blue!15}{1-1}{2-2}
		\rectanglecolor{green!15}{3-1}{4-2}
		\rectanglecolor{orange!25}{1-3}{2-4}
		\rectanglecolor{red!15}{3-3}{4-4}
		\Body   
		\RowStyle[nb-rows=4]{\bfseries} % make all bold
		8&7&5&3\\
		12&9&5&7\\
		13&2&10&3\\
		9&4&5&14\\      
	\end{NiceTabular}
}

\newcommand{\matrixB}{% 
	\setlength{\tabcolsep}{0pt}
	\begin{NiceTabular}{*{2}{c}}[hvlines,rules/width=1.6pt,cell-space-limits=1.0ex,columns-width=4ex]
		\CodeBefore % color the cells
		\cellcolor{blue!15}{1-1}
		\cellcolor{orange!15}{1-2}
		\cellcolor{green!15}{2-1}
		\cellcolor{red!15}{2-2}
		\Body   
		\RowStyle[nb-rows=2]{\bfseries} % make all bold
		12&7\\
		13&14\\     
	\end{NiceTabular}
}

\newcommand{\matrixC}{% 
	\setlength{\tabcolsep}{0pt}
	\begin{NiceTabular}{*{2}{c}}[hvlines,rules/width=1.6pt,cell-space-limits=1.0ex, columns-width=4ex]
		\CodeBefore % color the cells
		\cellcolor{blue!15}{1-1}
		\cellcolor{orange!15}{1-2}
		\cellcolor{green!15}{2-1}
		\cellcolor{red!15}{2-2}
		\Body   
		\RowStyle[nb-rows=2]{\bfseries} % make all bold
		9&5\\
		7&8\\       
	\end{NiceTabular}
}

\begin{figure}
	\centering
\begin{tikzpicture}
	% layout the matrices
	\node (matA) {\matrixA};
	\node[ above right = -30pt and 150pt of matA, scale=1.2, anchor = south west] (matB) {\matrixB};
	\node[ below right = -20pt and 150pt of matA, scale=1.2, anchor =north  west] (matC) {\matrixC};
	% a large parenthesis
	\node (paren) [right = 120pt of matA] {$\left(\rule{0pt}{100pt}\right.$};
	% the arrow
	\draw[-latex,ultra thick,  shorten >=3mm, shorten <=2mm,] (matA.east) -- (paren.center)  node[midway,above, text width=3cm,     font= \bfseries, text centered] {2x2 pooling, stride 2};    
	% add the labels
	\node[above =  -3pt of matB.north,font=\bfseries]{Max pooling};
	\node[above =  -3pt of matC.north,font=\bfseries]{Average pooling};
\end{tikzpicture}  
\caption{Comparación de técnicas de \textit{pooling}. Arriba, \textit{max pooling} y abajo, \textit{average pooling}, ambas con un filtro de 2x2 y \textit{stride} de 2, mostrando la transformación de la matriz original.}
\end{figure}

%La \textbf{normalización} es un paso crucial en muchas CNNs que ayuda a acelerar la convergencia del entrenamiento y reduce la sensibilidad a la inicialización de los parámetros de la red. La normalización por lotes es una técnica común que normaliza las salidas de la capa anterior por mini-batch, ajustando y escalando las activaciones para tener una media cero y una varianza unitaria.
%
%\subsubsection{Capa totalmente conectada}
%
%Las capas FC en las CNNs desempeñan un papel importante al transformar las características extraídas por las capas convolucionales y de \textit{pooling} en predicciones finales. Al final de la arquitectura de la red, estas capas funcionan como clasificadores, interpretando las características y tomando decisiones basadas en ellas para clasificar la entrada en diferentes categorías. Además, en la \textbf{transferencia de aprendizaje} o \textbf{\textit{transfer learning}}, estas capas pueden ser reemplazadas o ajustadas para adaptarse a nuevas tareas, mientras que las capas convolucionales anteriores se utilizan para aprovechar las características genéricas aprendidas.

\section{Modelos y estado del arte en CNNs}

Desde su origen, las arquitecturas de CNNs han experimentado un desarrollo significativo, marcado por una serie de innovaciones que han mejorado su rendimiento y eficiencia. Como vimos anteriormente, la historia de las CNNs comenzó con el neocognitrón de Fukushima en los años 80 \cite{fukushima1980neocognitron}, un modelo novedoso que introdujo el concepto de capas convolucionales y de \textit{pooling}. Posteriormente, la introducción de LeNet-5 por LeCun et al. en los años 90 \cite{lecun1998gradient} demostró la eficacia de las CNNs en tareas de reconocimiento de dígitos y documentos, mostrando que este tipo de arquitecturas podían aprender a resolver problemas de aprendizaje supervisado. Posteriormente AlexNet, desarrollada por Krizhevsky et al. en 2012 \cite{krizhevsky2012imagenet}, revolucionó el campo del aprendizaje profundo, utilizando técnicas como paralización del entrenamiento, profundización de la red, ReLU y \textit{dropout} para ganar el desafío de ImageNet con una reducción drástica en la tasa de error. A partir de ahí, surgieron modelos más sofisticados como VGG \cite{simonyan2014very}, GoogLeNet con su módulo Inception \cite{szegedy2015going}, o ResNet, que introdujo las conexiones residuales permitiendo entrenar redes mucho más profundas \cite{he2016deep}. Cada una de estas arquitecturas ha contribuido a comprender mejor cómo diseñar redes eficientes para procesar y aprender de imágenes a gran escala. En la actualidad, modelos más recientes como DenseNet, que conecta cada capa directamente con todas las anteriores \cite{huang2017densely}, o EfficientNet, que escala de manera uniforme todas las dimensiones de la red \cite{tan2019efficientnet}, han logrado mejoras notables tanto en la eficiencia como en la capacidad de clasificación de las CNNs. Desde entonces, diferentes modelos de redes convolucioneales como EfficientNetV2 \cite{tan2021efficientnetv2smallermodelsfaster}, mejoras en el proceso de entrenamiento \cite{pham2021metapseudolabels} o el uso de novedosas arquitecturas como \textit{Transformer} \cite{NIPS2017_3f5ee243, dosovitskiy2021imageworth16x16words} continúan empujando los límites de lo que las CNNs pueden lograr, mejorando su capacidad de clasificación y optimizando su rendimiento y eficiencia para emplearlos en aplicaciones en tiempo real y en dispositivos con recursos limitados.

\subsection{ResNet}

Las \textbf{Redes Neuronales Residuales} (ResNet) introducidas por He et al. en 2015 \cite{he2016deep}, supusieron un avance significativo en la arquitectura de redes profundas para el reconocimiento de imágenes. A diferencia de sus predecesores como AlexNet y VGG, ResNet aborda el problema del desvanecimiento del gradiente que suele presentarse en redes muy profundas mediante la introducción de una conexión de identidad que salta una o más capas.

La clave de la arquitectura de ResNet es el \textbf{bloque residual}, que incorpora una \textbf{conexión residual} directamente conectando la entrada del bloque a su salida, lo que permite que la señal se propague directamente a través de la red. El bloque residual se puede expresar como:
\begin{equation}
	H(\mathbf{x}_l) = \mathcal{F}(\mathbf{x}_l, \{W_i\}) + \mathbf{x}_l
\end{equation}
donde \(\mathbf{x}_l\) y \(H(\mathbf{x}_l)\) son la entrada y la salida del $l$-ésimo bloque residual respectivamente, \(\mathcal{F}\) representa las capas intermedias de la red y \(\{W_i\}\) denota el conjunto de pesos de estas capas.

\begin{figure}
	\centering
\begin{tikzpicture}
	
	\node[fill=orange!50] (l1) {Capa 1};
	\node[blue!50!black, right=of l1, label={below:Activación}] (act1) {$\theta(\mathbf{x})$};
	\node[fill=teal!50, right=of act1] (l2) {Capa 2};
	\node[right=of l2, font=\Large, label={below:Suma}, inner sep=0, pin={60:$\mathcal F(\mathbf{x}) + \mathbf{x}$}] (add) {$\oplus$};
	\node[blue!50!black, right=of add, label={below:Activación}] (act2) {$\theta(\mathbf{x})$};
	
	\draw[->] (l1) -- (act1);
	\draw[->] (act1) -- (l2);
	\draw[<-] (l1) -- ++(-2,0) node[below, pos=0.8] {$\mathbf{x}$};
	\draw[->] (l2) -- (act2) node[above, pos=0.8] {};
	\draw[->] ($(l1)-(1.5,0)$) to[out=90, in=90] node[below=1ex, midway, align=center] {Conexión residual\\(identidad)} node[above, midway] {$\mathbf{x}$} (add);
	\draw[decorate, decoration={brace, amplitude=1ex, raise=1cm}] (l2.east) -- node[midway, below=1.2cm] {$\mathcal F(\mathbf{x})$} (l1.west);
	
\end{tikzpicture}
\caption{Diagrama de una conexión residual. La entrada $\mathbf{x}$ pasa a través de la Capa 1 y una función de activación, fluyendo luego hacia la Capa 2. Paralelamente, $\mathbf{x}$ se suma directamente al final de la Capa 2 mediante una conexión residual. }
\end{figure}

A diferencia de AlexNet, que tiene 8 capas, y VGG, que tiene 16 o 19 capas, ResNet se diseñó con capacidades mucho más profundas, con versiones que van desde 18 hasta 152 capas. Lo revolucionario de ResNet no es simplemente añadir más capas, sino su habilidad para entrenar redes muy profundas sin sufrir desvanecimiento del gradiente gracias a sus conexiones residuales. Estas conexiones ayudan a preservar el gradiente a lo largo del proceso de aprendizaje, lo que permite entrenar redes más profundas que las posibles anteriormente.

El diseño del bloque residual permite que ResNet no solo evite el problema del desvanecimiento del gradiente sino que también mejore la eficiencia del entrenamiento. Los experimentos demuestran que las redes con bloques residuales superan a las arquitecturas convencionales en grandes conjuntos de datos de imágenes como ImageNet.

\subsection{DenseNet}

Las \textbf{Redes Convolucionales Densamente Conectadas} (DenseNet) propuestas por Huang et al. en 2017 \cite{huang2017densely}, representan otra evolución significativa en el diseño de redes neuronales profundas. DenseNet mejora la idea de conexiones de salto de ResNet mediante la integración de cada capa directamente con todas las capas posteriores de una manera densamente conectada.

La principal innovación de DenseNet es su estructura de \textbf{conexiones densas}, donde cada capa recibe como entrada todas las salidas de las capas anteriores, concatenando estas salidas. Esto se formula como sigue:
\begin{equation}
	\mathbf{x}_l = H_{l}([\mathbf{x}_0, \mathbf{x}_1, \dots, \mathbf{x}_{l-1}])
\end{equation}
donde \(\mathbf{x}_l\) es la salida de la capa \(l\), \([ \cdot ]\) denota la operación de concatenación y \(H_l\) es una función que representa las operaciones dentro de la capa \(l\), normalmente compuesta por operaciones de \textit{Batch Normalization}, activación ReLU, y convolución.

\begin{figure}
	\label{key}
	\centering
	\includegraphics[width=100mm]{img/densenet.png}
	\caption{Arquitectura de DenseNet. La figura ilustra la configuración de la red DenseNet, mostrando cómo las capas están conectadas entre sí, donde cada capa recibe como entrada todas las salidas de las capas anteriores. Fuente \cite{huang2017densely}.}
\end{figure}

Aunque tanto ResNet como DenseNet utilizan conexiones que saltan capas, DenseNet ofrece una mejora en la eficiencia y la efectividad del entrenamiento al promover la reutilización de características de mayor manera. Mientras que las conexiones de salto en ResNet suman entradas, en DenseNet la información fluye a través de la red mediante la concatenación de características, lo que resulta en una mejora del flujo de información y gradientes a través de la red, reduciendo así el problema del desvanecimiento del gradiente de manera más efectiva.

DenseNet ha demostrado ser especialmente eficaz en conjuntos de datos como ImageNet y en aplicaciones donde la conservación de la información a lo largo de la red es crítica. Además, DenseNet tiende a ser más eficiente en términos de parámetros que ResNet debido a su capacidad para reutilizar características, lo que permite la construcción de redes profundas que son tanto compactas como potentes.

\subsection{EfficientNet}

\textbf{EfficientNet}, introducido por Tan y Le en 2019 \cite{tan2019efficientnet}, es un ejemplo de cómo se puede mejorar la eficiencia y la efectividad de las redes neuronales mediante una cuidadosa optimización de sus dimensiones. Lo innovador de EfficientNet es su metodología de escalado compuesto, que escala uniformemente todas las dimensiones de la red (profundidad, ancho y resolución de la imagen) con un conjunto fijo de coeficientes de escalado. La fórmula de escalado se representa como:

\begin{figure}
	\centering
	\includegraphics[width=140mm]{img/mbconv.jpg}
	\caption{Estructura de un bloque MBConv, la unidad básica en la arquitectura de EfficientNet.  Incluye capas convolucionales con activaciones Swish y normalización por lotes, intercaladas con capas SE para un refinamiento de las características. La secuencia de procesamiento termina en una capa de \textit{dropout} antes de la salida final. Fuente \cite{TANG2024105605}.}
\end{figure}

\begin{equation}
	\text{profundidad}: d = \alpha^\phi, \quad \text{anchura}: w = \beta^\phi, \quad \text{resolución}: r = \gamma^\phi,
\end{equation}
tal que 
\[
\alpha \cdot \beta^2 \cdot \gamma^2 \approx 2, \quad \alpha \geq 1, \quad \beta \geq 1, \quad \gamma \geq 1,
\]

donde \(\phi\) es un coeficiente que determina la cantidad de recursos disponibles para el escalado, y \(\alpha\), \(\beta\), \(\gamma\) son constantes que definen cómo deben escalar la profundidad, el ancho y la resolución, respectivamente, para lograr un equilibrio óptimo entre precisión y eficiencia.

\textbf{EfficientNet-B0} es el modelo base en la familia de modelos EfficientNet, que se caracteriza por aplicar un escalado uniforme a una arquitectura optimizada mediante \textbf{búsqueda de arquitectura neuronal} (NAS). La arquitectura inicia con una capa convolucional que maneja la imagen de entrada, aplicando la función de activación Swish y normalización por lotes para preparar las características para las etapas siguientes. A continuación, EfficientNet-B0 implementa una serie de \textbf{bloques MBConv}. Cada bloque MBConv sigue un patrón específico:

\begin{enumerate}
	\item \textbf{Expansión:} Primero, una convolución \(1 \times 1\) incrementa el número de canales. Esta técnica de expansión prepara los canales para una manipulación más intensiva, facilitando la extracción de características.
	
	\item \textbf{Convolución en profundidad:} A continuación, se aplica una convolución en profundidad con un filtro \(3 \times 3\) o \(5 \times 5\). Esta técnica permite la extracción de características locales de manera eficiente, sin un gran incremento en el número de parámetros como se observaría con convoluciones tradicionales.
	
	\item \textbf{Capa \textit{Squeeze and Excitation} (SE):} Esta capa sigue a la convolución en profundidad. Funciona primero reduciendo espacialmente cada canal a un valor escalar (\textit{squeeze}), que luego se utiliza para recalibrar los canales (\textit{excitation}) mediante operaciones de reescalado. Este proceso permite al modelo aprender a enfocar dinámicamente su atención en características informativas y suprimir las menos útiles.
	
	\item \textbf{Compresión:} Después de la capa SE, una convolución \(1 \times 1\) reduce el número de canales, consolidando las características importantes, lo cual asegura la eficiencia de la red y prepara la salida para la siguiente etapa del procesamiento.
\end{enumerate}

Las capas de reducción, situadas entre grupos de bloques MBConv, emplean convoluciones con un paso de 2 para reducir las dimensiones espaciales, aumentando el nivel de abstracción y reduciendo la carga computacional en las capas más profundas.

Finalmente, conexiones residuales se incorporan en cada bloque MBConv para facilitar el entrenamiento de la red al prevenir el desvanecimiento del gradiente. La red termina con una capa de muestreo promedio global que transforma la salida de los bloques MBConv en un vector único por imagen, el cual es procesado por una capa FC para realizar la clasificación final.

%A diferencia de ResNet y DenseNet, que principalmente se enfocan en mejorar la profundidad de la red o la densidad de las conexiones, EfficientNet proporciona un marco holístico que ajusta de manera equilibrada todas las dimensiones de la red. Esto no solo mejora el rendimiento sino que también aumenta la eficiencia del modelo, permitiendo que EfficientNet supere a modelos anteriores en precisión con un número significativamente menor de parámetros y una menor cantidad de operaciones de punto flotante (FLOPs).

\begin{figure}[hbt!]
	\centering
	\includegraphics[width=100mm]{img/cnn-sota.png}
	\caption{Comparación de tamaño de modelo y precisión en ImageNet. EfficientNet supera notablemente al resto de modelos de CNNs. En general, EfficientNet muestra una mejora destacable en eficiencia y rendimiento comparado con modelos previos como ResNet-152. Fuente \cite{tan2019efficientnet}.}
\end{figure}

En la actualidad, modelos como ResNet, DenseNet y EfficientNet han establecido nuevos estándares de precisión en \textit{benchmarks} como \textbf{ImageNet}, donde las tasas de error se han reducido de manera significativa en comparación con los métodos tradicionales basados en características manuales. Estas arquitecturas avanzadas han demostrado no solo una gran capacidad de generalización sobre grandes conjuntos de datos, sino también robustez frente a variaciones y perturbaciones en las imágenes. Además, la integración de técnicas como el aumento de datos y la \textbf{transferencia de aprendizaje} o \textbf{\textit{transfer learning}}, han permitido aplicar modelos entrenados en un dominio específico a nuevos conjuntos de datos con poco o ningún ajuste adicional. Este progreso ha transformado la clasificación de imágenes de ser un desafío técnico a una herramienta útil, con aplicaciones en múltiples industrias, desde el reconocimiento automático de contenido en redes sociales hasta sistemas avanzados de asistencia al conductor en vehículos autónomos.

\endinput
%--------------------------------------------------------------------
% FIN DEL CAPÍTULO. 
%--------------------------------------------------------------------

% !TeX root = ../tfg.tex
% !TeX encoding = utf8

\chapter{Marco experimental y metodología}
\label{chapter:methodology}

Como se ha podido observar en el \autoref{chapter:tda}, el TDA proporciona una serie de mecanismos que nos permiten estudiar con bastante fidelidad la topología de los datos en CNNs. Además, hemos visto que dichas redes tienden a simplificar la forma de los datos durante su paso a través de las distintas capas y lo hacen de manera diferente en función de la arquitectura propuesta. En este capítulo se describe la metodología así como las herramientas empleadas para analizar distintas CNNs en profundidad.
%En este capítulo, aplicaremos el TDA para estudiar a fondo cómo afectan las distintas componentes y mecanismos comúnmente empleados en el entrenamiento de las CNNs en el marco de la topología. Además, trataremos de analizar cómo afecta la topología de los datos a la capacidad de generalización de los modelos y a su transferencia de conocimiento. Nuestra hipótesis en base a los estudios previamente citados apuntan a que una complejidad topológica menor en las etapas finales del modelos podrían mejorar su rendimiento en clasificación, mientras que una complejidad mayor podría facilitar la transferencia de conocimiento.

\section{Experimentos}

Los experimentos se han dividido en tres etapas con el fin de comprender cómo las CNNs transforman la topología de los datos. Para ello, se han entrenado una serie de modelos pertenecientes a tres familias de arquitecturas diferentes de CNNs: ResNet-18, DenseNet-121 y EfficientNet-B0. El principal motivo para seleccionar estos representantes y no alternativas más complejas de la misma familia se debe a que an mostrado una buena capacidad de generalización y eficiencia.

La \textbf{primera etapa} ha consistido en la comparación de la complejidad topológica de los datos a lo largo de la red en función de la arquitectura, el tamaño de lote y el optimizador escogido.% Para ello, se han evaluado los tamaños de lote 8, 16, 32 y 64 y los optimizadores SGD y Adam.
	
En la \textbf{segunda etapa}, se ha escogido el mejor modelo para cada una de las dos granularidades del conjunto de datos tratado y se ha comparado la topología de los datos con los mismos modelos entrenados con aumento de datos, la granuralidad del conjunto de datos y la partición de datos.
	
En la \textbf{última etapa}, se han realizado dos experimentos en base a las observaciones realizadas en las anteriores etapas. Como veremos en la \autoref{sec:homology-analysis}, nuestra hipótesis en base a los estudios previamente citados apuntan a que una complejidad topológica menor en las etapas finales del modelos podrían mejorar su rendimiento en clasificación, mientras que una complejidad mayor podría facilitar la transferencia de conocimiento a conjuntos de datos más complejos. Por ello, se propone tratar de mejorar la clasificación del modelo mediante el uso del regularizador topológico.

\section{Entorno de experimentación}

Para la realización de este proyecto se seleccionó Python como lenguaje de programación principal, dada su amplia disponibilidad de herramientas de análisis de datos \cite{10.5555/1593511}. En particular, se utilizó la versión 3.10.14 por su compatibilidad con una variedad de bibliotecas importantes para la investigación realizada. Para la gestión del entorno de desarrollo se empleó Anaconda \cite{anaconda}, facilitando así la configuración reproducible de los entornos de trabajo.

Los modelos de aprendizaje profundo se implementaron utilizando PyTorch 1.13.1 \cite{NEURIPS2019_9015}, una biblioteca de cálculo tensorial que soporta la aceleración por GPU. PyTorch es conocido por su extensiva colección de herramientas que simplifican el desarrollo de modelos de aprendizaje profundo, siendo ampliamente reconocido y utilizado en la investigación en inteligencia artificial.

El análisis de homología persistente se llevó a cabo mediante la biblioteca \texttt{giotto-tda} 0.6.0 \cite{giotto-tda}, que se apoya en \texttt{ripser} \cite{ctralie2018ripser} por su eficiencia en la ejecución y el uso de memoria en comparación con otras opciones disponibles \cite{Otter_2017}. Además, esta herramienta permite el procesamiento multihilo, lo que es un beneficio adicional importante. Para complementar el análisis, se utilizó \texttt{scikit-learn} 1.5.0 para la implementación del algoritmo de LLE.

Para la visualización de datos, se optó principalmente por \texttt{matplotlib} 3.8.4 y \texttt{seaborn} 0.13.2, herramientas que ofrecen funcionalidades avanzadas para la creación de gráficos y visualizaciones complejas. Adicionalmente, se integraron bibliotecas auxiliares como \texttt{yaml} 0.2.5 para la gestión de archivos de configuración, \texttt{argparse} para la manipulación de argumentos de línea de comandos, y \texttt{pandas} 2.2.2 y \texttt{numpy} 1.26.4 para cálculos numéricos adicionales\footnote{Una lista completa de paquetes y requisitos se encuentra disponible en el archivo \href{https://github.com/pab1s/tda-nn-analysis/blob/main/environment.yaml}{\texttt{environment.yaml}} en GitHub.}.

Los experimentos se realizaron en el \textit{switch} Dionisio de la partición Dios del clúster de servidores GPU ubicado en el edificio CPD Santa Lucía de la Universidad de Granada, perteneciente al Instituto Andaluz Interuniversitario en \textit{Data Science and Computational Intelligence}\footnote{\href{https://dasci.es/}{https://dasci.es/}} (DASCI). Está equipado con dos CPUs Intel Xeon Silver 4216 @ 2.10GHz, 512 Gb de RAM y dos GPUs Quadro RTX 8000.

\section{Conjunto de datos}

Durante el desarrollo de los experimentos se ha empleado una versión modificada del conjunto de datos \textbf{\textit{Vehicle Identification}}\footnote{\href{https://dasci.es/es/transferencia/open-data/vehicleid-es/}{https://dasci.es/es/transferencia/open-data/vehicleid-es/}} publicado por el DASCI. Este conjunto de datos está compuesto de imágenes frontales de coches en primer plano proveniente de diversas fuentes. El interés de este conjunto de datos en particular radica en la su disposición de \textbf{distintas granularidades}. Es decir, disponemos de distintas clasificaciones en función de lo \enquote{fina} que sea nuestra predicción sobre los detalles de los datos. Esta versión se divide en dos conjuntos en función de su especificidad:

\begin{itemize}
	\item \textbf{Especificidad Marca}: consiste en 3232 imágenes etiquetadas en 34 clases que indican la marca del fabricante del vehículo.
	\item \textbf{Especificidad Marca-Modelo}: consiste en 2701 imágenes etiquetadas en 152 clases identificando la marca y el modelo concreto del vehículo.
\end{itemize}

El principal motivo por el que se ha seleccionado este conjunto de datos es por su disposición en las dos especificidades propuestas. Dado que las instancias son comunes en la mayoría de los casos, resultará de gran interés saber cómo clasificar conjuntos de datos prácticamente idénticos con distintas etiquetas puede afectar al proceso de generalización y a la topología de los datos.

\begin{figure}[H]
	\centering
	\includegraphics[width=120mm]{img/marca-example.png}
	\caption{Ejemplos de instancias por clase en la especificidad de Marca.}
	\label{fig:example-m}
\end{figure}

\begin{figure}[H]
	\centering
	\includegraphics[width=120mm]{img/marca-modelo-example.png}
	\caption{Ejemplos de instancias por clase en la especificidad de Marca-Modelo.}
	\label{fig:example-mm}
\end{figure}

La \autoref{fig:imb-m} muestra un claro desbalance entre las distintas clases ambos conjuntos de datos. De hecho podemos observar que en ciertas ocasiones el número de muestras por clase es tan solo de una instancia, mientras que otras clases están claramente sobrerrepresentadas.

\begin{figure}[H]
	\centering
	\includegraphics[width=100mm]{img/imbalance-marca.png}
	\caption{Número de instancias por clase en la especificidad de Marca.}
	\label{fig:imb-m}
\end{figure}

Este problema es aún peor cuando nos fijamos en la especificidad Marca-Modelo, tal y como muestra la \autoref{fig:imb-mm}. Por este motivo, será interesante emplear métricas cuya sensibilidad respecto a clases desbalanceadas sea mayor.

\begin{figure}
	\centering
	\includegraphics[width=110mm]{img/imbalance-marca-modelo-1.png}
\end{figure}

\begin{figure}
	\centering
	\includegraphics[width=110mm]{img/imbalance-marca-modelo-2.png}
	\caption{Número de instancias por clase en la especificidad de Marca-Modelo.}
	\label{fig:imb-mm}
\end{figure}

 Ambos \textit{datasets} se dividirán en conjuntos de entrenamiento, validación y test con una distribución 80-10-10 respectivamente durante el proceso de experimentación. El conjunto de entrenamiento se empleará para ajustar los pesos de los modelos directamente de las instancias que lo componen. El conjunto de validación se empleará durante el proceso de entrenamiento para obtener una evaluación no sesgada de los modelos con el objetivo de evitar el sobreajuste y mejorar su capacidad de generalización. Finalmente, emplearemos el conjunto de test para obtener una evaluación final de los modelos entrenados. Es importante destacar la importancia y diferencia de estos dos últimos conjuntos de datos. Si bien ambos se emplean para evaluar los modelos, prescindir del conjunto de validación para emplear también el de test supondría un caso de \textbf{\textit{data snooping}} \cite{white2000reality}. Esto significa que nuestros resultados finales se verían sesgados de manera indirecta por las decisiones tomadas sobre las evaluaciones del conjunto de test, de forma que nuestros resultados también estarían sesgados y nuestras conclusiones sobre la generalización de los modelos serían incorrectas.

En un inicio, se valoró la posibilidad de realizar una validación cruzada para comparar los resultados de las redes entrenadas. No obstante, dado que nuestro objetivo es estudiar la topología de los datos en estas CNNs respecto a los conjuntos de datos que disponemos y no necesariamente obtener el mejor modelo posible, se descartó dicha opción. Además, el gran coste computacional del entrenamiento de estos modelos, junto a los numerosos entrenamientos realizados, reforzaron esta postura. Este enfoque permite centrar los recursos computacionales en el análisis de la topología de los datos y la realización de los experimentos.

\section{Preprocesamiento de datos} \label{sec:data-aug}

Dada la naturaleza heterogénea de las imágenes empleadas, todas las imágenes se han redimensionado a $224 \times 224$ con un total de $3$ canales normalizado en un rango entre $[0,1]$ para los colores RGB. Al estar los modelos preentrenados con ImageNet, los colores de las imágenes se han tipificado respecto a la media y desviación típica de dicho conjunto. Esto es, una media $\mu = (0.485, 0.456, 0.406)$ y una desviación típica $\sigma = (0.229, 0.224, 0.225)$.

Para los modelos entrenados empleando aumento de datos, se han estudiado las siguientes transformaciones:

\begin{itemize}
	\item \textbf{Fluctuación del color}: aplica alteraciones en el brillo, contraste y saturación base de las imágenes. Dada una imagen, se modifica cada uno de estos valores $\alpha$ en un factor obtenido de una distribución uniforme $\mathcal{U}([u, v])$, siendo $u = \max\{0, 1 - \alpha\}$ y $v = 1 + \alpha$.
	\item \textbf{Desenfoque Gaussiano}: se desenfoca la imagen base mediante el uso de una convolución con un \textit{kernel} Gaussiano $5 \times 5$ y una desviación típica escogida con una probabilidad obtenida de una distribución uniforme discreta para los valores $0.1$ y $2$.
	\item \textbf{Transformación especular}: se obtiene la imagen especular con una probabilidad de $0.5$.
	\item \textbf{Rotación}: se rotan las imágenes originales un número de grados en cualquiera de los sentidos obtenidos de $\mathcal{U}([-10,10])$.
	\item \textbf{Recorte}: se recortan porciones de $224 \times 224$ píxeles de la imagen original con un acercamiento de un factor hasta de $1$ y un alejamiento de hasta $0.08$.
\end{itemize}

\section{Métricas de evaluación}

En el ámbito de los problemas de clasificación binaria, existen numerosas métricas que pueden emplearse para estudiar cómo de bueno es un modelo y hacer un seguimiento de su entrenamiento. Las más populares son aquellas obtenidas a partir de la \textbf{matriz de confusión}, la cual muestra el rendimiento del modelo comparando las predicciones realizadas con las etiquetas originales. Para la elaboración de las métricas, los valores de la matriz de confusión se segmentan en cuatro clases:

\begin{itemize}
	\item \textbf{Verdaderos positivos} (TP): son los casos en los que el modelo predice correctamente la clase positiva. Es decir, cuando el modelo predice que un ejemplo es positivo y realmente lo es.
	\item \textbf{Verdaderos negativos} (TN): son los casos en los que el modelo predice correctamente la clase negativa. Es decir, cuando el modelo predice que un ejemplo es negativo y realmente lo es.
	\item \textbf{Falsos positivos} (FP): son los casos en los que el modelo predice incorrectamente la clase positiva. Es decir, cuando el modelo predice que un ejemplo es positivo, pero en realidad es negativo. Esto también se conoce como error de tipo I.
	\item \textbf{Falsos negativos} (FN): son los casos en los que el modelo predice incorrectamente la clase negativa. Es decir, cuando el modelo predice que un ejemplo es negativo, pero en realidad es positivo. Esto también se conoce como error de tipo II.
\end{itemize}

En los problemas de clasificación multiclase, esta metodología es adaptada para que tenga sentido en el caso donde el número de clases es mayor que 2. Para ello, se utiliza un enfoque conocido como \enquote{uno contra todos} (\textit{one-vs-all}), donde cada clase se compara contra todas las demás clases. En este contexto, se considera positivo cuando la etiqueta predicha se corresponde con su valor original y negativa si ha predicho cualquiera de las clases restantes.

\begin{figure}[H]
	\centering
	\includegraphics[width=100mm]{img/confusion-matrix.png}
	\caption{Martiz de confusión multiclase. Dada una predicción de una instancia de la clase $C_k$, la tabla muestra cuando se considera TP (verde), TN (naranja), FP (ocre) y FN (rojo). Fuente \cite{venkataramana2023}.}
	\label{fig:confusion-matrix}
\end{figure}

Estas clases son posteriormente empleadas para calcular diferentes métricas en función de los requisitos que debe cumplir el modelo. Entre ellas, las más empleadas son las siguientes:

\begin{itemize}
	\item \textbf{Exactitud o \textit{accuracy}}: Es el cociente entre el número de predicciones correctas y el total de predicciones realizadas. Se calcula como:
	\[
	\text{Exactitud} = \frac{TP + TN}{TP + TN + FP + FN}.
	\]
	\item \textbf{Precisión o \textit{precision}}: Es el cociente entre el número de verdaderos positivos y el total de predicciones positivas. Indica la exactitud de las predicciones positivas del modelo. Se calcula como:
	\[
	\text{Precisión} = \frac{TP}{TP + FP}.
	\]
	\item \textbf{Sensibilidad o \textit{recall}}: Es el cociente entre el número de verdaderos positivos y el total de casos reales positivos. Indica la capacidad del modelo para identificar todos los casos positivos. Se calcula como:
	\[
	\text{Sensibilidad} = \frac{TP}{TP + FN}
	\]
	\item \textbf{F1-Score}: Es la media armónica entre la precisión y la sensibilidad. Proporciona una única métrica que equilibra las dos anteriores y es especialmente útil cuando hay un desequilibrio entre las clases. Se calcula como:
	\[
	F1 = 2 \cdot \frac{\text{Precisión} \cdot \text{Sensibilidad}}{\text{Precisión} + \text{Sensibilidad}}.
	\]
\end{itemize}

Por otra parte, se ha empleado la persistencia total como métrica para medir la persistencia homológica de los datos en los distintos niveles de la red, además de nuestra propuesta, la persistencia total normalizada.

\section{Proceso de entrenamiento} \label{sec:train}

En esta etapa se ha optado por entrenar los modelos ResNet-18, DenseNet-121 y EfficientNet-B0 bajo las siguientes condiciones para hacer un exhaustivo análisis viendo cómo afectan a la topología de los datos y su rendimiento.

\begin{enumerate}
	\item \textbf{Optimización de Hiperparámetros:}
	Se realizó una búsqueda en rejilla (\textit{grid search}) para determinar los tamaños de lote óptimos (8, 16, 32 y 64). Además, se emplearon los optimizadores Adam y SGD. Para Adam, se emplearon los hiperparámetros $\beta_1 = 0.9$, $\beta_2 = 0.999$ y $\varepsilon = 10^{-8}$.
	
	\item \textbf{Aumento de Datos:}
	Posteriormente, se aplicaron técnicas de aumento de datos al mejor modelo de cada familia identificado anteriormente. Las transformaciones incluyeron las descritas en la \autoref{sec:data-aug} para determinar las más beneficiosas en el contexto del entrenamiento final de los modelos, con el fin de mejorar su generalización.
	
	\item \textbf{Incorporación de Regularización Topológica:} Finalmente, se ha aplicado una búsqueda en rejilla al término $\alpha$ en el regularizador de dos formas. Se ha empleado $0, 0.001, 0.005, 0.01, 0.05, 0.1, 0.5$ y $1$ con los extractores de características de EfficientNet-B0 y DenseNet-121 congelados y para transferir DenseNet-121 de la especificidad Marca-Modelo a Marca. Estos mismos valores negados se han empleado para transferir EfficientNet-B0 de Marca a Marca-Modelo.
\end{enumerate}

Todas las tasas de aprendizaje fueron obtenidas a partir de una implementación realizada del método propuesto por Leslie N. Smith \cite{smith2017cyclicallearningratestraining}. Dicha técnica comienza aplicando una tasa de aprendizaje muy baja que va aumentando exponencialmente en cada lote de entrenamiento hasta un máximo establecido y registrando el error de entrenamiento asociado a cada tasa. Finalmente, se determina el valor óptimo de la tasa escogiéndolo en algún punto previo al mínimo donde la pendiente en la gráfica de error frente a tasa de aprendizaje es más negativa. En particular, para la búsqueda se han empleado un mínimo de $10^{-7}$ y un máximo de $1$. Las Tablas \ref{tab:bs-optim-marca} y \ref{tab:bs-optim-marca-modelo} muestran las tasas de aprendizaje empleadas para cada alternativa.

\begin{table}[H]
	\centering
	\begin{adjustbox}{max width=\textwidth}
		\begin{tabular}{|c|c|c|c|c|c|c|c|c|c|}
			\hline
			\textbf{Modelo} & \multicolumn{4}{c|}{\textbf{SGD}} & \multicolumn{4}{c|}{\textbf{Adam}} \\ \cline{2-9} 
			& \textbf{Lote 8} & \textbf{Lote 16} & \textbf{Lote 32} & \textbf{Lote 64} & \textbf{Lote 8} & \textbf{Lote 16} & \textbf{Lote 32} & \textbf{Lote 64} \\ \hline
			\textbf{ResNet} & 0.005 & 0.001 & 0.01 & 0.01 & 0.0005 & 0.001 & 0.001 & 0.001 \\ \hline
			\textbf{DenseNet} & 0.005 & 0.01 & 0.01 & 0.01 & 0.0005 & 0.001 & 0.001 & 0.001 \\ \hline
			\textbf{EfficientNet} & 0.001 & 0.05 & 0.05 & 0.05 & 0.001 & 0.001 & 0.001 & 0.001 \\ \hline
		\end{tabular}
	\end{adjustbox}
	\caption{Tasas de aprendizaje escogidas para los diferentes modelos, optimizadores y tamaños de lote en el conjunto de datos Marca.}
	\label{tab:bs-optim-marca}
\end{table}

\begin{table}[H]
	\centering
	\begin{adjustbox}{max width=\textwidth}
		\begin{tabular}{|c|c|c|c|c|c|c|c|c|c|}
			\hline
			\textbf{Modelo} & \multicolumn{4}{c|}{\textbf{SGD}} & \multicolumn{4}{c|}{\textbf{Adam}} \\ \cline{2-9} 
			& \textbf{Lote 8} & \textbf{Lote 16} & \textbf{Lote 32} & \textbf{Lote 64} & \textbf{Lote 8} & \textbf{Lote 16} & \textbf{Lote 32} & \textbf{Lote 64} \\ \hline
			\textbf{ResNet} & 0.005 & 0.01 & 0.01 & 0.05 & 0.0005 & 0.001 & 0.001 & 0.001 \\ \hline
			\textbf{DenseNet} & 0.005 & 0.01 & 0.01 & 0.01 & 0.0005 & 0.001 & 0.001 & 0.005 \\ \hline
			\textbf{EfficientNet} & 0.05 & 0.05 & 0.05 & 0.05 & 0.001 & 0.005 & 0.005 & 0.005 \\ \hline
		\end{tabular}
	\end{adjustbox}
	\caption{Tasas de aprendizaje escogidas para los diferentes modelos, optimizadores y tamaños de lote en el conjunto de datos Marca-Modelo.}
	\label{tab:bs-optim-marca-modelo}
\end{table}

Se ha empleado \textit{early stopping} como criterio de parada con una paciencia de $5$ épocas respecto a los valores de pérdida del conjunto de validación. Todos los modelos han sido entrenados sobre los conjuntos de datos de Marca y Marca-Modelo.

\section{Postprocesamiento de resultados}

Para cada una de las redes entrenadas y seleccionadas en la \autoref{sec:train}, hemos medido la persistencia de las clases de homología de dimensión 0 y 1 respecto a los conjuntos de datos de entrenamiento, validación y test. En un inicio, se valoró estudiar la topología respecto al conjunto de entrenamiento al completo. Debido al gran coste en memoria que suponía, finalmente se optó por evaluarla para un subconjunto escogido de manera aleatoria de los datos de entrenamiento con el mismo número de instancias que los conjuntos de validación y test.

La visualización de los datos a través de la red se ha realizado mediante el uso de un algoritmo de reducción no lineal de la dimensionalidad conocido como \textbf{incrustación lineal local} o \textbf{\textit{locally lineal embedding}} (LLE). Este método consiste en describir cada instancia del conjunto en coordenadas de sus $k$ vecinos más cercanos, siendo $k \in \N$ la dimensionalidad de la nueva representación. Como dichos vecinos no tienen por qué formar una base, las coordenadas se escogen de manera que la combinación lineal realizada minimice el error cuadrático medio respecto a las coordenadas originales.

Para la visualización de la homología persistente se ha empleado el código de barras descrito en la \autoref{sec:barcode}.

\endinput
%--------------------------------------------------------------------
% FIN DEL CAPÍTULO. 
%--------------------------------------------------------------------

% !TeX root = ../tfg.tex
% !TeX encoding = utf8

\chapter{Resultados experimentales}
\label{chapter:analisis}

\section{Rendimiento y selección de modelos}

A continuación se presentan los resultados de las métricas respecto a los
conjuntos de validación obtenidas tras el entrenamiento de las redes. El objetivo
de esta sección será seleccionar los modelos que emplearemos para comparar la evolución
de la homología persistente entre ellos.

\subsection{Selección de hiperparámetros}
\label{subsec:hiperparam}

La \autoref{fig:loss-optim-m} muestra la evolución de la función de pérdida
promedio respecto al optimizador a lo largo del entrenamiento. En ella, podemos
observar que los modelos entrenados con SGD para los distintos tamaños de lote y
arquitecturas han mostrado unas tasas de pérdida más bajas y mayor regularidad
en el conjunto de validación. En general, la gráfica muestra un desarrollo
correcto del proceso de entrenamiento.

\begin{figure}[H]
	\centering
	\includegraphics[width=0.8\textwidth]{img/loss-optimizer-marca.png}
	\caption{Función de pérdida promedio en entrenamiento y validación de los
		distintos modelos entrenados en función del optimizador empleado: SGD y Adam.
		Se muestra de forma sombreada la desviación típica de la función de pérdida.}
	\label{fig:loss-optim-m}
\end{figure}

Por otro lado, la \autoref{fig:loss-batch-m} muestra como existe una mayor
variabilidad en el proceso de entrenamiento entre los modelos con un mismo
tamaño de lote. Entre ellos, el que parece presentar una mayor estabilidad para
las distintas arquitecturas es con un tamaño de lote 64, pues muestra una menor desviación
típica y valores bajos en la función de pérdida.

\begin{figure}[H]
	\centering
	\includegraphics[width=0.8\textwidth]{img/loss-batch-marca.png}
	\caption{Función de pérdida promedio en entrenamiento y validación de los
		distintos modelos entrenados en función delos tamaños de lote empleados: 8, 16
		,32 y 64. Se muestra de forma sombreada la desviación típica de la función de pérdida.}
	\label{fig:loss-batch-m}
\end{figure}

Finalmente, las Figuras \ref{fig:loss-optim-mm} y \ref{fig:loss-batch-mm},
muestran un comportamiento similar a las recién vistas, indicando un correcto desarrollo
de la fase de entrenamiento. A diferencia de las anteriores, si bien SGD ha vuelto
a mostrar una mejor evolución de la curva de pérdida, es esta vez el tamaño de
lote 32 el que ha mostrado unas tasas de pérdida promedio más baja. Otra observación
interesante es la disminución en la pendiente de la curva, reflejando una mayor
dificultad en el aprendizaje en la especificidad Marca-Modelo.

\begin{figure}[H]
	\centering
	\includegraphics[width=0.8\textwidth]{img/loss-optimizer-marca-modelo.png}
	\caption{Función de pérdida promedio en entrenamiento y validación de los
		distintos modelos entrenados en función del optimizador empleado: SGD y Adam.
		Se muestra de forma sombreada la desviación típica de la función de pérdida.}
	\label{fig:loss-optim-mm}
\end{figure}

\begin{figure}[H]
	\centering
	\includegraphics[width=0.8\textwidth]{img/loss-batch-marca-modelo.png}
	\caption{Función de pérdida promedio en entrenamiento y validación de los
		distintos modelos entrenados en función de los tamaños de lote empleados: 8,
		16 ,32 y 64. Se muestra de forma sombreada la desviación típica de la función de
		pérdida.}
	\label{fig:loss-batch-mm}
\end{figure}

Las Tablas \ref{tab:sgd_metrics-marca} y \ref{tab:adam_metrics-marca} muestran
las métricas obtenidas durante el proceso de entrenamiento tanto para SGD como
Adam respectivamente para la especificidad de Marca. Todas ellas han alcanzado el
criterio de parada en menos de 2 horas y de 50 épocas. En general, podemos observar
que ambos optimizadores logran una puntuación superior al $95\%$ en la gran mayoría
de las métricas. Es interesante observar cómo pese al desbalance de los datos, el
F1-Score sigue mostrando resultados excelentes. Entre todos los modelos vemos
que los mejores resultados los ha obtenido EfficientNet-B0 con el optimizador
SGD y un tamaño de lote 8 y 64. Sin embargo, finalmente se ha optado por escoger
el mismo modelo con un tamaño de lote 64, pues su resultados son muy similares y
su F1-Score es ligeramente mayor.

\begin{table}[H]
	\begin{adjustbox}
		{width=1\textwidth}
		\begin{tabular}{|c|c|c|c|c|c|}
			\hline
			\textbf{Modelo}             & \textbf{Tamaño de Lote} & \textbf{Exactitud} & \textbf{Precisión} & \textbf{Sensibilidad} & \textbf{F1-Score} \\
			\hline
			& 8                       & 0.9536             & 0.9416             & 0.9474                & 0.9443            \\
			\cline{2-6} ResNet-18       & 16                      & 0.9102             & 0.8834             & 0.8953                & 0.8891            \\
			\cline{2-6}                 & 32                      & 0.9721             & 0.9785             & 0.9753                & 0.9769            \\
			\cline{2-6}                 & 64                      & 0.9814             & 0.9758             & 0.9775                & 0.9767            \\
			\hline
			& 8                       & 0.9814             & 0.9850             & 0.9828                & 0.9838            \\
			\cline{2-6} DenseNet-121    & 16                      & 0.9752             & 0.9755             & 0.9783                & 0.9768            \\
			\cline{2-6}                 & 32                      & 0.9845             & 0.9931             & 0.9863                & 0.9897            \\
			\cline{2-6}                 & 64                      & 0.9814             & 0.9738             & 0.9753                & 0.9746            \\
			\hline
			& 8                       & \textbf{0.9907}    & 0.9907             & \textbf{0.9917}       & 0.9911            \\
			\cline{2-6} EfficientNet-B0 & 16                      & 0.9474             & 0.9512             & 0.9515                & 0.9512            \\
			\cline{2-6}                 & 32                      & 0.9690             & 0.9798             & 0.9689                & 0.9743            \\
			\cline{2-6}                 & 64                      & 0.9859             & \textbf{0.9925}    & 0.9907                & \textbf{0.9916}   \\
			\hline
		\end{tabular}
	\end{adjustbox}
	\caption{Métricas de validación para los modelos optimizados con SGD en la
		especificidad Marca.}
	\label{tab:sgd_metrics-marca}
\end{table}

\begin{table}[H]
	\begin{adjustbox}
		{width=1\textwidth}
		\begin{tabular}{|c|c|c|c|c|c|}
			\hline
			\textbf{Modelo}             & \textbf{Tamaño de Lote} & \textbf{Exactitud} & \textbf{Precisión} & \textbf{Sensibilidad} & \textbf{F1-Score} \\
			\hline
			& 8                       & 0.9536             & 0.9496             & 0.9508                & 0.9501            \\
			\cline{2-6} ResNet-18       & 16                      & 0.9536             & 0.9486             & 0.9467                & 0.9475            \\
			\cline{2-6}                 & 32                      & 0.9381             & 0.9319             & 0.9235                & 0.9275            \\
			\cline{2-6}                 & 64                      & 0.9721             & 0.9713             & 0.9633                & 0.9672            \\
			\hline
			& 8                       & 0.9350             & 0.9428             & 0.9363                & 0.9393            \\
			\cline{2-6} DenseNet-121    & 16                      & 0.9536             & 0.9465             & 0.9463                & 0.9462            \\
			\cline{2-6}                 & 32                      & 0.9659             & 0.9750             & 0.9694                & 0.9721            \\
			\cline{2-6}                 & 64                      & \textbf{0.9876}    & \textbf{0.9916}    & \textbf{0.9869}       & \textbf{0.9892}   \\
			\hline
			& 8                       & 0.9598             & 0.9588             & 0.9576                & 0.9580            \\
			\cline{2-6} EfficientNet-B0 & 16                      & 0.9752             & 0.9768             & 0.9757                & 0.9762            \\
			\cline{2-6}                 & 32                      & 0.9783             & 0.9720             & 0.9698                & 0.9709            \\
			\cline{2-6}                 & 64                      & 0.9567             & 0.9330             & 0.9332                & 0.9331            \\
			\hline
		\end{tabular}
	\end{adjustbox}
	\caption{Métricas de validación para los modelos optimizados con Adam en la
		especificidad Marca.}
	\label{tab:adam_metrics-marca}
\end{table}

Por otro lado, los modelos entrenados sobre el conjunto de datos Marca-Modelo han
mostrado mayores dificultades durante el proceso de aprendizaje, tal y como
muestran los resultados de las Tablas \ref{tab:sgd_metrics_mm} y
\ref{tab:adam_metrics_mm}. Pese a ello, la configuración que ha mostrado mejores
resultados en las métricas evaluadas ha sido DenseNet-121 con SGD y un tamaño de
lote 32. Esta configuración ha mostrado ser la mejor tanto en exactitud como en
F1-Score, además de tener resultados competentes en el resto de métricas. Por
ello, emplearemos dicho modelo en las siguientes etapas.

\begin{table}[H]
	\centering
	\begin{adjustbox}
		{width=\textwidth}
		\begin{tabular}{|c|c|c|c|c|c|}
			\hline
			\textbf{Modelo}             & \textbf{Tamaño de Lote} & \textbf{Exactitud} & \textbf{Precisión} & \textbf{Sensibilidad} & \textbf{F1-Score} \\
			\hline
			& 8                       & 0.8000             & 0.7934             & 0.7968                & 0.7949            \\
			\cline{2-6} ResNet-18       & 16                      & 0.8037             & 0.7866             & 0.7969                & 0.7916            \\
			\cline{2-6}                 & 32                      & 0.9000             & 0.8743             & 0.8912                & 0.8826            \\
			\cline{2-6}                 & 64                      & 0.8741             & 0.8510             & 0.8540                & 0.8524            \\
			\hline
			& 8                       & 0.8667             & 0.8569             & 0.8624                & 0.8595            \\
			\cline{2-6} DenseNet-121    & 16                      & 0.8741             & 0.8739             & 0.8732                & 0.8734            \\
			\cline{2-6}                 & 32                      & \textbf{0.9407}    & \textbf{0.9176}    & \textbf{0.9314}       & \textbf{0.9243}   \\
			\cline{2-6}                 & 64                      & 0.9111             & 0.8985             & 0.9009                & 0.8997            \\
			\hline
			& 8                       & 0.9111             & 0.9122             & 0.9116                & 0.9118            \\
			\cline{2-6} EfficientNet-B0 & 16                      & 0.8000             & 0.8020             & 0.8066                & 0.8040            \\
			\cline{2-6}                 & 32                      & 0.9148             & 0.9213             & 0.9181                & 0.9197            \\
			\cline{2-6}                 & 64                      & 0.9000             & 0.8941             & 0.8920                & 0.8929            \\
			\hline
		\end{tabular}
	\end{adjustbox}
	\caption{Métricas de validación para los modelos optimizados con SGD en la
		especificidad Marca-Modelo.}
	\label{tab:sgd_metrics_mm}
\end{table}

\begin{table}[H]
	\centering
	\begin{adjustbox}
		{width=\textwidth}
		\begin{tabular}{|c|c|c|c|c|c|}
			\hline
			\textbf{Modelo}             & \textbf{Tamaño de Lote} & \textbf{Exactitud} & \textbf{Precisión} & \textbf{Sensibilidad} & \textbf{F1-Score} \\
			\hline
			& 8                       & 0.8556             & 0.8526             & 0.8536                & 0.8529            \\
			\cline{2-6} ResNet-18       & 16                      & 0.8556             & 0.8471             & 0.8543                & 0.8506            \\
			\cline{2-6}                 & 32                      & 0.7741             & 0.7367             & 0.7561                & 0.7461            \\
			\cline{2-6}                 & 64                      & 0.8556             & 0.8493             & 0.8438                & 0.8464            \\
			\hline
			& 8                       & 0.8741             & 0.8672             & 0.8725                & 0.8696            \\
			\cline{2-6} DenseNet-121    & 16                      & 0.8667             & 0.8685             & 0.8687                & 0.8685            \\
			\cline{2-6}                 & 32                      & 0.9037             & 0.8943             & 0.9018                & 0.8979            \\
			\cline{2-6}                 & 64                      & \textbf{0.9201}    & \textbf{0.9263}    & 0.9103                & \textbf{0.9182}   \\
			\hline
			& 8                       & 0.8481             & 0.8384             & 0.8460                & 0.8420            \\
			\cline{2-6} EfficientNet-B0 & 16                      & 0.8926             & 0.8946             & 0.8924                & 0.8935            \\
			\cline{2-6}                 & 32                      & 0.9148             & 0.9044             & \textbf{0.9143}       & 0.9093            \\
			\cline{2-6}                 & 64                      & 0.8111             & 0.7869             & 0.7906                & 0.7887            \\
			\hline
		\end{tabular}
	\end{adjustbox}
	\caption{Métricas de validación para los modelos optimizados con Adam en la
		especificidad Marca-Modelo.}
	\label{tab:adam_metrics_mm}
\end{table}

\subsection{Selección de transformaciones de datos}

Tras la anterior etapa se ha procedido a realizar un aumento de datos sobre los
dos conjuntos seleccionados: EfficientNet-B0 con SGD y un tamaño de lote 64 para
la especificidad Marca, y DenseNet-121 con SGD y tamaño de lote 32 para la
especificidad Marca-Modelo. Las Tablas \ref{tab:transformation_metrics-m} y \ref{tab:transformation_metrics-mm}
muestran los resultados obtenidos para cada una de las transformaciones descritas
en la \autoref{sec:data-aug}.

Si bien el modelo entrenado para el conjunto de datos Marca con las transformaciones
no ha mejorado significativamente, vemos que las transformaciones sobre el
modelo entrenado para Marca-Modelo han logrado unas mejoras notablemente. El objetivo
del aumento de datos en las CNNs es mejorar la capacidad de generalización de los
modelos. Sin embargo, el primer modelo ya gozaba de muy buenas métricas por lo que
un empeoramiento general al realizar aumento de datos podría significar un sobreajuste
del modelo. Por tanto, entrenaremos el modelo de nuevo con las transformaciones que
mejor hayan funcionado para estudiar cómo afectan a la topología.

\begin{table}[H]
	\centering
	\begin{adjustbox}
		{width=0.8\textwidth}
		\begin{tabular}{|c|c|c|c|c|c|}
			\hline
			\textbf{Transformación} & \textbf{Exactitud} & \textbf{Precisión} & \textbf{Sensibilidad} & \textbf{F1-Score} \\
			\hline
			Fluctuación             & \textbf{0.9814}    & \textbf{0.9869}    & \textbf{0.9844}       & \textbf{0.9856}   \\
			\hline
			Desenfoque              & 0.9567             & 0.9360             & 0.9388                & 0.9373            \\
			\hline
			Simetría                & 0.9721             & 0.9634             & 0.9585                & 0.9609            \\
			\hline
			Recorte                 & 0.9783             & 0.9765             & 0.9709                & 0.9737            \\
			\hline
			Rotación                & 0.9783             & 0.9706             & 0.9657                & 0.9682            \\
			\hline
		\end{tabular}
	\end{adjustbox}
	\caption{Métricas de validación para EfficientNet-B0 con transformaciones de
		datos en la especificidad Marca.}
	\label{tab:transformation_metrics-m}
\end{table}

\begin{table}[H]
	\centering
	\begin{adjustbox}
		{width=0.8\textwidth}
		\begin{tabular}{|c|c|c|c|c|c|}
			\hline
			\textbf{Transformación} & \textbf{Exactitud} & \textbf{Precisión} & \textbf{Sensibilidad} & \textbf{F1-Score} \\
			\hline
			Fluctuación             & 0.9783             & 0.9728             & 0.9745                & 0.9736            \\
			\hline
			Desenfoque              & 0.9783             & 0.9790             & 0.9750                & 0.9769            \\
			\hline
			Simetría                & 0.9814             & 0.9804             & 0.9775                & 0.9789            \\
			\hline
			Recorte                 & 0.9845             & 0.9863             & 0.9817                & 0.9840            \\
			\hline
			Rotación                & \textbf{0.9876}    & \textbf{0.9914}    & \textbf{0.9874}       & \textbf{0.9894}   \\
			\hline
		\end{tabular}
	\end{adjustbox}
	\caption{Métricas de validación para DenseNet-121 con transformaciones de
		datos en la especificidad Marca-Modelo.}
	\label{tab:transformation_metrics-mm}
\end{table}

Finalmente, la \autoref{tab:model_comparison_val} muestra los resultados de los
modelos reentrenados con las transformaciones escogidas: EfficientNet-B0 con fluctuaciones
en el color, recortes aleatorios y rotaciones aleatorias en Marca; y DenseNet-121
con todas las transformaciones propuestas en Marca-Modelo.

\begin{table}[H]
	\centering
	\begin{adjustbox}
		{width=0.85\textwidth}
		\begin{tabular}{|c|c|c|c|c|}
			\hline
			\textbf{Modelo}                   & \textbf{Exactitud} & \textbf{Precisión} & \textbf{Sensibilidad} & \textbf{F1-Score} \\
			\hline
			EfficientNet-B0 Trans (Marca)     & 0.9505             & 0.9627             & 0.9501                & 0.9562            \\
			\hline
			\hline
			DenseNet-121 Trans (Marca-Modelo) & 0.7962             & 0.7660             & 0.7769                & 0.7712            \\
			\hline
		\end{tabular}
	\end{adjustbox}
	\caption{Rendimiento final de los modelos con aumento de datos en el conjunto
		de validación para EfficientNet-B0 en la especificidad Marca y para DenseNet-121
		en la especificidad Marca-Modelo. }
	\label{tab:model_comparison_val}
\end{table}

Una vez obtenidos los modelos, podemos observar en la \autoref{tab:model_comparison_test_aug}
las métricas obtenidas en cada modelo en el conjunto de test, tanto el modelo
base como el de aumento de datos. Es especialmente interesante el comportamiento
de DenseNet-121 con aumento de datos. Pese a tener bajos resultados en el
conjunto de validación, ha mostrado una buena generalización en el conjunto de test,
incluso superando al modelo base.

\begin{table}[H]
	\centering
	\begin{adjustbox}
		{width=0.85\textwidth}
		\begin{tabular}{|c|c|c|c|c|}
			\hline
			\textbf{Modelo}                   & \textbf{Exactitud} & \textbf{Precisión} & \textbf{Sensibilidad} & \textbf{F1-Score} \\
			\hline
			EfficientNet-B0 Base (Marca)      & \textbf{0.9505}    & \textbf{0.9462}    & \textbf{0.9386}       & \textbf{0.9423}   \\
			\hline
			EfficientNet-B0 Trans (Marca)     & 0.9474             & 0.9319             & 0.9271                & 0.9294            \\
			\hline
			\hline
			DenseNet-121 Base (Marca-Modelo)  & 0.9185             & 0.9077             & 0.9126                & 0.9101            \\
			\hline
			DenseNet-121 Trans (Marca-Modelo) & \textbf{0.9474}    & \textbf{0.9319}    & \textbf{0.9271}       & \textbf{0.9294}   \\
			\hline
		\end{tabular}
	\end{adjustbox}
	\caption{Rendimiento final de los modelos con y sin aumento de datos en el
		conjunto de test para EfficientNet-B0 en la especificidad Marca y para
		DenseNet-121 en la especificidad Marca-Modelo.}
	\label{tab:model_comparison_test_aug}
\end{table}

\section{Análisis de la homología persistente}
\label{sec:homology-analysis}

Una vez habiendo entrenado todos los modelos necesarios, procederemos analizando
los resultados obtenidos en función de la homología persistente. Para ello, se ha
calculado la persistencia total de una muestra de 128 instancias (debido al
coste en memoria) del conjunto de test tras cada activación no lineal presente en
la red evaluada. Las gráficas empleadas a lo largo de toda la sección registran
tanto la persistencia total como la normalizada de la muestra frente a su
posición relativa en su paso por la red expresada en porcentaje. Además, las
figuras muestran la curva de regresión cuadrática calculada a partir de la
persistencia con el fin de una visualización más clara de la tendencia de dichos
valores. Las áreas sombreadas indican el intervalo de confianza del 95\%.

Las Secciones \ref{subsec:arch}, \ref{subsec:optim} y \ref{subsec:batch} realizan
un estudio comparativo en función de la arquitectura, optimizador y tamaño de
lote descritos en el \autoref{chapter:methodology}. Posteriormente, en las
Secciones \ref{subsec:aug}, \ref{subsec:grano} y \ref{subsec:set}, el análisis
se realiza a parir del mejor modelo para cada especificidad con el fin de
analizar cómo afectan el aumento de datos, la granularidad del conjunto y la partición
de datos escogida a la topología de estos.

\subsection{Comparación según la arquitectura}
\label{subsec:arch}

\paragraph{Especificidad Marca}

Comencemos analizando la \autoref{fig:m-homology-arch-1}. La gráfica muestra una
tendencia clara entre los modelos estudiados con los diferentes hiperparámetros:
una disminución pronunciada de la persistencia total con un leve repunte en la
etapa final de ejecución. Esto implica que las transformaciones que realiza el modelo
sobre el conjunto de instancias (como los cambios en dimensionalidad y las
funciones aplicadas sobre ellos) tienden tanto a disminuir la persistencia de las
clases de homología como a reducir el número de ellos que se generan, claramente
simplificando los datos. Un comportamiento interesante es el obtenido al final de
las ejecuciones, donde los modelos complican la homología persistente de los
datos de cara a la tarea de clasificación final, aumentando la persistencia de componentes
conexas y otras características topológicas más complejas, con el fin de
facilitar la separación de clases final.

Asimismo, la \autoref{fig:m-homology-arch-2} muestra una conclusiones similares con
otra tendencia: la persistencia homológica de los datos crece durante las
primeras fases de la inferencia, mientras que desciende de cara al final de la ejecución.
Es decir, las clases de homología persistente tienden a ser más homogéneas en el
punto medio de la ejecución, de forma que los datos están más dispersos y
desordenados. De no ser así, las componentes conexas y clases de homología
persistentes en dimensión 1 morirían en fases tempranas de la filtración, lo que
daría una persistencia total normalizada más baja.

\begin{figure}[H]
	\centering
	\begin{subfigure}
		{.5\textwidth}
		\centering
		\includegraphics[width=\linewidth]{img/m_arch.png}
		\caption{Persistencia total según el porcentaje de avance en las redes para
			los modelos ResNet-18, DenseNet-121 y EfficientNet-B0.}
		\label{fig:m-homology-arch-1}
	\end{subfigure}%
	\begin{subfigure}
		{.5\textwidth}
		\centering
		\includegraphics[width=\linewidth]{img/m_arch_norm.png}
		\caption{Persistencia total normalizada según el porcentaje de avance en las
			redes para los modelos ResNet-18, DenseNet-121 y EfficientNet-B0.}
		\label{fig:m-homology-arch-2}
	\end{subfigure}
	\caption{Comparación de la persistencia total (a) y la persistencia total
		normalizada (b) de diferentes arquitecturas de redes neuronales en función del
		porcentaje de avance de los datos a través de la red para la especificidad
		Marca-Modelo.}
	\label{fig:m-homology-arch}
\end{figure}

\paragraph{Especificidad Marca-Modelo}

Los resultados de las gráficas de la \autoref{fig:mm-homology-arch} muestran una
tendencia similar. En este caso, las tendencias del modelo DenseNet-121 en la
\autoref{fig:mm-homology-arch-1} muestran una pendiente más pronunciada, indicando
transformaciones más agresivas en la homología persistente de los datos. Por
otro lado, la \autoref{fig:mm-homology-arch-2} muestra un comportamiento algo diferente
respecto a la obtenida en la especificidad Marca, obteniendo unos valores de
persistencia total normalizada notablemente superiores al final de las
ejecuciones. Dicha consecuencia puede deberse al aumento en la granularidad de la
clasificación, haciendo necesarias un mayor número de componentes conexas para
facilitar dicha tarea.

\begin{figure}[H]
	\centering
	\begin{subfigure}
		{.5\textwidth}
		\centering
		\includegraphics[width=\linewidth]{img/mm_arch.png}
		\caption{Persistencia total según el porcentaje de avance en las redes para
			los modelos ResNet-18, DenseNet-121 y EfficientNet-B0.}
		\label{fig:mm-homology-arch-1}
	\end{subfigure}%
	\begin{subfigure}
		{.5\textwidth}
		\centering
		\includegraphics[width=\linewidth]{img/mm_arch_norm.png}
		\caption{Persistencia total normalizada según el porcentaje de avance en las
			redes para los modelos ResNet-18, DenseNet-121 y EfficientNet-B0.}
		\label{fig:mm-homology-arch-2}
	\end{subfigure}
	\caption{Comparación de la persistencia total (a) y la persistencia total
		normalizada (b) de diferentes arquitecturas de redes neuronales en función del
		porcentaje de avance de los datos a través de la red para la especificidad
		Marca-Modelo.}
	\label{fig:mm-homology-arch}
\end{figure}

Es claro que las Figuras \ref{fig:m-homology-arch} y \ref{fig:mm-homology-arch} indican
que la arquitectura escogida es un factor determinante en las transformaciones que
los datos sufren desde el punto de vista topológico. Todos los modelos
entrenados con la misma arquitectura muestran evoluciones muy similares incluso
para las distintas especificidades, donde si que se aprecia un desplazamiento vertical
de la homología persistente en función de la granularidad de las clases del conjunto.

\subsection{Comparación según el optimizador}
\label{subsec:optim}

\paragraph{Especificidad Marca}

A diferencia de la comparativa en función de la arquitectura, la \autoref{fig:m-homology-optim}
no muestra patrones tan claros en cómo afecta el optimizador a la persistencia homológica.
La \autoref{fig:m-homology-optim-1} muestra cómo los modelos que presentan una
menor persistencia total al inicio, como ResNet-18, tienden a presentar una
persistencia inicial todavía menor para el optimizador SGD. Sin embargo, esta
tendencia empieza a cambiar cuando vamos pasando a modelos de mayor complejidad
topológica como DenseNet-121, donde en general Adam parece mostrar una menor persistencia
inicial. Por otro lado, la \autoref{fig:m-homology-optim-2} no parece mostrar ningún
patrón que nos indique que la elección del optimizador sea relevante para la modificación
de la homología persistente de los datos.

\begin{figure}[H]
	\centering
	\begin{subfigure}
		{.5\textwidth}
		\centering
		\includegraphics[width=\linewidth]{img/m_optim.png}
		\caption{Persistencia total según el porcentaje de avance en las redes para
			optimizadores SGD y Adam.}
		\label{fig:m-homology-optim-1}
	\end{subfigure}%
	\begin{subfigure}
		{.5\textwidth}
		\centering
		\includegraphics[width=\linewidth]{img/m_optim_norm.png}
		\caption{Persistencia total normalizada según el porcentaje de avance en las
			redes para SGD y Adam.}
		\label{fig:m-homology-optim-2}
	\end{subfigure}
	\caption{Comparación de la persistencia total (a) y la persistencia total
		normalizada (b) de diferentes optimizadores de redes neuronales en función del
		porcentaje de avance de los datos a través de la red para la especificidad
		Marca.}
	\label{fig:m-homology-optim}
\end{figure}

\paragraph{Especificidad Marca-Modelo}

De nuevo, los resultados obtenidos en la \autoref{fig:mm-homology-optim} son poco
esclarecedores acerca de la homología de los datos. No obstante, el aumento en el
número de clases parece haber homogeneizado las diferencias observadas en la persistencia
total, tal y como muestra la \autoref{fig:mm-homology-optim-1}. Además, las
tendencias en la \autoref{fig:mm-homology-optim-2} muestran una mayor
persistencia total normalizada al final de la inferencia para los modelos
entrenados con Adam.

\begin{figure}[H]
	\centering
	\begin{subfigure}
		{.5\textwidth}
		\centering
		\includegraphics[width=\linewidth]{img/mm_optim.png}
		\caption{Persistencia total normalizada según el porcentaje de avance en la
			red para optimizadores SGD y Adam.}
		\label{fig:mm-homology-optim-1}
	\end{subfigure}%
	\begin{subfigure}
		{.5\textwidth}
		\centering
		\includegraphics[width=\linewidth]{img/mm_optim_norm.png}
		\caption{Persistencia total normalizada según el porcentaje de avance en la
			red para optimizadores SGD y Adam.}
		\label{fig:mm-homology-optim-2}
	\end{subfigure}
	\caption{Comparación de la persistencia total (a) y la persistencia total
		normalizada (b) de diferentes optimizadores de redes neuronales en función del
		porcentaje de avance de los datos a través de la red para la especificidad
		Marca-Modelo.}
	\label{fig:mm-homology-optim}
\end{figure}

Los resultados recién vistos sobre las Figuras \ref{fig:m-homology-optim} y \ref{fig:mm-homology-optim}
parecen mostrar que el optimizador escogido (al menos, en el caso de los dos empleados)
no es un factor especialmente relevante a la hora de modificar la \enquote{forma}
de los datos. A pesar de ello, las tendencias observadas al inicio de la red en la
persistencia total y al final de ella en la persistencia total normalizada muestran
de manera débil ciertos patrones que podrían estudiarse en más profundidad.

\subsection{Comparación según el tamaño de lote}
\label{subsec:batch}

\paragraph{Especificidad Marca}

Las gráficas de la \autoref{fig:m-homology-batch} no muestran ninguna influencia
directa o significativa de la elección del tamaño de lote en el ámbito de la
topología de los datos durante la etapa de entrenamiento. Las curvas de regresión
se muestran muy entrelazadas y las nubes de puntos no muestran claras distinciones
o agrupaciones.

\begin{figure}[H]
	\centering
	\begin{subfigure}
		{.5\textwidth}
		\centering
		\includegraphics[width=\linewidth]{img/m_batch.png}
		\caption{Persistencia total según el porcentaje de avance en las redes
			entrenadas para diferentes tamaños de lote.}
		\label{fig:m-homology-batch-1}
	\end{subfigure}%
	\begin{subfigure}
		{.5\textwidth}
		\centering
		\includegraphics[width=\linewidth]{img/m_batch_norm.png}
		\caption{Persistencia total normalizada según el porcentaje de avance en las
			redes para diferentes tamaños de lote.}
		\label{fig:m-homology-batch-2}
	\end{subfigure}
	\caption{Comparación de la persistencia total (a) y la persistencia total
		normalizada (b) para diferentes tamaños de lote en función del porcentaje de
		avance de los datos a través de la red para la especificidad Marca.}
	\label{fig:m-homology-batch}
\end{figure}

\paragraph{Especificidad Marca-Modelo}

La misma observación se obtiene para los modelos entrenados en la especificidad
Marca-Modelo, donde las gráficas de la \autoref{fig:mm-homology-batch} muestran un
patrón bastante similar a las recién comentadas.

\begin{figure}[H]
	\centering
	\begin{subfigure}
		{.5\textwidth}
		\centering
		\includegraphics[width=\linewidth]{img/mm_batch.png}
		\caption{Persistencia total según el porcentaje de avance en las redes para
			diferentes tamaños de lote.}
		\label{fig:mm-homology-batch-1}
	\end{subfigure}%
	\begin{subfigure}
		{.5\textwidth}
		\centering
		\includegraphics[width=\linewidth]{img/mm_batch_norm.png}
		\caption{Persistencia total normalizada según el porcentaje de avance en las
			redes para diferentes tamaños de lote.}
		\label{fig:mm-homology-batch-2}
	\end{subfigure}
	\caption{Comparación de la persistencia total (a) y la persistencia total
		normalizada (b) para diferentes tamaños de lote en función del porcentaje de
		avance de los datos a través de la red para la especificidad Marca-Modelo.}
	\label{fig:mm-homology-batch}
\end{figure}

Es curioso observar que esta elección de hiperparámetros no muestre alteraciones
en la homología persistente de los datos de test estudiados. Este hecho podría implicar
que los métodos de optimización empleados son poco sensibles al tamaño de lote escogido
a la hora de inferir la variedad subyacente de los datos.

\subsection{Comparación en presencia de aumento de datos}
\label{subsec:aug}

A continuación trabajaremos sobre los modelos escogidos en la
\autoref{subsec:hiperparam} para cada especificidad.

\paragraph{Especificidad Marca: EfficientNet-B0}

La \autoref{fig:m-homology} muestra los resultados de persistencia total y
persistencia total normalizada para el modelo base de EfficientNet-B0 y su
variante con aumento de datos. Podemos observar que el modelo que obtuvo mejores
métricas en el conjunto de test, el modelo base, presenta una persistencia total
superior tanto al inicio como al final respecto al modelo con aumento de datos,
mientras que es menor en el punto medio de la ejecución, tal y como se ve en la \autoref{fig:m-homology-1}.
Esto es, el modelo con mejores métricas muestra transformaciones más agresivas sobre
la variedad subyacente de los datos.

En cuanto a la \autoref{fig:m-homology-2}, vemos que el modelo con aumento de
datos presenta una persistencia total normalizada inicial más alta que el modelo
base y una final más baja. Además, vemos que el máximo de persistencia lo
alcanza antes que el modelo base, mostrando una traslación del proceso de aumento
de la complejidad en homología persistente a etapas más tempranas de la ejecución.

\begin{figure}[H]
	\centering
	\begin{subfigure}
		{.5\textwidth}
		\centering
		\includegraphics[width=\linewidth]{img/m.png}
		\caption{Persistencia total según el porcentaje de avance en la red para el
			modelo base entrenado EfficientNet-B0 y su versión con aumento de datos.}
		\label{fig:m-homology-1}
	\end{subfigure}%
	\begin{subfigure}
		{.5\textwidth}
		\centering
		\includegraphics[width=\linewidth]{img/m_norm.png}
		\caption{Persistencia total normalizada según el porcentaje de avance en la
			red para el modelo base entrenado EfficientNet-B0 y su versión con aumento
			de datos.}
		\label{fig:m-homology-2}
	\end{subfigure}
	\caption{Comparación de la persistencia total (a) y la persistencia total
		normalizada (b) para EfficientNet-B0 Base y EfficientNet-B0 Transforms en
		función del porcentaje de avance de los datos a través de la red para la
		especificidad Marca.}
	\label{fig:m-homology}
\end{figure}

\paragraph{Especificidad Marca-Modelo: DenseNet-121}

Estas observaciones se ven reforzadas en la \autoref{fig:mm-homology}. En particular,
la \autoref{fig:mm-homology-1} muestra dichas observaciones de una manera más
agresiva. Vemos que el mejor modelo, DenseNet-121 con aumento de datos, muestra una
persistencia total mayor al inicio de la red y al final. Además, la \autoref{fig:mm-homology-2}
muestra como el aumento de datos traslada de nuevo el máximo a momentos más tempranos
de ejecución y una fuerte reducción de la complejidad total normalizada al final
de la red.

\begin{figure}[H]
	\centering
	\begin{subfigure}
		{.5\textwidth}
		\centering
		\includegraphics[width=\linewidth]{img/mm.png}
		\caption{Persistencia total según el porcentaje de avance en la red para el
			modelo base entrenado EfficientNet-B0 y su versión con aumento de datos.}
		\label{fig:mm-homology-1}
	\end{subfigure}%
	\begin{subfigure}
		{.5\textwidth}
		\centering
		\includegraphics[width=\linewidth]{img/mm_norm.png}
		\caption{Persistencia total normalizada según el porcentaje de avance en la
			red para el modelo base entrenado EfficientNet-B0 y su versión con aumento
			de datos.}
		\label{fig:mm-homology-2}
	\end{subfigure}
	\caption{Comparación de la persistencia total (a) y la persistencia total
		normalizada (b) para EfficientNet-B0 Base y EfficientNet-B0 Transforms en
		función del porcentaje de avance de los datos a través de la red para la
		especificidad Marca-Modelo.}
	\label{fig:mm-homology}
\end{figure}

Los resultados recién vistos muestran patrones interesantes en el comportamiento
del modelo cuando realizamos aumento de datos: tiende a realizar modificaciones más
intensas en fases anteriores de la red y simplificar las estructuras al final del
modelo. Este comportamiento muestra que las componentes conexas al final de la
red son más compactas y están mejor definidas, lo que lleva a una reducción de la
complejidad topológica. No solo eso, si no que al tener en cuenta las clases de persistencia
homológicas de dimensión 1, también se reducen el número de componentes conexas
que forman bucles, lo que evita entrelazamientos entre éstas.

\subsection{Comparación según la granularidad de las clases}
\label{subsec:grano}

\paragraph{EfficientNet-B0}

EfficientNet-B0 ha mostrado el mejor rendimiento en las métricas empleadas para
la especificidad Marca. La persistencia total del modelo entrenado en Marca-Modelo
en la \autoref{fig:efficientnet-1} muestra transformaciones más agresivas en la homología
persistente de los datos. Aquí, en los extremos de la ejecución los datos presentan
una persistencia total más alta y un mínimo inferior al del modelo entrenado para
Marca.

Por su parte, la persistencia total normalizada muestra valores regularmente
superiores para la especificidad Marca-Modelo a los de Marca (\autoref{fig:efficientnet-2}).
Estos hechos parecen coherentes, pues a pesar de corresponderse con el mismo
conjunto de datos, la necesidad de etiquetar una mayor variedad de clases implica
un desglose más fino de las características de los datos y en consecuencia, una
complejidad topológica mayor.

\begin{figure}[H]
	\centering
	\begin{subfigure}
		{.5\textwidth}
		\centering
		\includegraphics[width=\linewidth]{img/general_efficientnet.png}
		\caption{Persistencia total según el porcentaje de avance en la red para el
			modelo EfficientNet-B0 entrenado para las especificidades Marca y Marca-Modelo.}
		\label{fig:efficientnet-1}
	\end{subfigure}%
	\begin{subfigure}
		{.5\textwidth}
		\centering
		\includegraphics[width=\linewidth]{img/general_efficientnet_norm.png}
		\caption{Persistencia total normalizada según el porcentaje de avance en la
			red para el modelo EfficientNet-B0 entrenado para las especificidades Marca
			y Marca-Modelo.}
		\label{fig:efficientnet-2}
	\end{subfigure}
	\caption{Comparación de la persistencia total (a) y la persistencia total
		normalizada (b) para EfficientNet-B0 entrenado con SGD y un tamaño de lote 64
		en función del porcentaje de avance de los datos a través de la red para las
		especificidades Marca y Marca-Modelo.}
	\label{fig:efficientnet}
\end{figure}

\paragraph{DenseNet-121}

En cuanto a DenseNet-121, vemos en la \autoref{fig:densenet-1} que ambas curvas
se asemejan más que en el caso anterior. Es interesante ver como en el último instante
de la red, el modelo entrenado para la especificidad Marca-Modelo presenta una
subida de persistencia considerable, mientras que la de Marca es más modesta.
Este hecho muestra claramente la necesidad del modelo de deshacer la
simplificación realizada con el fin de separar los datos para la clasificación.

Además, las muestras tomadas en los distintos instantes de la red muestran cuatro
puntos con un aumento considerable de la persistencia total. Estos instantes
coinciden con los puntos donde se encuentran las activaciones de transición
entre los cuatro bloques densos que presenta DenseNet-121, mostrando cómo las conexiones
residuales entre dichos bloques aumenta la complejidad topológica de los datos.

Por otro lado, la \autoref{fig:densenet-2} nos muestra cómo, de nuevo, la
persistencia total normalizada presenta en general valores inferiores para la
especificidad Marca. Otra observación relevante es la reducción de complejidad más
progresiva y gradual que muestra el modelo de Marca en la segunda mitad de la inferencia.
El hecho de que DenseNet-121 entrenado con una mayor granularidad requiera de
mayor persistencia durante más tiempo puede deberse a la clara dificultad
añadida por el aumento de clases.

\begin{figure}[H]
	\centering
	\begin{subfigure}
		{.5\textwidth}
		\centering
		\includegraphics[width=\linewidth]{img/general_densenet.png}
		\caption{Persistencia total según el porcentaje de avance en la red para el
			modelo DenseNet-121 entrenado para las especificidades Marca y Marca-Modelo.}
		\label{fig:densenet-1}
	\end{subfigure}%
	\begin{subfigure}
		{.5\textwidth}
		\centering
		\includegraphics[width=\linewidth]{img/general_densenet_norm.png}
		\caption{Persistencia total normalizada según el porcentaje de avance en la
			red para el modelo DenseNet-121 entrenado para las especificidades Marca y
			Marca-Modelo.}
		\label{fig:densenet-2}
	\end{subfigure}
	\caption{Comparación de la persistencia total (a) y la persistencia total
		normalizada (b) para DenseNet-121 entrenado con SGD y un tamaño de lote 32 en
		función del porcentaje de avance de los datos a través de la red para las
		especificidades Marca y Marca-Modelo.}
	\label{fig:densenet}
\end{figure}

\subsection{Comparación según el subconjunto de datos}
\label{subsec:set}

Hasta ahora hemos estado viendo como distintos modelos con distintos
hiperparámetros y granularidad durante la etapa de entrenamiento afectan a la
variedad subyacente de los datos empleados. A continuación, compararemos cómo
transforma un mismo modelo los propios datos en función del subconjunto al que
pertenecen: entrenamiento, validación y test.

\paragraph{Especificidad Marca: EfficientNet-B0}

Lo primero que observamos en las Figuras \ref{fig:m_set_base} y
\ref{fig:m_set_trans} es una mayor persistencia total al inicio sobre el
conjunto de entrenamiento respecto al de validación y test en ambos modelos. Por
otro lado, en la \autoref{fig:m_set_base}, que muestra los resultados con mejores
métricas, los conjuntos de validación y test muestran una persistencia total
prácticamente idéntica, mientras que el modelo con aumento de datos presenta mayores
discrepancias al respecto.

Por lo que se refiere a la persistencia total normalizada (Figuras
\ref{fig:m_set_base_norm} y \ref{fig:m_set_trans_norm}), observamos tendencias
muy similares, donde las curvas de validación y test muestran un ajuste más fino
al de entrenamiento para el modelo base que para el modelo con aumento de datos.

\begin{figure}[H]
	\centering
	\begin{subfigure}
		{.45\textwidth}
		\centering
		\includegraphics[width=\linewidth]{img/m_set_base.png}
		\caption{Persistencia total según el porcentaje de avance en la red para los
			conjuntos de entrenamiento, validación y test.}
		\label{fig:m_set_base}
	\end{subfigure}
	\begin{subfigure}
		{.45\textwidth}
		\centering
		\includegraphics[width=\linewidth]{img/m_set_base_norm.png}
		\caption{Persistencia total normalizada según el porcentaje de avance en la
			red para los conjuntos de entrenamiento, validación y test.}
		\label{fig:m_set_base_norm}
	\end{subfigure}
	\begin{subfigure}
		{.45\textwidth}
		\centering
		\includegraphics[width=\linewidth]{img/m_set_trans.png}
		\caption{Persistencia total según el porcentaje de avance en la red con
			aumento de datos para los conjuntos de entrenamiento, validación y test.}
		\label{fig:m_set_trans}
	\end{subfigure}%
	\begin{subfigure}
		{.45\textwidth}
		\centering
		\includegraphics[width=\linewidth]{img/m_set_trans_norm.png}
		\caption{Persistencia total normalizada según el porcentaje de avance en la
			red con aumento de datos para los conjuntos de entrenamiento, validación y test.}
		\label{fig:m_set_trans_norm}
	\end{subfigure}
	\caption{Comparación de la persistencia total (a, c) y la persistencia total normalizada
		(b, d) para los conjuntos de entrenamiento, validación y test para la especificidad
		Marca. Las Figuras (a) y (b) representan el modelo base, mientras que las Figuras
		(c) y (d) representan el modelo con aumento de datos.}
	\label{fig:m-set}
\end{figure}

\paragraph{Especificidad Marca-Modelo: DenseNet-121}

Las Figuras \ref{fig:mm_set_base} y \ref{fig:mm_set_trans} muestran un ajuste de
la persistencia total prácticamente perfecto entre los tres subconjuntos de
datos. Más aún, podemos ver cómo el modelo que ha presentado mejores métricas en
el conjunto de test, es decir, el modelo con aumento de datos, presenta un ajuste
mucho más fino entre los valores de persistencia total normalizados para los tres
subconjuntos (\autoref{fig:mm_set_trans_norm}). Es interesante observar como en
esta ocasión, los puntos obtenidos se comportan de manera más homogénea independientemente
del subconjunto escogido, a diferencia de lo observado en la
\autoref{fig:mm_set_base_norm}.

\begin{figure}[H]
	\centering
	\begin{subfigure}
		{.45\textwidth}
		\centering
		\includegraphics[width=\linewidth]{img/mm_set_base.png}
		\caption{Persistencia total según el porcentaje de avance en la red para los
			conjuntos de entrenamiento, validación y test.}
		\label{fig:mm_set_base}
	\end{subfigure}
	\begin{subfigure}
		{.45\textwidth}
		\centering
		\includegraphics[width=\linewidth]{img/mm_set_base_norm.png}
		\caption{Persistencia total normalizada según el porcentaje de avance en la
			red para los conjuntos de entrenamiento, validación y test.}
		\label{fig:mm_set_base_norm}
	\end{subfigure}
	\begin{subfigure}
		{.45\textwidth}
		\centering
		\includegraphics[width=\linewidth]{img/mm_set_trans.png}
		\caption{Persistencia total según el porcentaje de avance en la red con
			aumento de datos para los conjuntos de entrenamiento, validación y test.}
		\label{fig:mm_set_trans}
	\end{subfigure}%
	\begin{subfigure}
		{.45\textwidth}
		\centering
		\includegraphics[width=\linewidth]{img/mm_set_trans_norm.png}
		\caption{Persistencia total normalizada según el porcentaje de avance en la
			red con aumento de datos para los conjuntos de entrenamiento, validación y test.}
		\label{fig:mm_set_trans_norm}
	\end{subfigure}
	\caption{Comparación de la persistencia total (a, c) y la persistencia total normalizada
		(b, d) para los conjuntos de entrenamiento, validación y test para la especificidad
		Marca-Modelo. Las Figuras (a) y (b) representan el modelo base, mientras que las
		Figuras (c) y (d) representan el modelo con aumento de datos.}
	\label{fig:mm-set}
\end{figure}

Las Figuras \ref{fig:m-set} y \ref{fig:mm-set} muestran cómo las redes mejor entrenadas
obtienen una coincidencia mayor entre los distintos subconjuntos de datos empleados
típicamente en el entrenamiento de las CNNs, lo que indica un mejor aprendizaje
de la variedad subyacente de la que parten los datos.

\subsection{Reflexiones sobre la topología en CNNs}

El análisis recién realizado sobre cómo diferentes configuraciones de modelos de
CNNs afectan las características topológicas de los datos ha mostrado patrones claros.
Las arquitecturas estudiadas comienzan simplificando los datos, probablemente como
consecuencia de los cambios de la dimensionalidad de las activaciones y para eliminar
ruido o detalles innecesarios. Sin embargo, hacia el final del proceso de aprendizaje,
la persistencia total aumenta nuevamente. Esta dinámica parece ser una estrategia
para crear representaciones más ricas y detalladas que ayuden a diferenciar mejor
entre las clases durante la clasificación de instancias.

El tipo de optimizador utilizado, como Adam o SGD, también juega un papel relevante,
pero su impacto varía a lo largo del entrenamiento. Adam, que ajusta
dinámicamente las tasas de aprendizaje, parece ser especialmente efectivo en las
fases iniciales, facilitando una rápida reducción de la persistencia total. En cambio,
SGD adopta un enfoque diferente, partiendo de una mayor complejidad y realizando
transformaciones más agresivas en la topología de los datos.

Sorprendentemente, el tamaño del lote mostró tener un impacto mínimo en la
homología persistente. Esto indica que los ajustes en la topología de los datos son
bastante robustos a variaciones en el tamaño del lote, siendo este un factor a
tener en cuenta a la hora de trabajar con la topología de los datos.

El aumento de datos es otro factor importante. Al entrenar modelos con una mayor
variedad de datos durante el entrenamiento, se promueve una adaptación más
temprana y efectiva a la complejidad relativa de los datos, lo que lleva a una menor
necesidad de trabajar los datos de manera tan intensiva en etapas posteriores.
Esto no solo ayuda a generalizar mejor sino también a simplificar eficazmente
esa complejidad hacia el final del entrenamiento, preparando el modelo para tareas
de clasificación más precisas.

La granularidad en la clasificación necesita que los modelos manejen y preserven
una complejidad topológica mayor durante más tiempo, manteniendo las
distinciones claras entre múltiples categorías. Al comparar cómo se comportan los
modelos en diferentes subconjuntos de datos, es notable que aquellos con mejores
métricas ajustan más precisamente sus características topológicas entre los
conjuntos de entrenamiento, validación y prueba, mostrando un aprendizaje más
consistente y profundo de la estructura subyacente de los datos.

\section{Propuesta de mejora: regularización topológica}
\label{subsec:proposal}

El análisis realizado sobre la homología persistente de los datos ha mostrado ciertos
patrones y resultados que podrían aprovecharse durante el entrenamiento de las CNNs.
En esta sección aplicamos las propuestas realizadas con el fin de confirmar
nuestras hipótesis y hallar nuevas técnicas que mejoren y faciliten el
aprendizaje a las CNNs.

\subsection{Mejora en clasificación}

\paragraph{Especificidad Marca: EfficientNet-B0}

La propuesta de emplear el regularizador topológico para mejorar la tasa de clasificación
en EfficientNet-B0 parece haber mostrado una leve mejoría respecto al modelo base
y el modelo refinado sin emplear el regularizador (\autoref{tab:efficientnet-refined}).

\begin{table}[H]
	\centering
	\begin{adjustbox}
		{width=0.8\textwidth}
		\begin{tabular}{|c|c|c|c|c|c|}
			\hline
			\textbf{Modelo}          & $\alpha$ & \textbf{Exactitud} & \textbf{Precisión} & \textbf{Sensibilidad} & \textbf{F1-Score} \\
			\hline
			EfficientNet-B0 Base     & -        & 0.9505             & 0.9462             & 0.9386                & 0.9423            \\
			\hline
			EfficientNet-B0 Refinado & 0.0      & 0.9598             & 0.9512             & 0.9445                & 0.9478            \\
			\hline
			EfficientNet-B0 Refinado & 0.001    & 0.9598             & 0.9496             & 0.9448                & 0.9472            \\
			\hline
			EfficientNet-B0 Refinado & 0.005    & \textbf{0.9659}    & \textbf{0.9545}    & \textbf{0.9514}       & \textbf{0.9529}   \\
			\hline
			EfficientNet-B0 Refinado & 0.01     & 0.9598             & 0.9517             & 0.9481                & 0.9499            \\
			\hline
			EfficientNet-B0 Refinado & 0.05     & 0.9536             & 0.9515             & 0.9412                & 0.9462            \\
			\hline
			EfficientNet-B0 Refinado & 0.1      & 0.9567             & 0.9475             & 0.9413                & 0.9444            \\
			\hline
			EfficientNet-B0 Refinado & 0.5      & 0.9505             & 0.9486             & 0.9369                & 0.9426            \\
			\hline
			EfficientNet-B0 Refinado & 1.0      & 0.9505             & 0.9453             & 0.9413                & 0.9432            \\
			\hline
		\end{tabular}
	\end{adjustbox}
	\caption{Comparación de métricas tras un proceso de refinamiento en la misma
		especificidad Marca para distintos valores de $\alpha$ en el término de
		regularización del modelo EfficientNet-B0.}
	\label{tab:efficientnet-refined}
\end{table}

Observando la \autoref{fig:efficientnet-refine-1} vemos que la persistencia
total del modelo refinado sin regularizador es generalmente inferior a la del
modelo regularizado. Sin embargo, en las etapas finales muestra un leve repunte
frente al modelo sin regularizar. La persistencia total normalizada sigue mostrando
un comportamiento similar en ambos modelos, siendo algo inferior en la salida
del modelo regularizado (\autoref{fig:efficientnet-refine-2}) .

\begin{figure}[H]
	\centering
	\begin{subfigure}
		{.5\textwidth}
		\centering
		\includegraphics[width=\linewidth]{img/exp_refine_efficientnet.png}
		\caption{Persistencia total según el porcentaje de avance en la red para el
			modelo EfficientNet-B0 refinado para la especificidad Marca.}
		\label{fig:efficientnet-refine-1}
	\end{subfigure}%
	\begin{subfigure}
		{.5\textwidth}
		\centering
		\includegraphics[width=\linewidth]{img/exp_refine_efficientnet_norm.png}
		\caption{Persistencia total normalizada según el porcentaje de avance en la
			red para el modelo EfficientNet-B0 refinado para Marca.}
		\label{fig:efficientnet-refine-2}
	\end{subfigure}
	\caption{Comparación de la persistencia total (a) y la persistencia total
		normalizada (b) para EfficientNet-B0 refinado con SGD y un tamaño de lote 64
		en función del porcentaje de avance de los datos a través de la red para las
		especificidad Marca con y sin regularizador.}
	\label{fig:efficientnet-refine}
\end{figure}

El ejemplo que se ve en la \autoref{fig:ex-efficientnet-refine} muestra la
diferencia entre las activaciones en ambas versiones. En la versión regularizada
desaparecen ciertas activaciones pequeñas para centrarse principalmente en el logo
del coche.

\begin{figure}[H]
	\centering
	\begin{subfigure}
		{.45\textwidth}
		\centering
		\includegraphics[width=\linewidth]{img/192_false.png}
		\caption{Ejemplo mal clasificado para el modelo EfficientNet-B0 refinado
			para la especificidad Marca.}
		\label{fig:ex-efficientnet-refine-1}
	\end{subfigure}%
	\qquad
	\begin{subfigure}
		{.45\textwidth}
		\centering
		\includegraphics[width=\linewidth]{img/192_true.png}
		\caption{Ejemplo bien clasificado para el modelo EfficientNet-B0 refinado
			con regularización para Marca.}
		\label{fig:ex-efficientnet-refine-2}
	\end{subfigure}
	\caption{Comparación de clasificación para el caso sin regularizar (a) frente
		al regularizado (b) para EfficientNet-B0.}
	\label{fig:ex-efficientnet-refine}
\end{figure}

Es interesante ver cómo pese a la gran similaridad que presentan ambas
alternativas en sus código de barras, pequeñas diferencias logran marcar la diferencia
(\autoref{fig:efficientnet-refine-bc}). Estos detalles son más notables al 100\%
de la ejecución, donde el modelo regularizado topológicamente muestra intervalos
de persistencia de dimensión 1 más cortos.

\paragraph{Especificidad Marca-Modelo: DenseNet-121}

Las observaciones recién realizadas sobre los valores de las métricas se repiten
con mayor fuerza para DenseNet-121 en Marca-Modelo (\autoref{tab:densenet-refine}).
Esta aportación, para el caso donde $\alpha = 0.1$, lleva a una mejora por encima
del 2\% en la exactitud del modelo, mientras que F1-Score llega a aumentar hasta
un 3\% sobre el modelo base. No sólo eso, sino que estos valores superan a los
obtenidos a los del modelo refinado sin el término de regularización, mostrando su
eficacia.

\begin{table}[h]
	\centering
	\begin{adjustbox}
		{width=0.8\textwidth}
		\begin{tabular}{|c|c|c|c|c|c|}
			\hline
			\textbf{Modelo}       & $\alpha$ & \textbf{Exactitud} & \textbf{Precisión} & \textbf{Sensibilidad} & \textbf{F1-Score} \\
			\hline
			DenseNet-121 Base     & -        & 0.9185             & 0.9077             & 0.9126                & 0.9101            \\
			\hline
			DenseNet-121 Refinado & 0.0      & 0.9333             & 0.9220             & 0.9286                & 0.9253            \\
			\hline
			DenseNet-121 Refinado & 0.001    & 0.9333             & 0.9220             & 0.9286                & 0.9253            \\
			\hline
			DenseNet-121 Refinado & 0.005    & 0.9370             & 0.9269             & 0.9336                & 0.9302            \\
			\hline
			DenseNet-121 Refinado & 0.01     & 0.9333             & 0.9220             & 0.9286                & 0.9253            \\
			\hline
			DenseNet-121 Refinado & 0.05     & 0.9370             & 0.9284             & 0.9334                & 0.9308            \\
			\hline
			DenseNet-121 Refinado & 0.1      & \textbf{0.9407}    & \textbf{0.9384}    & \textbf{0.9414}       & \textbf{0.9399}   \\
			\hline
			DenseNet-121 Refinado & 0.5      & 0.9370             & 0.9297             & 0.9330                & 0.9313            \\
			\hline
			DenseNet-121 Refinado & 1.0      & 0.9000             & 0.8997             & 0.8945                & 0.8970            \\
			\hline
		\end{tabular}
	\end{adjustbox}
	\caption{Comparación de métricas tras un proceso de refinamiento en la misma
		especificidad Marca-Modelos para distintos valores de $\alpha$ en el término
		de regularización del modelo DenseNet-121.}
	\label{tab:densenet-refine}
\end{table}

\begin{figure}[H]
	\centering
	% First Column
	\begin{subfigure}
		{.5\textwidth}
		\centering
		\includegraphics[width=\linewidth]{img/bar_effcientnet_base_0.01.png}
	\end{subfigure}%
	\begin{subfigure}
		{.5\textwidth}
		\centering
		\includegraphics[width=\linewidth]{img/bar_effcientnet_refine_0.01.png}
	\end{subfigure}
	\begin{subfigure}
		{.5\textwidth}
		\centering
		\includegraphics[width=\linewidth]{img/bar_effcientnet_base_0.50.png}
	\end{subfigure}%
	\begin{subfigure}
		{.5\textwidth}
		\centering
		\includegraphics[width=\linewidth]{img/bar_effcientnet_refine_0.50.png}
	\end{subfigure}
	\begin{subfigure}
		{.5\textwidth}
		\centering
		\includegraphics[width=\linewidth]{img/bar_effcientnet_base_1.00.png}
	\end{subfigure}%
	\begin{subfigure}
		{.5\textwidth}
		\centering
		\includegraphics[width=\linewidth]{img/bar_effcientnet_refine_1.00.png}
	\end{subfigure}
	
	\caption{Comparación de los códigos de barras del modelo EfficientNet-B0 para
		el refinamiento en la especificidad Marca sin regularización (izquierda)
		frente al modelo refinado con regularización (derecha). La figura muestra los códigos
		de barras para el conjunto de test tras avanzar un $1\%$ (arriba), un $50\%$ (centro)
		y el $100\%$ (abajo) de la red.}
	\label{fig:efficientnet-refine-bc}
\end{figure}

Las gráficas de persistencia (\autoref{fig:densenet-norm}) apenas muestran diferencias
para ambas versiones, con la excepción de la \autoref{fig:densenet-norm-2}, donde
el modelo regularizado muestra al final del proceso una persistencia total normalizada
ligeramente inferior. Este hecho es especialmente curioso si observamos los
códigos de barras obtenidos en la \autoref{fig:densenet-refine-bc}. Si bien los intervalos
de persistencia son similares a comienzos y mediados de la inferencia, al final
de la red observamos un menor número de componentes conexas persistentes y más
duraderas. Es más, observamos también que aumentan el número y la longitud de
clases de homología persistente de dimensión 1, lo que choca con las hipótesis planteadas.
Un motivo podría deberse a que la importante reducción en el número de
componentes conexas persistentes haya beneficiado más a la reducción de
persistencia total normalizada, de forma que el sacrificio en homología persistente
en dimensión 1 siga siendo una mejor opción en el cómputo total.

\begin{figure}[H]
	\centering
	\begin{subfigure}
		{.5\textwidth}
		\centering
		\includegraphics[width=\linewidth]{img/exp_refine_densenet.png}
		\caption{Persistencia total normalizada según el porcentaje de avance en la
			red para el modelo DenseNet-121 refinado para Marca-Modelo.}
		\label{fig:densenet-norm-1}
	\end{subfigure}%
	\begin{subfigure}
		{.5\textwidth}
		\centering
		\includegraphics[width=\linewidth]{img/exp_refine_densenet_norm.png}
		\caption{Persistencia total normalizada según el porcentaje de avance en la
			red para el modelo DenseNet-121 refinado para Marca-Modelo.}
		\label{fig:densenet-norm-2}
	\end{subfigure}
	\caption{Comparación de la persistencia total (a) y la persistencia total
		normalizada (b) para DenseNet-121 refinado con SGD y un tamaño de lote 32 en
		función del porcentaje de avance de los datos a través de la red para la
		especificidad Marca-Modelo con y sin regularizador.}
	\label{fig:densenet-norm}
\end{figure}

\begin{figure}[H]
	\centering
	\begin{subfigure}
		{.45\textwidth}
		\centering
		\includegraphics[width=\linewidth]{img/220_false.png}
		\caption{Ejemplo mal clasificado para el modelo DenseNet-121 refinado para
			la especificidad Marca-Modelo.}
		\label{fig:ex-densenet-refine-1}
	\end{subfigure}%
	\qquad
	\begin{subfigure}
		{.45\textwidth}
		\centering
		\includegraphics[width=\linewidth]{img/220_true.png}
		\caption{Ejemplo bien clasificado para el modelo DenseNet-121 refinado con
			regularización para Marca-Modelo.}
		\label{fig:ex-densenet-refine-2}
	\end{subfigure}
	\caption{Comparación de clasificación para el caso sin regularizar (a) frente
		al regularizado (b) para DenseNet-121 para la especificidad Marca-Modelo.}
	\label{fig:ex-densenet-refine}
\end{figure}

Esta vez, al tratar con una especificidad más fina, el logotipo de la marca no parece
ser suficiente criterio para detecta también el modelo, ampliando el área de
estudio (\autoref{fig:ex-densenet-refine}). Al emplear el regularizador topológico
vemos que el modelo se centra más en otros aspectos como los retrovisores o el techo
del coche frente a la versión sin regularizar.

\begin{figure}[H]
	\centering
	% First Column
	\begin{subfigure}
		{.5\textwidth}
		\centering
		\includegraphics[width=\linewidth]{img/bar_densenet_base_0.01.png}
	\end{subfigure}%
	\begin{subfigure}
		{.5\textwidth}
		\centering
		\includegraphics[width=\linewidth]{img/bar_densenet_refine_0.01.png}
	\end{subfigure}
	\begin{subfigure}
		{.5\textwidth}
		\centering
		\includegraphics[width=\linewidth]{img/bar_densenet_base_0.50.png}
	\end{subfigure}%
	\begin{subfigure}
		{.5\textwidth}
		\centering
		\includegraphics[width=\linewidth]{img/bar_densenet_refine_0.50.png}
	\end{subfigure}
	\begin{subfigure}
		{.5\textwidth}
		\centering
		\includegraphics[width=\linewidth]{img/bar_densenet_base_1.00.png}
	\end{subfigure}%
	\begin{subfigure}
		{.5\textwidth}
		\centering
		\includegraphics[width=\linewidth]{img/bar_densenet_refine_1.00.png}
	\end{subfigure}
	
	\caption{Comparación de los códigos de barras del modelo DenseNet-121 para el
		refinamiento en la especificidad Marca-Modelo sin regularización (izquierda)
		frente al modelo refinado con regularización (derecha). La figura muestra los códigos
		de barras para el conjunto de test tras avanzar un $1\%$ (arriba), un $50\%$ (centro)
		y el $100\%$ (abajo) de la red.}
	\label{fig:densenet-refine-bc}
\end{figure}

\subsection{Mejora en transferibilidad}

\paragraph{Transferencia EfficientNet-B0: Marca $\to$ Marca-Modelo}

La \autoref{tab:efficientnet-transfer} muestra resultados interesantes en relación
a cómo la topología afecta a la transferencia de conocimiento. Al aumentar la
granularidad del conjunto de clases y la acción del regularizador topológico
para aumentar la persistencia total normalizada de los datos, vemos que la capacidad
de clasificación del modelo mejora notablemente frente a la alternativa sin regularizar.
En el caso donde el valor $\alpha = -0.01$, el modelo llega a mejorar la exactitud
del modelo más de un 4\% y el F1-Score más de un 5\% por encima de la transferencia
base.

\begin{table}[H]
	\centering
	\begin{adjustbox}
		{width=0.8\textwidth}
		\begin{tabular}{|c|c|c|c|c|c|}
			\hline
			\textbf{Modelo}           & $\alpha$ & \textbf{Exactitud} & \textbf{Precisión} & \textbf{Sensibilidad} & \textbf{F1-Score} \\
			\hline
			EfficientNet-B0 Fine-Tune & 0.0      & 0.8333             & 0.7724             & 0.8053                & 0.7884            \\
			\hline
			EfficientNet-B0 Fine-Tune & -0.001   & 0.8407             & 0.7882             & 0.8162                & 0.8018            \\
			\hline
			EfficientNet-B0 Fine-Tune & -0.005   & 0.8407             & 0.7959             & 0.8193                & 0.8074            \\
			\hline
			EfficientNet-B0 Fine-Tune & -0.01    & \textbf{0.8741}    & \textbf{0.8276}    & \textbf{0.8522}       & \textbf{0.8397}   \\
			\hline
			EfficientNet-B0 Fine-Tune & -0.05    & 0.7963             & 0.7379             & 0.7624                & 0.7499            \\
			\hline
			EfficientNet-B0 Fine-Tune & -0.1     & 0.6000             & 0.5200             & 0.5631                & 0.5405            \\
			\hline
			EfficientNet-B0 Fine-Tune & -0.5     & 0.0222             & 0.0006             & 0.0218                & 0.0011            \\
			\hline
			EfficientNet-B0 Fine-Tune & -1.0     & 0.0148             & 0.0010             & 0.0103                & 0.0016            \\
			\hline
		\end{tabular}
	\end{adjustbox}
	\caption{Comparación de métricas tras un proceso de \textit{fine-tuning} del modelo
		EfficientNet-B0 desde la especificidad Marca hacia la especificidad Marca-Modelo
		para distintos valores de $\alpha$ en el término de regularización.}
	\label{tab:efficientnet-transfer}
\end{table}

Los resultados de las métricas se ven reforzados por las tendencias en la
\autoref{fig:efficientnet-transfer}. Aunque apenas veamos diferencias en la evolución
de la persistencia total (\autoref{fig:efficientnet-transfer-1}), la
persistencia total normalizada muestra unos valores mayores en la parte final
respecto a la transferencia base (\autoref{fig:efficientnet-transfer-2}). Este hecho
es resultado del esfuerzo en aumentar la complejidad topológica, mostrando una
relación directa entre una clasificación más fina y la topología de los datos. Una
diferencia que vemos respecto a los resultados obtenidos en la \autoref{fig:efficientnet-2}
es la ausencia de una mayor complejidad en otras etapas de la red. Al estar
considerando solamente la salida final de la red en el regularizador, el optimizador
solo parece centrarse en complicar la topología de dichas salidas. En
consecuencia, podría ser interesante plantear este concepto con activaciones de varias
capas de la red.

\begin{figure}[H]
	\centering
	\begin{subfigure}
		{.5\textwidth}
		\centering
		\includegraphics[width=\linewidth]{img/exp_transfer_efficientnet.png}
		\caption{Persistencia total normalizada según el porcentaje de avance en la
			red para el modelo EfficientNet-B0 transferido a Marca-Modelo.}
		\label{fig:efficientnet-transfer-1}
	\end{subfigure}%
	\begin{subfigure}
		{.5\textwidth}
		\centering
		\includegraphics[width=\linewidth]{img/exp_transfer_efficientnet_norm.png}
		\caption{Persistencia total normalizada según el porcentaje de avance en la
			red para el modelo EfficientNet-B0 transferido a Marca-Modelo.}
		\label{fig:efficientnet-transfer-2}
	\end{subfigure}
	\caption{Comparación de la persistencia total (a) y la persistencia total
		normalizada (b) para EfficientNet-B0 transferido con SGD y un tamaño de lote
		32 en función del porcentaje de avance de los datos a través de la red desde
		la especificidad Marca a Marca-Modelo.}
	\label{fig:efficientnet-transfer}
\end{figure}

Si observamos la \autoref{fig:ex-efficientnet-transfer}, vemos que la transferencia
hacia el mismo \textit{dataset} con una clasificación más fina, el modelo necesita
ampliar el área donde se enfoca. En particular, vemos que el modelo regularizado
logra tener un enfoque más homogéneo en la parrilla del vehículo a cambio de introducir
ciertos artefactos lineales.

\begin{figure}[H]
	\centering
	\begin{subfigure}
		{.45\textwidth}
		\centering
		\includegraphics[width=\linewidth]{img/154_false.png}
		\caption{Ejemplo mal clasificado para el modelo EfficientNet-B0 transferido
			desde la especificidad Marca a Marca-Modelo.}
		\label{fig:ex-efficientnet-transfer-1}
	\end{subfigure}%
	\qquad
	\begin{subfigure}
		{.45\textwidth}
		\centering
		\includegraphics[width=\linewidth]{img/154_true.png}
		\caption{Ejemplo bien clasificado para el modelo EfficientNet-B0 transferido
			desde la especificidad Marca a Marca-Modelo.}
		\label{fig:ex-efficientnet-transfer-2}
	\end{subfigure}
	\caption{Comparación de clasificación para el caso sin regularizar (a) frente
		al regularizado (b) para la transferencia de EfficientNet-B0 a Marca-Modelo.}
	\label{fig:ex-efficientnet-transfer}
\end{figure}

Como consecuencia directa de aumentar la complejidad topológica de los datos, los
códigos de barras en el final de la red regularizada muestra intervalos algo más
numerosos y de longitud similar, de forma que disponemos de una mayor
persistencia total normalizada (\autoref{fig:efficientnet-transfer-bc}).

\begin{figure}[H]
	\centering
	% First Column
	\begin{subfigure}
		{.5\textwidth}
		\centering
		\includegraphics[width=\linewidth]{img/bar_efficientnet_trans_base_0.01.png}
	\end{subfigure}%
	\begin{subfigure}
		{.5\textwidth}
		\centering
		\includegraphics[width=\linewidth]{img/bar_efficientnet_trans_reg_0.01.png}
	\end{subfigure}
	\begin{subfigure}
		{.5\textwidth}
		\centering
		\includegraphics[width=\linewidth]{img/bar_efficientnet_trans_base_0.50.png}
	\end{subfigure}%
	\begin{subfigure}
		{.5\textwidth}
		\centering
		\includegraphics[width=\linewidth]{img/bar_efficientnet_trans_reg_0.50.png}
	\end{subfigure}
	\begin{subfigure}
		{.5\textwidth}
		\centering
		\includegraphics[width=\linewidth]{img/bar_efficientnet_trans_base_1.00.png}
	\end{subfigure}%
	\begin{subfigure}
		{.5\textwidth}
		\centering
		\includegraphics[width=\linewidth]{img/bar_efficientnet_trans_reg_1.00.png}
	\end{subfigure}
	
	\caption{Comparación de los códigos de barras del modelo EfficientNet-B0 para
		la transferencia de la especificidad Marca a Marca-Modelo sin regularización (izquierda)
		frente al modelo transferido con regularización (derecha). La figura muestra los
		códigos de barras para el conjunto de test tras avanzar un $1\%$ (arriba), un
		$50\%$ (centro) y el $100\%$ (abajo) de la red.}
	\label{fig:efficientnet-transfer-bc}
\end{figure}

\paragraph{Transferencia Densenet-121: Marca-Modelo $\to$ Marca}

Por su parte, la transferencia desde Marca-Modelo a Marca ha mostrado métricas
más mucho más altas que las vistas en la transferencia opuesta (\autoref{tab:densenet-transfer}).
A pesar de la similitud entre los resultados obtenidos, vemos que los mejores
resultados los han obtenido las variantes para los valores de $\alpha$ iguales a
0.001 y 0.005. Más aún, es interesante ver cómo alteraciones similares de la
topología de los datos pueden llevar los resultados obtenidos a los mismos
valores. %Una hipótesis podría sugerir que la alteración de la homología persistente en CNNs converge de manera discreta a las mismas estructuras en función de varios umbrales de $\alpha$.

\begin{table}[H]
	\centering
	\begin{adjustbox}
		{width=0.8\textwidth}
		\begin{tabular}{|c|c|c|c|c|c|}
			\hline
			\textbf{Modelo}        & $\alpha$ & \textbf{Exactitud} & \textbf{Precisión} & \textbf{Sensibilidad} & \textbf{F1-Score} \\
			\hline
			DenseNet-121 Fine-Tune & 0.0      & 0.9907             & 0.9816             & 0.9864                & 0.9840            \\
			\hline
			DenseNet-121 Fine-Tune & 0.001    & \textbf{0.9938}    & \textbf{0.9834}    & \textbf{0.9882}       & \textbf{0.9857}   \\
			\hline
			DenseNet-121 Fine-Tune & 0.005    & \textbf{0.9938}    & \textbf{0.9834}    & \textbf{0.9882}       & \textbf{0.9857}   \\
			\hline
			DenseNet-121 Fine-Tune & 0.01     & 0.9876             & 0.9747             & 0.9812                & 0.9779            \\
			\hline
			DenseNet-121 Fine-Tune & 0.05     & 0.9443             & 0.8895             & 0.9077                & 0.8983            \\
			\hline
			DenseNet-121 Fine-Tune & 0.1      & 0.8111             & 0.6628             & 0.7252                & 0.6919            \\
			\hline
			DenseNet-121 Fine-Tune & 0.5      & 0.0194             & 0.0738             & 0.0738                & 0.0287            \\
			\hline
			DenseNet-121 Fine-Tune & 1.0      & 0.0743             & 0.0048             & 0.0546                & 0.0086            \\
			\hline
		\end{tabular}
	\end{adjustbox}
	\caption{Comparación de métricas tras un proceso de \textit{fine-tuning} del modelo
		DenseNet-121 desde la especificidad Marca-Modelo hacia la especificidad Marca para
		distintos valores de $\alpha$ en el término de regularización.}
	\label{tab:densenet-transfer}
\end{table}

Pese a la poca diferencia entre métricas, los cambios en homología persistente muestran
varios aspectos de interés. En esta ocasión, la simplificación de la persistencia
total normalizada se ha reflejado en la persistencia total a través de una simplificación
al comienzo de la red, tal y como muestra la \autoref{fig:densenet-transfer-1}. Por
otra parte, la \autoref{fig:densenet-transfer-2} muestra claramente como el
regularizador ha funcionado como se esperaba. Es curioso ver cómo para los casos
donde $\alpha = 0.001$ y $\alpha = 0.005$, cuyas métricas son idénticas, obtenemos
diferencias en complejidad topológica tan notables.

\begin{figure}[H]
	\centering
	\begin{subfigure}
		{.5\textwidth}
		\centering
		\includegraphics[width=\linewidth]{img/exp_transfer_densenet.png}
		\caption{Persistencia total según el porcentaje de avance en la red para el
			modelo DenseNet-121 transferido desde la especificidad Marca-Modelo a Marca.}
		\label{fig:densenet-transfer-1}
	\end{subfigure}%
	\begin{subfigure}
		{.5\textwidth}
		\centering
		\includegraphics[width=\linewidth]{img/exp_transfer_densenet_norm.png}
		\caption{Persistencia total normalizada según el porcentaje de avance en la
			red para el modelo DenseNet-121 transferido desde la especificidad Marca-Modelo
			a Marca.}
		\label{fig:densenet-transfer-2}
	\end{subfigure}
	\caption{Comparación de la persistencia total (a) y la persistencia total
		normalizada (b) para DenseNet-121 transferido con SGD y un tamaño de lote 32
		en función del porcentaje de avance de los datos a través de la red desde la
		especificidad Marca-Modelo a Marca.}
	\label{fig:densenet-transfer}
\end{figure}

Finalmente, los códigos de barras en la \autoref{fig:densenet-transfer-bc} reflejan
el comportamiento esperado en las capas finales, donde el modelo regularizado
muestra intervalos de persistencia más cortos que su par (\autoref{fig:densenet-transfer-bc}).
Pese al aumento de persistencia total en las capas iniciales, los códigos de
barras no parecen dar una respuesta a esta cuestión.

\begin{figure}[H]
	\centering
	% First Column
	\begin{subfigure}
		{.5\textwidth}
		\centering
		\includegraphics[width=\linewidth]{img/bar_densenet_trans_base_0.01.png}
	\end{subfigure}%
	\begin{subfigure}
		{.5\textwidth}
		\centering
		\includegraphics[width=\linewidth]{img/bar_densenet_trans_reg_0.01.png}
	\end{subfigure}
	\begin{subfigure}
		{.5\textwidth}
		\centering
		\includegraphics[width=\linewidth]{img/bar_densenet_trans_base_0.50.png}
	\end{subfigure}%
	\begin{subfigure}
		{.5\textwidth}
		\centering
		\includegraphics[width=\linewidth]{img/bar_densenet_trans_reg_0.50.png}
	\end{subfigure}
	\begin{subfigure}
		{.5\textwidth}
		\centering
		\includegraphics[width=\linewidth]{img/bar_densenet_trans_base_1.00.png}
	\end{subfigure}%
	\begin{subfigure}
		{.5\textwidth}
		\centering
		\includegraphics[width=\linewidth]{img/bar_densenet_trans_reg_1.00.png}
	\end{subfigure}
	
	\caption{Comparación de los códigos de barras del modelo DenseNet-121 para la
		transferencia de la especificidad Marca-Modelo a Marca sin regularización (izquierda)
		frente al modelo con regularización (derecha). La figura muestra los códigos de
		barras para el conjunto de test tras avanzar un $1\%$ (arriba), un $50\%$ (centro)
		y el $100\%$ (derecha) de la red.}
	\label{fig:densenet-transfer-bc}
\end{figure}

A diferencia de la transferencia anterior, al reducir el número de clases vemos
que el modelo focaliza su análisis en un área más concentrada (\autoref{fig:ex-densenet-transfer}).
La \autoref{fig:ex-densenet-transfer-2} muestra un área de enfoque más reducida que
el caso sin regularizar, centrándose en las características más relevantes y
evitando otras que podrían introducir ruido.

\begin{figure}[H]
	\centering
	\begin{subfigure}
		{.45\textwidth}
		\centering
		\includegraphics[width=\linewidth]{img/94_false.png}
		\caption{Ejemplo mal clasificado para el modelo DenseNet-121 transferido
			desde la especificidad Marca-Modelo a Marca.}
		\label{fig:ex-densenet-transfer-1}
	\end{subfigure}%
	\qquad
	\begin{subfigure}
		{.45\textwidth}
		\centering
		\includegraphics[width=\linewidth]{img/94_true.png}
		\caption{Ejemplo bien clasificado para el modelo DenseNet-121 transferido
			desde la especificidad Marca-Modelo a Marca.}
		\label{fig:ex-densenet-transfer-2}
	\end{subfigure}
	\caption{Comparación de clasificación para el caso sin regularizar (a) frente
		al regularizado (b) para la transferencia de DenseNet-121 a Marca.}
	\label{fig:ex-densenet-transfer}
\end{figure}

\subsection{Impacto de la regularización topológica en CNNs}

La implementación de un regularizador topológico en modelos como EfficientNet-B0
y DenseNet-121 ha demostrado ser prometedora, mejorando tanto la clasificación como
la transferencia de conocimiento. Este enfoque ajusta la complejidad topológica
de los datos, lo que parece reflejarse en la mejora general del rendimiento del modelo.

Para el modelo EfficientNet-B0, el uso del regularizador topológico ha mostrado
una mejora leve pero significativa en comparación con versiones sin
regularización. Las observaciones realizadas indican que ajustar la topología,
principalmente en etapas finales del aprendizaje, puede afinar la capacidad del modelo
para diferenciar entre clases y por tanto haciendo más efectiva la clasificación
final. Esto se ve en cómo el modelo regularizado presenta intervalos más cortos
de persistencia en instantes más cercanos al final del proceso de inferencia.

En el caso de DenseNet-121, la aplicación del regularizador ha tenido un impacto
considerable, optimizando de manera efectiva la estructura de los datos para
mejorar el rendimiento del modelo. Los cambios sutiles en la homología al final de
la red han demostrado ser importantes, reduciendo la cantidad de componentes
conexas y aumentando la duración de las clases de homología persistente de
dimensiones superiores, lo que sugiere una manipulación estratégica de la topología
para beneficiar la tarea de clasificación.

En términos de transferibilidad, los estudios sobre EfficientNet-B0 y DenseNet-121
muestran que el regularizador topológico no solo mejora la clasificación en el
dominio original, sino que también facilita la transferencia de conocimiento al menos
hacia dominios de diferente granularidad. Esto se muestra en la mejora de la
clasificación cuando se aumenta la granularidad de las clases, y cómo la complejidad
topológica ajustada al final del proceso contribuye de manera notable a estos resultados.
La relación entre una clasificación más precisa y la topología de los datos
sugiere que enfocarse en optimizar la topología en las salidas finales puede ser
una estrategia eficaz.

Sin embargo, la selección del valor de $\alpha$ para el regularizador ha mostrado
ser relevante y debe hacerse con cuidado. Un valor inadecuado puede llevar a
resultados subóptimos, indicando la sensibilidad de la técnica y la necesidad de
entender profundamente cómo las modificaciones topológicas influencian el aprendizaje.
Por este motivo, el estudio realizado sugiere escoger valores de $\alpha$ entre $0
.001$ y $0.1$.

\endinput
%--------------------------------------------------------------------
% FIN DEL CAPÍTULO.
%--------------------------------------------------------------------
% !TeX root = ../tfg.tex
% !TeX encoding = utf8

\chapter{Conclusión}

\section{Trabajo futuro}
\section{Conclusión}

\endinput
%--------------------------------------------------------------------
% FIN DEL CAPÍTULO. 
%--------------------------------------------------------------------


% -------------------------------------------------------------------
% APPENDIX: Opcional
% -------------------------------------------------------------------

\appendix % Reinicia la numeración de los capítulos y usa letras para numerarlos
\pdfbookmark[-1]{Apéndices}{appendix} % Alternativamente podemos agrupar los apéndices con un nuevo \part{Apéndices}

%\input{apendices/apendice-ejemplo}
% Añadir tantos apéndices como sea necesario 

% -------------------------------------------------------------------
% GLOSARIO: Opcional
% -------------------------------------------------------------------

%\input{glosario} 

% -------------------------------------------------------------------
% BACKMATTER
% -------------------------------------------------------------------

\backmatter % Desactiva la numeración de los capítulos
\pdfbookmark[-1]{Referencias}{BM-Referencias}

% BIBLIOGRAFÍA
%-------------------------------------------------------------------

\bibliographystyle{alpha-es} 
\begin{small} 
  \bibliography{library.bib}
\end{small}


\end{document}
